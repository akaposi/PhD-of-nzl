\documentclass[12pt, mathserif]{beamer} 
\usepackage{etex}

\usecolortheme{lily}

\usepackage[latin1]{inputenc}
\usetheme{AnnArbor} 
\usecolortheme{sidebartab}
\setbeamercovered{highly dynamic} 
\usepackage{xspace}
\usepackage{ifthen}
\usepackage{amsmath}
\usepackage{amssymb}
\usepackage{stmaryrd}

\usepackage[english]{babel}
\usepackage{graphicx}
%\usepackage{proof}
\usepackage{multimedia}
\usepackage{animate}
\usepackage[all]{xy}
\usepackage{booktabs}
\usepackage[nobibnewpage,notocbib]{apacite}
\usepackage[absolute,overlay,quiet]{textpos}

\definecolor{blueish}{rgb}{0.2, 0.2, 0.7}

\usepackage[T1]{fontenc}
\usepackage{cmbright}
\usepackage{fancybox}

\usepackage[inference]{semantic}


%\definecolor{orangecol}{RGB}{70,30,30}


\setbeamercolor{yellow}{fg=black,bg=yellow}
\setbeamercolor{lightyellow}{fg=black,bg=yellow!40}
\setbeamercolor{orange}{fg=black,bg=orange}
\setbeamercolor{lightorange}{fg=black,bg=orange!40}
\setbeamercolor{green}{fg=black,bg=green}
\setbeamercolor{lightgreen}{fg=black,bg=green!40}
\setbeamercolor{blue}{fg=black,bg=blue}
\setbeamercolor{lightblue}{fg=black,bg=blue!40}
\setbeamercolor{red}{fg=black,bg=red}
\setbeamercolor{lightred}{fg=black,bg=red!40}

\setbeamercovered{invisible}


\newenvironment{colorblock}[2]
{\setbeamercolor{item}{fg=#1,bg=#1}\begin{beamerboxesrounded}[upper=#1,lower=#2,shadow=true]}
{\end{beamerboxesrounded}}

\newcommand{\type}{\textit{ : type}}
\newcommand{\set}{\textit{Set}}
\newcommand{\ruleType}[2]{#1 \vdash #2 \, \type}
\newcommand{\ruleTerm}[3]{#1 \vdash #2 \, : \, #3}
\newcommand{\ruleTypeE}[1]{\vdash #1 \, \type}
\newcommand{\ruleTermE}[2]{\vdash #1 \, : \, #2}
\newcommand{\refl}[1]{\textit{refl}_#1}
\newcommand{\judgeTerm}[2]{#1 \, : \, #2}
\newcommand{\judgeType}[1]{#1 \, : \, \textit{type}}

%\addtolength{\itemsep}{0.5\baselineskip}

\author{
Nicolai Kraus
}

%\institute{
%School of Computer Science \\  University of Nottingham, UK
%}






\date[16/11/12]{16/11/12}
\title{Homotopy Type Theory and Hedberg's Theorem}
\subtitle[Birmingham]{}


\addtobeamertemplate{sidebar right}{}{\vspace{-2cm}}
\addtobeamertemplate{footline}{}{\vspace{-1cm}\hskip2pt(\insertframenumber/\inserttotalframenumber) \insertshortsubtitle{} -- \insertshortdate\vskip2.2pt}
\setbeamercolor{footline}{fg=blueish}

%\includeonlyframes{current}


\begin{document}
\frame{\titlepage}


\begin{frame}
\frametitle{Reminder: Type Theory}
% \pause{
\begin{colorblock}{green}{lightyellow}{\center{Intensional Type Theory}}
\center{
\pause {a formal system} \\
\pause {\ldots and a possible foundation of (constructive) mathematics} \\
\pause{\ldots for proof assistants and (dependently typed) programming} \\
\pause{\ldots as used for Coq and Agda}
}
\end{colorblock}
%}

\vspace{15pt}

\pause{\center e.g.\\
\qquad \qquad $\lambda f \, \to \, \lambda a \, \to \, f \, a \, a \: : \: (A \to A \to B) \to A \to B$
}


\end{frame}


\section{Equality}

\begin{frame}
\frametitle{Reminder: Equality}
% \pause{
\begin{colorblock}{green}{lightyellow}{\center{Definitional Equality}
}
\center{
Decidable equality for typechecking \& computation; e.\,g. $(\lambda a . b) x =_\beta b[x/a]$}
\end{colorblock}
%}

\vspace{15pt}

\pause{
\begin{colorblock}{green}{lightyellow}{\center{Propositional Equality}}
\center{Equality needing a proof, i.\,e. a term of the identity type }
\end{colorblock}
}
\end{frame}



\begin{frame}
\frametitle{Reminder: Identity Types}



\begin{colorblock}{green}{lightyellow}{\center{Propositional equality}}
\center
\ldots is just an inductive type
\end{colorblock}


\begin{columns}
\begin{column}{.4\textwidth}

\begin{pause}
\begin{colorblock}{orange}{lightyellow}{\center{Formation}}
\center
$\inference
{\judgeTerm {a, b} A}{\judgeType {a \equiv b}}$ 
\end{colorblock}
\end{pause} 



\vspace{22pt} % ???

\begin{pause}
\begin{colorblock}{orange}{lightyellow}{\center{Introduction}}
\center
$\inference
{\judgeTerm a A}{\judgeTerm {\refl a} {a \equiv a}}$ 
\end{colorblock}
\end{pause}



\end{column}
\hfill
\begin{column}{.55\textwidth}


\begin{pause}
\begin{colorblock}{orange}{lightyellow}{\center{Elimination ($J$)}}
\center
\begin{equation*}
  \inference{ 
      \judgeTerm P {(a, b : A) \to a\equiv b \to \set} \\
      m : \forall a . P(a, a, \refl a)
  }
  { \judgeTerm {J_{(a, b, q)}} {P(a, b, q)} }
\end{equation*}
\end{colorblock}
\end{pause}

\vspace{18pt}

\begin{pause}
\begin{colorblock}{orange}{lightyellow}{\center{Computation ($\beta$)}}
\center
$J_{(a, a, \refl a)} =_\beta m a$
\end{colorblock}
\end{pause}

\end{column}%
\end{columns}
\end{frame}


\begin{frame}
\frametitle{Uniqueness of Identity Proofs (UIP)}
\center {
Given $a : \, A$ and $p : \, a \equiv a$, 
can we prove $p \equiv \refl{a}$?\\
}
\vspace{18pt}



\begin{pause}
\begin{colorblock}{orange}{lightyellow}{\center{Axiom UIP}}
\center
\begin{equation*}
  \inference{ 
      \judgeTerm {p, \, q} {a \equiv b} 
  }
  { \judgeTerm {\textit{uip}} \, p \equiv q}
\end{equation*}
\end{colorblock}
\end{pause}

\vspace{12pt}

\begin{columns}
\begin{column}{.4\textwidth}


\begin{pause}
\begin{colorblock}{green}{lightyellow}{\center{Advantages}}
\center
Simple, \\
Good computational properties
\end{colorblock}
\end{pause}


\end{column}
\hfill
\begin{column}{.5\textwidth}
\begin{pause}
\begin{colorblock}{green}{lightyellow}{\center{Disadvantages}}
\center
Intuitively wrong, \\
impossible to express statements about equality, \\
isomorphic sets cannot be equal
\end{colorblock}
\end{pause}
\end{column}
\end{columns}

\end{frame}



\begin{frame}
\frametitle{Homotopic Model - technical details}

\center


\begin{itemize}
\item[] Voevodsky (and Awodey, independently, and others): \\
\vspace{8pt}

\item[] Without UIP: new model of Type Theory 
\item[] \qquad \qquad (types as weak $\omega$-groupoids)
\vspace{8pt}

\item[] \begin{itemize}
         \item \pause{best expressible in Simplicial Sets \emph{SSets} (the topos $\textit{Sets}^{\Delta^{\textit{op}}}$)}
	 \item \pause{realization functor $R : \, \textit{SSets} \, \to \, \textit{Top}$}
	 \item \pause{$R$ is a \emph{Quillen equivalence} of model categories}
	 \item \pause{$\Rightarrow$ (more or less) a model that uses topological spaces as types}
        \end{itemize}
\end{itemize}

\end{frame}






% \begin{frame}
% \frametitle{$J$ versus $K$ revisited}
% \center 
% 
% Fix $a : A$.
% 
% \vspace{12pt}
% 
% \begin{colorblock}{orange}{lightyellow}{\center{$J$ (variant of Paulin-Mohring)}}
% \center
% To show $\forall (b, q)_{: \Sigma (b : A).a \equiv b} \, . \, P(b,q)$, just prove $P(a, \refl a)$.
% \end{colorblock}
% 
% \vspace{12pt}
% 
% 
% \begin{colorblock}{orange}{lightyellow}{\center{$K$ (reformulated)}}
% \center
% $\forall q_{: a \equiv a} \, . \, q \equiv \refl a$
% \end{colorblock}
% \end{frame}




\begin{frame}
\frametitle{Hedberg's theorem}
\center
Fix a type $A$.
\vspace{8pt}

\center 
\begin{colorblock}{orange}{lightyellow}{\center{Decidable Equality}}
\center
$\textit{DecidableEquality} \: := \: \forall \, a \, b \, \to \left(a \equiv b \: + \: \neg \, a \equiv b\right)$
\end{colorblock}
\vspace{12pt}

\begin{colorblock}{orange}{lightyellow}{\center{Hedberg's theorem}}
\center
$\textit{DecidableEquality} \: \longrightarrow \: \textit{UIP}$
\end{colorblock}


\end{frame}


\begin{frame}
\frametitle{Hedberg's theorem}
\center


\begin{colorblock}{orange}{lightyellow}{\center{Constant Function}}
\center
$\textit{const} (f) \: := \: \forall \, a \, b \, \to \, fa \equiv fb$
\end{colorblock}
\vspace{12pt}

\begin{colorblock}{orange}{lightyellow}{\center{Constant Endofunction on Path Spaces}}
\center
$g \: : \: \forall \, a \, b \, \to \, a \equiv b \, \to \, a \equiv b$ \\
$\textit{path-const} (g) \, := \, \forall \, a \, b \, \to \textit{const} \, g_{ab}$
\end{colorblock}

\end{frame}

\begin{frame}
\frametitle{Hedberg's theorem}

\begin{colorblock}{lightred}{lightyellow}{\center{Lemma 1}}
\center
$\textit{DecidableEquality} \: \longrightarrow \: \Sigma_{g} \, . \,  \textit{path-const} (g)$
\end{colorblock}

\pause 
\emph{Proof. }
\begin{itemize}
\item Given ${\color{red}\textit{dec}} : \, \forall \, a \, b \, \to \left(a \equiv b \: + \: \neg \, a \equiv b\right)$.
\pause \item Given {\color{red}$a, b$}, we want: ${\color{red}g_{ab}} : \, a\equiv b \, \to \, a \equiv b$.
\vspace*{4pt}

\pause \item If ${\color{red}\textit{dec} \, a \, b = \textit{inr} \, \_ } \: $, then nothing to do
\pause \item If ${\color{red}\textit{dec} \, a \, b = \textit{inl} \, p } \: $, then ${\color{red}g_{ab} (\_) \, = \, p }$
\qquad $\qed$
\end{itemize}

\end{frame}



\begin{frame}
\frametitle{Hedberg's theorem}

\begin{colorblock}{lightred}{lightyellow}{\center{Lemma 2}}
\center
$\Sigma_{g} \, . \,  \textit{path-const} (g) \: \longrightarrow \: \textit{UIP}$
\end{colorblock}

\pause
\emph{Proof. }
\begin{itemize}
\item Given ${\color{red}\textit{g}} \: : \: \forall \, a \, b \, \to \, a \equiv b \, \to \, a \equiv b$ which is constant
\pause \item Given any {\color{red}$a, b$} $\, : \, A$ and {\color{red}$p, q$}$\,  : \, a \equiv b$.
\vspace*{4pt}

\pause \item Claim: ${\color{red}p \equiv \, (g_{aa}\refl a)^{-1} \circ g_{ab}(p)}$
\pause \item Proof with $J$: Just do it for ${\color{red}(a, a, \refl a)}$. That's true!
\vspace*{4pt}
\pause \item Same for $q$. But $g_{aa}$ and $g_{ab}$ are constant.\qquad $\qed$
\end{itemize}

\end{frame}


\begin{frame}
\frametitle{Generalizations of Hedberg's theorem}
\center
We have seen 
\vspace*{2pt}

\begin{colorblock}{lightred}{lightyellow}{\center{Lemma 1}}
\center
$\textit{DecidableEquality} \: \longrightarrow \: \Sigma_{g} \, . \,  \textit{path-const} (g)$
\end{colorblock}
\vspace*{4pt}

$\textit{DecidableEquality}$ is a very strong property. How about something weaker? 
\pause For example:
\vspace*{4pt}

\begin{colorblock}{orange}{lightyellow}{\center{Separated}}
\center
$\forall \, a \, b \, \to \, \neg\neg(a \equiv b) \, \to \, a \equiv b$
\end{colorblock}
\vspace*{4pt}


\begin{colorblock}{orange}{lightyellow}{\center{``general''}}
\center
$\forall \, a \, b \, \to \, \textit{[propositional evidence for } a\equiv b\textit{]} \, \to \, a \equiv b$
\end{colorblock}


\end{frame}



\begin{frame}
\frametitle{Propositions}
\center
So, what is ``propositional evidence''?
\vspace*{4pt}

\begin{pause}
\begin{colorblock}{orange}{lightyellow}{\center{Type $A$ is a Proposition if}}
\center
$\textit{prop}_A \, = \, \forall \, a \, b \, \to \, a\equiv b$
\end{colorblock}
\vspace*{4pt}

``at most one inhabitant''
\vspace*{4pt}
\end{pause}

\pause{Write \textbf{Prop} for this ``subset'' of \textbf{Type}}

\end{frame}



\begin{frame}
\frametitle{H-Propositional Reflection}
\center
$A$ some type. We want a way to say that $A$ is inhabited without giving away a specific inhabitant.
\vspace*{4pt}

\begin{pause}
\begin{colorblock}{orange}{lightyellow}{\center{H-propositional reflection}}
\center
$\,^* \, : \, \textbf{Type} \, \to \, \textbf{Prop}$ \\
is defined to be the left adjoint of $\textit{emb: }\textbf{Prop} \, \hookrightarrow \, \textbf{Type}$
\end{colorblock}
\end{pause}
\vspace*{4pt}

\begin{pause}
 This means: \\
 \vspace*{4pt}

 \center
 \begin{itemize}
  \item $A^*$ is in \textbf{Prop}\\
  \item $\eta \, : \, A \, \to \, A^*$
  \item if $P$ is a proposition and $A \, \to \, P$, then $A^* \, \to \, P$
 \end{itemize}

\end{pause}
\end{frame}


\begin{frame}
\frametitle{Generalizations of Hedberg's Theorem}
\center
``Propositional evidence for $a \equiv b$'' is now just [an inhabitant of] $(a \equiv b)^*$.
 \vspace*{4pt}

\begin{colorblock}{orange}{lightyellow}{\center{H-Separated}}
\center
$\forall \, a \, b \, \to \, (a \equiv b)^* \, \to \, a \equiv b$
\end{colorblock}
\vspace*{12pt}

\pause
\begin{colorblock}{lightred}{lightyellow}{\center{Theorem}}
\center
$\textit{h-separated}_A \, \longleftrightarrow \Sigma_{g} \, . \,  \textit{path-const} (g) \, \longleftrightarrow \, \textit{UIP}_A$
\end{colorblock}
\vspace*{4pt}

 

\end{frame}







\begin{frame}
\frametitle{Nearly uncountable many things to be done \ldots}
\begin{itemize}
\item Higher Inductive Types (see Mike Shulman's work)
\item Model construction with modern abstract (not point-set) homotopy theory
\item Constructive Simplicial Sets (the combinatorial version of what I have shown; see Thierry Coquand's / Simon Huber's work)
\item Univalent foundations / Univalence (``alternative'' to $K$) in general (see Voevodsky)
\item \ldots and possible computational properties (Thorsten?)
%\item Combine canonicity, strong normalization, extensionality?
\end{itemize}

\begin{center}
THANK YOU!
\end{center}
\end{frame}








\end{document}