\documentclass[12pt,pdflatex,hyperref={pdfstartview=Fit,bookmarks=true,bookmarksopen=true,pdfpagemode=None,colorlinks=true,linkcolor=unserblau,urlcolor=unserblau},notes=hide,t,handout]{beamer}

\usepackage{etex}

\usepackage[latin1]{inputenc}
%\usepackage{beamerthemelined}

\usetheme[width=0cm]{Boadilla}
%\usetheme[width=0cm]{CamebridgeUS}

\usecolortheme{sidebartab}
\setbeamercovered{highly dynamic} 
\usepackage{xspace}
\usepackage{ifthen}
\usepackage{amsmath}
\usepackage{amssymb}
\usepackage{stmaryrd}
\usepackage{graphicx}
\usepackage{color}
%\usepackage[usenames,dvipsnames]{color}

\usepackage[english]{babel}
%\usepackage[T1]{fontenc}
\usepackage{graphicx}
\usepackage{multimedia}

\usepackage[all]{xy}
\usepackage{proof}

\definecolor{unserblau}{rgb}{0.21, 0.21, 0.70}

\usepackage{booktabs}
\usepackage[nobibnewpage,notocbib]{apacite}
\usepackage[absolute,overlay,quiet]{textpos}
%\usepackage{ngerman}


\pagestyle{empty}

\author[Nicolai Kraus]{Nicolai Kraus}

\institute{Functional Programming Laboratory Away Day}
\date[2011-06-8]{8th July 2011}
\title{Homotopy Type Theory}
\subtitle[FP Away Day 2011]{}

% \pgfdeclareimage[width=2cm]{logo}{hulogo} \logo{\pgfuseimage{logo}}


\addtobeamertemplate{sidebar right}{}{\vspace{-2cm}}
\addtobeamertemplate{footline}{}{\vspace{-1cm}\hskip2pt(\insertframenumber/\inserttotalframenumber) \insertshortsubtitle{} -- \insertshortdate\vskip2.2pt}
\setbeamercolor{footline}{fg=unserblau}
%\setbeamercolor*{alerted text}{fg=unserblau}

%\includeonlyframes{current}

\begin{document}
\frame{\titlepage}

% \section*{Overview}
% \[squeeze]{\frametitle{Overview}\tableofcontents}



\section{Introduction}

\begin{frame}

\frametitle{What is it all about?}
\begin{center}{A connection...}\end{center}
\begin{columns}[u] % the "c" option specifies center vertical alignment 

\column{.5\textwidth} 

\vspace{3pt}

\begin{block}{} \Large \begin{center}
\textcolor{red}{\bf Type Theory}
\end{center}
\end{block}
\only<2->{related to logic}

\column{.5\textwidth} 

\vspace{3pt}
\begin{block}{} \Large \begin{center}
\textcolor{red}{\bf Topology}
\end{center}
\end{block}
\only<3->{related to
\begin{itemize}
\item Algebra
\item Analysis
\item Geometry
\item \ldots
\end{itemize}}
%not related to:
%\begin{itemize}
%\item logic
%\end{itemize}

\end{columns}


\end{frame}



\begin{frame}
\frametitle{So, what is Topology actually?}

\only<2->{\center{A major area of mathematics that examines continuity!}}

\vspace{0.8cm}

\only<3->{Sets often have a natural notion of \textcolor{red}{\emph {open}} subsets, e.g. in $\mathbb R$: 
\vspace{0.8cm}

$(1,2) := \{x \ | \ 1<x<2 \}$ is open, but $[1,2] := \{x \ | \ 1 \leq x \leq 2 \}$ is not.}
\vspace{0.8cm}

\only<4->{\begin{block}{Definition (Continuity of $f : X \rightarrow Y$):} $f$ is continuous iff inverse images of open sets are open, i.e. if $V \subset Y$ is open, so is $f^{-1}(V)$.
\end{block}}

\end{frame}


\begin{frame}
\frametitle{Remark}
The word "Continuity" is used a lot:
\begin{itemize}
\only<2->{\item functions $\mathbb R \rightarrow \mathbb R$. "Definition": $\forall x, \epsilon > 0 . \exists \delta > 0 . |x-x_0| < \delta \Rightarrow |f(x) - f(x_0)| < \epsilon$}
\only<3->{\item complete partial orders: $f ( \textit{sup } D) = \textit{sup }f(D)$ for every directed $D$}
\only<4->{\item "Computable functions are continuous."}
\only<4->{\item \ldots}
\end{itemize}

\only<5->{
all notions can be broken down to:

\begin{block}{} \center{\bf{Inverse images of open sets are open.}}
\end{block}
}

\end{frame}


\section{Identity Types}

\begin{frame}

\frametitle{Identity Types without UIP - a Reminder}

\begin{columns}[c] % the "c" option specifies center vertical alignment 

\column{.5\textwidth} 

\vspace{30pt}

\[
\infer{a \equiv b \quad \mbox{Type}}{a, b : A}
\]

\vspace{30pt}

\[
\infer{\mbox{refl}_a : a \equiv a}{}
\]


\column{.5\textwidth} 

\vspace{30pt}

\[
\begin{array}{c@{ \ : \ }l}
\noalign{\medskip}
p & a \equiv b \\
\noalign{\medskip}

p^{-1} & b \equiv a \\
\noalign{\medskip}

q & b \equiv c \\
\noalign{\medskip}

q \circ p & a \equiv c \\
\end{array}
\]

\end{columns}

\end{frame}


\begin{frame}

\frametitle{Identity Types without UIP - a Reminder}


\end{frame}


\begin{frame}
\frametitle{Back to Topology}

\end{frame}




\begin{frame}
\only<1-7,9-11,13->{\frametitle{A disc}}
\only<8>{\frametitle{Another set}}
\only<12>{\frametitle{A ring}}
\begin{columns}[c]

\column{.3\textwidth} 

\begin{itemize}
\only<1>{\item[] a type - we call it $X$}
\only<2>{\item[] two terms}
\only<3>{\item[] ?}
\only<4>{\item[] $a, b \, : \, X$ \item[] $p \, : \, a \equiv b$}
\only<5>{\item[] $p^{-1} \, : \, b \equiv a$}
\only<6>{\item[] $p \, : \, a \equiv b$ \item[] $q \, : \, b \equiv c$}
\only<7>{\item[] $q \circ p \, : \, a \equiv c$}
\only<8>{\item[] $a \equiv c$ not inhabited}
\only<9>{\item[] $p, p' \, : \, a \equiv b$}
\only<10,11,13,16>{\item[] $H \, : \, p \equiv p'$}
\only<14>{\item[] $H' \, : \, p \equiv p'$}
\only<15>{\item[] $K \, : \, H' \equiv H$}
\end{itemize}

\column{.4\textwidth}

\includegraphics<1>[width=\textwidth]{pictures/1circle}
\includegraphics<2>[width=\textwidth]{pictures/2circle_points}
\includegraphics<3>[width=\textwidth]{pictures/3circle_line}
\includegraphics<4>[width=\textwidth]{pictures/4circle_named}
\includegraphics<5>[width=\textwidth]{pictures/5circle_named_inversed}
\includegraphics<6>[width=\textwidth]{pictures/6circle_three}
\includegraphics<7>[width=\textwidth]{pictures/7circle_threetwo}
\includegraphics<8>[width=\textwidth]{pictures/8not_connected}
\includegraphics<9>[width=\textwidth]{pictures/9circle2paths}
\includegraphics<10,11,13,16>[width=\textwidth]{pictures/10circleHomAlt}
\includegraphics<12>[width=\textwidth]{pictures/11ring}
%\includegraphics<12>[scale=0.3]{pictures/10circleHomAlt}
\includegraphics<14>[width=\textwidth]{pictures/12circleHom2}
\includegraphics<15>[width=\textwidth]{pictures/13circleHomhom}


\column{.35\textwidth} 
\begin{itemize}
\only<1>{\item[] a topological space - we call it $X$}
\only<2>{\item[] two points}
\only<3>{\item[] a path}
\only<4>{\item[] $a, b \, \in \, X$ \noalign{\medskip} \item[] $p \, : \, [0,1] \rightarrow X$ \item[] $p (0) = a$ \item[] $p (1) = b$}
\only<5>{\item[]  \item[] $p^{-1} \, : \, [0,1] \rightarrow X$ \item[] $p^{-1} (t) = p(1-t)$}
\only<6>{\item[] $a, b \, \in \, X$ \noalign{\medskip} \item[] $p \, : \, [0,1] \rightarrow X$ \item[] $p  (0) = a$ \item[] $p (1) = b$ \noalign{\medskip} \item[] $q \, : \, [0,1] \rightarrow X$ \item[] $q (0) = b$ \item[] $q (1) = c$}
%\only<6>{\item[] $a, b \, \in \, X$ \item[] $p \, : \, [0,1] \rightarrow X$ \item[] $p \, 0 = a$ \item[] $p \, 1 = b$ \item[] $q \, : \, [0,1] \rightarrow X$ \item[] $q \, 0 = b$ \item[] $q \, 1 = c$}
\only<7>{\item[] $q \circ p \, : \,$ \item[] $ \ [0,1] \rightarrow X$  \item[] $x \mapsto \begin{cases} p(2x), x < 0.5 \\ q(2x-1), \mbox{else} \end{cases}$}
\only<8>{\item[] not path-connected}
\only<9>{\item[] $p, p' \, : \, [0,1] \rightarrow X$}
\only<10,11,13,16>{\item[] $H \, : \, [0,1]^2 \rightarrow X$ \item[] $H(0,\cdot) = p$ \item[] $H(1,\cdot) = p'$ \item[] $H(t,0) = a$ \item[] $H(t,1) = b$}
\only<11>{\vspace{1cm} \item[] $p \, : \, [0,1]^1 \rightarrow X$ \item[] $a \, : \, [0,1]^0 \rightarrow X$}
\only<14>{\item[] $H' \, : \, [0,1]^2 \rightarrow X$ \item[] $H'(0,\cdot) = p$ \item[] $H'(1,\cdot) = p'$ \item[] $H'(t,0) = a$ \item[] $H'(t,1) = b$}
\only<15>{\item[] $K \, : \, [0,1]^3 \rightarrow X$ \item[] $K(0,\cdot, \cdot) = H'$ \item[] \ldots}
\end{itemize}

\end{columns}

\end{frame}


\begin{frame}

\frametitle{Putting it together}

\begin{columns}[c] % the "c" option specifies center vertical alignment 

\column{.5\textwidth} 
\begin{center}
\xymatrix{ 
a 
\ar@/_4pc/[ddd]^p="p" 
\ar@/^4pc/[ddd]_{p'}="p'" 
\\
\\
\\
b
\ar@2{->} @/^2pc/ "p";"p'" _{H}="H"
\ar@2{->} @/_2pc/ "p";"p'" ^{H'}="H'"
{\ar@3{<-} "H";"H'"_{K}}
}
\end{center}

\column{.5\textwidth} 

\includegraphics[scale=0.3]{pictures/13circleHomhom}

\end{columns}

\end{frame}

\begin{frame}
\frametitle{So, which types can we get?}

\vspace{1.5cm}

\begin{columns}[c] % the "c" option specifies center vertical alignment 
\column{.25\textwidth} 
\includegraphics[width=\textwidth]{pictures/1circle}

\column{.25\textwidth} 
\includegraphics[width=\textwidth]{pictures/11ringb}

\column{.25\textwidth} 
\includegraphics[width=\textwidth]{pictures/wikisphere}

\column{.25\textwidth} 
\includegraphics[width=\textwidth]{pictures/wikitorus}

\end{columns}

\center{any CW complex?}

\end{frame}

\begin{frame}
\frametitle{Where is it going?}

\begin{center} All has been done in abstract homotopy theory. \end{center}

\vspace{1pt}

\only<2->{What I (at the moment) hope:
\vspace{1pt}

\begin{itemize}
\item Creating a simple model
\item that is complete
\item and easy to understand and to use
\end{itemize}
}
\end{frame}


\begin{frame}
\frametitle{Where is it going?}
\vspace{1pt}

e.g. for this problem (Thorsten): 
\vspace{1pt}

\begin{itemize}
\item subst-refl P (subst P (refl x) p)
\item[] and
\item cong (subst P (refl x)) (subst-refl P p)

%\only<2->{
\item[] both prove
\item subst P (refl x) (subst P (refl x) p) $\equiv$ subst P (refl x) p
\item[] But are they equal?
%}
\end{itemize}
\end{frame}


%
%\begin{frame}
%\begin{block}{Definition (Topological space):} A set $X$ together with $\mathfrak T \subset P(X)$ that satisfies $ \emptyset, X \in \mathfrak T$, and $\mathfrak T$ is closed under arbitrary union and finite intersection.
%\end{block}
%examples: basically any set, especially $\mathbb R^n$, $\mathbb C^n$, sets of functions between those, ordinals, any complete partial order, \ldots
%
%\begin{block}{Definition (Continuity of $f : X \rightarrow Y$):} Inverse images of open sets are open, i.e. if $V \subset Y$ is open, so is $f^{-1}(V)$.
%\end{block}
%\end{frame}


\end{document}