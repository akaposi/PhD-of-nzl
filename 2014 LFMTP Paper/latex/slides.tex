\documentclass[12pt, mathserif,handout]{beamer}

%\documentclass[12pt, mathserif]{beamer}

\usepackage{etex}


\usepackage{agda}

\usepackage{mypack}

\usepackage{ucs}
\usepackage[utf8x]{inputenc}

\usepackage{relsize}
\usepackage{autofe}
\usepackage{textgreek}

\setbeamertemplate{navigation symbols}{}


\usetheme{Madrid}
\setbeamercovered{highly dynamic} 
\usepackage{xspace}
\usepackage{ifthen}
\usepackage{amsmath}
\usepackage{amssymb}
\usepackage{stmaryrd}
\usepackage{proof}

\usepackage[english]{babel}
\usepackage{graphicx}
\usepackage{multimedia}
\usepackage{animate}
\usepackage[all]{xy}
\usepackage{booktabs}
\usepackage[nobibnewpage,notocbib]{apacite}
\usepackage[absolute,overlay,quiet]{textpos}

\definecolor{blueish}{rgb}{0.2, 0.2, 0.7}

\usepackage[T1]{fontenc}
\usepackage{cmbright}
\usepackage{fancybox}

\usepackage[inference]{semantic}

% Background Color

\definecolor{rice}{RGB}{245,245,220}
\setbeamercolor{background canvas}{bg=rice}

\setbeamercolor{yellow}{fg=black,bg=yellow}
\setbeamercolor{lightyellow}{fg=black,bg=yellow!40}
\setbeamercolor{orange}{fg=black,bg=orange}
\setbeamercolor{lightorange}{fg=black,bg=orange!40}
\setbeamercolor{green}{fg=black,bg=green}
\setbeamercolor{lightgreen}{fg=black,bg=green!40}
\setbeamercolor{blue}{fg=black,bg=blue}
\setbeamercolor{lightblue}{fg=black,bg=blue!40}
\setbeamercolor{red}{fg=black,bg=red}
\setbeamercolor{lightred}{fg=black,bg=red!40}
\setbeamercolor{lightpink}{fg=black,bg=pink!40}


%\setbeamercolor{frametitle}{fg=white,bg=orange}
%\setbeamercolor{title}{fg=white,bg=orange}

\setbeamercovered{invisible}

%\begin{colorblock}{orange}{lightyellow}{\center{Heterogeneous equality for Tm}}


\newenvironment{colorblock}[2]
{\setbeamercolor{item}{fg=#1,bg=#1}\begin{beamerboxesrounded}[upper=#1,lower=#2,shadow=true]}
{\end{beamerboxesrounded}}



% shorthands


\author{Thorsten Altenkirch \& Nuo Li \& Ond\v{r}ej Ryp\'a\v{c}ek}

\author[Thorsten Altenkirch \& Nuo Li \& Ond\v{r}ej
Ryp\'a\v{c}ek]{Thorsten Altenkirch \inst{1} \and Nuo Li \inst{1} \and
  Ond\v{r}ej Ryp\'a\v{c}ek \inst{2}}
\institute[shortinst]{\inst{1} 
School of Computer Science \\  University of Nottingham, UK \and %
                      \inst{2} University of Oxford, UK}






\date[17/07/14]{17/07/14}
\title{Some constructions on $\omega$-groupoids}
\subtitle[LFMTP 2014]{}


\addtobeamertemplate{sidebar right}{}{\vspace{-2cm}}
\addtobeamertemplate{footline}{}{\vspace{-1cm}\hskip2pt(\insertframenumber/\inserttotalframenumber) \insertshortsubtitle{} -- \insertshortdate\vskip2.2pt}
\setbeamercolor{footline}{fg=purple}


\begin{document}

% 1
\frame{\titlepage}


% 2

\begin{frame}{Outline}
\begin{itemize}
\item Introduction to \wog
\item Basic syntax of \tig (describing the structure of \wog)
\item Heterogeneous equality for syntactic terms
\item Suspension and replacement
\item Coherences constructions
\item Semantic interpretation
\end{itemize}
\end{frame}




\begin{frame}[allowframebreaks,t]{Introduction to \wog}

\begin{itemize}
\item To internalize equality in types : setoids, groupoids

\item A new interpreted of types : \wog (In \hott)
\begin{itemize}
\item \wogs: A higher dimensional category ($\omega$-category) where every
  morphism is an weak equivalence (equivalence for short)
\item Equivalence: invertible morphism up to all higher
  equivalence (generalization of isomorphism)
\item Weak: equality in coherence laws are up to equivalence (not
    strictly equal) e.g. $(f \circ g) \circ h \to f \circ (g \circ h)$


% \begin{center}
% \xymatrix{ 
% a
% \ar@/_4pc/[ddd]^p="p" 
% \\
% \\
% \\
% b 
% \ar@/_4pc/[uuu]^{p^{-1}}="p^{-1}"" 
% \only<2->{\ar@2{->} @/^2pc/ "p";"p^{-1}" _{H}="H"}
% \only<3->{\ar@2{<-} @/_2pc/ "p";"p^{-1}" ^{H ^{-1}}="H^{-1}"}
% \only<4->{\ar@3{->} @/^1pc/ "H";"H^{-1}"_{...}="A"}
% \only<5->{\ar@3{<-}@/_1pc/  "H";"H^{-1}"^{...}="A"}
% }
% \end{center}

\end{itemize}

\item How can we formalize \wog in type theory?
\begin{itemize}
 \item Warren's \emph{strict} $\omega$-groupoid model
\item Altenkirch and Rypacek's syntactic approach
\item Brunerie's syntactic approach: \tig

\end{itemize}
% \item Altenkirch and Rypacek's syntactic approach
% \item Brunerie's syntactic approach to describe the internal structure
%   of \wog using a type theory called \tig

% \item In 1983, Grothendieck's definition of \wog
% \item Maltsiniotis and Ara's simplified versions

% \item Related work: ($\omega$-)groupoid model of type theory
% \begin{itemize}
% \item Hofmann and Streicher's groupoid model
% \item Warren's \emph{strict} $\omega$-groupoid model
% \item Altenkirch and Rypacek's syntactic approach
% \item Brunerie's syntactic approach to describe the internal structure
%   of \wog using a type theory called \tig
%\end{itemize}

\end{itemize}


\end{frame}



\begin{frame}[allowframebreaks,t]{Basic syntax of \tig}




\begin{itemize}

\item We use \tig to describe the internal structure of a \wogs

\begin{code}\>\<
\\
\>\AgdaKeyword{data} \AgdaDatatype{Con} \<[19]%
\>[19]\AgdaSymbol{:} \AgdaPrimitiveType{Set}\<%
\\
\>\AgdaKeyword{data} \AgdaDatatype{Ty} \AgdaSymbol{(}\AgdaBound{Γ} \AgdaSymbol{:} \AgdaDatatype{Con}\AgdaSymbol{)} \<[19]%
\>[19]\AgdaSymbol{:} \AgdaPrimitiveType{Set}\<%
\\
\>\AgdaKeyword{data} \AgdaDatatype{Tm} \<[19]%
\>[19]\AgdaSymbol{:} \AgdaSymbol{\{}\AgdaBound{Γ} \AgdaSymbol{:} \AgdaDatatype{Con}\AgdaSymbol{\}(}\AgdaBound{A} \AgdaSymbol{:} \AgdaDatatype{Ty} \AgdaBound{Γ}\AgdaSymbol{)} \AgdaSymbol{→} \AgdaPrimitiveType{Set}\<%
\>\<\end{code}

\item The structure is inductively defined as
\small{
\begin{code}\>\<%
\\
\>\AgdaKeyword{data} \AgdaDatatype{Ty} \AgdaBound{Γ} \AgdaKeyword{where}\<%
\\
\>[0]\AgdaIndent{2}{}\<[2]%
\>[2]\AgdaInductiveConstructor{*} \<[6]%
\>[6]\AgdaSymbol{:} \AgdaDatatype{Ty} \AgdaBound{Γ}\<%
\\
\>[0]\AgdaIndent{2}{}\<[2]%
\>[2]\AgdaInductiveConstructor{\_=h\_} \AgdaSymbol{:} \AgdaSymbol{\{}\AgdaBound{A} \AgdaSymbol{:} \AgdaDatatype{Ty} \AgdaBound{Γ}\AgdaSymbol{\}(}\AgdaBound{a} \AgdaBound{b} \AgdaSymbol{:} \AgdaDatatype{Tm} \AgdaBound{A}\AgdaSymbol{)} \AgdaSymbol{→} \AgdaDatatype{Ty} \AgdaBound{Γ}\<%
\\
\end{code}
}

\end{itemize}
\end{frame}


\begin{frame}[allowframebreaks,t]{Structure of \wog}

\begin{itemize}

\item Setoid: 
\begin{equation*}
\begin{array}[b]{l} 
\text{id} : x = x \\
\_^{-1} : x= y \to y  = x \\
\_ \circ\_ : y = z \to x = y \to x = z
 \end{array}
\end{equation*}

\item Groupoid: 

\begin{equation*}
\begin{array}[b]{l}
\lambda : id \circ p = p \\
\rho : p \circ id = p \\
\alpha : p \circ (q \circ r) = (p \circ q) \circ r \\
\kappa : p^{-1}\circ p = id \\
\kappa' : p \circ p^{-1} = id
 \end{array}
\end{equation*}
\item In \wog, we have more higher coherence laws which are weak (up
  to higher equivalences)
\item How can we describe all these coherence constants?
\end{itemize}

\end{frame}




\begin{frame}[allowframebreaks,t]{Contractible Contexts and Coherences}

\begin{itemize}

\item Definition of a \textbf{contractible contexts}:
\begin{itemize}
\item $\epsilon, * $
\item $\epsilon,x : *, y : *, \alpha : x = y$
\item ... $\Gamma, y: A , \alpha : x = y$ (Given $\Gamma \vdash A$ and
  $\Gamma \vdash x : A$)
\end{itemize}


\item Contexts for 0-level coherences in minimum contractible contexts:
\begin{itemize}
\item $\epsilon,x : * \vdash x = x$ ($id^{*}$)
\item $\epsilon,x : *, y : *, \alpha : x = y \vdash y = x$ ($\_^{-1*}$)
\item $\epsilon,x : *, y : *, \alpha : x = y, z: A,
  \beta : y = z \vdash x = z$ ($\_\circ^{*}\_$)
\end{itemize}

\item If we ``replace'' $\epsilon$ by arbitrary context $\Gamma$, and
  $*$ by arbitrary $A$, we obtain contexts for general 0-level coherences:
\begin{itemize}
\item $\Gamma,x : A \vdash x = x$ (id)
\item $\Gamma,x : A, y : A, \alpha : x = y \vdash y = x$ ($\_^{-1}$)
\item $\Gamma,x : A, y : A, \alpha : x = y, z: A,
  \beta : y = z \vdash x = z$ ($\_\circ\_$)
\end{itemize}

\item Indeed any coherence constant exists in a context which we can
  substitute into a contractible context

\small{
\begin{code}\>\<%
\\
\>\AgdaKeyword{data} \AgdaDatatype{Tm} \AgdaKeyword{where}\<%
\\
\>[0]\AgdaIndent{2}{}\<[2]%
\>[2]\AgdaInductiveConstructor{var} \<[7]%
\>[7]\AgdaSymbol{:} \AgdaSymbol{∀\{}\AgdaBound{Γ}\AgdaSymbol{\}\{}\AgdaBound{A} \AgdaSymbol{:} \AgdaDatatype{Ty} \AgdaBound{Γ}\AgdaSymbol{\}} \AgdaSymbol{→} \AgdaDatatype{Var} \AgdaBound{A} \AgdaSymbol{→} \AgdaDatatype{Tm} \AgdaBound{A}\<%
\\
\>[0]\AgdaIndent{2}{}\<[2]%
\>[2]\AgdaInductiveConstructor{coh} \<[7]%
\>[7]\AgdaSymbol{:} \AgdaSymbol{∀\{}\AgdaBound{Γ} \AgdaBound{Δ}\AgdaSymbol{\}} \AgdaSymbol{→} \AgdaDatatype{isContr} \AgdaBound{Δ} \AgdaSymbol{→} \AgdaSymbol{(}\AgdaBound{δ} \AgdaSymbol{:} \AgdaBound{Γ} \AgdaDatatype{⇒} \AgdaBound{Δ}\AgdaSymbol{)} \<[42]%
\>[42]\<%
\\
\>[2]\AgdaIndent{7}{}\<[7]%
\>[7]\AgdaSymbol{→} \AgdaSymbol{(}\AgdaBound{A} \AgdaSymbol{:} \AgdaDatatype{Ty} \AgdaBound{Δ}\AgdaSymbol{)} \AgdaSymbol{→} \AgdaDatatype{Tm} \AgdaSymbol{(}\AgdaBound{A} \AgdaFunction{[} \AgdaBound{δ} \AgdaFunction{]T}\AgdaSymbol{)}\<%
\\
\>\<\end{code}
}

\item Anything in a contractible context is a coherence constant:
  intuitively $J$ helps us derive everything in contractible contexts

Example: Assume $\epsilon,x : * \vdash x = x$ (weakening)
$\Rightarrow \epsilon,x : *, x : *, \alpha : x = x \vdash x =
x$ (J-eliminator)
$\Rightarrow \epsilon,x : *, y : *, \alpha : x = y \vdash y = x$

\end{itemize}

\end{frame}


\begin{frame}

\frametitle{Reasoning about syntactic terms}
\begin{itemize}
\item Using homogeneous equality to reason about syntactic terms, we
  have to deal with \textbf{subst} e.g.\ $\text{subst}~ p~ x = y$, $\text{subst} ~p
  (\text{subst} ~p^{-1}~ x) = x$
\item Heterogeneous equality for Tm

\begin{code}\>\<%
\\
\>\AgdaKeyword{data} \AgdaDatatype{\_≅\_} \AgdaSymbol{\{}\AgdaBound{Γ} \AgdaSymbol{:} \AgdaDatatype{Con}\AgdaSymbol{\}\{}\AgdaBound{A} \AgdaSymbol{:} \AgdaDatatype{Ty} \AgdaBound{Γ}\AgdaSymbol{\}} \<[29]%
\>[29]\<%
\\
\>[2]\AgdaIndent{9}{}\<[9]%
\>[9]\AgdaSymbol{:} \AgdaSymbol{\{}\AgdaBound{B} \AgdaSymbol{:} \AgdaDatatype{Ty} \AgdaBound{Γ}\AgdaSymbol{\}} \AgdaSymbol{→} \AgdaDatatype{Tm} \AgdaBound{A} \AgdaSymbol{→} \AgdaDatatype{Tm} \AgdaBound{B} \AgdaSymbol{→} \AgdaPrimitiveType{Set} \AgdaKeyword{where}\<%
\\
\>[0]\AgdaIndent{2}{}\<[2]%
\>[2]\AgdaInductiveConstructor{refl} \AgdaSymbol{:} \AgdaSymbol{(}\AgdaBound{b} \AgdaSymbol{:} \AgdaDatatype{Tm} \AgdaBound{A}\AgdaSymbol{)} \AgdaSymbol{→} \AgdaBound{b} \AgdaDatatype{≅} \AgdaBound{b}\<%
\\
\>\<\end{code}

\item Justification: The equality of inductively defined types are
  decidable, hence from Hedberg's Theorem they have UIP
\end{itemize}
\end{frame}


\begin{frame}[allowframebreaks,t]{Construction of coherences}

\begin{itemize}
\item To obtain a coherence term for arbitrary context $\Gamma$ in two steps:
\begin{enumerate}
\item a coherence term in a contractible context $\Delta$
\item a substitution $\Gamma \Rightarrow \Delta$
\end{enumerate}

\item To do the second step: \emph{replacement} and suspension

\item \textbf{Replacement}: Given an arbitrary type $A$ in arbitrary context
  $\Gamma$, we can replace $*$ in a contractible context $\Delta$ by
  $A$ and paste it onto $A$ in $\Gamma$, such that we can obtain coherence $B$ in $\Delta$ for type $A$

\begin{equation*}
\infer{\Gamma, \Delta^{A} \vdash B^{A}}
{\Gamma \vdash A & \vdash \Delta~ \text{contractible} & \Delta \vdash B} 
\end{equation*}

An example of $\Gamma, \Delta^{A}$: $\Gamma, y : A, \alpha : x = y$

\item Intuitively, we can filter out variables in $\Gamma$ which are
  unrelated to $A$. However it is very difficult to do that. Instead
  we build a new context using 

\item \textbf{Suspension}: build a minimum contractible context for type
  $A$ of level $n$:

\begin{itemize}
\item $(x_0: *)$ (the one-variable context) for $n=0$; 
\item $(x_0 : *, x_1 : *, x_2 : x_0\,=_\mathsf{h}\,x_1)$ for
$n=1$; 
\item $(x_0 : *, x_1 : *, x_2 : x_0\,=_\mathsf{h}\,x_1, x_3 :
x_0\,=_{\mathsf{h}}\,x_1, x_4 : x_2\,=_\mathsf{h}\,x_3)$ for $n=2$,
etc. 
\end{itemize}

with $\Delta$ whose $*$ is replaced by $A$

\begin{equation*}
\infer{\Sigma A, \Delta^{A} \vdash B^{A}}
{\Gamma \vdash A &  \vdash \Delta ~\text{contractible} & \Delta \vdash B} 
\end{equation*}


\item Thus we can define a substitution from a \emph{replaced context} to a
  \emph{suspended context} called \textbf{filter}

$$\Gamma, \Delta^{A} \Rightarrow \Sigma A , \Delta^{A}$$

Note: The suspended context is contractible because one-step
suspension proves to preserve contractibility.

\framebreak

\item Case: \textbf{Reflexivity}

\begin{itemize}

\item 1st step: reflexivity in a minimum contractible context
\begin{code}\>\<%
\\
\>\AgdaFunction{refl*-Tm} \AgdaSymbol{:} \AgdaDatatype{Tm} \AgdaSymbol{\{}\AgdaFunction{x:*}\AgdaSymbol{\}} \AgdaSymbol{(}\AgdaInductiveConstructor{var} \AgdaInductiveConstructor{v0} \AgdaInductiveConstructor{=h} \AgdaInductiveConstructor{var} \AgdaInductiveConstructor{v0}\AgdaSymbol{)}\<%
\\
\>\AgdaFunction{refl*-Tm} \AgdaSymbol{=} \AgdaFunction{Coh-Contr} \AgdaInductiveConstructor{c*}\<%
\\
\>\<\end{code}

\item 2nd step: reflexivity for arbitrary type $A$ in arbitrary
  context $\Gamma$

\begin{code}\>\<%
\\
\>\AgdaFunction{refl-Tm} \<[11]%
\>[11]\AgdaSymbol{:} \AgdaSymbol{\{}\AgdaBound{Γ} \AgdaSymbol{:} \AgdaDatatype{Con}\AgdaSymbol{\}(}\AgdaBound{A} \AgdaSymbol{:} \AgdaDatatype{Ty} \AgdaBound{Γ}\AgdaSymbol{)} \<[33]%
\>[33]\<%
\\
\>[9]\AgdaIndent{11}{}\<[11]%
\>[11]\AgdaSymbol{→} \AgdaDatatype{Tm} \AgdaSymbol{(}\AgdaFunction{rpl-T} \AgdaSymbol{\{}Δ \AgdaSymbol{=} \AgdaFunction{x:*}\AgdaSymbol{\}} \AgdaBound{A} \AgdaSymbol{(}\AgdaInductiveConstructor{var} \AgdaInductiveConstructor{v0} \AgdaInductiveConstructor{=h} \AgdaInductiveConstructor{var} \AgdaInductiveConstructor{v0}\AgdaSymbol{))}\<%
\\
\>\AgdaFunction{refl-Tm} \AgdaBound{A} \<[11]%
\>[11]\AgdaSymbol{=} \AgdaFunction{rpl-tm} \AgdaBound{A} \AgdaFunction{refl*-Tm}\<%
\\
\>\<\end{code}

\end{itemize}


\end{itemize}
\end{frame}



\begin{frame}{Semantics}

\begin{itemize}

\item A syntactic Grothendieck \wog is a globular set with an
  interpretation of syntactic coherence terms (coh)

\item A globular set $A$ consists coinductively of:
\begin{itemize}
\item A set obj$_{A}$
\item For every $x,y : \text{obj}_{A}$, a globular set Hom$_{A}(x, y)$
\end{itemize}

\item  Example: the identity globular set $Id^{\omega} A$
\begin{itemize}
\item obj$_{Id^{\omega} A} = A$
\item Hom$_{Id^{\omega} A}(a, b) = Id^{\omega} A (a = b)$
\end{itemize}

\item The interpretation of contexts, types and terms

\end{itemize}

\end{frame}











\begin{frame}{Conclusion}


\begin{itemize}

\item Types bear the structure of weak $\omega$-groupoids: the tower
  of iterated identity types

\item An implementation of syntactic \wog in Agda
  \begin{itemize}
  \item {Basic syntax of the type theory \tig}
  \item Heterogeneous equality for terms
  \item Constructions of coherences
  \item Semantic interpretation with globular sets
  \end{itemize}

\item To complete a \wogs model of type theory
\item A computational interpretation of univalent axiom in \itt

\end{itemize}




\end{frame}




\end{document}