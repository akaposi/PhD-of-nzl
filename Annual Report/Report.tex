\documentclass{article}
\def\textmu{}
\author{Li Nuo}
\title{First Year PhD Annual Report}


%\institute{University of Nottingham}

\usepackage{dsfont}
\usepackage{amsthm}


%% ODER: format ==         = "\mathrel{==}"
%% ODER: format /=         = "\neq "
%
%
\makeatletter
\@ifundefined{lhs2tex.lhs2tex.sty.read}%
  {\@namedef{lhs2tex.lhs2tex.sty.read}{}%
   \newcommand\SkipToFmtEnd{}%
   \newcommand\EndFmtInput{}%
   \long\def\SkipToFmtEnd#1\EndFmtInput{}%
  }\SkipToFmtEnd

\newcommand\ReadOnlyOnce[1]{\@ifundefined{#1}{\@namedef{#1}{}}\SkipToFmtEnd}
\usepackage{amstext}
\usepackage{amssymb}
\usepackage{stmaryrd}
\DeclareFontFamily{OT1}{cmtex}{}
\DeclareFontShape{OT1}{cmtex}{m}{n}
  {<5><6><7><8>cmtex8
   <9>cmtex9
   <10><10.95><12><14.4><17.28><20.74><24.88>cmtex10}{}
\DeclareFontShape{OT1}{cmtex}{m}{it}
  {<-> ssub * cmtt/m/it}{}
\newcommand{\texfamily}{\fontfamily{cmtex}\selectfont}
\DeclareFontShape{OT1}{cmtt}{bx}{n}
  {<5><6><7><8>cmtt8
   <9>cmbtt9
   <10><10.95><12><14.4><17.28><20.74><24.88>cmbtt10}{}
\DeclareFontShape{OT1}{cmtex}{bx}{n}
  {<-> ssub * cmtt/bx/n}{}
\newcommand{\tex}[1]{\text{\texfamily#1}}	% NEU

\newcommand{\Sp}{\hskip.33334em\relax}


\newcommand{\Conid}[1]{\mathit{#1}}
\newcommand{\Varid}[1]{\mathit{#1}}
\newcommand{\anonymous}{\kern0.06em \vbox{\hrule\@width.5em}}
\newcommand{\plus}{\mathbin{+\!\!\!+}}
\newcommand{\bind}{\mathbin{>\!\!\!>\mkern-6.7mu=}}
\newcommand{\rbind}{\mathbin{=\mkern-6.7mu<\!\!\!<}}% suggested by Neil Mitchell
\newcommand{\sequ}{\mathbin{>\!\!\!>}}
\renewcommand{\leq}{\leqslant}
\renewcommand{\geq}{\geqslant}
\usepackage{polytable}

%mathindent has to be defined
\@ifundefined{mathindent}%
  {\newdimen\mathindent\mathindent\leftmargini}%
  {}%

\def\resethooks{%
  \global\let\SaveRestoreHook\empty
  \global\let\ColumnHook\empty}
\newcommand*{\savecolumns}[1][default]%
  {\g@addto@macro\SaveRestoreHook{\savecolumns[#1]}}
\newcommand*{\restorecolumns}[1][default]%
  {\g@addto@macro\SaveRestoreHook{\restorecolumns[#1]}}
\newcommand*{\aligncolumn}[2]%
  {\g@addto@macro\ColumnHook{\column{#1}{#2}}}

\resethooks

\newcommand{\onelinecommentchars}{\quad-{}- }
\newcommand{\commentbeginchars}{\enskip\{-}
\newcommand{\commentendchars}{-\}\enskip}

\newcommand{\visiblecomments}{%
  \let\onelinecomment=\onelinecommentchars
  \let\commentbegin=\commentbeginchars
  \let\commentend=\commentendchars}

\newcommand{\invisiblecomments}{%
  \let\onelinecomment=\empty
  \let\commentbegin=\empty
  \let\commentend=\empty}

\visiblecomments

\newlength{\blanklineskip}
\setlength{\blanklineskip}{0.66084ex}

\newcommand{\hsindent}[1]{\quad}% default is fixed indentation
\let\hspre\empty
\let\hspost\empty
\newcommand{\NB}{\textbf{NB}}
\newcommand{\Todo}[1]{$\langle$\textbf{To do:}~#1$\rangle$}

\EndFmtInput
\makeatother
%
%
%
%
%
%
% This package provides two environments suitable to take the place
% of hscode, called "plainhscode" and "arrayhscode". 
%
% The plain environment surrounds each code block by vertical space,
% and it uses \abovedisplayskip and \belowdisplayskip to get spacing
% similar to formulas. Note that if these dimensions are changed,
% the spacing around displayed math formulas changes as well.
% All code is indented using \leftskip.
%
% Changed 19.08.2004 to reflect changes in colorcode. Should work with
% CodeGroup.sty.
%
\ReadOnlyOnce{polycode.fmt}%
\makeatletter

\newcommand{\hsnewpar}[1]%
  {{\parskip=0pt\parindent=0pt\par\vskip #1\noindent}}

% can be used, for instance, to redefine the code size, by setting the
% command to \small or something alike
\newcommand{\hscodestyle}{}

% The command \sethscode can be used to switch the code formatting
% behaviour by mapping the hscode environment in the subst directive
% to a new LaTeX environment.

\newcommand{\sethscode}[1]%
  {\expandafter\let\expandafter\hscode\csname #1\endcsname
   \expandafter\let\expandafter\endhscode\csname end#1\endcsname}

% "compatibility" mode restores the non-polycode.fmt layout.

\newenvironment{compathscode}%
  {\par\noindent
   \advance\leftskip\mathindent
   \hscodestyle
   \let\\=\@normalcr
   \let\hspre\(\let\hspost\)%
   \pboxed}%
  {\endpboxed\)%
   \par\noindent
   \ignorespacesafterend}

\newcommand{\compaths}{\sethscode{compathscode}}

% "plain" mode is the proposed default.
% It should now work with \centering.
% This required some changes. The old version
% is still available for reference as oldplainhscode.

\newenvironment{plainhscode}%
  {\hsnewpar\abovedisplayskip
   \advance\leftskip\mathindent
   \hscodestyle
   \let\hspre\(\let\hspost\)%
   \pboxed}%
  {\endpboxed%
   \hsnewpar\belowdisplayskip
   \ignorespacesafterend}

\newenvironment{oldplainhscode}%
  {\hsnewpar\abovedisplayskip
   \advance\leftskip\mathindent
   \hscodestyle
   \let\\=\@normalcr
   \(\pboxed}%
  {\endpboxed\)%
   \hsnewpar\belowdisplayskip
   \ignorespacesafterend}

% Here, we make plainhscode the default environment.

\newcommand{\plainhs}{\sethscode{plainhscode}}
\newcommand{\oldplainhs}{\sethscode{oldplainhscode}}
\plainhs

% The arrayhscode is like plain, but makes use of polytable's
% parray environment which disallows page breaks in code blocks.

\newenvironment{arrayhscode}%
  {\hsnewpar\abovedisplayskip
   \advance\leftskip\mathindent
   \hscodestyle
   \let\\=\@normalcr
   \(\parray}%
  {\endparray\)%
   \hsnewpar\belowdisplayskip
   \ignorespacesafterend}

\newcommand{\arrayhs}{\sethscode{arrayhscode}}

% The mathhscode environment also makes use of polytable's parray 
% environment. It is supposed to be used only inside math mode 
% (I used it to typeset the type rules in my thesis).

\newenvironment{mathhscode}%
  {\parray}{\endparray}

\newcommand{\mathhs}{\sethscode{mathhscode}}

% texths is similar to mathhs, but works in text mode.

\newenvironment{texthscode}%
  {\(\parray}{\endparray\)}

\newcommand{\texths}{\sethscode{texthscode}}

% The framed environment places code in a framed box.

\def\codeframewidth{\arrayrulewidth}
\RequirePackage{calc}

\newenvironment{framedhscode}%
  {\parskip=\abovedisplayskip\par\noindent
   \hscodestyle
   \arrayrulewidth=\codeframewidth
   \tabular{@{}|p{\linewidth-2\arraycolsep-2\arrayrulewidth-2pt}|@{}}%
   \hline\framedhslinecorrect\\{-1.5ex}%
   \let\endoflinesave=\\
   \let\\=\@normalcr
   \(\pboxed}%
  {\endpboxed\)%
   \framedhslinecorrect\endoflinesave{.5ex}\hline
   \endtabular
   \parskip=\belowdisplayskip\par\noindent
   \ignorespacesafterend}

\newcommand{\framedhslinecorrect}[2]%
  {#1[#2]}

\newcommand{\framedhs}{\sethscode{framedhscode}}

% The inlinehscode environment is an experimental environment
% that can be used to typeset displayed code inline.

\newenvironment{inlinehscode}%
  {\(\def\column##1##2{}%
   \let\>\undefined\let\<\undefined\let\\\undefined
   \newcommand\>[1][]{}\newcommand\<[1][]{}\newcommand\\[1][]{}%
   \def\fromto##1##2##3{##3}%
   \def\nextline{}}{\) }%

\newcommand{\inlinehs}{\sethscode{inlinehscode}}

% The joincode environment is a separate environment that
% can be used to surround and thereby connect multiple code
% blocks.

\newenvironment{joincode}%
  {\let\orighscode=\hscode
   \let\origendhscode=\endhscode
   \def\endhscode{\def\hscode{\endgroup\def\@currenvir{hscode}\\}\begingroup}
   %\let\SaveRestoreHook=\empty
   %\let\ColumnHook=\empty
   %\let\resethooks=\empty
   \orighscode\def\hscode{\endgroup\def\@currenvir{hscode}}}%
  {\origendhscode
   \global\let\hscode=\orighscode
   \global\let\endhscode=\origendhscode}%

\makeatother
\EndFmtInput
%
%
\ReadOnlyOnce{agda.fmt}%


\RequirePackage[T1]{fontenc}
\RequirePackage[utf8x]{inputenc}
\RequirePackage{ucs}
\RequirePackage{amsfonts}

\providecommand\mathbbm{\mathbb}

% TODO: Define more of these ...
\DeclareUnicodeCharacter{737}{\textsuperscript{l}}
\DeclareUnicodeCharacter{8718}{\ensuremath{\blacksquare}}
\DeclareUnicodeCharacter{8759}{::}
\DeclareUnicodeCharacter{9669}{\ensuremath{\triangleleft}}
\DeclareUnicodeCharacter{8799}{\ensuremath{\stackrel{\scriptscriptstyle ?}{=}}}
\DeclareUnicodeCharacter{10214}{\ensuremath{\llbracket}}
\DeclareUnicodeCharacter{10215}{\ensuremath{\rrbracket}}

% TODO: This is in general not a good idea.
\providecommand\textepsilon{$\epsilon$}
\providecommand\textmu{$\mu$}


%Actually, varsyms should not occur in Agda output.

% TODO: Make this configurable. IMHO, italics doesn't work well
% for Agda code.

\renewcommand\Varid[1]{\mathord{\textsf{#1}}}
\let\Conid\Varid
\newcommand\Keyword[1]{\textsf{\textbf{#1}}}
\EndFmtInput




\usepackage{color}
\usepackage{amsmath}
\usepackage{amsfonts}
\usepackage{amssymb}
\usepackage{xypic}

%\newtheorem{theorem}{Theorem}[section]
%\newtheorem{lemma}[theorem]{Lemma}
\theoremstyle{definition}
\newtheorem{definition}{Definition}[section]
%\newtheorem{proposition}[theorem]{Proposition}

\DeclareUnicodeCharacter{955}{$\lambda$}
\DeclareUnicodeCharacter{8988}{$\ulcorner$}
\DeclareUnicodeCharacter{8990}{$\llcorner$}
\DeclareUnicodeCharacter{8989}{$\urcorner$}
\DeclareUnicodeCharacter{8991}{$\lrcorner$}
\DeclareUnicodeCharacter{946}{$\beta$}

%\DeclareUnicodeCharacter{9666}{$\blacktriangleleft$}
%\DeclareUnicodeCharacter{9667}{$\triangleleft$}


% Editing and debugging
%\hfuzz 0.1pt
%\overfullrule=15pt
%\brokenpenalty=10000

\newcommand{\todo}[1]{\textcolor{red}{TO~DO:~#1}}


\newcommand{\ed}[1]{\textcolor{blue}{#1}}


%\newtheorem{assumption}[theorem]{Assumption}

\newcommand{\N}{\mathbb{N}}
\newcommand{\Q}{\mathbb{Q}}
\newcommand{\R}{\mathbb{R}}
\newcommand{\Z}{\mathbb{Z}}


\newcommand{\dotph}{\,\cdot\,}
\newcommand{\dotop}{\mathrel{.}}
\providecommand{\abs}  [1]{\lvert#1\rvert}
\providecommand{\norm} [1]{\lVert#1\rVert}
\providecommand{\class}[1]{[#1]}
\providecommand{\set}  [1]{\left\{#1\right\}}
\providecommand{\dlift}[1]{\widehat{#1}}

\DeclareMathOperator{\Prop}{\mathbf{Prop}}
\DeclareMathOperator{\Set}{\mathbf{Set}}
\DeclareMathOperator{\Ext}{Ext}
\DeclareMathOperator{\Bool}{Bool}
\DeclareMathOperator{\id}{id}
\DeclareMathOperator{\sound}{sound}
\DeclareMathOperator{\qelimbeta}{qelim-\beta}
\DeclareMathOperator{\qind}{qind}
\DeclareMathOperator{\exact}{exact}
\DeclareMathOperator{\subst}{subst}
\DeclareMathOperator{\emb}{emb}
\DeclareMathOperator{\complete}{complete}
\DeclareMathOperator{\stable}{stable}
\DeclareMathOperator{\List}{List}
\DeclareMathOperator{\Fin}{Fin}
\DeclareMathOperator{\now}{now}
\DeclareMathOperator{\later}{later}
\DeclareMathOperator{\nowequal}{now_\sqsubseteq}
\DeclareMathOperator{\laterequal}{later_\sqsubseteq}
\DeclareMathOperator{\laterleft}{later_{left}}
\DeclareMathOperator{\inl}{inl}
\DeclareMathOperator{\inr}{inr}
\DeclareMathOperator{\qelim}{qelim}
\DeclareMathOperator{\lift}{lift}
\DeclareMathOperator{\LC}{LC}
\DeclareMathOperator{\liftbeta}{lift-\beta}
\DeclareMathOperator{\Bijection}{Bijection}
\DeclareMathOperator{\true}{true}
\DeclareMathOperator{\false}{false}
\DeclareMathOperator{\sort}{sort}
\DeclareMathOperator{\length}{length}
\DeclareMathOperator{\nub}{nub}
\DeclareMathOperator{\suc}{suc\,}
\DeclareMathOperator{\defi}{\stackrel{\text{\tiny def}}{=}}
\DeclareMathOperator{\coprime}{Coprime}

\DeclareMathOperator{\A/sim}{A\,/ \sim}

\newcommand{\eqqm}{\overset{\text{\tiny ?}}{=}}
\newcommand{\sep}{\mathrel{\sharp}}


% For xy matrices
%\newcommand{\pullbackcorner}[1][dr]{\save*!/#1-1.2pc/#1:(-1,1)@^{|-}\restore}
%\newcommand{\pushoutcorner} [1][dr]{\save*!/#1+1.2pc/#1:(1,-1)@^{|-}\restore}

\newcommand{\itt}{intensional Type Theory}
\newcommand{\ett}{extensional Type Theory}
\newcommand{\mltt}{Martin-L\"{o}f type theory}

%\newtheorem{theorem}{Theorem}[section]
%\newtheorem{lemma}[theorem]{Lemma}
%\theoremstyle{definition}
%\newtheorem{definition}{Definition}[section]
%\newtheorem{proposition}[theorem]{Proposition}

\usepackage{varioref}

\begin{document}

\maketitle

\tableofcontents

\newpage

\begin{abstract}
%\todo{enhance the connection within between ideas. Split the two
%ideas in the abstract: why in general quotient is useful and
%implementing in Agda.}


% use of quotients is a 
% types before objects rather than objects before in set theory
%  what is itt what is ett what is the diff and what implementation is
%  itt or ett? more detail and technical?
% always be more precise , give explanation, examples, list things,
% not use complicated
% what i want to do, plan, proposal, explain the context of that paper
% why it is good to use the quotient, examples. 
%suggestion: 
% 1. extend the txa:1999 to quotients, integrate quotients into these
% 2. to prove the conservativity of itt + quotients ove ett,
% formalisation and detail.
% 3. smaller goals on the way
% 4. what I have done formalise proofs in Agda. Develop the
% background. how I have done related to others, motivation, why it is
% interesting what I have read, diff between ett and itt. in my
% undergraduate thesis.
% 5. Plan and steps to achieve these.

In set theory, given a set equipped with an equivalence relation, one can form its
quotient set, that is the set of equivalence classes. Reinterpreting
this notion in type theory, the implementation of quotient set is
called quotient type. However in the quotient type is still
unavailable in type theories like \itt{}. Quotients are very common in
mathematics and computer science. The introduction of quotients could
be very helpful. Some quotient types are less effective to define than
being based on their base types and equivalence relations, such as the
set of integers, even some others are impossible to define without
quotients in current implementations of \itt{} such as the set of real
numbers. Quotient types are often more difficult to reason than their
base types. Therefore it is more convenient to manipulate the base
types and lift the operations and propositions using the properties of
quotients. 
% Also we can still benefit from the interaction of
% types within some quotients without using quotient types. Because some
% base types are simpler to deal with or have better features while what
% we want to use is the new type, for instance the integers represented
% by a pair of natural numbers is easier to handled compared to the
% normal form integers. 
Therefore we conduct a research project on the implementation of quotients in
\itt{}. The work of this project will be divided into several phases. This report aims at
introducing the basic notions in the project like type theory and
quotient types, discuss some work related to this
topic and conclude some result of the first phase. The results done by
Altenkirch, Anberr\'{e}e and me in \cite{aan} will be explained with a
few instances of quotients.

%For some definable quotients like the integers and the rational numbers,
%it is unnecessary to formalise them based on setoid. But we found the quotient %interface could provide more convenience if we prove that they are definable %quotients. While some other types such as real numbers are undefinable, %but they are the quotient of dividing Cauchy sequences of rational numbers %by the equivalence relation that two sequences converges. 

\end{abstract}


\section{Introduction}


In mathematics, a quotient represents the result of division. The
notion of quotient is extended to other more abstract branches of mathematics. For example, we
have quotient set, quotient group, quotient space, quotient category
etc. They are all defined similarly. Some group of objects are divided
by some equivalence relation and the group of the equivalent classes
is the quotient sets or other algebraic structure.

Actually, we can find quotients in our daily life. When you take a
picture by a digital camera, the real scene is divided into each pixel
on the picture, in other words, the real scene within a certain area is
equated. The digital picture which is the set of the pixel is just the
quotient of the real scene. Since things are equated
with respect to certain rules, quotient is usually related to
information hiding or information losing. Quotients also exist in
computer science. Users are usually concerned the extensional use of
softwares rather than the intensional implementation of them,
different implementations of softwares doing the same tasks are
treated as the same to them even though 
some of them are programmed in different languages. Another example is
the application of \emph{interface} in JAVA. All different classes implementing the same
interface \ensuremath{\Conid{A}} are extensionally equal and are treated as the same when
we call some object of \ensuremath{\Conid{A}}.

In this report, I will mainly discuss the quotients in type theory
which are usually named as quotient types. Although set theory and
type theory have different fundamentals, they have many similar
notions like product and disjoint union.  In this case, quotient type
is also an interpretation of quotient set in set theory which may be
more familiar to some of you. Because of this reason, we start from
quotient set in set theory and then move on to type theory.

\subsection{Quotient Set} 

The division of sets is different from division of numbers. We divide a
given set into small groups according to a given equivalence relation
and the quotient is the set of these groups.

Formally, given a set $A$ and an equivalence relation $\sim$ on $A$, the equivalence class for each $a \,\in A$ is,

\[
\class a = \{ b \in A \, \vert \, b \sim a \}
\]

The quotient set denoted as $\A/sim$ is the set of equivalence classes of $\sim$,


\[
\A/sim =\{[a] \in \wp(A) \, \vert \, a \in A\}
\]

%why quotients is useful?

There are many mathematical notions which can be constructed as quotient sets
from some other sets. Some are more natural to come up with,
such as integers modulo some number
$n$ is the quotient set constructed by quotienting the set of all
integers $\Z$ with the congruence relation which equates two integers sharing the same remainders when divided by
$n$.
The set of rational numbers $\Q$ is defined as the set of numbers
which can be expressed as fractions, but different fractions like
$\frac{1}{2}$ and $\frac{2}{4}$ can be
reduced to the same rational number. In another word, $\Q$ can
be constructed by quotienting the set of pairs of integers, while the
second is non-zero integer, with the equivalence
relation which equates fractions sharing the same irreducible forms.
%The set of canonical forms of rational numbers are subset of the pairs of integers. 
A less common example is the set of integers $\Z$, which can also been
obtained from quotienting the set of pairs of natural numbers $\N \times \N$ which represent integers as the result of subtraction
between two natural numbers within each pair. Furthermore real
numbers can be represented by Cauchy sequences of
rational numbers, hence the set of real numbers $\R$ is the quotient
set of the set of Cauchy sequences of rational numbers with the
equivalence relation that the distance between two sequences converges
to zero. There are more examples of quotient sets, but the main
topic of this report is quotients in \emph{type theory}. 

\subsection{Type Theory}

The theory of types was first introduced by Russell \cite{rus:1903} as
an alternative to naive set theory. After that, mathematicians or
computer scientists have developed a number of variants of type
theory. The type theory in this discourse is the one developed by Per
Martin-L\"{o}f \cite{per:71,per:82} which is also called
intuitionistic type theory. It is based on the Curry-Howard
isomorphism between propositions and the types of its proofs such that
it can served as a formalisation of mathematics. For detailed
introduction, please refer to\cite{nor:00}. In this report, Type
Theory specially indicates \mltt{}.


Per Martin-L\"{o}f proposed both an intensional and an extensional
variants of his intuitionistic type theory. The distinction between them is whether definitional equality is distinguished with
propositional equality. In \itt{}, definitional equality exists
between two intensionally identical objects, but propositional
equality is a type which requires proof terms. Any thing is only
definitionally equal to itself and all terms that can be normalised to
it which means that definitional equality is decidable in \itt{}. Therefore type checking which
depends on definitional equality is decidable as well  \cite{alt:99}
. The propositional equality we use in \itt{} is written
as $Id(A,a,b)$ \cite{nor:90} which is also called
\emph{intensional equality}. In Agda, an implementation of \itt{},  it is
redefined as \ensuremath{\Varid{a}\;\Varid{≡}\;\Varid{b}} with the type \ensuremath{\Conid{A}}implicitly, and this set has an
unique element \ensuremath{\Varid{refl}} only if \ensuremath{\Varid{a}} and \ensuremath{\Varid{b}} are definitionally equal. However
in \ett{}, they are not distinguished so that if we have $p \,\colon Eq
(A,a,b)$ which is called extensional equality, $a$ and $b$ are
definitional equal. Terms which have different normal forms may be
definitional equal or not. In other words, definitional equality is
undecidable and type checking become undecidable as well.
Altenkirch and McBride \cite{alt:06} introduce a variant of \ett{} called
\emph{Observational Type Theory} in which definitional equality is
decidable and propositional equality is extensional.

%The propositional equality in \ett{} is written as $Eq(A,a,b)$ and is
%also \emph{extensional equality}. \todo{Why?}


% There are also two different propositions expressing the equality
% between two elements in Type Theory \cite{nor:90}. Both of them
% require the types of two elements are definitionally equal.
% One is intensional equality written as $Id(A,a,b)$ and it is inhabited
% only we have a proof showing $a$ and $b$ are definitionally equal. The
% other is extensional equality written as $Eq(A,a,b)$, the elements
% do not depend on an element of $A$ and the largest difference is if
% $Eq(A,a,b)$ is inhabited, then $a$ converts to $b$ and vice versa. The
% latter one will make type-checking undecidable so we usually use the
% first one which is available in \itt{}. For example in Agda,


Type theory can also serve as a programming language in
which the evaluation of well-typed program always terminates
\cite{nor:90}. 
There are a few implementation based on different type theories, such as
NuPRL, Coq and Agda.
Agda is one of the most recent implementation of intensional version
of \mltt{}. As we have seen that \mltt{} is based on the Curry-Howard
isomorphism, types are identified with propositions and programs or
terms are identified with proofs. Therefore it is not only a programming language but also a
theorem prover which allows user to verify Agda programs in
itself. Compared to other implementations, it has a bundle of good
features like pattern matching, unicode input, implicit arguments etc
\cite{bov:09}  but it does not have tactics such that the proofs are less
readable. Since this project is based on \mltt{}, it is a good choice to implement our definitions
and verify propositions in Agda.

Although type theory has some similarities to set theory, they are
fundamentally different. Types play a similar role to sets and are
also called sets in many situations. However we can only create
elements after we declare their types, while in set theory elements are there before
we have sets. For example,
we have type $\N$ for natural numbers corresponding to the set of
natural numbers in set theory. In set theory, $2$ is a shared element
of the set of natural numbers and the set of integers. While in type
theory, $\N$ provides us two constructors
$zero \,\colon\N$ and $suc\,\colon\N\to\N$, and $2$ can be constructed
as $suc\,(suc\,\,zero)$ which is of type $\N$ and does not have any other
types like $\Z$. Because different sets may contain the same elements, we
have the subset relation such that we can construct equivalence
classes and quotient set. In type theory we have to give constructors
for any type before we can construct elements which is different to the situation in set theory that
elements exist before we construct quotient sets. The approach in set
theory fails here. So how can we reinterpret quotient sets in type theory? 


\subsection{Quotient Types}

Following the correspondence between sets and types, many notions from
set theory can be reinterpreted in type theory. The product of
sets can be formed by $\Sigma-Type$ and the functions can be
formed by $\Pi-Type$ \cite{nor:00}. However in \itt{} quotient
types are still unavailable and it is a problematic issue to interpret
quotients.

Alternatively, in \itt{} we have the ingredients of quotients as follows,

\begin{definition}
A setoid $(A,\sim)\,\colon\Set_1$ is a set $A\,\colon\Set$ equipped with an equivalence relation ${\,\sim\,}\colon A \to A \to \Prop$.
\end{definition}

Here we assume $\Set$ means type and for any $p\,q\colon\,\Prop$, $p = q$. In Agda, we define a setoid as

\begin{hscode}\SaveRestoreHook
\column{B}{@{}>{\hspre}l<{\hspost}@{}}%
\column{3}{@{}>{\hspre}l<{\hspost}@{}}%
\column{5}{@{}>{\hspre}l<{\hspost}@{}}%
\column{19}{@{}>{\hspre}l<{\hspost}@{}}%
\column{E}{@{}>{\hspre}l<{\hspost}@{}}%
\>[B]{}\Keyword{record}\;\Conid{Setoid}\;\mathbin{:}\;\Conid{Set₁}\;\Keyword{where}{}\<[E]%
\\
\>[B]{}\hsindent{3}{}\<[3]%
\>[3]{}\Keyword{infix}\;\Varid{4}\;\Varid{\char95 ≈\char95 }{}\<[E]%
\\
\>[B]{}\hsindent{3}{}\<[3]%
\>[3]{}\Keyword{field}{}\<[E]%
\\
\>[3]{}\hsindent{2}{}\<[5]%
\>[5]{}\Conid{Carrier}\;{}\<[19]%
\>[19]{}\mathbin{:}\;\Conid{Set}{}\<[E]%
\\
\>[3]{}\hsindent{2}{}\<[5]%
\>[5]{}\Varid{\char95 ≈\char95 }\;{}\<[19]%
\>[19]{}\mathbin{:}\;\Conid{Carrier}\;\Varid{→}\;\Conid{Carrier}\;\Varid{→}\;\Conid{Set}{}\<[E]%
\\
\>[3]{}\hsindent{2}{}\<[5]%
\>[5]{}\Varid{isEquivalence}\;\mathbin{:}\;\Conid{IsEquivalence}\;\Varid{\char95 ≈\char95 }{}\<[E]%
\ColumnHook
\end{hscode}\resethooks

Setoid could be universe polymorphic.

%. If we use the setoid $(\N \times \N, \sim)$ for integers, 
We can use setoids to represent quotients. However setoids is
different from sets so that we have to redefine all the operations on sets
and it is unsafe \cite{aan}. Another interesting problem is how to
represent quotients if the $A$ is already a setoid. It means that it
is better that if the base type $A$ is of type $\Set$ then the type of
quotient should also be of type $\Set$. It should be
universe polymorphic as well. From mathematical perspective, we also
found the structure of the base object is always the same as the
structure of the result quotient object. So what should be a quotient type?
%Quotient types should enable users to implement quotients in \itt{}
%which means the quotient types based on the setoid should have type
%$\Set$. 

\begin{definition}
Given a setoid $(A,\sim)\,\colon\Set_1$, there is a type
$Q\,\colon\Set$ which is an implementation of the quotient set $A/\sim$, $Q$ is called the quotient type of $(A,\sim)$
\end{definition}


For example, given a setoid $(\N\times\N , \sim)$ where $\sim$ is
defined as

\[(a , b) \sim (c , d) \defi  a + d \equiv c + b \]

and $\sim$ is proved to be an equivalence relation, 
the quotient type corresponding to this setoid is just the set of
integers $\Z\,\colon\Set$. 

Quotient types can be very useful. The encoding of quotients in mathematics is not all what quotient
types can do. It is a type theoretical notion which means some notions in Type
Theory or in programming languages can also be treated as quotient
types. For example partiality monad divided by a weak
similarity ignoring finite delays \cite{aan}, propositions divided by
$\iff$  and the set of extensionally equal functions. Also
set-theoretical finite sets can be implemented as the quotient of
lists in Type Theory. 
Furthermore given any function $f \,\colon\, A \to B$, we obtain an
equivalence relation $\sim \,\colon A \to A \to \Prop$ called \emph{kernel}
of $f$ which is defined as $a \sim b \defi f \,a \equiv f \,b$. Based on
this setoid $(A,\sim)$ we can form a quotient.
Indeed all types can be seen as quotient types of itself with the
intensional equality \ensuremath{\Varid{≡}}.

%How to obtain quotient type? 

In the definition of quotient types, we do not provide an approach to
construct them from given setoids. Indeed how to obtain a
quotient type of a given setoid is one of the main topic of this
project.

One feasible approach in current setting of \itt{} is to manually construct
the quotient type and prove it is the required quotient type. For instances in
\cite{nuo:10}, the normal form integers, irreducible rational numbers
are definable and proved to construct quotients with respect to the
corresponding setoids. The quotient interfaces introduced in
\cite{aan} requires the necessary properties for some type $Q$ to be
the quotient type of some setoid $(A , \sim)$. Since they are basics
of quotients, we can use them to lift operations and prove some
general theorems. However, this approach is
inefficient because the quotient types and the properties have to be
manually figured out rather than automatically derived. 
%The lifting of functions also  proofs that these functions or
%predicates respect the equivalence relation\cite{hof:95:sm}.
Furthermore, some quotients like real numbers are undefinable even
though we can define the base type and equivalence relation for them
\cite{nuo:10} . Although it has some drawbacks, it is feasible without
extending \itt{} and it provides some convenience in practice. There
has been some results on this \cite{aan}. I will discuss them later. 

%In this report, I will use the symbol $\A/sim$ for the quotient based
%on a given setoid  $(A,\sim)$. 
%To make the difference between setoids and quotient types clear, we
%use an analogy, $8\div2=4$. The number 4 is the quotient because $4
%\times 2 = 8$, and we cannot recover the dividend and the divisor
%from the quotient $4$ or manipulate $8$ or $2$ separately. Similarly,
%setoids contain pairs of dividend and divisor, but quotient sets do
%not include all the information from the setoids. Furthermore one set can be the quotient sets of several different setoids. 

The ideal approach should be an axiomatised type former for quotient
types. It means that we have to extend \itt{} with the introduction
rules and elimination rules of quotient types. However there are
many problems arising, for example the constructors for quotient types, the
definitional equality of quotient types etc.
% As we have seen before, all types in \itt{} can be seen as quotient
% types while the equivalence relation is the identity relation. Based
% on this idea, new quotient types can be seen as replacing the
% underlying equivalence relation with new one. In \itt{}, type checking which depends on definitional
% equality does not identify propositionally equal terms such that the
% new quotient type fails to work as $\Set$. 

Quotient types can be seen as the result of replacing the equivalence
relation of given types. This operation does not work in \itt{}, but it seems
easier to manage in \ett{} where
propositional equal terms are also definitional equal. Nevertheless there
are still some problems which we discuss in the literature review.

%[(S m) , (S p)] -> [m , p] 
%[(S(S(S m))) , m] -> [(S(S(S m))) m] 
%Vec (S m + n) = Vec S (m + n).

%  In this way, the constructors for quotient
% type are automatically generated which means we have to use the
% constructors for base type with some symbols to represent the
% corresponding element in quotient type. Because of that, each element
% of quotient type sometimes has more than one normal forms which are not
% $\beta$-equivalent terms. As a result, the different representations
% for the same term of quotient type are only propositionally equal with
% respect to the equivalence relation rather than definitionally equal. 
%Therefore it still remains a difficult problem in \itt{}.
% Maybe we can write some new beta-conversion rules for quotient
% types, then different terms maybe definitionally equal to each other.




% \subsection{The relation between equality and quotient types}



\subsection{Functional extensionality and quotient types}

% equality $Id(A,a,b)$ consists of
%proof terms that $a$ and $b$ are definitionally equal in all cases. 

In \itt{} intensional propositional equality $Id(A,a,b)$ is inhabited
if and only if $a$ and $b$ are definitional equal terms,

\begin{hscode}\SaveRestoreHook
\column{B}{@{}>{\hspre}l<{\hspost}@{}}%
\column{3}{@{}>{\hspre}l<{\hspost}@{}}%
\column{E}{@{}>{\hspre}l<{\hspost}@{}}%
\>[B]{}\Keyword{data}\;\Conid{Id}\;(\Conid{A}\;\mathbin{:}\;\Conid{Set})\;\mathbin{:}\;\Conid{A}\;\Varid{→}\;\Conid{A}\;\Varid{→}\;\Conid{Set}\;\Keyword{where}{}\<[E]%
\\
\>[B]{}\hsindent{3}{}\<[3]%
\>[3]{}\Varid{refl}\;\mathbin{:}\;(\Varid{a}\;\mathbin{:}\;\Conid{A})\;\Varid{→}\;\Conid{Id}\;\Conid{A}\;\Varid{a}\;\Varid{a}{}\<[E]%
\ColumnHook
\end{hscode}\resethooks

 but the intensional propositional
equalities of functions are not inhabited \cite{alt:99}. When talking about the equality of functions, they are
usually treated extensionally as black boxes. If two functions pointwise
generate the same outputs for the same inputs, they are equivalent
even though their inside definitions may differ. This is called
functional extensionality  which is not inhabited in original
\itt{} and can be expressed as following,

given two types $A$ and $B$, and two functions $f,\,g\,\colon A \to B$,

\[Ext = \forall\, x\colon A, f x = g x \to f = g\]


If seems that we just need to add the constant $Ext$ to \itt{}.
However postulate something new may cause inconsistence. If we postulate $Ext$, then the theory is said to be not
adequate which means it is possible to define irreducible terms. It
can be easily verified in Agda through formalising a non-canonical term
of natural number by an eliminator of intensional equality. 
Firstly we postulate $Ext$

\begin{hscode}\SaveRestoreHook
\column{B}{@{}>{\hspre}l<{\hspost}@{}}%
\column{3}{@{}>{\hspre}l<{\hspost}@{}}%
\column{9}{@{}>{\hspre}l<{\hspost}@{}}%
\column{E}{@{}>{\hspre}l<{\hspost}@{}}%
\>[B]{}\Keyword{postulate}{}\<[E]%
\\
\>[B]{}\hsindent{3}{}\<[3]%
\>[3]{}\Conid{Ext}\;\mathbin{:}\;\{\mskip1.5mu \Conid{A}\;\mathbin{:}\;\Conid{Set}\mskip1.5mu\}\;\{\mskip1.5mu \Conid{B}\;\mathbin{:}\;\Conid{A}\;\Varid{→}\;\Conid{Set}\mskip1.5mu\}\;\{\mskip1.5mu \Varid{f}\;\Varid{g}\;\mathbin{:}\;(\Varid{x}\;\mathbin{:}\;\Conid{A})\;\Varid{→}\;\Conid{B}\;\Varid{x}\mskip1.5mu\}\;{}\<[E]%
\\
\>[3]{}\hsindent{6}{}\<[9]%
\>[9]{}\Varid{→}\;((\Varid{x}\;\mathbin{:}\;\Conid{A})\;\Varid{→}\;\Conid{Id}\;(\Conid{B}\;\Varid{x})\;(\Varid{f}\;\Varid{x})\;(\Varid{g}\;\Varid{x}))\;{}\<[E]%
\\
\>[3]{}\hsindent{6}{}\<[9]%
\>[9]{}\Varid{→}\;\Conid{Id}\;((\Varid{x}\;\mathbin{:}\;\Conid{A})\;\Varid{→}\;\Conid{B}\;\Varid{x})\;\Varid{f}\;\Varid{g}{}\<[E]%
\ColumnHook
\end{hscode}\resethooks

And the eliminator of the intensional equality is

\begin{hscode}\SaveRestoreHook
\column{B}{@{}>{\hspre}l<{\hspost}@{}}%
\column{5}{@{}>{\hspre}l<{\hspost}@{}}%
\column{E}{@{}>{\hspre}l<{\hspost}@{}}%
\>[B]{}\Conid{J}\;\mathbin{:}\;(\Conid{A}\;\mathbin{:}\;\Conid{Set})\;(\Conid{P}\;\mathbin{:}\;(\Varid{a}\;\Varid{b}\;\mathbin{:}\;\Conid{A})\;\Varid{→}\;\Conid{Id}\;\Conid{A}\;\Varid{a}\;\Varid{b}\;\Varid{→}\;\Conid{Set})\;{}\<[E]%
\\
\>[B]{}\hsindent{5}{}\<[5]%
\>[5]{}\Varid{→}\;((\Varid{a}\;\mathbin{:}\;\Conid{A})\;\Varid{→}\;\Conid{P}\;\Varid{a}\;\Varid{a}\;(\Varid{refl}\;\Varid{a}))\;{}\<[E]%
\\
\>[B]{}\hsindent{5}{}\<[5]%
\>[5]{}\Varid{→}\;(\Varid{a}\;\Varid{b}\;\mathbin{:}\;\Conid{A})\;(\Varid{p}\;\mathbin{:}\;\Conid{Id}\;\Conid{A}\;\Varid{a}\;\Varid{b})\;\Varid{→}\;\Conid{P}\;\Varid{a}\;\Varid{b}\;\Varid{p}{}\<[E]%
\\
\>[B]{}\Conid{J}\;\Conid{A}\;\Conid{P}\;\Varid{m}\;\Varid{.b}\;\Varid{b}\;(\Varid{refl}\;\Varid{.b})\;\mathrel{=}\;\Varid{m}\;\Varid{b}{}\<[E]%
\ColumnHook
\end{hscode}\resethooks

Finally we can construct a irreducible term of natural number as

\begin{hscode}\SaveRestoreHook
\column{B}{@{}>{\hspre}l<{\hspost}@{}}%
\column{E}{@{}>{\hspre}l<{\hspost}@{}}%
\>[B]{}\Varid{irr}\;\mathbin{:}\;\Conid{ℕ}{}\<[E]%
\\
\>[B]{}\Varid{irr}\;\mathrel{=}\;\Conid{J}\;(\Conid{ℕ}\;\Varid{→}\;\Conid{ℕ})\;(\Varid{λ}\;\Varid{a}\;\Varid{b}\;\Varid{x}\;\Varid{→}\;\Conid{ℕ})\;(\Varid{λ}\;\Varid{a}\;\Varid{→}\;\Varid{0})\;(\Varid{λ}\;\Varid{x}\;\Varid{→}\;\Varid{x})\;(\Varid{λ}\;\Varid{x}\;\Varid{→}\;\Varid{x})\;(\Conid{Ext}\;\Varid{refl}){}\<[E]%
\ColumnHook
\end{hscode}\resethooks

Because of the inconsistency, the definitional equality and type checking become undecidable. 

Altenkirch investigates this issue and gives a solution in
\cite{alt:99}. He propose an extension of \itt{} by a universe of
propositions $\Prop$ in which all proofs of same propositions are
definitionally equal, namely it is proof irrelevant. At the same time,
a setoid model where types are interpreted by a type and an equivalence relation acts as the metatheory and $\eta$-rules for
$\Pi$-types and $\Sigma$-types hold in the metatheory. The extended type
theory generated from the metatheory is decidable and adequate, $Ext$ is
inhabited and it permits large elimination. Within this type theory,
introduction of quotient types is straightforward. 
The set of functions are naturally quotient types, the hidden information is the
definition of the functions and the equivalence relation is the
functional extensionality.
% extension

There are more problems concerning quotient types and most
of them are related to equality. One of the main problems is how to lift the functions for
base types to the ones for quotient types. Only functions respects the
equivalence relation can be lifted. Even in \ett{}, the implementation
of quotient types does not stop at replacing equality of the types. We will discussed these later.

\subsection{Literature Review}

% Why I mention about this article
% More description about these articles
% in a more comprehensive way, tell a story
% compare and link between literatures
In \cite{cab}, Mendler et al. have firstly considered building new types from a
given type using a quotient operator $//$. Their work is based on an
implementation of \ett{}, NuPRL. In NuPRL, every type
comes with its own equality relation, so the quotient operator can be
seen as a way of redefining equality in a type. But it is not all
about building new types. They also discuss the problems arises from
defining functions on the new type. The problem can be illustrated using a simple example. 

Assume the base type is $A$ and the new equality relation is $E$, the new
type can be formed as $A//E$. If we want to define a function $f
\,\colon\, A//E \to Bool$,  Assume we have two $a, b \,\colon A$ such that $E\,a\,b$ but $f\,a \not= f\,b$, then it
becomes inconsistent since $E\,a\,b$ implies $a$ converts to $b$ in \ett{}, the left
hand side $f\,a$ can be converted to $f\,b$ namely $f\,b \not= f\,b$
which is contradicted with the equality reflection rule. Only a function respects the equivalence
relation, namely

$$\forall \, a\,b\,\colon A, E\,a\,b \to f\,a = f\,b$$

$f$ is said to be well-defined on the new type. We call it \emph{sound} in
\cite{aan} and this project.

 After the introduction of quotient types, Mendler further investigates
 this topic from a categorical perspective in ~\cite{men:90}. He use
 the correspondence between quotient types in \mltt{} and coequalizers
 in a category of types to define a notion called \emph{squash types}
 which is further discussed by Nogin.


To adding quotient types to \mltt{}, Hofmann proposes  three models for
quotient types in his PhD thesis \cite{hof:phd}. The first one is a setoid model for
quotient types. In this model all types are attached with partial
equivalence relations, namely all types are setoids rather than
sets. Types without specific equivalence relation can be translated as
setoids with the basic intensional equality. It looks a bit like
\ett{}. The second one is groupoid model which solves some problems
but it is not definable in \itt{}. He also proposes a third model to
combine the advantages of the first two models, but it also has some
disadvantages. Later in \cite{hof:95:sm} he gives a simple model in which we have type dependency only at the propositional level, he also shows that extensional Type Theory is conservative over \itt {}  extended with quotient types and a universe \cite{hof:95:con}.

Nogin \cite{nog:02} considers a modular approach to axiomatizing the
same quotient types in NuPRL as well. Despite the ease of constructing new types
from base types, he also discusses some
problems about quotient types. For example, since the the equality is
extensional, we can not recover the
witness of the equality.  He suggests to include more axioms to
conceptualise quotients. He decomposes the formalisation of quotient type
into several smaller primitives such that they can be handled much
simpler.

Homeier \cite{hom} axiomatises quotient types in Higher Order Logic
(HOL), which is also a theorem prover. He creates a tool package to
construct quotient types as a conservative extension of HOL such that
users are able to define new types in HOL. Next he defines the
normalisation functions and proves several properties of
these. Finally he discussed the issues when quotienting on the
aggregate types such as lists and pairs.


Courtieu \cite{cou:01} shows an extension of Calculus of Inductive Constructions
with \emph{Normalised Types} which are similar to quotient types, but equivalence relations are replaced by normalisation functions. 
However not all quotient types have normal forms. Normalised types are
proper subsets of quotient types, because we can easily recover a quotient
type from a normalised type as below
\[ (A, Q, \class\dotph \colon A \to Q) \Rightarrow(A, \lambda \,a \,b\to \class a = \class b)\]


Barthe and Geunvers \cite{bar:96} also proposes a new notion called
\emph{congruence types}, which is also a special class of quotient
types, in which the base type are inductively defined and with a set
of reduction rules called the term-rewriting system. The idea behind
it is the $\beta$-equivalence is replaced by a set of
$\beta$-conversion rules. Congruence types can be treated as an
alternative to pattern matching introduced in \cite{coq:92}. The main
purpose of introduction of congruence types is to solve problems in
term rewriting systems rather than to implement quotient types.


\section{Aims and Objectives of the Project}

The objective of this project is to investigate and explore the ways of
implementing quotients in \mltt{}, especially in intensional
one where type checking always terminates.
As we have seen quotients can enable defining various kinds of
mathematical notions or programming datatypes, the introduction of
quotient types will be quite beneficial in theorem provers and
programming languages based on Type Theory. 

The ultimate aim is to extend \itt{} such that all quotients can be
defined and handled easily and correctly without losing consistency
and good features of \itt{}.

The project will be undertaken step by step. Firstly, we should make
the basic notions clear, for example what is quotients and if we want
quotients in Type Theory what kind of problems need to be solved. We
also need to do research on related works on this topic as much as possible.
The second step is working in current setting of \itt{},
investigating some definable quotients, and building the module structure of quotients. The module structure
and some research on definable quotients has been done in
\cite{aan}.

Next we need to investigate some undefinable quotients such as the set
of real numbers $\R$ and partiality monads and prove why they are undefinable. The
key different characters between definable and undefinable quotients
will be studied. You can find the proof of why $\R$ is undefinable in
\cite{aan} as well.

The development of framework of quotient types in \itt{} is one of the major
objective. We need to propose a set of rules to axiomatise quotient
types in \itt{}. To test our approach with a few typical quotients
to explore its potential benefits. The correctness of axiomatisation
and the consistency of extended \itt{} require formal proofs.

A possible intermediate result is to extend the addition of extensionality in \itt{}
\cite{txa:99}. The conservativity of \itt{} with quotient types over
\ett{} could be proved following the work in \cite{hof:95}. 

Finally, we will summarise the approach of defining quotient types in
\itt{}, the benefits of it and the application of it into a PhD Thesis.


\section{Theoretical Methods}

To do this research, we need to review several related work, compare
the existing approaches in different implementations of \mltt{}
and try to figure out the best approach by testing it in real cases.

As we have mentioned before, Agda is a good implementation of
\itt{}. Conducting this research in Agda will be very efficient and
useful. We can verify our proofs in it and try to apply quotient types
to a lot of practical examples.

We also need to advertise our work, get feedbacks from the users and
improve our approaches such that they are more applicable and easier
to use.

\section{Results and Discussion}


\subsection{Definitions}

Currently, we are on the first stage and there are some progression on
definable quotients in \itt{} \cite{aan}. Here I will present some necessary knowledge from that paper.

%It is about the definable quotients and some undefinable
%quotients. Here we only talk about the quotient set, but it is
%universal polymorphic. 

During the first stage, the aim is to explore the potential to define quotients
in current setting of \itt{}. 


Given a setoid $(A,\sim)$, we know what is a quotient type but we can
not define it from the setoid because there is no axiomatised quotient
types. If means that we can just prove some type is quotient type of given
setoid. Therefore the only way to introduce possible quotient types $Q
\,\colon \Set$ is to define it by ourselves. With defined $Q$ and
$(A,\sim)$, we are required construct some structures of quotients in
\cite{aan} which consists of a set of essential properties of quotients.

Here I will explain these structures by using the example of integers
in Agda. All integers are the result of subtraction between two
natural numbers. Therefore we can use a pair of natural numbers in a subtraction
expression to represent the resulting integer.
For example, $1 - 4 = - 3$ says that the pair of natural numbers $(1,4)$
represents the integer $- 3$. Assuming we have the necessary definitions
of natural numbers, the base type of this quotient is

$$\Z_0=\N \times \N$$

Mathematically we know that for any two pairs of natural numbers $(n_1, n_2)$ and $(n_3, n_4)$, 
$$ n_1 - n_2 = n_3 - n_4\iff n_1 + n_4 = n_3 + n_2$$

Because the results of subtraction are the same, we can infer that the
two pairs represent the same integer, so the equivalence relation
$\sim$ for $\Z_0$ could be defined as

\begin{hscode}\SaveRestoreHook
\column{B}{@{}>{\hspre}l<{\hspost}@{}}%
\column{E}{@{}>{\hspre}l<{\hspost}@{}}%
\>[B]{}\Varid{\char95 ∼\char95 }\;\mathbin{:}\;\Conid{Rel}\;\Conid{ℤ₀}\;\Varid{zero}{}\<[E]%
\\
\>[B]{}(\Varid{x+},\Varid{x-})\;\Varid{∼}\;(\Varid{y+},\Varid{y-})\;\mathrel{=}\;(\Varid{x+}\;\Conid{ℕ+}\;\Varid{y-})\;\Varid{≡}\;(\Varid{y+}\;\Conid{ℕ+}\;\Varid{x-}){}\<[E]%
\ColumnHook
\end{hscode}\resethooks

Here \ensuremath{\Varid{≡}} is the propositional equality. Of course we must prove \ensuremath{\Varid{∼}} is an equivalence relation then we can define the setoid $(\Z_0,\sim)$ in Agda as

\begin{hscode}\SaveRestoreHook
\column{B}{@{}>{\hspre}l<{\hspost}@{}}%
\column{3}{@{}>{\hspre}l<{\hspost}@{}}%
\column{19}{@{}>{\hspre}l<{\hspost}@{}}%
\column{E}{@{}>{\hspre}l<{\hspost}@{}}%
\>[B]{}\Conid{ℤ-Setoid}\;\mathbin{:}\;\Conid{Setoid}{}\<[E]%
\\
\>[B]{}\Conid{ℤ-Setoid}\;\mathrel{=}\;\Keyword{record}{}\<[E]%
\\
\>[B]{}\hsindent{3}{}\<[3]%
\>[3]{}\{\mskip1.5mu \Conid{Carrier}\;{}\<[19]%
\>[19]{}\mathrel{=}\;\Conid{ℤ₀}{}\<[E]%
\\
\>[B]{}\hsindent{3}{}\<[3]%
\>[3]{};\Varid{\char95 ≈\char95 }\;{}\<[19]%
\>[19]{}\mathrel{=}\;\Varid{\char95 ∼\char95 }{}\<[E]%
\\
\>[B]{}\hsindent{3}{}\<[3]%
\>[3]{};\Varid{isEquivalence}\;\mathrel{=}\;\Varid{\char95 ∼\char95 isEquivalence}{}\<[E]%
\\
\>[B]{}\hsindent{3}{}\<[3]%
\>[3]{}\mskip1.5mu\}{}\<[E]%
\ColumnHook
\end{hscode}\resethooks

In set theory, we can immediately derive the quotient set from this
setoid which is the set of integers $\Z$, but in current setting of \itt{},
we need to define $\Z$ as follows

\begin{hscode}\SaveRestoreHook
\column{B}{@{}>{\hspre}l<{\hspost}@{}}%
\column{3}{@{}>{\hspre}l<{\hspost}@{}}%
\column{9}{@{}>{\hspre}l<{\hspost}@{}}%
\column{E}{@{}>{\hspre}l<{\hspost}@{}}%
\>[B]{}\Keyword{data}\;\Conid{ℤ}\;\mathbin{:}\;\Conid{Set}\;\Keyword{where}{}\<[E]%
\\
\>[B]{}\hsindent{3}{}\<[3]%
\>[3]{}\Varid{+\char95 }\;{}\<[9]%
\>[9]{}\mathbin{:}\;(\Varid{n}\;\mathbin{:}\;\Conid{ℕ})\;\Varid{→}\;\Conid{ℤ}{}\<[E]%
\\
\>[B]{}\hsindent{3}{}\<[3]%
\>[3]{}\Varid{-suc\char95 }\;\mathbin{:}\;(\Varid{n}\;\mathbin{:}\;\Conid{ℕ})\;\Varid{→}\;\Conid{ℤ}{}\<[E]%
\ColumnHook
\end{hscode}\resethooks

This is called normal form or canonical form of integers.

The next step is to prove that it is the quotient type of the setoid $(\Z_0,\sim)$.
To relate the setoid and the potential quotient type, we need to
provide a mapping function from the base type $\Z_0$ to the target
type $\Z$ which should be the normalisation function

\begin{hscode}\SaveRestoreHook
\column{B}{@{}>{\hspre}l<{\hspost}@{}}%
\column{23}{@{}>{\hspre}l<{\hspost}@{}}%
\column{E}{@{}>{\hspre}l<{\hspost}@{}}%
\>[B]{}[\mskip1.5mu \anonymous \mskip1.5mu]\;{}\<[23]%
\>[23]{}\mathbin{:}\;\Conid{ℤ₀}\;\Varid{→}\;\Conid{ℤ}{}\<[E]%
\\
\>[B]{}[\mskip1.5mu \Varid{m},\Varid{0}\mskip1.5mu]\;{}\<[23]%
\>[23]{}\mathrel{=}\;\Varid{+}\;\Varid{m}{}\<[E]%
\\
\>[B]{}[\mskip1.5mu \Varid{0},\Conid{ℕ.suc}\;\Varid{n}\mskip1.5mu]\;{}\<[23]%
\>[23]{}\mathrel{=}\;\Varid{-suc}\;\Varid{n}{}\<[E]%
\\
\>[B]{}[\mskip1.5mu \Conid{ℕ.suc}\;\Varid{m},\Conid{ℕ.suc}\;\Varid{n}\mskip1.5mu]\;\mathrel{=}\;[\mskip1.5mu \Varid{m},\Varid{n}\mskip1.5mu]{}\<[E]%
\ColumnHook
\end{hscode}\resethooks

The first property to prove is the \emph{sound} property,

\begin{hscode}\SaveRestoreHook
\column{B}{@{}>{\hspre}l<{\hspost}@{}}%
\column{23}{@{}>{\hspre}l<{\hspost}@{}}%
\column{E}{@{}>{\hspre}l<{\hspost}@{}}%
\>[B]{}\Varid{sound}\;{}\<[23]%
\>[23]{}\mathbin{:}\;\Varid{∀}\;\{\mskip1.5mu \Varid{x}\;\Varid{y}\mskip1.5mu\}\;\Varid{→}\;\Varid{x}\;\Varid{∼}\;\Varid{y}\;\Varid{→}\;[\mskip1.5mu \Varid{x}\mskip1.5mu]\;\Varid{≡}\;[\mskip1.5mu \Varid{y}\mskip1.5mu]{}\<[E]%
\\
\>[B]{}\Varid{sound}\;\{\mskip1.5mu \Varid{x}\mskip1.5mu\}\;\{\mskip1.5mu \Varid{y}\mskip1.5mu\}\;\Varid{x∼y}\;\mathrel{=}\;\Varid{⌞}\;\Varid{compl}\;\Varid{>∼<}\;\Varid{x∼y}\;\Varid{>∼<}\;\Varid{compl'}\;\Varid{⌟}{}\<[E]%
\ColumnHook
\end{hscode}\resethooks

The normalised results of two propositional equal elements of $\Z_0$
should be the same. With this property, we are able to form a prequotient which is defined as


\begin{hscode}\SaveRestoreHook
\column{B}{@{}>{\hspre}l<{\hspost}@{}}%
\column{3}{@{}>{\hspre}l<{\hspost}@{}}%
\column{5}{@{}>{\hspre}l<{\hspost}@{}}%
\column{9}{@{}>{\hspre}l<{\hspost}@{}}%
\column{11}{@{}>{\hspre}l<{\hspost}@{}}%
\column{E}{@{}>{\hspre}l<{\hspost}@{}}%
\>[B]{}\Keyword{record}\;\Conid{PreQu}\;(\Conid{S}\;\mathbin{:}\;\Conid{Setoid})\;\mathbin{:}\;\Conid{Set₁}\;\Keyword{where}{}\<[E]%
\\
\>[B]{}\hsindent{3}{}\<[3]%
\>[3]{}\Varid{constructor}\;{}\<[E]%
\\
\>[3]{}\hsindent{2}{}\<[5]%
\>[5]{}\Conid{Q:\char95 }\;[\mskip1.5mu \mskip1.5mu]\;\Conid{:\char95 sound:\char95 }{}\<[E]%
\\
\>[B]{}\hsindent{3}{}\<[3]%
\>[3]{}\Keyword{private}{}\<[E]%
\\
\>[3]{}\hsindent{2}{}\<[5]%
\>[5]{}\Conid{A}\;{}\<[9]%
\>[9]{}\mathrel{=}\;\Conid{Carrier}\;\Conid{S}{}\<[E]%
\\
\>[3]{}\hsindent{2}{}\<[5]%
\>[5]{}\Varid{\char95 ∼\char95 }\;\mathrel{=}\;\Varid{\char95 ≈\char95 }\;\Conid{S}{}\<[E]%
\\
\>[B]{}\hsindent{3}{}\<[3]%
\>[3]{}\Keyword{field}{}\<[E]%
\\
\>[3]{}\hsindent{2}{}\<[5]%
\>[5]{}\Conid{Q}\;{}\<[11]%
\>[11]{}\mathbin{:}\;\Conid{Set}{}\<[E]%
\\
\>[3]{}\hsindent{2}{}\<[5]%
\>[5]{}[\mskip1.5mu \anonymous \mskip1.5mu]\;{}\<[11]%
\>[11]{}\mathbin{:}\;\Conid{A}\;\Varid{→}\;\Conid{Q}{}\<[E]%
\\
\>[3]{}\hsindent{2}{}\<[5]%
\>[5]{}\Varid{sound}\;\mathbin{:}\;\Varid{∀}\;\{\mskip1.5mu \Varid{a}\;\Varid{b}\;\mathbin{:}\;\Conid{A}\mskip1.5mu\}\;\Varid{→}\;\Varid{a}\;\Varid{∼}\;\Varid{b}\;\Varid{→}\;[\mskip1.5mu \Varid{a}\mskip1.5mu]\;\Varid{≡}\;[\mskip1.5mu \Varid{b}\mskip1.5mu]{}\<[E]%
\ColumnHook
\end{hscode}\resethooks

and the prequotient of integers is,

\begin{hscode}\SaveRestoreHook
\column{B}{@{}>{\hspre}l<{\hspost}@{}}%
\column{3}{@{}>{\hspre}l<{\hspost}@{}}%
\column{13}{@{}>{\hspre}l<{\hspost}@{}}%
\column{E}{@{}>{\hspre}l<{\hspost}@{}}%
\>[B]{}\Conid{ℤ-PreQu}\;\mathbin{:}\;\Conid{PreQu}\;\Conid{ℤ-Setoid}{}\<[E]%
\\
\>[B]{}\Conid{ℤ-PreQu}\;\mathrel{=}\;\Keyword{record}{}\<[E]%
\\
\>[B]{}\hsindent{3}{}\<[3]%
\>[3]{}\{\mskip1.5mu \Conid{Q}\;{}\<[13]%
\>[13]{}\mathrel{=}\;\Conid{ℤ}{}\<[E]%
\\
\>[B]{}\hsindent{3}{}\<[3]%
\>[3]{};[\mskip1.5mu \anonymous \mskip1.5mu]\;{}\<[13]%
\>[13]{}\mathrel{=}\;[\mskip1.5mu \anonymous \mskip1.5mu]{}\<[E]%
\\
\>[B]{}\hsindent{3}{}\<[3]%
\>[3]{};\Varid{sound}\;{}\<[13]%
\>[13]{}\mathrel{=}\;\Varid{sound}{}\<[E]%
\\
\>[B]{}\hsindent{3}{}\<[3]%
\>[3]{}\mskip1.5mu\}{}\<[E]%
\ColumnHook
\end{hscode}\resethooks

To form quotients we have several different definitions as written in \cite{aan},

\begin{enumerate}

\item \emph{Quotient with a dependent eliminator}

\begin{hscode}\SaveRestoreHook
\column{B}{@{}>{\hspre}l<{\hspost}@{}}%
\column{3}{@{}>{\hspre}l<{\hspost}@{}}%
\column{5}{@{}>{\hspre}l<{\hspost}@{}}%
\column{11}{@{}>{\hspre}l<{\hspost}@{}}%
\column{13}{@{}>{\hspre}l<{\hspost}@{}}%
\column{17}{@{}>{\hspre}l<{\hspost}@{}}%
\column{50}{@{}>{\hspre}l<{\hspost}@{}}%
\column{E}{@{}>{\hspre}l<{\hspost}@{}}%
\>[B]{}\Keyword{record}\;\Conid{Qu}\;\{\mskip1.5mu \Conid{S}\;\mathbin{:}\;\Conid{Setoid}\mskip1.5mu\}\;(\Conid{PQ}\;\mathbin{:}\;\Conid{PreQu}\;\Conid{S})\;\mathbin{:}\;\Conid{Set₁}\;\Keyword{where}{}\<[E]%
\\
\>[B]{}\hsindent{3}{}\<[3]%
\>[3]{}\Varid{constructor}\;{}\<[E]%
\\
\>[3]{}\hsindent{2}{}\<[5]%
\>[5]{}\Varid{qelim:\char95 qelim-β:\char95 }{}\<[E]%
\\
\>[B]{}\hsindent{3}{}\<[3]%
\>[3]{}\Keyword{private}{}\<[E]%
\\
\>[3]{}\hsindent{2}{}\<[5]%
\>[5]{}\Conid{A}\;{}\<[11]%
\>[11]{}\mathrel{=}\;\Conid{Carrier}\;\Conid{S}{}\<[E]%
\\
\>[3]{}\hsindent{2}{}\<[5]%
\>[5]{}\Varid{\char95 ∼\char95 }\;{}\<[11]%
\>[11]{}\mathrel{=}\;\Varid{\char95 ≈\char95 }\;\Conid{S}{}\<[E]%
\\
\>[3]{}\hsindent{2}{}\<[5]%
\>[5]{}\Conid{Q}\;{}\<[11]%
\>[11]{}\mathrel{=}\;\Conid{Q'}\;\Conid{PQ}{}\<[E]%
\\
\>[3]{}\hsindent{2}{}\<[5]%
\>[5]{}[\mskip1.5mu \anonymous \mskip1.5mu]\;{}\<[11]%
\>[11]{}\mathrel{=}\;\Varid{nf}\;\Conid{PQ}{}\<[E]%
\\
\>[3]{}\hsindent{2}{}\<[5]%
\>[5]{}\Varid{sound}\;\mathbin{:}\;\Varid{∀}\;\{\mskip1.5mu \Varid{a}\;\Varid{b}\;\mathbin{:}\;\Conid{A}\mskip1.5mu\}\;\Varid{→}\;(\Varid{a}\;\Varid{∼}\;\Varid{b})\;\Varid{→}\;[\mskip1.5mu \Varid{a}\mskip1.5mu]\;\Varid{≡}\;[\mskip1.5mu \Varid{b}\mskip1.5mu]{}\<[E]%
\\
\>[3]{}\hsindent{2}{}\<[5]%
\>[5]{}\Varid{sound}\;\mathrel{=}\;\Varid{sound'}\;\Conid{PQ}{}\<[E]%
\\
\>[B]{}\hsindent{3}{}\<[3]%
\>[3]{}\Keyword{field}{}\<[E]%
\\
\>[3]{}\hsindent{2}{}\<[5]%
\>[5]{}\Varid{qelim}\;{}\<[13]%
\>[13]{}\mathbin{:}\;\{\mskip1.5mu \Conid{B}\;\mathbin{:}\;\Conid{Q}\;\Varid{→}\;\Conid{Set}\mskip1.5mu\}\;{}\<[E]%
\\
\>[13]{}\Varid{→}\;(\Varid{f}\;\mathbin{:}\;(\Varid{a}\;\mathbin{:}\;\Conid{A})\;\Varid{→}\;\Conid{B}\;[\mskip1.5mu \Varid{a}\mskip1.5mu])\;{}\<[E]%
\\
\>[13]{}\Varid{→}\;((\Varid{a}\;\Varid{b}\;\mathbin{:}\;\Conid{A})\;\Varid{→}\;(\Varid{p}\;\mathbin{:}\;\Varid{a}\;\Varid{∼}\;\Varid{b})\;{}\<[E]%
\\
\>[13]{}\hsindent{4}{}\<[17]%
\>[17]{}\Varid{→}\;\Varid{subst}\;\Conid{B}\;(\Varid{sound}\;\Varid{p})\;(\Varid{f}\;\Varid{a})\;\Varid{≡}\;\Varid{f}\;\Varid{b})\;{}\<[E]%
\\
\>[13]{}\Varid{→}\;(\Varid{q}\;\mathbin{:}\;\Conid{Q})\;\Varid{→}\;\Conid{B}\;\Varid{q}{}\<[E]%
\\
\>[3]{}\hsindent{2}{}\<[5]%
\>[5]{}\Varid{qelim-β}\;\mathbin{:}\;\Varid{∀}\;\{\mskip1.5mu \Conid{B}\;\Varid{a}\;\Varid{f}\mskip1.5mu\}\;\Varid{q}\;\Varid{→}\;\Varid{qelim}\;\{\mskip1.5mu \Conid{B}\mskip1.5mu\}\;\Varid{f}\;\Varid{q}\;[\mskip1.5mu \Varid{a}\mskip1.5mu]\;{}\<[50]%
\>[50]{}\Varid{≡}\;\Varid{f}\;\Varid{a}{}\<[E]%
\ColumnHook
\end{hscode}\resethooks


\item \emph{Exact (or efficient) quotient}

\begin{hscode}\SaveRestoreHook
\column{B}{@{}>{\hspre}l<{\hspost}@{}}%
\column{3}{@{}>{\hspre}l<{\hspost}@{}}%
\column{5}{@{}>{\hspre}l<{\hspost}@{}}%
\column{10}{@{}>{\hspre}l<{\hspost}@{}}%
\column{11}{@{}>{\hspre}l<{\hspost}@{}}%
\column{E}{@{}>{\hspre}l<{\hspost}@{}}%
\>[B]{}\Keyword{record}\;\Conid{QuE}\;\{\mskip1.5mu \Conid{S}\;\mathbin{:}\;\Conid{Setoid}\mskip1.5mu\}\;\{\mskip1.5mu \Conid{PQ}\;\mathbin{:}\;\Conid{PreQu}\;\Conid{S}\mskip1.5mu\}\;(\Conid{QU}\;\mathbin{:}\;\Conid{Qu}\;\Conid{PQ})\;\mathbin{:}\;\Conid{Set₁}\;\Keyword{where}{}\<[E]%
\\
\>[B]{}\hsindent{3}{}\<[3]%
\>[3]{}\Varid{constructor}\;{}\<[E]%
\\
\>[3]{}\hsindent{2}{}\<[5]%
\>[5]{}\Varid{exact:\char95 }{}\<[E]%
\\
\>[B]{}\hsindent{3}{}\<[3]%
\>[3]{}\Keyword{private}{}\<[E]%
\\
\>[3]{}\hsindent{2}{}\<[5]%
\>[5]{}\Conid{A}\;{}\<[11]%
\>[11]{}\mathrel{=}\;\Conid{Carrier}\;\Conid{S}{}\<[E]%
\\
\>[3]{}\hsindent{2}{}\<[5]%
\>[5]{}\Varid{\char95 ∼\char95 }\;{}\<[10]%
\>[10]{}\mathrel{=}\;\Varid{\char95 ≈\char95 }\;\Conid{S}{}\<[E]%
\\
\>[3]{}\hsindent{2}{}\<[5]%
\>[5]{}[\mskip1.5mu \anonymous \mskip1.5mu]\;{}\<[10]%
\>[10]{}\mathrel{=}\;\Varid{nf}\;\Conid{PQ}{}\<[E]%
\\
\>[B]{}\hsindent{3}{}\<[3]%
\>[3]{}\Keyword{field}{}\<[E]%
\\
\>[3]{}\hsindent{2}{}\<[5]%
\>[5]{}\Varid{exact}\;\mathbin{:}\;\Varid{∀}\;\{\mskip1.5mu \Varid{a}\;\Varid{b}\;\mathbin{:}\;\Conid{A}\mskip1.5mu\}\;\Varid{→}\;[\mskip1.5mu \Varid{a}\mskip1.5mu]\;\Varid{≡}\;[\mskip1.5mu \Varid{b}\mskip1.5mu]\;\Varid{→}\;\Varid{a}\;\Varid{∼}\;\Varid{b}{}\<[E]%
\ColumnHook
\end{hscode}\resethooks

\item \emph{Quotient with a non-dependent eliminator and induction principle}

\begin{hscode}\SaveRestoreHook
\column{B}{@{}>{\hspre}l<{\hspost}@{}}%
\column{3}{@{}>{\hspre}l<{\hspost}@{}}%
\column{5}{@{}>{\hspre}l<{\hspost}@{}}%
\column{11}{@{}>{\hspre}l<{\hspost}@{}}%
\column{12}{@{}>{\hspre}l<{\hspost}@{}}%
\column{42}{@{}>{\hspre}l<{\hspost}@{}}%
\column{45}{@{}>{\hspre}l<{\hspost}@{}}%
\column{48}{@{}>{\hspre}l<{\hspost}@{}}%
\column{E}{@{}>{\hspre}l<{\hspost}@{}}%
\>[B]{}\Keyword{record}\;\Conid{QuH}\;\{\mskip1.5mu \Conid{S}\;\mathbin{:}\;\Conid{Setoid}\mskip1.5mu\}\;(\Conid{PQ}\;\mathbin{:}\;\Conid{PreQu}\;\Conid{S})\;\mathbin{:}\;\Conid{Set₁}\;\Keyword{where}{}\<[E]%
\\
\>[B]{}\hsindent{3}{}\<[3]%
\>[3]{}\Varid{constructor}\;{}\<[E]%
\\
\>[3]{}\hsindent{2}{}\<[5]%
\>[5]{}\Varid{lift:\char95 lift-β:\char95 qind:\char95 }{}\<[E]%
\\
\>[B]{}\hsindent{3}{}\<[3]%
\>[3]{}\Keyword{private}{}\<[E]%
\\
\>[3]{}\hsindent{2}{}\<[5]%
\>[5]{}\Conid{A}\;{}\<[12]%
\>[12]{}\mathrel{=}\;\Conid{Carrier}\;\Conid{S}{}\<[E]%
\\
\>[3]{}\hsindent{2}{}\<[5]%
\>[5]{}\Varid{\char95 ∼\char95 }\;{}\<[12]%
\>[12]{}\mathrel{=}\;\Varid{\char95 ≈\char95 }\;\Conid{S}{}\<[E]%
\\
\>[3]{}\hsindent{2}{}\<[5]%
\>[5]{}\Conid{Q}\;{}\<[12]%
\>[12]{}\mathrel{=}\;\Conid{Q'}\;\Conid{PQ}{}\<[E]%
\\
\>[3]{}\hsindent{2}{}\<[5]%
\>[5]{}[\mskip1.5mu \anonymous \mskip1.5mu]\;{}\<[12]%
\>[12]{}\mathrel{=}\;\Varid{nf}\;\Conid{PQ}{}\<[E]%
\\
\>[B]{}\hsindent{3}{}\<[3]%
\>[3]{}\Keyword{field}{}\<[E]%
\\
\>[3]{}\hsindent{2}{}\<[5]%
\>[5]{}\Varid{lift}\;{}\<[12]%
\>[12]{}\mathbin{:}\;\{\mskip1.5mu \Conid{B}\;\mathbin{:}\;\Conid{Set}\mskip1.5mu\}\;{}\<[E]%
\\
\>[12]{}\Varid{→}\;(\Varid{f}\;\mathbin{:}\;\Conid{A}\;\Varid{→}\;\Conid{B})\;{}\<[E]%
\\
\>[12]{}\Varid{→}\;((\Varid{a}\;\Varid{b}\;\mathbin{:}\;\Conid{A})\;\Varid{→}\;(\Varid{a}\;\Varid{∼}\;\Varid{b})\;\Varid{→}\;\Varid{f}\;\Varid{a}\;{}\<[42]%
\>[42]{}\Varid{≡}\;{}\<[45]%
\>[45]{}\Varid{f}\;\Varid{b})\;{}\<[E]%
\\
\>[12]{}\Varid{→}\;\Conid{Q}\;\Varid{→}\;\Conid{B}{}\<[E]%
\\
\>[3]{}\hsindent{2}{}\<[5]%
\>[5]{}\Varid{lift-β}\;\mathbin{:}\;\Varid{∀}\;\{\mskip1.5mu \Conid{B}\;\Varid{a}\;\Varid{f}\;\Varid{q}\mskip1.5mu\}\;\Varid{→}\;\Varid{lift}\;\{\mskip1.5mu \Conid{B}\mskip1.5mu\}\;\Varid{f}\;\Varid{q}\;[\mskip1.5mu \Varid{a}\mskip1.5mu]\;{}\<[48]%
\>[48]{}\Varid{≡}\;\Varid{f}\;\Varid{a}{}\<[E]%
\\
\>[3]{}\hsindent{2}{}\<[5]%
\>[5]{}\Varid{qind}\;{}\<[11]%
\>[11]{}\mathbin{:}\;(\Conid{P}\;\mathbin{:}\;\Conid{Q}\;\Varid{→}\;\Conid{Set})\;{}\<[E]%
\\
\>[11]{}\hsindent{1}{}\<[12]%
\>[12]{}\Varid{→}\;(\Varid{∀}\;\Varid{x}\;\Varid{→}\;(\Varid{p}\;\Varid{p'}\;\mathbin{:}\;\Conid{P}\;\Varid{x})\;\Varid{→}\;\Varid{p}\;\Varid{≡}\;\Varid{p'})\;{}\<[E]%
\\
\>[11]{}\hsindent{1}{}\<[12]%
\>[12]{}\Varid{→}\;(\Varid{∀}\;\Varid{a}\;\Varid{→}\;\Conid{P}\;[\mskip1.5mu \Varid{a}\mskip1.5mu])\;{}\<[E]%
\\
\>[11]{}\hsindent{1}{}\<[12]%
\>[12]{}\Varid{→}\;(\Varid{∀}\;\Varid{x}\;\Varid{→}\;\Conid{P}\;\Varid{x}){}\<[E]%
\ColumnHook
\end{hscode}\resethooks

\item \emph{Definable quotient}
 
\begin{hscode}\SaveRestoreHook
\column{B}{@{}>{\hspre}l<{\hspost}@{}}%
\column{3}{@{}>{\hspre}l<{\hspost}@{}}%
\column{5}{@{}>{\hspre}l<{\hspost}@{}}%
\column{11}{@{}>{\hspre}l<{\hspost}@{}}%
\column{14}{@{}>{\hspre}l<{\hspost}@{}}%
\column{E}{@{}>{\hspre}l<{\hspost}@{}}%
\>[B]{}\Keyword{record}\;\Conid{QuD}\;\{\mskip1.5mu \Conid{S}\;\mathbin{:}\;\Conid{Setoid}\mskip1.5mu\}\;(\Conid{PQ}\;\mathbin{:}\;\Conid{PreQu}\;\Conid{S})\;\mathbin{:}\;\Conid{Set₁}\;\Keyword{where}{}\<[E]%
\\
\>[B]{}\hsindent{3}{}\<[3]%
\>[3]{}\Varid{constructor}\;{}\<[E]%
\\
\>[3]{}\hsindent{2}{}\<[5]%
\>[5]{}\Varid{emb:\char95 complete:\char95 stable:\char95 }{}\<[E]%
\\
\>[B]{}\hsindent{3}{}\<[3]%
\>[3]{}\Keyword{private}{}\<[E]%
\\
\>[3]{}\hsindent{2}{}\<[5]%
\>[5]{}\Conid{A}\;{}\<[11]%
\>[11]{}\mathrel{=}\;\Conid{Carrier}\;\Conid{S}{}\<[E]%
\\
\>[3]{}\hsindent{2}{}\<[5]%
\>[5]{}\Varid{\char95 ∼\char95 }\;{}\<[11]%
\>[11]{}\mathrel{=}\;\Varid{\char95 ≈\char95 }\;\Conid{S}{}\<[E]%
\\
\>[3]{}\hsindent{2}{}\<[5]%
\>[5]{}\Conid{Q}\;{}\<[11]%
\>[11]{}\mathrel{=}\;\Conid{Q'}\;\Conid{PQ}{}\<[E]%
\\
\>[3]{}\hsindent{2}{}\<[5]%
\>[5]{}[\mskip1.5mu \anonymous \mskip1.5mu]\;{}\<[11]%
\>[11]{}\mathrel{=}\;\Varid{nf}\;\Conid{PQ}{}\<[E]%
\\
\>[B]{}\hsindent{3}{}\<[3]%
\>[3]{}\Keyword{field}{}\<[E]%
\\
\>[3]{}\hsindent{2}{}\<[5]%
\>[5]{}\Varid{emb}\;{}\<[14]%
\>[14]{}\mathbin{:}\;\Conid{Q}\;\Varid{→}\;\Conid{A}{}\<[E]%
\\
\>[3]{}\hsindent{2}{}\<[5]%
\>[5]{}\Varid{complete}\;\mathbin{:}\;\Varid{∀}\;\Varid{a}\;\Varid{→}\;\Varid{emb}\;[\mskip1.5mu \Varid{a}\mskip1.5mu]\;\Varid{∼}\;\Varid{a}{}\<[E]%
\\
\>[3]{}\hsindent{2}{}\<[5]%
\>[5]{}\Varid{stable}\;{}\<[14]%
\>[14]{}\mathbin{:}\;\Varid{∀}\;\Varid{q}\;\Varid{→}\;[\mskip1.5mu \Varid{emb}\;\Varid{q}\mskip1.5mu]\;\Varid{≡}\;\Varid{q}{}\<[E]%
\ColumnHook
\end{hscode}\resethooks

\end{enumerate}

We have proved that the first and third definitions are equivalent and
the last one is the most strongest definition which can generate any
other from it \cite{aan}.

For integers, it is natural to define a function to choose a representative for each element in $\Z$,

\begin{hscode}\SaveRestoreHook
\column{B}{@{}>{\hspre}l<{\hspost}@{}}%
\column{12}{@{}>{\hspre}l<{\hspost}@{}}%
\column{E}{@{}>{\hspre}l<{\hspost}@{}}%
\>[B]{}\Varid{⌜\char95 ⌝}\;{}\<[12]%
\>[12]{}\mathbin{:}\;\Conid{ℤ}\;\Varid{→}\;\Conid{ℤ₀}{}\<[E]%
\\
\>[B]{}\Varid{⌜}\;\Varid{+}\;\Varid{n}\;\Varid{⌝}\;{}\<[12]%
\>[12]{}\mathrel{=}\;\Varid{n},\Varid{0}{}\<[E]%
\\
\>[B]{}\Varid{⌜}\;\Varid{-suc}\;\Varid{n}\;\Varid{⌝}\;\mathrel{=}\;\Varid{0},\Conid{ℕ.suc}\;\Varid{n}{}\<[E]%
\ColumnHook
\end{hscode}\resethooks

Now we need to prove \ensuremath{\Varid{⌜\char95 ⌝}} is the required embedding function, namely it is the inverse function of \ensuremath{[\mskip1.5mu \anonymous \mskip1.5mu]}.
Firstly \ensuremath{\Varid{⌜\char95 ⌝}} is left inverse of \ensuremath{[\mskip1.5mu \anonymous \mskip1.5mu]},


\begin{hscode}\SaveRestoreHook
\column{B}{@{}>{\hspre}l<{\hspost}@{}}%
\column{25}{@{}>{\hspre}l<{\hspost}@{}}%
\column{46}{@{}>{\hspre}l<{\hspost}@{}}%
\column{E}{@{}>{\hspre}l<{\hspost}@{}}%
\>[B]{}\Varid{compl}\;{}\<[25]%
\>[25]{}\mathbin{:}\;\Varid{∀}\;\{\mskip1.5mu \Varid{n}\mskip1.5mu\}\;\Varid{→}\;\Varid{⌜}\;[\mskip1.5mu \Varid{n}\mskip1.5mu]\;\Varid{⌝}\;\Varid{∼}\;\Varid{n}{}\<[E]%
\\
\>[B]{}\Varid{compl}\;\{\mskip1.5mu \Varid{x},\Varid{0}\mskip1.5mu\}\;{}\<[25]%
\>[25]{}\mathrel{=}\;\Varid{refl}{}\<[E]%
\\
\>[B]{}\Varid{compl}\;\{\mskip1.5mu \Varid{0},\Varid{nsuc}\;\Varid{y}\mskip1.5mu\}\;{}\<[25]%
\>[25]{}\mathrel{=}\;\Varid{refl}{}\<[E]%
\\
\>[B]{}\Varid{compl}\;\{\mskip1.5mu \Varid{nsuc}\;\Varid{x},\Varid{nsuc}\;\Varid{y}\mskip1.5mu\}\;\mathrel{=}\;\Varid{compl}\;\{\mskip1.5mu \Varid{x},\Varid{y}\mskip1.5mu\}\;\Varid{>∼<}\;{}\<[46]%
\>[46]{}\Varid{⟨}\;\Conid{ℕ.sm+n≡m+sn}\;\Varid{x}\;\Varid{y}\;\Varid{⟩}{}\<[E]%
\ColumnHook
\end{hscode}\resethooks

This is called the \emph{complete} property. 

Secondly \ensuremath{\Varid{⌜\char95 ⌝}} is right inverse of \ensuremath{[\mskip1.5mu \anonymous \mskip1.5mu]},

\begin{hscode}\SaveRestoreHook
\column{B}{@{}>{\hspre}l<{\hspost}@{}}%
\column{19}{@{}>{\hspre}l<{\hspost}@{}}%
\column{E}{@{}>{\hspre}l<{\hspost}@{}}%
\>[B]{}\Varid{stable}\;{}\<[19]%
\>[19]{}\mathbin{:}\;\Varid{∀}\;\{\mskip1.5mu \Varid{n}\mskip1.5mu\}\;\Varid{→}\;[\mskip1.5mu \Varid{⌜}\;\Varid{n}\;\Varid{⌝}\mskip1.5mu]\;\Varid{≡}\;\Varid{n}{}\<[E]%
\\
\>[B]{}\Varid{stable}\;\{\mskip1.5mu \Varid{+}\;\Varid{n}\mskip1.5mu\}\;{}\<[19]%
\>[19]{}\mathrel{=}\;\Varid{refl}{}\<[E]%
\\
\>[B]{}\Varid{stable}\;\{\mskip1.5mu \Varid{-suc}\;\Varid{n}\mskip1.5mu\}\;\mathrel{=}\;\Varid{refl}{}\<[E]%
\ColumnHook
\end{hscode}\resethooks

This is called the \emph{stable} property.

Now we can form the definable quotient structure with the prequotient we have,

\begin{hscode}\SaveRestoreHook
\column{B}{@{}>{\hspre}l<{\hspost}@{}}%
\column{3}{@{}>{\hspre}l<{\hspost}@{}}%
\column{15}{@{}>{\hspre}l<{\hspost}@{}}%
\column{E}{@{}>{\hspre}l<{\hspost}@{}}%
\>[B]{}\Conid{ℤ-QuD}\;\mathbin{:}\;\Conid{QuD}\;\Conid{ℤ-PreQu}{}\<[E]%
\\
\>[B]{}\Conid{ℤ-QuD}\;\mathrel{=}\;\Keyword{record}{}\<[E]%
\\
\>[B]{}\hsindent{3}{}\<[3]%
\>[3]{}\{\mskip1.5mu \Varid{emb}\;{}\<[15]%
\>[15]{}\mathrel{=}\;\Varid{⌜\char95 ⌝}{}\<[E]%
\\
\>[B]{}\hsindent{3}{}\<[3]%
\>[3]{};\Varid{complete}\;{}\<[15]%
\>[15]{}\mathrel{=}\;\Varid{λ}\;\Varid{z}\;\Varid{→}\;\Varid{compl}{}\<[E]%
\\
\>[B]{}\hsindent{3}{}\<[3]%
\>[3]{};\Varid{stable}\;{}\<[15]%
\>[15]{}\mathrel{=}\;\Varid{λ}\;\Varid{z}\;\Varid{→}\;\Varid{stable}{}\<[E]%
\\
\>[B]{}\hsindent{3}{}\<[3]%
\>[3]{}\mskip1.5mu\}{}\<[E]%
\ColumnHook
\end{hscode}\resethooks

Now we have the mapping between the base type $\Z_0$ and the target type $\Z$, and have proved that \ensuremath{[\mskip1.5mu \anonymous \mskip1.5mu]} is a normalisation function.

We can obtain the dependent and non-dependent eliminators by translate the definable quotient into other definitions,

\begin{hscode}\SaveRestoreHook
\column{B}{@{}>{\hspre}l<{\hspost}@{}}%
\column{E}{@{}>{\hspre}l<{\hspost}@{}}%
\>[B]{}\Conid{ℤ-Qu}\;\mathrel{=}\;\Conid{QuD→Qu}\;\Conid{ℤ-QuD}{}\<[E]%
\\[\blanklineskip]%
\>[B]{}\Conid{ℤ-QuE}\;\mathrel{=}\;\Conid{QuD→QuE}\;\{\mskip1.5mu \anonymous \mskip1.5mu\}\;\{\mskip1.5mu \anonymous \mskip1.5mu\}\;\{\mskip1.5mu \Conid{ℤ-Qu}\mskip1.5mu\}\;\Conid{ℤ-QuD}{}\<[E]%
\\[\blanklineskip]%
\>[B]{}\Conid{ℤ-QuH}\;\mathrel{=}\;\Conid{QuD→QuH}\;\Conid{ℤ-QuD}{}\<[E]%
\ColumnHook
\end{hscode}\resethooks

It is not easy to find what benefit we obtain from constructing
quotients. The real benefit is generated from the interaction between
setoids and quotient types.
Firstly the setoid definitions are usually more simpler than the normal
definitions. In the case of integers, the normal form have two
constructors. For propositions with only one argument, sometimes we
have to prove them for both cases in the canonical definition. With
the increasing number of arguments in propositions, the number of
cases we need to prove would increase exponentially. A real case is
when trying to prove the distributivity of multiplication over
addition for integers, there are too many cases which makes the
proving cumbersome and we can hardly save any effort from any theorems
we proved. However for the setoid definition of integers, a
proposition can be converted into another proposition on natural
numbers which is much convenient to prove because we do not need to
prove case by case and we have a bundle of
theorems for natural numbers. For example,


\begin{hscode}\SaveRestoreHook
\column{B}{@{}>{\hspre}l<{\hspost}@{}}%
\column{3}{@{}>{\hspre}l<{\hspost}@{}}%
\column{6}{@{}>{\hspre}l<{\hspost}@{}}%
\column{E}{@{}>{\hspre}l<{\hspost}@{}}%
\>[B]{}\Varid{distʳ}\;\mathbin{:}\;\Varid{\char95 *\char95 }\;\Conid{DistributesOverʳ}\;\Varid{\char95 +\char95 }{}\<[E]%
\\
\>[B]{}\Varid{distʳ}\;(\Varid{a},\Varid{b})\;(\Varid{c},\Varid{d})\;(\Varid{e},\Varid{f})\;\mathrel{=}\;{}\<[E]%
\\
\>[B]{}\hsindent{3}{}\<[3]%
\>[3]{}\Conid{ℕ.dist-lemʳ}\;\Varid{a}\;\Varid{b}\;\Varid{c}\;\Varid{d}\;\Varid{e}\;\Varid{f}\;\Varid{+=}\;{}\<[E]%
\\
\>[B]{}\hsindent{3}{}\<[3]%
\>[3]{}\Varid{⟨}\;{}\<[6]%
\>[6]{}\Conid{ℕ.dist-lemʳ}\;\Varid{b}\;\Varid{a}\;\Varid{c}\;\Varid{d}\;\Varid{e}\;\Varid{f}\;\Varid{⟩}{}\<[E]%
\ColumnHook
\end{hscode}\resethooks

Moreover, as we have constructed the semiring of natural numbers, it
is even simpler to use an automatic prover \emph{ring solver} to prove simple equation of natural numbers. 

The rest we have to do is to lift the properties proved for setoid definition to the ones for canonical definition.
We can easily lift n-ary operators defined for $\Z_0$ to the ones for $\Z$ by

\begin{hscode}\SaveRestoreHook
\column{B}{@{}>{\hspre}l<{\hspost}@{}}%
\column{E}{@{}>{\hspre}l<{\hspost}@{}}%
\>[B]{}\Varid{liftOp}\;\mathbin{:}\;\Varid{∀}\;\Varid{n}\;\Varid{→}\;\Conid{Op}\;\Varid{n}\;\Conid{ℤ₀}\;\Varid{→}\;\Conid{Op}\;\Varid{n}\;\Conid{ℤ}{}\<[E]%
\\
\>[B]{}\Varid{liftOp}\;\Varid{0}\;\Varid{op}\;\mathrel{=}\;[\mskip1.5mu \Varid{op}\mskip1.5mu]{}\<[E]%
\\
\>[B]{}\Varid{liftOp}\;(\Conid{ℕ.suc}\;\Varid{n})\;\Varid{op}\;\mathrel{=}\;\Varid{λ}\;\Varid{x}\;\Varid{→}\;\Varid{liftOp}\;\Varid{n}\;(\Varid{op}\;\Varid{⌜}\;\Varid{x}\;\Varid{⌝}){}\<[E]%
\ColumnHook
\end{hscode}\resethooks

However, this lift function is unsafe because some operations on
$\N\times\N$ do not make sense by applying this function. It is
similar to the situation when defining functions on types with
replaced equality in \ett{}. The solution is to lift functions which
respects the equivalence relation. I only define the two most commonly
used safe lifting functions

\begin{hscode}\SaveRestoreHook
\column{B}{@{}>{\hspre}l<{\hspost}@{}}%
\column{15}{@{}>{\hspre}l<{\hspost}@{}}%
\column{16}{@{}>{\hspre}l<{\hspost}@{}}%
\column{E}{@{}>{\hspre}l<{\hspost}@{}}%
\>[B]{}\Varid{liftOp1safe}\;\mathbin{:}\;(\Varid{f}\;\mathbin{:}\;\Conid{Op}\;\Varid{1}\;\Conid{ℤ₀})\;\Varid{→}\;{}\<[E]%
\\
\>[B]{}\hsindent{15}{}\<[15]%
\>[15]{}(\Varid{∀}\;\{\mskip1.5mu \Varid{a}\;\Varid{b}\mskip1.5mu\}\;\Varid{→}\;\Varid{a}\;\Varid{∼}\;\Varid{b}\;\Varid{→}\;\Varid{f}\;\Varid{a}\;\Varid{∼}\;\Varid{f}\;\Varid{b})\;\Varid{→}\;{}\<[E]%
\\
\>[15]{}\hsindent{1}{}\<[16]%
\>[16]{}\Conid{Op₁}\;\Conid{ℤ}{}\<[E]%
\\
\>[B]{}\Varid{liftOp1safe}\;\Varid{f}\;\Varid{cong}\;\mathrel{=}\;\Varid{λ}\;\Varid{n}\;\Varid{→}\;[\mskip1.5mu \Varid{f}\;\Varid{⌜}\;\Varid{n}\;\Varid{⌝}\mskip1.5mu]{}\<[E]%
\\[\blanklineskip]%
\>[B]{}\Varid{liftOp2safe}\;\mathbin{:}\;(\Varid{op}\;\mathbin{:}\;\Conid{Op}\;\Varid{2}\;\Conid{ℤ₀})\;\Varid{→}\;{}\<[E]%
\\
\>[B]{}\hsindent{15}{}\<[15]%
\>[15]{}(\Varid{∀}\;\{\mskip1.5mu \Varid{a}\;\Varid{b}\;\Varid{c}\;\Varid{d}\mskip1.5mu\}\;\Varid{→}\;\Varid{a}\;\Varid{∼}\;\Varid{b}\;\Varid{→}\;\Varid{c}\;\Varid{∼}\;\Varid{d}\;\Varid{→}{}\<[E]%
\\
\>[15]{}\hsindent{1}{}\<[16]%
\>[16]{}\Varid{op}\;\Varid{a}\;\Varid{c}\;\Varid{∼}\;\Varid{op}\;\Varid{b}\;\Varid{d})\;\Varid{→}\;{}\<[E]%
\\
\>[B]{}\hsindent{15}{}\<[15]%
\>[15]{}\Conid{Op₂}\;\Conid{ℤ}{}\<[E]%
\\
\>[B]{}\Varid{liftOp2safe}\;\Varid{\char95 op\char95 }\;\Varid{cong}\;\mathrel{=}\;\Varid{λ}\;\Varid{m}\;\Varid{n}\;\Varid{→}\;[\mskip1.5mu \Varid{⌜}\;\Varid{m}\;\Varid{⌝}\;\Varid{op}\;\Varid{⌜}\;\Varid{n}\;\Varid{⌝}\mskip1.5mu]{}\<[E]%
\ColumnHook
\end{hscode}\resethooks

Then we can obtain the $\beta$-laws which are very useful,

\begin{hscode}\SaveRestoreHook
\column{B}{@{}>{\hspre}l<{\hspost}@{}}%
\column{13}{@{}>{\hspre}l<{\hspost}@{}}%
\column{14}{@{}>{\hspre}l<{\hspost}@{}}%
\column{E}{@{}>{\hspre}l<{\hspost}@{}}%
\>[B]{}\Varid{liftOp1-β}\;\mathbin{:}\;(\Varid{f}\;\mathbin{:}\;\Conid{Op}\;\Varid{1}\;\Conid{ℤ₀})\;\Varid{→}\;{}\<[E]%
\\
\>[B]{}\hsindent{13}{}\<[13]%
\>[13]{}(\Varid{cong}\;\mathbin{:}\;\Varid{∀}\;\{\mskip1.5mu \Varid{a}\;\Varid{b}\mskip1.5mu\}\;\Varid{→}\;\Varid{a}\;\Varid{∼}\;\Varid{b}\;\Varid{→}\;\Varid{f}\;\Varid{a}\;\Varid{∼}\;\Varid{f}\;\Varid{b})\;\Varid{→}\;{}\<[E]%
\\
\>[B]{}\hsindent{13}{}\<[13]%
\>[13]{}\Varid{∀}\;\Varid{n}\;\Varid{→}\;\Varid{liftOp1safe}\;\Varid{f}\;\Varid{cong}\;[\mskip1.5mu \Varid{n}\mskip1.5mu]\;\Varid{≡}\;[\mskip1.5mu \Varid{f}\;\Varid{n}\mskip1.5mu]{}\<[E]%
\\
\>[B]{}\Varid{liftOp1-β}\;\Varid{f}\;\Varid{cong}\;\Varid{n}\;\mathrel{=}\;\Varid{sound}\;(\Varid{cong}\;\Varid{compl}){}\<[E]%
\\[\blanklineskip]%
\>[B]{}\Varid{liftOp2-β}\;\mathbin{:}\;(\Varid{op}\;\mathbin{:}\;\Conid{Op}\;\Varid{2}\;\Conid{ℤ₀})\;\Varid{→}\;{}\<[E]%
\\
\>[B]{}\hsindent{13}{}\<[13]%
\>[13]{}(\Varid{cong}\;\mathbin{:}\;\Varid{∀}\;\{\mskip1.5mu \Varid{a}\;\Varid{b}\;\Varid{c}\;\Varid{d}\mskip1.5mu\}\;\Varid{→}\;\Varid{a}\;\Varid{∼}\;\Varid{b}\;\Varid{→}\;\Varid{c}\;\Varid{∼}\;\Varid{d}\;\Varid{→}{}\<[E]%
\\
\>[13]{}\hsindent{1}{}\<[14]%
\>[14]{}\Varid{op}\;\Varid{a}\;\Varid{c}\;\Varid{∼}\;\Varid{op}\;\Varid{b}\;\Varid{d})\;\Varid{→}\;{}\<[E]%
\\
\>[B]{}\hsindent{13}{}\<[13]%
\>[13]{}\Varid{∀}\;\Varid{m}\;\Varid{n}\;\Varid{→}\;\Varid{liftOp2safe}\;\Varid{op}\;\Varid{cong}\;[\mskip1.5mu \Varid{m}\mskip1.5mu]\;[\mskip1.5mu \Varid{n}\mskip1.5mu]\;\Varid{≡}\;[\mskip1.5mu \Varid{op}\;\Varid{m}\;\Varid{n}\mskip1.5mu]{}\<[E]%
\\
\>[B]{}\Varid{liftOp2-β}\;\Varid{op}\;\Varid{cong}\;\Varid{m}\;\Varid{n}\;\mathrel{=}\;\Varid{sound}\;(\Varid{cong}\;\Varid{compl}\;\Varid{compl}){}\<[E]%
\ColumnHook
\end{hscode}\resethooks

Now we can lift the negation easily

\begin{hscode}\SaveRestoreHook
\column{B}{@{}>{\hspre}l<{\hspost}@{}}%
\column{E}{@{}>{\hspre}l<{\hspost}@{}}%
\>[B]{}\Varid{-\char95 }\;\mathbin{:}\;\Conid{Op}\;\Varid{1}\;\Conid{ℤ}{}\<[E]%
\\
\>[B]{}\Varid{-\char95 }\;\mathrel{=}\;\Varid{liftOp1safe}\;\Conid{ℤ₀.-\char95 }\;\Conid{ℤ₀.⁻¹-cong}{}\<[E]%
\ColumnHook
\end{hscode}\resethooks

and the $\beta$-laws for negation can be proved as

\begin{hscode}\SaveRestoreHook
\column{B}{@{}>{\hspre}l<{\hspost}@{}}%
\column{E}{@{}>{\hspre}l<{\hspost}@{}}%
\>[B]{}\Varid{-β}\;\mathbin{:}\;\Varid{∀}\;\Varid{a}\;\Varid{→}\;\Varid{-}\;[\mskip1.5mu \Varid{a}\mskip1.5mu]\;\Varid{≡}\;[\mskip1.5mu \Conid{ℤ₀-}\;\Varid{a}\mskip1.5mu]{}\<[E]%
\\
\>[B]{}\Varid{-β}\;\mathrel{=}\;\Varid{liftOp1-β}\;\Conid{ℤ₀-\char95 }\;\Conid{ℤ₀.⁻¹-cong}{}\<[E]%
\ColumnHook
\end{hscode}\resethooks

When trying to prove theorems for canonical integers, we can lift
proved properties for the setoid integers, such as commutativity of
any binary operations,


\begin{hscode}\SaveRestoreHook
\column{B}{@{}>{\hspre}l<{\hspost}@{}}%
\column{12}{@{}>{\hspre}l<{\hspost}@{}}%
\column{E}{@{}>{\hspre}l<{\hspost}@{}}%
\>[B]{}\Varid{liftComm}\;\mathbin{:}\;\Varid{∀}\;\{\mskip1.5mu \Varid{op}\;\mathbin{:}\;\Conid{Op}\;\Varid{2}\;\Conid{ℤ₀}\mskip1.5mu\}\;\Varid{→}\;{}\<[E]%
\\
\>[B]{}\hsindent{12}{}\<[12]%
\>[12]{}\Conid{P.Commutative}\;\Varid{\char95 ∼\char95 }\;\Varid{op}\;\Varid{→}\;{}\<[E]%
\\
\>[B]{}\hsindent{12}{}\<[12]%
\>[12]{}\Conid{Commutative}\;(\Varid{liftOp}\;\Varid{2}\;\Varid{op}){}\<[E]%
\\
\>[B]{}\Varid{liftComm}\;\{\mskip1.5mu \Varid{op}\mskip1.5mu\}\;\Varid{comm}\;\Varid{x}\;\Varid{y}\;\mathrel{=}\;\Varid{sound}\;\mathbin{\$}\;\Varid{comm}\;\Varid{⌜}\;\Varid{x}\;\Varid{⌝}\;\Varid{⌜}\;\Varid{y}\;\Varid{⌝}{}\<[E]%
\ColumnHook
\end{hscode}\resethooks

The generalised lifting function for commutativity is also one of the
derived theorem of quotients as it only uses \ensuremath{\Varid{sound}} and \ensuremath{\Varid{⌜\char95 ⌝}} which are part of the
quotients.
then we can lift the commutativity of addition and multiplication,

\begin{hscode}\SaveRestoreHook
\column{B}{@{}>{\hspre}l<{\hspost}@{}}%
\column{11}{@{}>{\hspre}l<{\hspost}@{}}%
\column{E}{@{}>{\hspre}l<{\hspost}@{}}%
\>[B]{}\Varid{+-comm}\;\mathbin{:}\;\Conid{Commutative}\;\Varid{\char95 +\char95 }{}\<[E]%
\\
\>[B]{}\Varid{+-comm}\;\mathrel{=}\;\Varid{liftComm}\;\Conid{ℤ₀.+-comm}{}\<[E]%
\\[\blanklineskip]%
\>[B]{}\Varid{*-comm}\;\mathbin{:}\;{}\<[11]%
\>[11]{}\Conid{Commutative}\;\Varid{\char95 *\char95 }{}\<[E]%
\\
\>[B]{}\Varid{*-comm}\;\mathrel{=}\;\Varid{liftComm}\;\Conid{ℤ₀.*-comm}{}\<[E]%
\ColumnHook
\end{hscode}\resethooks

It is also much simpler to prove the complicated distributivity of multiplication over addition, 

\begin{hscode}\SaveRestoreHook
\column{B}{@{}>{\hspre}l<{\hspost}@{}}%
\column{15}{@{}>{\hspre}l<{\hspost}@{}}%
\column{E}{@{}>{\hspre}l<{\hspost}@{}}%
\>[B]{}\Varid{distʳ}\;\mathbin{:}\;\Varid{\char95 *\char95 }\;\Conid{DistributesOverʳ}\;\Varid{\char95 +\char95 }{}\<[E]%
\\
\>[B]{}\Varid{distʳ}\;\Varid{a}\;\Varid{b}\;\Varid{c}\;\mathrel{=}\;\Varid{sound}\;\mathbin{\$}\;\Conid{ℤ₀.*-cong}\;(\Varid{compl}\;\{\mskip1.5mu \Varid{⌜}\;\Varid{b}\;\Varid{⌝}\;\Conid{ℤ₀+}\;\Varid{⌜}\;\Varid{c}\;\Varid{⌝}\mskip1.5mu\})\;\Varid{zrefl}\;\Varid{>∼<}\;{}\<[E]%
\\
\>[B]{}\hsindent{15}{}\<[15]%
\>[15]{}\Conid{ℤ₀.distʳ}\;\Varid{⌜}\;\Varid{a}\;\Varid{⌝}\;\Varid{⌜}\;\Varid{b}\;\Varid{⌝}\;\Varid{⌜}\;\Varid{c}\;\Varid{⌝}\;\Varid{>∼<}\;{}\<[E]%
\\
\>[B]{}\hsindent{15}{}\<[15]%
\>[15]{}\Conid{ℤ₀.+-cong}\;\Varid{compl'}\;\Varid{compl'}{}\<[E]%
\ColumnHook
\end{hscode}\resethooks

There is no need to use pattern matching, namely prove the
propositions inductively. The simplicity of the proof is achieved by
applying the quotient properties such as \ensuremath{\Varid{sound}}, we can translate or
convert the proposition into the corresponding proposition for setoid
integers. We can further translating the propositions for setoid
integers into some easier propositions for natural numbers. The
connections between canonical integers and natural numbers is built by
the definition of quotient.

We can also lift a structure of properties such as monoid,

\begin{hscode}\SaveRestoreHook
\column{B}{@{}>{\hspre}l<{\hspost}@{}}%
\column{3}{@{}>{\hspre}l<{\hspost}@{}}%
\column{5}{@{}>{\hspre}l<{\hspost}@{}}%
\column{10}{@{}>{\hspre}l<{\hspost}@{}}%
\column{13}{@{}>{\hspre}l<{\hspost}@{}}%
\column{14}{@{}>{\hspre}l<{\hspost}@{}}%
\column{31}{@{}>{\hspre}l<{\hspost}@{}}%
\column{E}{@{}>{\hspre}l<{\hspost}@{}}%
\>[B]{}\Varid{liftId}\;\mathbin{:}\;\Varid{∀}\;\{\mskip1.5mu \Varid{op}\;\mathbin{:}\;\Conid{Op}\;\Varid{2}\;\Conid{ℤ₀}\mskip1.5mu\}\;(\Varid{e}\;\mathbin{:}\;\Conid{ℤ})\;\Varid{→}\;{}\<[E]%
\\
\>[B]{}\hsindent{10}{}\<[10]%
\>[10]{}\Conid{P.Identity}\;\Varid{\char95 ∼\char95 }\;\Varid{⌜}\;\Varid{e}\;\Varid{⌝}\;\Varid{op}\;\Varid{→}\;{}\<[E]%
\\
\>[B]{}\hsindent{10}{}\<[10]%
\>[10]{}\Conid{Identity}\;\Varid{e}\;(\Varid{liftOp}\;\Varid{2}\;\Varid{op}){}\<[E]%
\\
\>[B]{}\Varid{liftId}\;\Varid{e}\;(\Varid{idl},\Varid{idr})\;\mathrel{=}\;(\Varid{λ}\;\Varid{x}\;\Varid{→}\;\Varid{sound}\;(\Varid{idl}\;\Varid{⌜}\;\Varid{x}\;\Varid{⌝})\;\Varid{>≡<}\;\Varid{stable}),{}\<[E]%
\\
\>[B]{}\hsindent{3}{}\<[3]%
\>[3]{}(\Varid{λ}\;\Varid{x}\;\Varid{→}\;\Varid{sound}\;(\Varid{idr}\;\Varid{⌜}\;\Varid{x}\;\Varid{⌝})\;\Varid{>≡<}\;\Varid{stable}){}\<[E]%
\\[\blanklineskip]%
\>[B]{}\Varid{liftAssoc}\;\mathbin{:}\;\Varid{∀}\;\{\mskip1.5mu \Varid{op}\;\mathbin{:}\;\Conid{Op}\;\Varid{2}\;\Conid{ℤ₀}\mskip1.5mu\}\;(\Varid{cong}\;\mathbin{:}\;\Conid{Cong2}\;\Varid{op})\;\Varid{→}\;{}\<[E]%
\\
\>[B]{}\hsindent{13}{}\<[13]%
\>[13]{}\Conid{P.Associative}\;\Varid{\char95 ∼\char95 }\;\Varid{op}\;\Varid{→}\;{}\<[E]%
\\
\>[B]{}\hsindent{13}{}\<[13]%
\>[13]{}\Conid{Associative}\;(\Varid{liftOp2safe}\;\Varid{op}\;\Varid{cong}){}\<[E]%
\\
\>[B]{}\Varid{liftAssoc}\;\{\mskip1.5mu \Varid{op}\mskip1.5mu\}\;\Varid{cong}\;\Varid{assoc}\;\Varid{a}\;\Varid{b}\;\Varid{c}\;\mathrel{=}\;\Varid{sound}\;\mathbin{\$}\;\Varid{cong}\;(\Varid{compl}\;\{\mskip1.5mu \Varid{op}\;\Varid{⌜}\;\Varid{a}\;\Varid{⌝}\;\Varid{⌜}\;\Varid{b}{}\<[E]%
\\
\>[B]{}\hsindent{3}{}\<[3]%
\>[3]{}\Varid{⌝}\mskip1.5mu\})\;\Varid{zrefl}\;\Varid{>∼<}\;\Varid{assoc}\;\Varid{⌜}\;\Varid{a}\;\Varid{⌝}\;\Varid{⌜}\;\Varid{b}\;\Varid{⌝}\;\Varid{⌜}\;\Varid{c}\;\Varid{⌝}\;\Varid{>∼<}\;\Varid{cong}\;\Varid{zrefl}\;\Varid{compl'}{}\<[E]%
\\[\blanklineskip]%
\>[B]{}\Varid{liftMonoid}\;\mathbin{:}\;\{\mskip1.5mu \Varid{op}\;\mathbin{:}\;\Conid{Op}\;\Varid{2}\;\Conid{ℤ₀}\mskip1.5mu\}\;\{\mskip1.5mu \Varid{e}\;\mathbin{:}\;\Conid{ℤ}\mskip1.5mu\}\;(\Varid{cong}\;\mathbin{:}\;\Conid{Cong2}\;\Varid{op})\;\Varid{→}\;{}\<[E]%
\\
\>[B]{}\hsindent{14}{}\<[14]%
\>[14]{}\Conid{IsMonoid}\;\Varid{\char95 ∼\char95 }\;\Varid{op}\;{}\<[31]%
\>[31]{}\Varid{⌜}\;\Varid{e}\;\Varid{⌝}\;\Varid{→}\;{}\<[E]%
\\
\>[B]{}\hsindent{14}{}\<[14]%
\>[14]{}\Conid{IsMonoid}\;\Varid{\char95 ≡\char95 }\;(\Varid{liftOp}\;\Varid{2}\;\Varid{op})\;\Varid{e}{}\<[E]%
\\
\>[B]{}\Varid{liftMonoid}\;\{\mskip1.5mu \Varid{op}\mskip1.5mu\}\;\{\mskip1.5mu \Varid{e}\mskip1.5mu\}\;\Varid{cong}\;\Varid{im}\;\mathrel{=}\;\Keyword{record}{}\<[E]%
\\
\>[B]{}\hsindent{3}{}\<[3]%
\>[3]{}\{\mskip1.5mu \Varid{isSemigroup}\;\mathrel{=}\;\Keyword{record}{}\<[E]%
\\
\>[3]{}\hsindent{2}{}\<[5]%
\>[5]{}\{\mskip1.5mu \Varid{isEquivalence}\;\mathrel{=}\;\Varid{isEquivalence}{}\<[E]%
\\
\>[3]{}\hsindent{2}{}\<[5]%
\>[5]{};\Varid{assoc}\;\mathrel{=}\;\Varid{liftAssoc}\;\Varid{cong}\;(\Conid{IsMonoid.assoc}\;\Varid{im}){}\<[E]%
\\
\>[3]{}\hsindent{2}{}\<[5]%
\>[5]{};\Varid{∙-cong}\;\mathrel{=}\;\Varid{cong₂}\;(\Varid{liftOp}\;\Varid{2}\;\Varid{op}){}\<[E]%
\\
\>[3]{}\hsindent{2}{}\<[5]%
\>[5]{}\mskip1.5mu\}{}\<[E]%
\\
\>[B]{}\hsindent{3}{}\<[3]%
\>[3]{};\Varid{identity}\;\mathrel{=}\;\Varid{liftId}\;\{\mskip1.5mu \Varid{op}\mskip1.5mu\}\;\Varid{e}\;(\Conid{IsMonoid.identity}\;\Varid{im}){}\<[E]%
\\
\>[B]{}\hsindent{3}{}\<[3]%
\>[3]{}\mskip1.5mu\}{}\<[E]%
\ColumnHook
\end{hscode}\resethooks

These lift functions for operators and properties can be generalised
even further such that they can be applied to all quotients. They are
all derived theorems for quotients which can save a lot of work for
us. We can reuse them in the next example, the set of rational numbers $\Q$.

\subsection{Rational numbers}

The quotient of rational numbers is better known than the previous
quotient. We usually write two integers $m$ and $n$ ($n$ is not zero) in
fractional form $\frac{m}{n}$ to represent a rational number. Alternatively we
can use an integer and a positive natural number such that it is
simpler to exclude 0 in the denominator. Two fractions are equal if
they are reduced to the same irreducible term. If the numerator and
denominator of a fraction are coprime, it is said to be an irreducible
fraction. Based on this observation, it is naturally to form a definable quotient, where the base type is 

$$\Q_0 = \Z \times \N$$

The integer is \emph{numerator} and the natural number is \emph{denominator-1}. This approach avoids invalid fractions from construction. 

In Agda, to make the terms more meaningful we define it as

\begin{hscode}\SaveRestoreHook
\column{B}{@{}>{\hspre}l<{\hspost}@{}}%
\column{3}{@{}>{\hspre}l<{\hspost}@{}}%
\column{E}{@{}>{\hspre}l<{\hspost}@{}}%
\>[B]{}\Keyword{data}\;\Conid{ℚ₀}\;\mathbin{:}\;\Conid{Set}\;\Keyword{where}{}\<[E]%
\\
\>[B]{}\hsindent{3}{}\<[3]%
\>[3]{}\Varid{\char95 /suc\char95 }\;\mathbin{:}\;(\Varid{n}\;\mathbin{:}\;\Conid{ℤ})\;\Varid{→}\;(\Varid{d}\;\mathbin{:}\;\Conid{ℕ})\;\Varid{→}\;\Conid{ℚ₀}{}\<[E]%
\ColumnHook
\end{hscode}\resethooks

In mathematics, to judge the equality of two fractions, it is easier to conduct the following conversion,

$$ \frac{a}{b} = \frac{c}{d} \iff a \times d = c \times b $$

Therefore the equivalence relation can be defined as,

\begin{hscode}\SaveRestoreHook
\column{B}{@{}>{\hspre}l<{\hspost}@{}}%
\column{7}{@{}>{\hspre}l<{\hspost}@{}}%
\column{28}{@{}>{\hspre}l<{\hspost}@{}}%
\column{E}{@{}>{\hspre}l<{\hspost}@{}}%
\>[B]{}\Varid{\char95 ∼\char95 }\;{}\<[7]%
\>[7]{}\mathbin{:}\;\Conid{Rel}\;\Conid{ℚ₀}\;\Varid{zero}{}\<[E]%
\\
\>[B]{}\Varid{n1}\;\Varid{/suc}\;\Varid{d1}\;\Varid{∼}\;\Varid{n2}\;\Varid{/suc}\;\Varid{d2}\;\mathrel{=}\;{}\<[28]%
\>[28]{}\Varid{n1}\;\Conid{ℤ*ℕ}\;\Varid{suc}\;\Varid{d2}\;\Varid{≡}\;\Varid{n2}\;\Conid{ℤ*ℕ}\;\Varid{suc}\;\Varid{d1}{}\<[E]%
\ColumnHook
\end{hscode}\resethooks

The normal form of rational numbers, namely the quotient type in this quotient is the set of irreducible fractions. We only need to add a restriction that the numerator and denominator is coprime,

$$\Q = \Sigma (n \colon \Z) \Sigma (d \colon \N), \coprime \,n \,(d +1)$$

We can encode it using record type in Agda,

\begin{hscode}\SaveRestoreHook
\column{B}{@{}>{\hspre}l<{\hspost}@{}}%
\column{3}{@{}>{\hspre}l<{\hspost}@{}}%
\column{5}{@{}>{\hspre}l<{\hspost}@{}}%
\column{19}{@{}>{\hspre}l<{\hspost}@{}}%
\column{E}{@{}>{\hspre}l<{\hspost}@{}}%
\>[B]{}\Keyword{record}\;\Conid{ℚ}\;\mathbin{:}\;\Conid{Set}\;\Keyword{where}{}\<[E]%
\\
\>[B]{}\hsindent{3}{}\<[3]%
\>[3]{}\Keyword{field}{}\<[E]%
\\
\>[3]{}\hsindent{2}{}\<[5]%
\>[5]{}\Varid{numerator}\;{}\<[19]%
\>[19]{}\mathbin{:}\;\Conid{ℤ}{}\<[E]%
\\
\>[3]{}\hsindent{2}{}\<[5]%
\>[5]{}\Varid{denominator-1}\;\mathbin{:}\;\Conid{ℕ}{}\<[E]%
\\
\>[3]{}\hsindent{2}{}\<[5]%
\>[5]{}\Varid{isCoprime}\;{}\<[19]%
\>[19]{}\mathbin{:}\;\Conid{True}\;(\Conid{C.coprime?}\;\Varid{∣}\;\Varid{numerator}\;\Varid{∣}\;(\Varid{suc}\;\Varid{denominator-1})){}\<[E]%
\ColumnHook
\end{hscode}\resethooks

The normalisation function is an implementation of the reducing
process, the \ensuremath{\Varid{gcd}} function which calculates the greatest common
divisor can help us reduce the fraction and give us the proof of coprime, 

\begin{hscode}\SaveRestoreHook
\column{B}{@{}>{\hspre}l<{\hspost}@{}}%
\column{E}{@{}>{\hspre}l<{\hspost}@{}}%
\>[B]{}[\mskip1.5mu \anonymous \mskip1.5mu]\;\mathbin{:}\;\Conid{ℚ₀}\;\Varid{→}\;\Conid{ℚ}{}\<[E]%
\\
\>[B]{}[\mskip1.5mu (\Varid{+}\;\Varid{0})\;\Varid{/suc}\;\Varid{d}\mskip1.5mu]\;\mathrel{=}\;\Conid{ℤ.+\char95 }\;\Varid{0}\;\Varid{÷}\;\Varid{1}{}\<[E]%
\\
\>[B]{}[\mskip1.5mu (\Varid{+}\;(\Varid{suc}\;\Varid{n}))\;\Varid{/suc}\;\Varid{d}\mskip1.5mu]\;\Keyword{with}\;\Varid{gcd}\;(\Varid{suc}\;\Varid{n})\;(\Varid{suc}\;\Varid{d}){}\<[E]%
\\
\>[B]{}[\mskip1.5mu (\Varid{+}\;\Varid{suc}\;\Varid{n})\;\Varid{/suc}\;\Varid{d}\mskip1.5mu]\;\mid \;\Varid{di},\Varid{g}\;\mathrel{=}\;\Conid{GCD′→ℚ}\;(\Varid{suc}\;\Varid{n})\;(\Varid{suc}\;\Varid{d})\;\Varid{di}\;(\Varid{λ}\;())\;(\Conid{C.gcd-gcd′}\;\Varid{g}){}\<[E]%
\\
\>[B]{}[\mskip1.5mu (\Varid{-suc}\;\Varid{n})\;\Varid{/suc}\;\Varid{d}\mskip1.5mu]\;\Keyword{with}\;\Varid{gcd}\;(\Varid{suc}\;\Varid{n})\;(\Varid{suc}\;\Varid{d}){}\<[E]%
\\
\>[B]{}\Varid{...}\;\mid \;\Varid{di},\Varid{g}\;\mathrel{=}\;\Varid{-}\;\Conid{GCD′→ℚ}\;(\Varid{suc}\;\Varid{n})\;(\Varid{suc}\;\Varid{d})\;\Varid{di}\;(\Varid{λ}\;())\;(\Conid{C.gcd-gcd′}\;\Varid{g}){}\<[E]%
\ColumnHook
\end{hscode}\resethooks

The embedding function is simple. We only need to forget the coprime proof in the normal form,

\begin{hscode}\SaveRestoreHook
\column{B}{@{}>{\hspre}l<{\hspost}@{}}%
\column{E}{@{}>{\hspre}l<{\hspost}@{}}%
\>[B]{}\Varid{⌜\char95 ⌝}\;\mathbin{:}\;\Conid{ℚ}\;\Varid{→}\;\Conid{ℚ₀}{}\<[E]%
\\
\>[B]{}\Varid{⌜}\;\Varid{x}\;\Varid{⌝}\;\mathrel{=}\;(\Conid{ℚ.numerator}\;\Varid{x})\;\Varid{/suc}\;(\Conid{ℚ.denominator-1}\;\Varid{x}){}\<[E]%
\ColumnHook
\end{hscode}\resethooks

Similarly, we are able to construct the setoid, the prequotient and then the definable quotient of rational numbers. We can benefit from the ease of defining operators and proving theorems on setoids while still using the normal form of rational numbers which is safer. and the lifted operators and properties.





\subsection{Real numbers}

The previous quotient types are all definable in \itt{} so that we can
construct the definable quotients for them. However, there are some
types undefinable in \itt{}. The set of real numbers $\R$ has been proved to be undefinable in \cite{aan}.

We have several choices to represent real numbers. We choose Cauchy
sequences of rational numbers to represent real numbers \cite{bis:85}.

$$\R_{0} = \set{s : \N\to\Q \mid \forall\varepsilon
  :\Q,\varepsilon>0\to\exists m:\N, \forall i:\N, i>m\to \vert  s_i -
  s_m \vert  <\varepsilon}$$

We can implement it in Agda. First a sequence of elements of $A$ can be represented by a function from $\N$ to $A$.

\begin{hscode}\SaveRestoreHook
\column{B}{@{}>{\hspre}l<{\hspost}@{}}%
\column{E}{@{}>{\hspre}l<{\hspost}@{}}%
\>[B]{}\Conid{Seq}\;\mathbin{:}\;(\Conid{A}\;\mathbin{:}\;\Conid{Set})\;\Varid{→}\;\Conid{Set}{}\<[E]%
\\
\>[B]{}\Conid{Seq}\;\Conid{A}\;\mathrel{=}\;\Conid{ℕ}\;\Varid{→}\;\Conid{A}{}\<[E]%
\ColumnHook
\end{hscode}\resethooks

And a sequence of rational numbers converges to zero can be expressed as follows,

\begin{hscode}\SaveRestoreHook
\column{B}{@{}>{\hspre}l<{\hspost}@{}}%
\column{23}{@{}>{\hspre}l<{\hspost}@{}}%
\column{E}{@{}>{\hspre}l<{\hspost}@{}}%
\>[B]{}\Conid{Converge}\;\mathbin{:}\;\Conid{Seq}\;\Conid{ℚ₀}\;\Varid{→}\;\Conid{Set}{}\<[E]%
\\
\>[B]{}\Conid{Converge}\;\Varid{f}\;\mathrel{=}\;\Varid{∀}\;(\Varid{ε}\;\mathbin{:}\;\Conid{ℚ₀*})\;\Varid{→}\;\Varid{∃}\;\Varid{λ}\;\Varid{lb}\;\Varid{→}\;\Varid{∀}\;\Varid{m}\;\Varid{n}\;\Varid{→}\;{}\<[E]%
\\
\>[B]{}\hsindent{23}{}\<[23]%
\>[23]{}\Varid{∣}\;(\Varid{f}\;(\Varid{suc}\;\Varid{lb}\;\Varid{+}\;\Varid{m}))\;\Varid{-}\;(\Varid{f}\;(\Varid{suc}\;\Varid{lb}\;\Varid{+}\;\Varid{n}))\;\Varid{∣}\;\Varid{<'}\;\Varid{ε}{}\<[E]%
\ColumnHook
\end{hscode}\resethooks

Now we can write the Cauchy sequence of rational numbers,

\begin{hscode}\SaveRestoreHook
\column{B}{@{}>{\hspre}l<{\hspost}@{}}%
\column{3}{@{}>{\hspre}l<{\hspost}@{}}%
\column{5}{@{}>{\hspre}l<{\hspost}@{}}%
\column{E}{@{}>{\hspre}l<{\hspost}@{}}%
\>[B]{}\Keyword{record}\;\Conid{ℝ₀}\;\mathbin{:}\;\Conid{Set}\;\Keyword{where}{}\<[E]%
\\
\>[B]{}\hsindent{3}{}\<[3]%
\>[3]{}\Varid{constructor}\;\Varid{f:\char95 p:\char95 }{}\<[E]%
\\
\>[B]{}\hsindent{3}{}\<[3]%
\>[3]{}\Keyword{field}{}\<[E]%
\\
\>[3]{}\hsindent{2}{}\<[5]%
\>[5]{}\Varid{f}\;\mathbin{:}\;\Conid{Seq}\;\Conid{ℚ₀}{}\<[E]%
\\
\>[3]{}\hsindent{2}{}\<[5]%
\>[5]{}\Varid{p}\;\mathbin{:}\;\Conid{Converge}\;\Varid{f}{}\<[E]%
\ColumnHook
\end{hscode}\resethooks

To complete the setoid for real numbers, an equivalence relation is required. In mathematics two Cauchy sequences $\R_0$ are said to be equal if their pointwise difference converges to zero,

$$r \sim s = \forall\varepsilon :\Q,\varepsilon>0\to\exists m:\N,
\forall i:\N, i>m\to \vert  r_i - s_i \vert <\varepsilon$$

The Agda version is

\begin{hscode}\SaveRestoreHook
\column{B}{@{}>{\hspre}l<{\hspost}@{}}%
\column{4}{@{}>{\hspre}l<{\hspost}@{}}%
\column{8}{@{}>{\hspre}l<{\hspost}@{}}%
\column{E}{@{}>{\hspre}l<{\hspost}@{}}%
\>[4]{}\Varid{\char95 Diff\char95 on\char95 }\;\mathbin{:}\;\Conid{Seq}\;\Conid{ℚ₀}\;\Varid{→}\;\Conid{Seq}\;\Conid{ℚ₀}\;\Varid{→}\;\Conid{Seq}\;\Conid{ℚ₀*}{}\<[E]%
\\
\>[4]{}\Varid{f}\;\Conid{Diff}\;\Varid{g}\;\Varid{on}\;\Varid{m}\;\mathrel{=}\;\Varid{∣}\;\Varid{f}\;\Varid{m}\;\Varid{-}\;\Varid{g}\;\Varid{m}\;\Varid{∣}{}\<[E]%
\\[\blanklineskip]%
\>[4]{}\Varid{\char95 \char126 \char95 }\;\mathbin{:}\;\Conid{Rel}\;\Conid{ℝ₀}\;\Varid{zero}{}\<[E]%
\\
\>[4]{}(\Varid{f:}\;\Varid{f}\;\Varid{p:}\;\Varid{p})\;\mathord{\sim}\;(\Varid{f:}\;\Varid{f'}\;\Varid{p:}\;\Varid{p'})\;\mathrel{=}\;{}\<[E]%
\\
\>[4]{}\hsindent{4}{}\<[8]%
\>[8]{}\Varid{∀}\;(\Varid{ε}\;\mathbin{:}\;\Conid{ℚ₀*})\;\Varid{→}\;\Varid{∃}\;\Varid{λ}\;\Varid{lb}\;\Varid{→}\;\Varid{∀}\;\Varid{i}\;\Varid{→}\;(\Varid{lb}\;\Varid{<}\;\Varid{i})\;\Varid{→}\;\Varid{f}\;\Conid{Diff}\;\Varid{f'}\;\Varid{on}\;\Varid{i}\;\Varid{<'}\;\Varid{ε}{}\<[E]%
\ColumnHook
\end{hscode}\resethooks

In set theory we can construct  quotient set $\R_0 /\sim$. However
since real numbers have no normal forms we can not define the quotient
in \itt{}. Hence the definable quotient definition does not work for
it.The undefinability of the any type $\R$ which is the quotient type
of the setoid $(\R_0, \sim)$ is proved by local continuity \cite{aan}.

\subsection{All epimorphisms are split epimorphisms}

In addition we also proved that classically all epimorphisms are split epimorphisms.

A morphism $e$ is an epimorphism if it is right-cancellative 

\begin{hscode}\SaveRestoreHook
\column{B}{@{}>{\hspre}l<{\hspost}@{}}%
\column{7}{@{}>{\hspre}l<{\hspost}@{}}%
\column{E}{@{}>{\hspre}l<{\hspost}@{}}%
\>[B]{}\Conid{Epi}\;\mathbin{:}\;\{\mskip1.5mu \Conid{A}\;\Conid{B}\;\mathbin{:}\;\Conid{Set}\mskip1.5mu\}\;\Varid{→}\;(\Varid{e}\;\mathbin{:}\;\Conid{A}\;\Varid{→}\;\Conid{B})\;\Varid{→}\;(\Conid{C}\;\mathbin{:}\;\Conid{Set})\;\Varid{→}\;\Conid{Set}{}\<[E]%
\\
\>[B]{}\Conid{Epi}\;\{\mskip1.5mu \Conid{A}\mskip1.5mu\}\;\{\mskip1.5mu \Conid{B}\mskip1.5mu\}\;\Varid{e}\;\Conid{C}\;\mathrel{=}\;{}\<[E]%
\\
\>[B]{}\hsindent{7}{}\<[7]%
\>[7]{}\Varid{∀}\;(\Varid{f}\;\Varid{g}\;\mathbin{:}\;\Conid{B}\;\Varid{→}\;\Conid{C})\;\Varid{→}\;{}\<[E]%
\\
\>[B]{}\hsindent{7}{}\<[7]%
\>[7]{}(\Varid{∀}\;(\Varid{a}\;\mathbin{:}\;\Conid{A})\;\Varid{→}\;\Varid{f}\;(\Varid{e}\;\Varid{a})\;\Varid{≡}\;\Varid{g}\;(\Varid{e}\;\Varid{a}))\;\Varid{→}\;{}\<[E]%
\\
\>[B]{}\hsindent{7}{}\<[7]%
\>[7]{}\Varid{∀}\;(\Varid{b}\;\mathbin{:}\;\Conid{B})\;\Varid{→}\;\Varid{f}\;\Varid{b}\;\Varid{≡}\;\Varid{g}\;\Varid{b}{}\<[E]%
\ColumnHook
\end{hscode}\resethooks

If it has a right inverse it is called a split epi

\begin{hscode}\SaveRestoreHook
\column{B}{@{}>{\hspre}l<{\hspost}@{}}%
\column{E}{@{}>{\hspre}l<{\hspost}@{}}%
\>[B]{}\Conid{Split}\;\mathbin{:}\;\{\mskip1.5mu \Conid{A}\;\Conid{B}\;\mathbin{:}\;\Conid{Set}\mskip1.5mu\}\;\Varid{→}\;(\Varid{e}\;\mathbin{:}\;\Conid{A}\;\Varid{→}\;\Conid{B})\;\Varid{→}\;\Conid{Set}{}\<[E]%
\\
\>[B]{}\Conid{Split}\;\{\mskip1.5mu \Conid{A}\mskip1.5mu\}\;\{\mskip1.5mu \Conid{B}\mskip1.5mu\}\;\Varid{e}\;\mathrel{=}\;\Varid{∃}\;\Varid{λ}\;(\Varid{s}\;\mathbin{:}\;\Conid{B}\;\Varid{→}\;\Conid{A})\;\Varid{→}\;\Varid{∀}\;\Varid{b}\;\Varid{→}\;\Varid{e}\;(\Varid{s}\;\Varid{b})\;\Varid{≡}\;\Varid{b}{}\<[E]%
\ColumnHook
\end{hscode}\resethooks

We assume the axioms of classical logic

\begin{hscode}\SaveRestoreHook
\column{B}{@{}>{\hspre}l<{\hspost}@{}}%
\column{E}{@{}>{\hspre}l<{\hspost}@{}}%
\>[B]{}\Keyword{postulate}\;\Varid{classic}\;\mathbin{:}\;(\Conid{P}\;\mathbin{:}\;\Conid{Set})\;\Varid{→}\;\Conid{P}\;\Varid{∨}\;(\Varid{¬}\;\Conid{P}){}\<[E]%
\\[\blanklineskip]%
\>[B]{}\Varid{raa}\;\mathbin{:}\;\{\mskip1.5mu \Conid{P}\;\mathbin{:}\;\Conid{Set}\mskip1.5mu\}\;\Varid{→}\;\Varid{¬}\;(\Varid{¬}\;\Conid{P})\;\Varid{→}\;\Conid{P}{}\<[E]%
\\
\>[B]{}\Varid{raa}\;\{\mskip1.5mu \Conid{P}\mskip1.5mu\}\;\Varid{nnp}\;\Keyword{with}\;\Varid{classic}\;\Conid{P}{}\<[E]%
\\
\>[B]{}\Varid{raa}\;\Varid{nnp}\;\mid \;\Varid{inl}\;\Varid{y}\;\mathrel{=}\;\Varid{y}{}\<[E]%
\\
\>[B]{}\Varid{raa}\;\Varid{nnp}\;\mid \;\Varid{inr}\;\Varid{y}\;\Keyword{with}\;\Varid{nnp}\;\Varid{y}{}\<[E]%
\\
\>[B]{}\Varid{...}\;\mid \;(){}\<[E]%
\\[\blanklineskip]%
\>[B]{}\Varid{contrapositive}\;\mathbin{:}\;\Varid{∀}\;\{\mskip1.5mu \Conid{P}\;\Conid{Q}\;\mathbin{:}\;\Conid{Set}\mskip1.5mu\}\;\Varid{→}\;(\Varid{¬}\;\Conid{Q}\;\Varid{→}\;\Varid{¬}\;\Conid{P})\;\Varid{→}\;\Conid{P}\;\Varid{→}\;\Conid{Q}{}\<[E]%
\\
\>[B]{}\Varid{contrapositive}\;\Varid{nqnp}\;\Varid{p}\;\mathrel{=}\;\Varid{raa}\;(\Varid{λ}\;\Varid{nq}\;\Varid{→}\;\Varid{nqnp}\;\Varid{nq}\;\Varid{p}){}\<[E]%
\ColumnHook
\end{hscode}\resethooks

We also need one of the De Morgan's law in classical logic

\begin{hscode}\SaveRestoreHook
\column{B}{@{}>{\hspre}l<{\hspost}@{}}%
\column{E}{@{}>{\hspre}l<{\hspost}@{}}%
\>[B]{}\Keyword{postulate}\;\Conid{DeMorgan}\;\mathbin{:}\;\Varid{∀}\;\{\mskip1.5mu \Conid{A}\;\mathbin{:}\;\Conid{Set}\mskip1.5mu\}\;\{\mskip1.5mu \Conid{P}\;\mathbin{:}\;\Conid{A}\;\Varid{→}\;\Conid{Set}\mskip1.5mu\}\;\Varid{→}{}\<[E]%
\\
\>[B]{}\Varid{¬}\;(\Varid{∀}\;(\Varid{x}\;\mathbin{:}\;\Conid{A})\;\Varid{→}\;\Conid{P}\;\Varid{x})\;\Varid{→}\;\Varid{∃}\;\Varid{λ}\;(\Varid{x}\;\mathbin{:}\;\Conid{A})\;\Varid{→}\;\Varid{¬}\;\Conid{P}\;\Varid{x}{}\<[E]%
\ColumnHook
\end{hscode}\resethooks

What we need to prove is

\begin{hscode}\SaveRestoreHook
\column{B}{@{}>{\hspre}l<{\hspost}@{}}%
\column{E}{@{}>{\hspre}l<{\hspost}@{}}%
\>[B]{}\Conid{Epi→Split}\;\mathbin{:}\;\{\mskip1.5mu \Conid{A}\;\Conid{B}\;\mathbin{:}\;\Conid{Set}\mskip1.5mu\}\;\Varid{→}\;(\Varid{e}\;\mathbin{:}\;\Conid{A}\;\Varid{→}\;\Conid{B})\;\Varid{→}\;\Conid{Set₁}{}\<[E]%
\\
\>[B]{}\Conid{Epi→Split}\;\Varid{e}\;\mathrel{=}\;((\Conid{C}\;\mathbin{:}\;\Conid{Set})\;\Varid{→}\;\Conid{Epi}\;\Varid{e}\;\Conid{C})\;\Varid{→}\;\Conid{Split}\;\Varid{e}{}\<[E]%
\ColumnHook
\end{hscode}\resethooks

Because we have classical theorems, it is equivalent to prove the contrapositive of \ensuremath{\Conid{Epi→Split}}. To make the steps clear, we decompose the complicated proof. In order,
we postulate the following things first

\begin{hscode}\SaveRestoreHook
\column{B}{@{}>{\hspre}l<{\hspost}@{}}%
\column{E}{@{}>{\hspre}l<{\hspost}@{}}%
\>[B]{}\Keyword{postulate}\;\Conid{A}\;\Conid{B}\;\mathbin{:}\;\Conid{Set}{}\<[E]%
\\
\>[B]{}\Keyword{postulate}\;\Varid{e}\;\mathbin{:}\;\Conid{A}\;\Varid{→}\;\Conid{B}{}\<[E]%
\\[\blanklineskip]%
\>[B]{}\Keyword{postulate}\;\Varid{¬split}\;\mathbin{:}\;\Varid{¬}\;\Conid{Split}\;\Varid{e}{}\<[E]%
\ColumnHook
\end{hscode}\resethooks

Now from the assumption that e is not a split, we can find an element $b : B$ which is not the image of any element $a : A$ under $e$

\begin{hscode}\SaveRestoreHook
\column{B}{@{}>{\hspre}l<{\hspost}@{}}%
\column{E}{@{}>{\hspre}l<{\hspost}@{}}%
\>[B]{}\Varid{¬surj}\;\mathbin{:}\;\Varid{∃}\;\Varid{λ}\;\Varid{b}\;\Varid{→}\;\Varid{¬}\;(\Varid{∃}\;\Varid{λ}\;(\Varid{a}\;\mathbin{:}\;\Conid{A})\;\Varid{→}\;(\Varid{e}\;\Varid{a}\;\Varid{≡}\;\Varid{b})){}\<[E]%
\\
\>[B]{}\Varid{¬surj}\;\mathrel{=}\;\Conid{DeMorgan}\;(\Varid{λ}\;\Varid{x}\;\Varid{→}\;\Varid{¬split}\;((\Varid{λ}\;\Varid{b}\;\Varid{→}\;\Varid{proj₁}\;(\Varid{x}\;\Varid{b})),\Varid{λ}\;\Varid{b}\;\Varid{→}\;(\Varid{proj₂}\;(\Varid{x}\;\Varid{b})))){}\<[E]%
\\[\blanklineskip]%
\>[B]{}\Varid{b}\;\mathrel{=}\;\Varid{proj₁}\;\Varid{¬surj}{}\<[E]%
\\[\blanklineskip]%
\>[B]{}\Varid{ignore}\;\mathbin{:}\;\Varid{∀}\;(\Varid{a}\;\mathbin{:}\;\Conid{A})\;\Varid{→}\;\Varid{¬}\;(\Varid{e}\;\Varid{a}\;\Varid{≡}\;\Varid{b}){}\<[E]%
\\
\>[B]{}\Varid{ignore}\;\Varid{a}\;\Varid{eq}\;\mathrel{=}\;\Varid{proj₂}\;\Varid{¬surj}\;(\Varid{a},\Varid{eq}){}\<[E]%
\ColumnHook
\end{hscode}\resethooks

We can define a constant function

\begin{hscode}\SaveRestoreHook
\column{B}{@{}>{\hspre}l<{\hspost}@{}}%
\column{E}{@{}>{\hspre}l<{\hspost}@{}}%
\>[B]{}\Varid{f}\;\mathbin{:}\;\Conid{B}\;\Varid{→}\;\Conid{Bool}{}\<[E]%
\\
\>[B]{}\Varid{f}\;\Varid{x}\;\mathrel{=}\;\Varid{false}{}\<[E]%
\ColumnHook
\end{hscode}\resethooks

and postulate a function to decide whether $x : B$ is equal to b. The
reason to postulate it is we do not know the constructor of b and we
are sure that if $B$ is definable in Agda, the intensional equality
must be decidable,

\begin{hscode}\SaveRestoreHook
\column{B}{@{}>{\hspre}l<{\hspost}@{}}%
\column{E}{@{}>{\hspre}l<{\hspost}@{}}%
\>[B]{}\Keyword{postulate}\;\Varid{g}\;\mathbin{:}\;\Conid{B}\;\Varid{→}\;\Conid{Bool}{}\<[E]%
\\
\>[B]{}\Keyword{postulate}\;\Varid{gb}\;\mathbin{:}\;\Varid{g}\;\Varid{b}\;\Varid{≡}\;\Varid{true}{}\<[E]%
\\
\>[B]{}\Keyword{postulate}\;\Varid{gb'}\;\mathbin{:}\;\Varid{∀}\;\Varid{b'}\;\Varid{→}\;\Varid{¬}\;(\Varid{b'}\;\Varid{≡}\;\Varid{b})\;\Varid{→}\;\Varid{false}\;\Varid{≡}\;\Varid{g}\;\Varid{b'}{}\<[E]%
\ColumnHook
\end{hscode}\resethooks

Finally we can prove $e$ is not an epi

\begin{hscode}\SaveRestoreHook
\column{B}{@{}>{\hspre}l<{\hspost}@{}}%
\column{E}{@{}>{\hspre}l<{\hspost}@{}}%
\>[B]{}\Varid{¬epiBool}\;\mathbin{:}\;\Varid{¬}\;\Conid{Epi}\;\Varid{e}\;\Conid{Bool}{}\<[E]%
\\
\>[B]{}\Varid{¬epiBool}\;\Varid{epi}\;\Keyword{with}\;\Varid{assoc}\;(\Varid{epi}\;\Varid{f}\;\Varid{g}\;(\Varid{λ}\;\Varid{a}\;\Varid{→}\;\Varid{gb'}\;(\Varid{e}\;\Varid{a})\;(\Varid{ignore}\;\Varid{a}))\;\Varid{b})\;\Varid{gb}{}\<[E]%
\\
\>[B]{}\Varid{...}\;\mid \;(){}\<[E]%
\\[\blanklineskip]%
\>[B]{}\Varid{¬epi}\;\mathbin{:}\;\Varid{¬}\;((\Conid{C}\;\mathbin{:}\;\Conid{Set})\;\Varid{→}\;\Conid{Epi}\;\Varid{e}\;\Conid{C}){}\<[E]%
\\
\>[B]{}\Varid{¬epi}\;\Varid{epi}\;\mathrel{=}\;\Varid{¬epiBool}\;(\Varid{epi}\;\Conid{Bool}){}\<[E]%
\ColumnHook
\end{hscode}\resethooks

This proposition can only been applied to definable types $A$ $B$ in \itt{} and is
only proved to be true with classic axioms.Therefore it does not make
sense for the epimorphism from $\R_0$ to $\R$.

\section{Conclusion}

In the first phase of the project of quotient types in \itt{}, we
investigate the quotients which are definable in current setting of
\itt{}. The quotient types are separately defined and then proved to
be correct by forming definable quotients of some setoids. The
properties contained in the quotient structure is very helpful in
lifting functions and propositions for setoids to quotient types. This
approach provides us an alternative choice to define functions and prove propositions. It is probably simpler to define functions on setoids and we can reuses proved theorems for the setoids in many cases. However it is a little complicated to build the quotients and is only applicable to quotients which are definable in \itt{}.

In the next phase we will focus on the undefinable quotients and extending \itt{} with axiomatic quotient types. To implement undefinable quotients, a new type former with the essential introduction and elimination rules is unavoidable. Although the quotient structures only works for definable quotients, it can be a good guidance to axiomatise the quotient types. We can extend the work in \cite{alt:99} and find an approach to extend \itt{} without losing nice features such as termination and decidable type checking. Another possible task is to investigate the conservativity of \itt{} with quotient types over \ett{}.

%future : something on equality, complete preliminary work in Agda
%Extend without losing nice features of itt, termination, decidable type checking
%axiomatizing quotients types, adding rules, na 
% possible quotations, talk to Thorsten
% 
% pay attention to the connection,the flow of ideas throughout the article.
% one thing in one paragraph.





\newpage
\bibliography{quotients}
\bibliographystyle{plain}

\end{document}
