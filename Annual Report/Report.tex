%&latex

\documentclass[envcountsame]{llncs}

\pagestyle{headings}

\title{First Year PhD Annual Report}

%make a proper TOC despite llncs
\setcounter{tocdepth}{2}
\makeatletter
\renewcommand*\l@author[2]{}
\renewcommand*\l@title[2]{}
\makeatletter

\author{LI Nuo}

\institute{University of Nottingham}

\usepackage{dsfont}

%\usepackage{amsthm}


\usepackage{color}
\usepackage{amsmath}
\usepackage{amsfonts}
\usepackage{amssymb}
\usepackage{xypic}

% Editing and debugging
%\hfuzz 0.1pt
%\overfullrule=15pt
%\brokenpenalty=10000
\newcommand{\todo}[1]{\textcolor{red}{TO~DO:~#1}}


\newcommand{\ed}[1]{\textcolor{blue}{#1}}


\newtheorem{assumption}[theorem]{Assumption}

\newcommand{\N}{\mathbb{N}}
\newcommand{\Q}{\mathbb{Q}}
\newcommand{\R}{\mathbb{R}}
\newcommand{\Z}{\mathbb{Z}}


\newcommand{\dotph}{\,\cdot\,}
\newcommand{\dotop}{\mathrel{.}}
\providecommand{\abs}  [1]{\lvert#1\rvert}
\providecommand{\norm} [1]{\lVert#1\rVert}
\providecommand{\class}[1]{[#1]}
\providecommand{\set}  [1]{\left\{#1\right\}}
\providecommand{\dlift}[1]{\widehat{#1}}

\DeclareMathOperator{\Prop}{\mathbf{Prop}}
\DeclareMathOperator{\Set}{\mathbf{Set}}
\DeclareMathOperator{\Ext}{Ext}
\DeclareMathOperator{\Bool}{Bool}
\DeclareMathOperator{\id}{id}
\DeclareMathOperator{\sound}{sound}
\DeclareMathOperator{\qelimbeta}{qelim-\beta}
\DeclareMathOperator{\qind}{qind}
\DeclareMathOperator{\exact}{exact}
\DeclareMathOperator{\subst}{subst}
\DeclareMathOperator{\emb}{emb}
\DeclareMathOperator{\complete}{complete}
\DeclareMathOperator{\stable}{stable}
\DeclareMathOperator{\List}{List}
\DeclareMathOperator{\Fin}{Fin}
\DeclareMathOperator{\now}{now}
\DeclareMathOperator{\later}{later}
\DeclareMathOperator{\nowequal}{now_\sqsubseteq}
\DeclareMathOperator{\laterequal}{later_\sqsubseteq}
\DeclareMathOperator{\laterleft}{later_{left}}
\DeclareMathOperator{\inl}{inl}
\DeclareMathOperator{\inr}{inr}
\DeclareMathOperator{\qelim}{qelim}
\DeclareMathOperator{\lift}{lift}
\DeclareMathOperator{\LC}{LC}
\DeclareMathOperator{\liftbeta}{lift-\beta}
\DeclareMathOperator{\Bijection}{Bijection}
\DeclareMathOperator{\true}{true}
\DeclareMathOperator{\false}{false}
\DeclareMathOperator{\sort}{sort}
\DeclareMathOperator{\length}{length}
\DeclareMathOperator{\nub}{nub}
\DeclareMathOperator{\suc}{suc\,}
\DeclareMathOperator{\defi}{\stackrel{\text{\tiny def}}{=}}
\DeclareMathOperator{\coprime}{Coprime}

\DeclareMathOperator{\A/sim}{A\, / \sim}

\newcommand{\eqqm}{\overset{\text{\tiny ?}}{=}}
\newcommand{\sep}{\mathrel{\sharp}}


% For xy matrices
\newcommand{\pullbackcorner}[1][dr]{\save*!/#1-1.2pc/#1:(-1,1)@^{|-}\restore}
\newcommand{\pushoutcorner} [1][dr]{\save*!/#1+1.2pc/#1:(1,-1)@^{|-}\restore}

\newcommand{\itt}{intensional Type Theory}
\newcommand{\ett}{extensional Type Theory}
\newcommand{\mltt}{Martin-L\"{o}f type theory}

%\newtheorem{theorem}{Theorem}[section]
%\newtheorem{lemma}[theorem]{Lemma}
%\theoremstyle{definition}
%\newtheorem{definition}{Definition}[section]
%\newtheorem{proposition}[theorem]{Proposition}

\usepackage{varioref}

\begin{document}

\maketitle

%\newpage

\tableofcontents

\newpage 

\begin{abstract}
%\todo{enhance the connection within between ideas. Split the two ideas in the abstract: why in general quotient is useful and implementing in Agda.}


Given a setoid, that is a set equipped with an equivalence relation, one can form its quotient set, that is the set of equivalence classes. % When dividing a set by an equivalence relation on it, we get a new set called quotient set
Reinterpreting this idea in type theory, quotient type is formed by some type with its equivalence relation. However in \itt{}  the quotient type is still unavailable. The axiomatic quotient type enables us to define real numbers, functionals, and many other sets which are not definable in current \itt,  such as Agda which is also a theorem prover. Also we can benefit from the formalisation of some quotient types, since some base types are simpler to deal with or have better features, such as the integers represented by a pair of natural numbers.  Therefore, we undertake this project to investigate how to implement quotients in Agda. There are several schemes to do this, extending intensional Type Theory to quotient types, or we can define quotients without in current settings. In  \cite{aan}, we present some results of defining quotients in Agda and I will
give some explanations  and show some instances in this report as a complement to it.
%For some definable quotients like the integers and the rational numbers,
%it is unnecessary to formalise them based on setoid. But we found the quotient %interface could provide more convenience if we prove that they are definable %quotients. While some other types such as real numbers are undefinable, %but they are the quotient of dividing cauchy sequences of rational numbers %by the equivalence relation that two sequences converges. 

\end{abstract}


\section{Introduction}


\subsection{Quotient Set}
In mathematics, a quotient represents the result of division which is
between numbers and it is also a number.  However the concept of
division can be extended to other domains. In set theory, addition
and multiplication can be extended to sets as disjoint union and
product between sets. Based on this analogy, numbers can be
represented by the cardinality of following inductively generated
sets, which are called set-theoretical natural numbers,
\begin{align*}
\underline0 &= \O\\
\underline1 &= \{\underline0\}\\
\underline2 &= \{\underline0,\underline1\}\\
\vdots
\\\underline n &= \{\underline0,\underline1,\underline2,\cdots \underline{n-1}\}
\end{align*}
then the addition and multiplication between numbers can be can be
reinterpreted as the calculation on the cardinality when operating
disjoint union and product within these sets.

Similarly, we can also extend division to sets. In Set Theory, given a
set A and an equivalence relation $\sim$ on A, the equivalence class for each $a \,\in A$ is,

\[\class a =\{b \in A \,|\, b \sim a \}\  \]

The quotient set $\A/sim$ is the set of equivalence classes of $\sim$,

\[\A/sim =\{[a] \in \wp(A) \,|\, a \in A\}\  \]

The process of  formulating new set $\A/sim$ from base set $A$ can be
described as the \emph{extended division} of set $A$ by an equivalence
relation $\sim$ on $A$.

Dividing a natural number $a$, by another number $b$, can be
inductively defined. For example, natural numbers divided by $3$ can
be simulated by the following process. Assume $0$ divided by any
positive number gives $0$ and remainder is $0$. the quotient number
can be obtained from the cardinality of the last equivalence class of set $\underline a$
under congruence modulo $3$, that is the equivalence class whose
elements share the remainder $2$. The remainder can be obtained by
adding one to the remainder of the previous number in the set of ${0,1,2}$.

%why quotients is useful?

There are many mathematics notions can be constructed in the form of
quotient sets. Some are more natural to construct by quotients, such as integers modulo some
number and rational numbers as fractions. Integers modulo some number
$n$ is the set of equivalence classes of all integers $\Z$ under the equivalence relation
which equates two elements with the same remainders after divided by
$n$. Pairs of integers $\Z \times \Z$ can be used to represent rational numbers, but
eahc rational number can be represented by infinitely many internally
different pairs. In other words, the set of rational numbers is the
set of equivalence classes of $\Z \times \Z$ under the equivalence relation that equates
different forms of one rational number. Some other sets like
integers and real numbers are also intrinsically
quotients. Integers can be interpreted as pairs of natural numbers
$\N \times \N$ and real numbers can be represented by cauchy sequences of
rational numbers. They can be encoded in Type Theory as we will
discussed in detail later.

\subsection{Type Theory}

The theory of types was first introduced by Russell \cite{rus:1903} as
an alternative to naive set theory. However, the type theory discussed
here is the one developed by Per Martin-L\"{o}f \cite{per:71,per:82}. It is based
on the Curry-Howard isomorphism between propositions and the types of
its proofs such that it can served as a formalisation of
mathematics. For detailed introduction, please refer to\cite{nor:00}.


Per Martin-L\"{o}f proposed both an intensional and an extensional
variants of his Intuitionistic type theory. The distinction between them is whether definitional equality is
distinguished with propositional equality. In \itt{}, definitional equality exists
between two intensionally identical objects, but propositional
equality is a type which requires proof term. However
in \ett{}, they are not distinguished so that definitional equality is
undecidable. Since type checking only depends on definitional
equality\cite{alt:99}, it is decidable and terminates in \itt{} but
not in \ett{}.

Type theory can also serve as a programming language in
which the evaluation of well-typed program always terminates
\cite{nor:90}. There are a few implementation based on different type theories, such as
NuPRL, Coq and Agda. Agda is one of the most recent implementation of
intensional version of \mltt{}. As we have seen, Because of the identification of types
and propositions, it is not only a programming language but also a
theorem prover. Moreover we are able to verify our Agda programs in
the itself. It has a bundle of good features like pattern matching,
unicode input, implicit aruguments etc \cite{bov:09}. Since this
project is based on \mltt{}, it is ideal to implement our definitions
and verify propositions in Agda.

\subsection{Quotient Types}

In \itt{},  many notions from set theory and propositional logic can
be reinterpreted easily. For instances, the product of sets can be
formalised by $\Sigma-Type$ and the functions can be formalised by
$\Pi-Type$ \cite{nor:00}. However the reinterpretation of quotients in
Type Theory is still a problematic issue and quotient types are still unavailable.

Alternatively, in \itt{} we have the bases of quotients as follows,

\begin{definition}
A setoid $(A,\sim)$ is a set $A$ equipped with an equivalence relation ${\,\sim\,}\colon A \to A \to \Prop$.
\end{definition}

We can use setoids to represent quotients but they are not sets. If we
use the setoid $(\N \times \N, \sim)$ for integers, we have to
redefine set-based operations and it is unsafe \cite{aan}. Moreover to define quotients
based on this setoid, such as rational numbers, is
impossible. Therefore it is not a good solution.

Quotient types should enable users to implement quotients in
\itt{} which means the quotient types based on the setoid should have
type $\Set$. Moreover it should be universe polymorphic.


In this report, I will use the symbol $\A/sim$ for the quotient based
on a given setoid  $(A,\sim)$. To make the difference between setoids and quotient types clear, we use
an analogy, $8\div2=4$. The number 4 is the quotient because $4 \times
2 = 8$, and we cannot recover the dividend and the divisor from the
quotient $4$ or manipulate $8$ or $2$ separately. Similarly, setoids contain pairs of dividend and
divisor, but quotient sets do not include all the information from the
setoids. Futhermore one set can be the quotient sets of several different setoids. 


\itt{} has to be extended if we want to define quotient
types. Alternatively, some quotients are definable such that we can
construct it from scratch and prove it is the quotient based on a given setoid
\cite{aan}. For instances in \cite{nuo:10}, the normal form integers,
reducible rational numbers are definable and proved to construct
quotients with respect to the corresponding setoids,  such that we can
treat them as the quotient types. The quotient interfaces introduced in \cite{aan} do not
provide the access to the underlying setoids but include a set of
properties which serve as not only proofs but also tools. I will
explain the quotient interface and discuss some examples
later. The advantages of this idea are, it is feasible in current setting of
intensional Type Theory,  and we achieve some convenience
from constructing quotients. The disadvantages are, it only works for
the quotient types which are definable in \itt{} and the quotient
types and functions are constructed manually rather than derived
automatically. Hence the lifting of functions and predicates is a bit
complicated as it requires proofs that these functions or predicates respect the equivalence
relation\cite{hof:95:sm}.  Moreover, if the quotient sets are not
definable, for instance the set of real numbers $\R$, even though $\R$
can be seen as the quotient set of the Cauchy sequences of rational numbers $\Q$ and the equivalence
relation that two sequences converge to zero. 

Quotient types is not only a tool to implement quotients in
mathematics. It is based on \mltt{}, therefore some notions in Type
Theory or in programming languages are actually quotients. For example partiality monad divided by a weak
similarity ignoring finite delays \cite{aan}, propositions divided by
$\iff$  and the set of extensionally equal functions. Also
set-theoretical finite sets can be implemented as the quotient of
lists in Type Theory. 

Furthermore from any given function $f \,\colon\, A \to B$, we obtain an
equivalence relation $\sim \,\colon A \to A \to \Prop$ called $kernel$
of $f$ which is defined as $a \sim b \defi f \,a \equiv f \,b$. From
this setoid we can form a quotient.

Actually all types can be seen as quotient types. The types without
specified equivalence relation can be seen as the quotient of itself
by the trivial equivalence relation, namely the definitional equality.


\subsection{The relation between equality and quotient types}
As we have discussed before, we distinguish definitional equality
and propositional equality in \itt{}. Within the setoid, the
equivalence relation is usually propositional equality which is
non-trivial despite the case that it is the same with definitional
equality. Hence the type-checking which depends on definitional
equality does not respect the equivalence relation.

There are also two different propositions expressing the equailty
between two elements in Type Theory \cite{nor:90}. Both of them
require the types of two elements are definitionally equal.
One is intensional equality written as $Id(A,a,b)$ and it is inhabited
only we have a proof showing $a$ and $b$ are definitionally equal. The
other is extensional equality written as $Eq(A,a,b)$, the elements
do not depend on an element of $A$ and the largest difference is if
$Eq(A,a,b)$ is inhabited, then $a$ converts to $b$ and vice versa. The
latter one will make type-checking undecidable so we usually use the
first one which is available in \itt{}. For example in Agda, it is
redefined as $a \equiv b$ with the type $A$ implicitly, and it has an
unique element $refl$.

Intensional equality is enough for many quotients. However if
we want quotient types corresponding to the sets of functions, the
intensional equalities of functions are inhabited \cite{alt:99}. This
is because functions are equated with respect to extensional features
rather than intensional constructions, i.e.~normal forms.
Hence the functional extensionality which is not inhabited in original \itt{}
will be required. The introduction of functional extensionality is
another big problem of \itt{}, but fortunately Altenkirch introduces a
solution in \cite{alt:99}.

Most of the topics concerning quotient types are closely related to equality.
One of the main issues of quotient types is how to lift the functions for
base types to the ones for quotient types. Only functions respects the
equivalence relation can be lifted, even in the extensional Type
Theory as we will discussed later.

\subsection{Literature Review}

% Why I mention about this article
% More description about these articles
% in a more comprehensive way, tell a story
% compare and link between literatures
In \cite{cab}, Mendler et al. have considered building new types from a
given type by  a quotient operator $//$. Their work is based on an
implementation of extensional Type Theory, NuPRL. In NuPRL, every type
comes with its own equality relation, so the quotient operator can be
seen as a way of redefining equality in a type. But it is not all
about quotient types. They also discuss the problems arised when
defining functions on the new type.
We can illustrate this problem with a simple example. Assume the base
set is $A$ and the new equality relation is $E$, then the new
type can be represented by $A//E$. If we want to define a function $f
\,\colon\, A//E \to Bool$,  Assume we have two different elements in
A, $a, b \colon A$ such that $E\,a\,b$ but $f\,a \not= f\,b$, then it
becomes inconsistent since $E\,a\,b$ implies a converts to b, then
$f\,a = f\,b$ which contradicts with the assumption $f\,a \not=
f\,b$. Therefore even in \ett{}, the definition of functions on the
quotient types are not so simple. The functions have
to respect the equivalence relation, namely$$\forall \, a\,b\,\colon A, E\,a\,b \to f\,a = f\,b$$
then $f$ is well-defined on the new type. We call it \emph{sound} in
\cite{aan} and this project.

 After the introduction of quotient types, Mendler futher investigates
 this topics from a categorical perspective in ~\cite{men:90}. He use
 the correspondence between quotient types in \mltt{} and coequalizers
 in a category of types to define a notion called \emph{squash types}
 which is further discussed by Nogin.


Hofmann proposed  in his PhD thesis \cite{hof:phd} three models for
quotient types. The first one is a setoid model for
quotient types. In this model all types are attached with partial
equivalence relations, namely all types are setoids rather than
sets. Types without specific equivalence relation can be translated as setoids with trivial reflection equality. It is similar to NuPRL. While in \cite{hof:95:sm} he gives a simple model in which we have type dependency only at the propositional level, he also shows that extensional Type Theory is conservative over \itt {}  extended with quotient types and a universe \cite{hof:95:con}.

Nogin \cite{nog:02} considers a modular approach to axiomatizing the
same quotient types also in NuPRL. He also discusses a few complicated
problems about quotient types despite the ease of constructing new types
from base types. For example, since the the equality is
extensional, we can not recover the
witness of the equality. So he suggests to include more axioms to
conceptualise quotients. He decomposes the concept of quotient type
into several more primitive concepts such that the quotient types can
be formalised based on these concepts and can be handled much
simpler. 

Homeier \cite{hom} axiomatises quotient types in Higher Order Logic
(HOL), which is also a theorem prover. He creates a tool package to
construct quotient types as a conservative extension of HOL such that
users are able to define new types in HOL. Next he defines the
normalisation functions and proves several properties of
these. Finally he discussed the issues when quotienting on the
aggregate types such as lists and pairs.


Courtieu \cite{cou:01} shows an extension of Calculus of Inductive Constructions
with \emph{Normalised Types} which are similar to quotient types, but equivalence relations are replaced by normalisation functions. Normalised types are proper subsets of quotient types,
\[ (A, Q, \class\dotph \colon A \to Q) \Rightarrow(A, \lambda \,a \,b\to \class a = \class b)\]

However not all quotient types have normal forms. Therefore it only solves part of the problem.

Similarly, Barthe and Geunvers \cite{bar:96} also proposes \emph{congruence types}, which is also a special class of quotient types, in which the base type are inductively defined and with a set of reduction rules called the term-rewriting system. \ed{The idea behind it is the $\beta$-equivalence is replaced by a set of $\beta$-conversion rules.} The congruence types can be treated as an alternative to pattern matching introduced in \cite{coq:92}.
Hence it aims at solving problems in term rewriting systems rather than simply implementing quotient types.


\section{Aims and Objectives of the Project}

The objective of this project is to investigate and explore the ways to
implement quotient types in Type Theory, especially in intensional
one since type checking always terminates.
As we have seen quotients are quite useful in implementing
mathematical objects and programming datatypes, it will be very
helpful if we can define quotients in thoerem prover like Agda. Also
to implement some other undefinable quotients such as Real numbers, it
is an unvoidable issue to implement the idea of quotient.

The current aim is to implement some definable quotients, use the
quotient interfaces for them and study their benefits. We also need to do
research on the different definitions of quotients.

Next we need to investigate undefinable quotients such as the real
numbers and partiality monads and prove why they are undefinable. The
key different characters between definable and undefinable quotients
will be studied. 


\section{Theoretical Methods}
Part of our work is implemented in Agda, which is a dependent typed
programming language and mainly used as a theorem prover.

\todo{I have to write some other things here. Agda has been introduced
before}
It has dependent type so that we can use Curry-Howard correspondence between types and propositions. Since propositions can be represented as types, its type checker can verify the proof.

In this project the work will be proved in Agda and also verified in Agda since it is a good choice of intensional Type Theory.

\section{Results and Discussion}


\subsection{Definitions}

Currently, we have done some work on the framework of quotient.
We have submitted a paper \cite{aan} for APLAS 2011.
It is about the definable quotients and some undefinable quotients. Here we only talk about the quotient set, but it is universal polymorphic. 

To associate a setoid $(A,\sim)$ with a set Q, we have several definitions as in~\cite{aan},
I will not present it again but explain some ideas behind them.


Given a setoid $(A,\sim)$   , we denote the set of equivalence classes as $\A/sim$ and the normalisation function is $\class\dotph_{\sim} \colon A \to \A/sim $, assigning each elements to the set it is belonging to.
Hence we have

\begin{proposition}\label{prop:nf}
$\forall \, a \,, b \, \colon A, a \sim b \iff [ a ]_{\sim} = [b]_{\sim}$
\end{proposition}

And the normalisation function is surjective, hence we assume classically,



\begin{proposition}
$\forall \, e \,\colon \A/sim , \exists \, a \, \colon A , [a]_{\sim} = e$

\end{proposition}

Namely, the normalisation function is split,


\begin{proposition}
$\exists \, s \, \colon \A/sim \to A,\, \class\dotph_{\sim}  \circ s = 1_{\A/sim}$
\end{proposition}

\


Since then 

$$\class\dotph_{\sim}  \circ (s \circ \class\dotph_{\sim}) = \class\dotph_{\sim} \circ 1_{\A/sim} $$

And with Proposition~\ref{prop:nf}, we can prove that

\begin{proposition}\label{prop:stable}
$\forall \, a \,  \colon A,  (s \circ \, \class\dotph_{\sim}) \, a   \sim a$ 
\end{proposition}


Some of them are only classically true. However, we worked in intensional Type Theory which is constructive. What we do is to  associate a given set Q to the setoid or the quotient set. Given a function $\class\dotph\colon A \to Q$,

 \[\sound\colon  (a,b : A) \to a\sim b \to [a] = [b]\]

is a property which means that from the images of the elements from the same equivalence class are identical, namely $\class\dotph$ respects the equivalence relation. It is also equivalent to say that there is a naming function, $na \, \colon \A/sim  \to Q $ , such that the following diagram commutes,
\[\xymatrix{
A\ar[r]^{\class\dotph_{\sim}}\ar[dr]_{\class\dotph} & \A/sim \ar[d]^{na}\\
&Q
}\]

And $na$ can be constructed as $\class\dotph \, \circ s$. We can prove this diagram is commute as,

\[ na \circ  \class\dotph_{\sim} = \class\dotph \circ s  \circ \class\dotph_{\sim}
\]

Apply  Proposition~\ref{prop:stable} and  Proposition~\ref{prop:nf}, we know,
\[ \forall \,a \, \colon A, (na \circ  \class\dotph_{\sim}) \, a = (\class\dotph \circ s  \circ \class\dotph_{\sim}) \, a = \class\dotph \circ ((s  \circ \class\dotph_{\sim}) \, a) = \class\dotph \, a     
\]

Extensionally, we proved that the diagram commute.


However, with this property, we cannot confirm $Q$ is the required quotient set, we only construct a prequotient.
To complete a quotient, we require one eliminator
for every
$B\colon Q\to\Set$,
\begin{align*}
 \qelim_B\colon &(f\colon (a:A) \to B\,\class a) \\
        {\to}\, &((p:a\sim b) \to f\,a \simeq_{\sound\,p}f\,b)\\
        {\to}\, &((q:Q) \to B\,q)
 \end{align*}
such that $\qelimbeta\colon \qelim_B f \,p\,\class a\equiv f a$.

With this eliminator we can lift a function which takes in $a \,\colon A$ but the result is dependent on the $[a] \, \colon Q$  and identifies more than $\class\dotph$, namely for the elements in the same equivalence class by $\class\dotph$ , the result produced by $f$ is the same. Combining with this function, 

However this is not a exact quotient, since it is unnecessary for $na$ function to assign one \emph{name} ($q \,\colon Q$) to each equivalence class ($e \,\colon \A/sim$). It is very inefficient to define a too general quotient. Therefore we need a property to make a quotient exact,

\[\exact :\forall \,a \, b : A\,,\,  \class a = \class b \to a \sim b \]

equivalently, we have the property that $na$ is injective,

\begin{align*}
\forall \,e \, f : \A/sim \,,\, &na \,e = na \, f \Rightarrow
(\class\dotph \, \circ s) \, e = (\class\dotph \, \circ s) \,f \\
\Rightarrow &\class{s \, e} =  \class{s \, f} \Rightarrow s \, e \sim s \, f \Rightarrow [s \, e]_{\sim} = [s \, f]_{\sim} \Rightarrow e = f
\end{align*}

The alternative definition of quotient with non-dependent eliminator introduced in \cite{hof:phd}, and consists of,



\begin{align*}
\lift_B &\colon (f\colon A \to B) \to (\forall a,b\cdot a\sim b \to f\,a
\equiv f\,b) \to (Q \to B) \\
\liftbeta &\colon \lift_B f \,p\,\class a\equiv f a
\end{align*}



 for any $B\colon\Set,$ which can lift a function $f$ which respects the equivalence relation and the following diagram commute with respect to lift-$\beta$,

\[\xymatrix{
A\ar[r]^{\class\dotph}\ar[dr]_{f} & Q \ar[d]^{lift_{B} \, f \, p}\\
&B
}\]




In this definition we also need an introduction principle if $B$ in the dependent eliminator is a predicate on $Q$,

\[\qind_{P} \colon((a: A)\to P \,\class a)\to ((q : Q)\to P\,q)\]

The quotient with dependent eliminator and the one with non-dependent eliminator are actually equivalent. We prove this by formalise one by another in Agda. It is quite trivial to generate the non-dependent version from dependent version since $\lift_B$ and $\qind_{P}$ are both special cases of the dependent eliminator. However to recover dependent eliminator, it is a little complicated. We need a function $indep$ to transform the dependent $a \, \colon A \to B \,\class a $ into the non-dependent $A \to \Sigma \,Q \,B$ which is defined as $indep \, f \,a \mapsto \class a \,, f \,a$. Then we can use non-dependent eliminator to lift $indep \,f$ and the projection of the second component is the same as dependent function. You can check the detailed Agda proof in the Appendix.

When the quotient type is definable and we want the target type $Q$ is just the quotient type, which means

\[  Q \cong \A/sim \]

Therefore, to constructively define isomorphism in intensional Type Theory, we not only need $na$, but also the inverse function of it.
So the definable quotients in \cite{aan} is the prequotient with

\begin{align*}
\emb &: Q \to A\\
\complete &: (a : A) \to \emb {\class a} \sim a\\
\stable &: (q:Q) \to \class{\emb\,q} \equiv q\\
\end{align*}

$\emb$ is the embedding function which choose one representative element for each equivalence class. Hence the following diagram needs to commute,

\[ \xymatrix{
&A\ar@<-0.5ex>[dl]^{\class\dotph}\ar@<0.5ex>[dr]_{\class\dotph_{\sim}} \\
Q \ar@<1.5ex>[ur]^{\emb} \ar@<-0.5ex>[rr]_{na^{-1}}&&\ar@<-0.5ex>[ll]_{na}  \A/sim \ar@<-1.5ex>[ul]_{s}
}\]


Such that $ na^{-1} = \class\dotph_{\sim} \circ \emb$ is the inverse function of $na$. 



\begin{align*}
&na^{-1} \circ na = 1_{\A/sim} \Rightarrow \class\dotph_{\sim} \circ \emb \circ \class\dotph \circ s = 1_{\A/sim} \\
&\Rightarrow \forall a \colon A ,  a \sim (s \circ \,
\class\dotph_{\sim}) \, a \sim (s \circ 1_{\A/sim} \circ
\class\dotph_{\sim})  \, a \\
&\sim (s \circ (\class\dotph_{\sim} \circ \emb \circ \class\dotph \circ s) \circ \class\dotph_{\sim}) \, a \sim \emb \,\class{a} 
\end{align*}

So we need this property called completeness which ensures the correctness of emb.
Also,

\[ na \circ na^{-1} =1 _{Q} \Rightarrow\class\dotph \circ s \circ \class\dotph_{\sim} \circ \emb = \class\dotph  _{} \circ \emb = 1 _{Q}  \Rightarrow \forall \,q\, \colon Q, \class{\emb \,q} = q  \]

is needed called the stable property which ensures
$\class\dotph$ is surjective, Hence it is \emph{normalisation} function and the $Q$ is the quotient type without redundance.

With these two properties, we can conclude that $ \class\dotph_{\sim} \circ \emb $ is the inverse function of $na$ , hence $Q$ is isomorphic to $\A/sim$.
We can use it as the quotient type.

In category theory, coequalizers are the generalization of quotients.
We assume

$R = \Sigma a , b : A , a \sim b$  are the pairs of equivalent elements in $A$

$\pi_{0}\,,\pi_{1} \colon R \to A $ are the projection functions for $R$

$\class\dotph \colon A \to Q$ satisfies that $\sound \colon \forall\,a,b \colon A, a \sim b \to \class a = \class b$


\[\xymatrix{
R\ar@<0.5ex>[r]^{\pi_0}\ar@<-0.5ex>[r]_{\pi_1}& A\ar[r]^{\class\dotph}\ar[dr]_{f} & Q\ar@{-->}[d]^{\dlift f}\\
&&X
}\]


Since 
\begin{enumerate}
\item$(Q \,, \class\dotph)$ fulfils that $\class\dotph\ \circ \pi_0 = \class\dotph \circ \pi_1$, we can acquire this from applying the $\pi_{0}\, r ,\pi_{1}\,r$ for all $r $ to sound.
\item Given any $(X, f \colon A \to X)$, there exists a unique $\hat{f}$, such that the diagram above commutes. From the definition of quotients, we can use the eliminator to lift $f$ , namely $\hat{f} = \lift f,  $ and the $\beta$-law simply implies the diagram commutes. The uniqueness can be proved as follows

\[\forall \,g \, \colon Q \to X, g \circ \class\dotph = f \Rightarrow \forall \,a \,\colon A, g \,\class a = f a = \lift f \,q \,\class a  \Rightarrow g = \lift f \,q\]

\end{enumerate}

These two parts proved from quotients
exactly define a coequalizer. Also we can prove $\class\dotph$ is an epimorphism

\begin{align*}
&\forall\, g_1,g_2 : Q \to Z, g_1 \circ \class\dotph = g_2 \circ
\class\dotph\\
&\Rightarrow  \forall \,q \,\colon Q,  g_{1} \,q = \lift \, (g_1 \circ \class\dotph) \, q = \lift \, (g_2 \circ \class\dotph) \, q = g_2\, q \Rightarrow g_1 = g_2
\end{align*}

Also the exact quotient is equivalent to the exact coequalizer,

\[\xymatrix{
R\pullbackcorner\ar[r]^{\pi_2}\ar[d]_{\pi_1} & A\ar[d]^{\class\dotph} \\
A\ar[r]_{\class\dotph} & Q
}\]

\begin{enumerate}

\item This diagram commutes

$(\forall \,r\, \colon R, \pi_1 \, r \sim \pi_2 \, r \Rightarrow\class{\pi_1 \,r} = \class{\pi_2 \, r}) \Rightarrow \class\dotph \circ \pi_1 = \class\dotph\circ \pi_2$


\item 
\begin{align*}
&\forall (Z , z_1 \colon Z \to A, z_2 \colon Z \to A), \,
\class{\dotph} \circ z_1 = \class\dotph \circ z_2 \\
&\Rightarrow ( \exists \,u : Z \to R, \pi_1 \circ u = z_1 \wedge  \pi_2
\circ u = z \\
&\wedge \,\forall \,u' \colon Z \to R, \pi_1 \circ u' = z_1 \wedge
\pi_2 \circ u' = z_2 \\
&\Rightarrow u = u') 
\end{align*}

We  can construct the unique function as $u \,x \mapsto z_1 \,x \,, z_2\, x$, but we need to prove $z_1 \,x\sim z_2 \,x$ from exact property of quotient, 

$\class{\dotph} \circ z_1 = \class\dotph \circ z_2 \Rightarrow \forall \,x\, \colon Z, (\class{z_1\,x}=\class{z_2 \,x} \Rightarrow z_{1} \,x \sim z_2 \, x) $

$u$ is the function which makes the diagram commutes,

$$\forall \,x\,\colon Z, (\pi_1 \circ u) \, x = z_1 \,x$$

$$\forall \,x\,\colon Z, (\pi_2 \circ u) \, x = z_2 \,x$$

$u$ is unique,


\begin{align*}
&\forall \,u' \colon Z \to R, \pi_1 \circ u' = z_1 \wedge  \pi_2 \circ u'  = z_2\\ 
&\Rightarrow \forall \,x\,\colon Z, u' \,x = z_1 \,x\,,z_2\,x=u\,x \Rightarrow u' = u
\end{align*}

\end{enumerate}


\section{Examples}
We have already define the basic requirements to create quotients in intensional Type Theory, I will then present some concrete examples in \cite{nuo:10} to illustrate these ideas. They are implemented in Agda. 

\subsection{Integers}


All the result of subtraction between natural numbers are integers. Therefore it is naturally to define a pair of
natural numbers to represent integers. Hence the base type of the quotient is

$$\Z_0=\N \times \N$$

Mathematically, for any two pairs of natural numbers $(n_1, n_2)$ and $(n_3, n_4)$, 

$$ n_1 - n_2 = n_3 - n_4\iff n_1 + n_4 = n_3 + n_2$$
since the pair of integers represent the same result of subtraction, they define the same integer. Hence we can define an equivalence relation for $\N \times \N$ as

\[ (n_1, n_2) \sim (n_3, n_4) = n_1 + n_4 \equiv n_3 + n_2 
 \]


Here $\equiv$ is the propositional equality. so that the $\Z_0/\sim$ is the quotient integer.
Integer is also definable in intensional Type Theory as $\N+\N$ where we define two constructors

$ (n \colon \N) \Rightarrow + \,n \colon\Z$

$ (n \colon \N) \Rightarrow -\suc n \colon\Z$


Firstly to construct the prequotient based on the setoid $(\Z_0,\sim)$, we need to define the $\class\dotph \colon \Z_0 \to \Z$ as

\begin{align*}
\class{(a,0)} &= +\,a\\
\class{(0,\suc b)} &= -\suc b\\
\class{(\suc a,\suc b)} &= \class{(a,b)}\\
\end{align*}

and prove \emph{sound}. Then we define emb function and prove all the required properties for definable quotients


\begin{align*}
\emb \,(+ a) &= (a,0)\\
\emb (-\suc b) &= (0,b+1)\\
\end{align*}

We have done these in Agda \cite{nuo:10}. The quotients here are not just something relate the setoid with the quotient type, we use lift functions to define functions trivially and use properties to transform the proof term for the setoid to the quotient type. For instance, the addition of the setoid is defined as

$$(a,b){+_0}(a', b')= (a+a',b+b')$$

We can then define use the eliminator to lift the operator, or just define a lift function for binary operators,

$$ \lift \,* \,z_1 \,z_2 = \class{\emb \,z\textcolor[rgb]{0,0,0}{text}_1 \,*\,\emb \,z_2}$$

or a more general lift function for n-ary operators,



\begin{align*}
\lift' \,0 \,op &= \class{op}\\
\lift' \,(\suc n) \,op &= \lambda \,x \to \lift' n \,(op \,(\emb x))\\
\end{align*}
Then we don't need to define the addition of integers by several cases.

$$+ = \lift +_0$$

If we lift the operators in this way, we have to prove it respects the equivalence relation later. The main benefits from the quotients arise in proving properties. Because for normal form integers, we have two cases for each argument. The number of cases will expand exponentially if we can not combine cases. The proof of distributivity of multiplication over addition is so cumbersome
that it is hard to write and read. However, we could lift the proof for the setoid integers so that we could prove it in one case. This convenience is due to the simplicity of the proof for the setoid $(\Z_0,\sim)$. 

\subsection{Rational numbers}

The quotients of rational numbers is more natural to understand and the normalisation function is also commonly used in regular mathematics. Generally we can use a pair of integers to represent rational numbers. However, it is complicated to exclude 0 in the denominator. For simplicity, we just use one integer for \emph{numerator} and one natural number for \emph{denominator-1} to represent a rational number to avoid the invalid cases from construction. 

$$\Q_0 = \Z \times \N$$



The equivalence relation is

$$(n_1, d_1) \sim (n_2, d_2) = n_1 \times (d_2 + 1) \equiv n_2 \times (d_1 + 1) $$

The normal form of rational numbers can just be defined by adding a condition that the numerator and denominator are coprime.

$$\Q = \Sigma (n \colon \Z)(d \colon \N), \coprime \,n \,(d +1)$$

Since there are a set of \emph{gcd} (great common divisor) functions in Agda, it is possible to define the normalisation functions (See Appendix).
emb function can be trivially defined by forgetting coprime proof.

\subsection{Real numbers as cauchy sequences}

We can represent real numbers as cauchy sequences of rational numbers \cite{bis:85}.

$$\R_{0} = \set{s : \N\to\Q \mid \forall\varepsilon :\Q,\varepsilon>0\to\exists m:\N, \forall i:\N, i>m\to |s\,i - s\, m|<\varepsilon}$$

And we define the equivalence relation of two sequences by the proposition that their pointwise difference converges to 0.

$$r \sim s = \forall\varepsilon :\Q,\varepsilon>0\to\exists m:\N, \forall i:\N, i>m\to |r\,i - s\,i|<\varepsilon$$

Then $\R_0 /\sim$ is the quotient set of real numbers. However it is undefinable because real numbers have no normal forms.
Therefore we cannot use the definable quotient interface for it. The undefinability is proved in \cite{aan}.
Nevertheless, we could easily embedding rational numbers as the cauchy sequences of all the same rational numbers. But for irrational numbers, there is no such an uniform way to generate a sequences.
\section{Conclusion}

Currently we investigate the possible quotient definitions in \itt{} and present some examples and benefits from the definable quotients. For definable quotients, it provides an alternative choice to define functions or prove propositions which reuses things and could be simpler in most cases. However, to solve the problems arose from undefinable quotients, a new type former may be unavoidable.

\todo{future extension.}

%future : something on equality, complete preliminary work in Agda
%Extend without losing nice features of itt, termination, deciable type checking
%axiomatising quotients types, adding rules, na 
% possible quotions, talk to Thorsten
% 
% pay attention to the connection,the flow of ideas throughout the article.
% one thing in one paragraph.

If we axiomize quotient types in \itt, then every type can be seen as quotient type, when the default equivalence relation is just \ed{reflection equality}.
\newpage
\bibliography{quotients}
\bibliographystyle{plain}

\end{document}
