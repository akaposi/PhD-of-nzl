\documentclass[a4paper,12pt]{article}
\def\textmu{}
%% ODER: format ==         = "\mathrel{==}"
%% ODER: format /=         = "\neq "
%
%
\makeatletter
\@ifundefined{lhs2tex.lhs2tex.sty.read}%
  {\@namedef{lhs2tex.lhs2tex.sty.read}{}%
   \newcommand\SkipToFmtEnd{}%
   \newcommand\EndFmtInput{}%
   \long\def\SkipToFmtEnd#1\EndFmtInput{}%
  }\SkipToFmtEnd

\newcommand\ReadOnlyOnce[1]{\@ifundefined{#1}{\@namedef{#1}{}}\SkipToFmtEnd}
\usepackage{amstext}
\usepackage{amssymb}
\usepackage{stmaryrd}
\DeclareFontFamily{OT1}{cmtex}{}
\DeclareFontShape{OT1}{cmtex}{m}{n}
  {<5><6><7><8>cmtex8
   <9>cmtex9
   <10><10.95><12><14.4><17.28><20.74><24.88>cmtex10}{}
\DeclareFontShape{OT1}{cmtex}{m}{it}
  {<-> ssub * cmtt/m/it}{}
\newcommand{\texfamily}{\fontfamily{cmtex}\selectfont}
\DeclareFontShape{OT1}{cmtt}{bx}{n}
  {<5><6><7><8>cmtt8
   <9>cmbtt9
   <10><10.95><12><14.4><17.28><20.74><24.88>cmbtt10}{}
\DeclareFontShape{OT1}{cmtex}{bx}{n}
  {<-> ssub * cmtt/bx/n}{}
\newcommand{\tex}[1]{\text{\texfamily#1}}	% NEU

\newcommand{\Sp}{\hskip.33334em\relax}


\newcommand{\Conid}[1]{\mathit{#1}}
\newcommand{\Varid}[1]{\mathit{#1}}
\newcommand{\anonymous}{\kern0.06em \vbox{\hrule\@width.5em}}
\newcommand{\plus}{\mathbin{+\!\!\!+}}
\newcommand{\bind}{\mathbin{>\!\!\!>\mkern-6.7mu=}}
\newcommand{\rbind}{\mathbin{=\mkern-6.7mu<\!\!\!<}}% suggested by Neil Mitchell
\newcommand{\sequ}{\mathbin{>\!\!\!>}}
\renewcommand{\leq}{\leqslant}
\renewcommand{\geq}{\geqslant}
\usepackage{polytable}

%mathindent has to be defined
\@ifundefined{mathindent}%
  {\newdimen\mathindent\mathindent\leftmargini}%
  {}%

\def\resethooks{%
  \global\let\SaveRestoreHook\empty
  \global\let\ColumnHook\empty}
\newcommand*{\savecolumns}[1][default]%
  {\g@addto@macro\SaveRestoreHook{\savecolumns[#1]}}
\newcommand*{\restorecolumns}[1][default]%
  {\g@addto@macro\SaveRestoreHook{\restorecolumns[#1]}}
\newcommand*{\aligncolumn}[2]%
  {\g@addto@macro\ColumnHook{\column{#1}{#2}}}

\resethooks

\newcommand{\onelinecommentchars}{\quad-{}- }
\newcommand{\commentbeginchars}{\enskip\{-}
\newcommand{\commentendchars}{-\}\enskip}

\newcommand{\visiblecomments}{%
  \let\onelinecomment=\onelinecommentchars
  \let\commentbegin=\commentbeginchars
  \let\commentend=\commentendchars}

\newcommand{\invisiblecomments}{%
  \let\onelinecomment=\empty
  \let\commentbegin=\empty
  \let\commentend=\empty}

\visiblecomments

\newlength{\blanklineskip}
\setlength{\blanklineskip}{0.66084ex}

\newcommand{\hsindent}[1]{\quad}% default is fixed indentation
\let\hspre\empty
\let\hspost\empty
\newcommand{\NB}{\textbf{NB}}
\newcommand{\Todo}[1]{$\langle$\textbf{To do:}~#1$\rangle$}

\EndFmtInput
\makeatother
%
%
%
%
%
%
% This package provides two environments suitable to take the place
% of hscode, called "plainhscode" and "arrayhscode". 
%
% The plain environment surrounds each code block by vertical space,
% and it uses \abovedisplayskip and \belowdisplayskip to get spacing
% similar to formulas. Note that if these dimensions are changed,
% the spacing around displayed math formulas changes as well.
% All code is indented using \leftskip.
%
% Changed 19.08.2004 to reflect changes in colorcode. Should work with
% CodeGroup.sty.
%
\ReadOnlyOnce{polycode.fmt}%
\makeatletter

\newcommand{\hsnewpar}[1]%
  {{\parskip=0pt\parindent=0pt\par\vskip #1\noindent}}

% can be used, for instance, to redefine the code size, by setting the
% command to \small or something alike
\newcommand{\hscodestyle}{}

% The command \sethscode can be used to switch the code formatting
% behaviour by mapping the hscode environment in the subst directive
% to a new LaTeX environment.

\newcommand{\sethscode}[1]%
  {\expandafter\let\expandafter\hscode\csname #1\endcsname
   \expandafter\let\expandafter\endhscode\csname end#1\endcsname}

% "compatibility" mode restores the non-polycode.fmt layout.

\newenvironment{compathscode}%
  {\par\noindent
   \advance\leftskip\mathindent
   \hscodestyle
   \let\\=\@normalcr
   \let\hspre\(\let\hspost\)%
   \pboxed}%
  {\endpboxed\)%
   \par\noindent
   \ignorespacesafterend}

\newcommand{\compaths}{\sethscode{compathscode}}

% "plain" mode is the proposed default.
% It should now work with \centering.
% This required some changes. The old version
% is still available for reference as oldplainhscode.

\newenvironment{plainhscode}%
  {\hsnewpar\abovedisplayskip
   \advance\leftskip\mathindent
   \hscodestyle
   \let\hspre\(\let\hspost\)%
   \pboxed}%
  {\endpboxed%
   \hsnewpar\belowdisplayskip
   \ignorespacesafterend}

\newenvironment{oldplainhscode}%
  {\hsnewpar\abovedisplayskip
   \advance\leftskip\mathindent
   \hscodestyle
   \let\\=\@normalcr
   \(\pboxed}%
  {\endpboxed\)%
   \hsnewpar\belowdisplayskip
   \ignorespacesafterend}

% Here, we make plainhscode the default environment.

\newcommand{\plainhs}{\sethscode{plainhscode}}
\newcommand{\oldplainhs}{\sethscode{oldplainhscode}}
\plainhs

% The arrayhscode is like plain, but makes use of polytable's
% parray environment which disallows page breaks in code blocks.

\newenvironment{arrayhscode}%
  {\hsnewpar\abovedisplayskip
   \advance\leftskip\mathindent
   \hscodestyle
   \let\\=\@normalcr
   \(\parray}%
  {\endparray\)%
   \hsnewpar\belowdisplayskip
   \ignorespacesafterend}

\newcommand{\arrayhs}{\sethscode{arrayhscode}}

% The mathhscode environment also makes use of polytable's parray 
% environment. It is supposed to be used only inside math mode 
% (I used it to typeset the type rules in my thesis).

\newenvironment{mathhscode}%
  {\parray}{\endparray}

\newcommand{\mathhs}{\sethscode{mathhscode}}

% texths is similar to mathhs, but works in text mode.

\newenvironment{texthscode}%
  {\(\parray}{\endparray\)}

\newcommand{\texths}{\sethscode{texthscode}}

% The framed environment places code in a framed box.

\def\codeframewidth{\arrayrulewidth}
\RequirePackage{calc}

\newenvironment{framedhscode}%
  {\parskip=\abovedisplayskip\par\noindent
   \hscodestyle
   \arrayrulewidth=\codeframewidth
   \tabular{@{}|p{\linewidth-2\arraycolsep-2\arrayrulewidth-2pt}|@{}}%
   \hline\framedhslinecorrect\\{-1.5ex}%
   \let\endoflinesave=\\
   \let\\=\@normalcr
   \(\pboxed}%
  {\endpboxed\)%
   \framedhslinecorrect\endoflinesave{.5ex}\hline
   \endtabular
   \parskip=\belowdisplayskip\par\noindent
   \ignorespacesafterend}

\newcommand{\framedhslinecorrect}[2]%
  {#1[#2]}

\newcommand{\framedhs}{\sethscode{framedhscode}}

% The inlinehscode environment is an experimental environment
% that can be used to typeset displayed code inline.

\newenvironment{inlinehscode}%
  {\(\def\column##1##2{}%
   \let\>\undefined\let\<\undefined\let\\\undefined
   \newcommand\>[1][]{}\newcommand\<[1][]{}\newcommand\\[1][]{}%
   \def\fromto##1##2##3{##3}%
   \def\nextline{}}{\) }%

\newcommand{\inlinehs}{\sethscode{inlinehscode}}

% The joincode environment is a separate environment that
% can be used to surround and thereby connect multiple code
% blocks.

\newenvironment{joincode}%
  {\let\orighscode=\hscode
   \let\origendhscode=\endhscode
   \def\endhscode{\def\hscode{\endgroup\def\@currenvir{hscode}\\}\begingroup}
   %\let\SaveRestoreHook=\empty
   %\let\ColumnHook=\empty
   %\let\resethooks=\empty
   \orighscode\def\hscode{\endgroup\def\@currenvir{hscode}}}%
  {\origendhscode
   \global\let\hscode=\orighscode
   \global\let\endhscode=\origendhscode}%

\makeatother
\EndFmtInput
%
%
\ReadOnlyOnce{agda.fmt}%


\RequirePackage[T1]{fontenc}
\RequirePackage[utf8x]{inputenc}
\RequirePackage{ucs}
\RequirePackage{amsfonts}

\providecommand\mathbbm{\mathbb}

% TODO: Define more of these ...
\DeclareUnicodeCharacter{737}{\textsuperscript{l}}
\DeclareUnicodeCharacter{8718}{\ensuremath{\blacksquare}}
\DeclareUnicodeCharacter{8759}{::}
\DeclareUnicodeCharacter{9669}{\ensuremath{\triangleleft}}
\DeclareUnicodeCharacter{8799}{\ensuremath{\stackrel{\scriptscriptstyle ?}{=}}}
\DeclareUnicodeCharacter{10214}{\ensuremath{\llbracket}}
\DeclareUnicodeCharacter{10215}{\ensuremath{\rrbracket}}

% TODO: This is in general not a good idea.
\providecommand\textepsilon{$\epsilon$}
\providecommand\textmu{$\mu$}


%Actually, varsyms should not occur in Agda output.

% TODO: Make this configurable. IMHO, italics doesn't work well
% for Agda code.

\renewcommand\Varid[1]{\mathord{\textsf{#1}}}
\let\Conid\Varid
\newcommand\Keyword[1]{\textsf{\textbf{#1}}}
\EndFmtInput


\usepackage[utf8x]{inputenc}
\usepackage{ucs}
\usepackage{cite}
\usepackage{MnSymbol}


%\DeclareUnicodeCharacter{"2237}{\ensuremath{::}}
\DeclareUnicodeCharacter{"03BB}{\ensuremath{\lambda}}
\DeclareUnicodeCharacter{"03A3}{\ensuremath{\Sigma}}
\DeclareUnicodeCharacter{"03C6}{\ensuremath{\phi}}
\DeclareUnicodeCharacter{"03C8}{\ensuremath{\psi}}
\DeclareUnicodeCharacter{"231C}{\ensuremath{\ulcorner}}
\DeclareUnicodeCharacter{"231D}{\ensuremath{\urcorner}}
\DeclareUnicodeCharacter{"25B6}{\ensuremath{\filledmedtriangleright}}

\usepackage{color}
\newcommand{\txa}[1]{\textcolor{red}{\textbf{Thorsten:~}#1}}

\newcommand{\nzl}[1]{\textcolor{green}{\textbf{Nuo:~}#1}}

\begin{document}

\section{Quotient definition of Integers}
All the result of subtraction between natural numbers are integers. Therefore it is naturally to define a pair of
natural numbers to represent integers. Hence we can use a pair of natural nubmers as raw type \ensuremath{\Conid{ℤ₀}}.

\begin{hscode}\SaveRestoreHook
\column{B}{@{}>{\hspre}l<{\hspost}@{}}%
\column{E}{@{}>{\hspre}l<{\hspost}@{}}%
\>[B]{}\Conid{ℤ₀}\;\mathrel{=}\;\Conid{ℕ}\;\Varid{×}\;\Conid{ℕ}{}\<[E]%
\ColumnHook
\end{hscode}\resethooks

Mathematically, for two elements of \ensuremath{(\Varid{n1},\Varid{n2})} and \ensuremath{(\Varid{n3},\Varid{n4})}, when \ensuremath{\Varid{n1}\;\Varid{+}\;\Varid{n4}\;\mathrel{=}\;\Varid{n3}\;\Varid{+}\;\Varid{n2}}, they represent the same result of subtraction , namely the same integer. By these definition, we can define for an equivalence relation for \ensuremath{\Conid{ℕ}\;\Varid{×}\;\Conid{ℕ}}, when we want to use \ensuremath{\Conid{ℕ}\;\Varid{×}\;\Conid{ℕ}} to represent integers,


\begin{hscode}\SaveRestoreHook
\column{B}{@{}>{\hspre}l<{\hspost}@{}}%
\column{E}{@{}>{\hspre}l<{\hspost}@{}}%
\>[B]{}\Varid{\char95 ∼\char95 }\;\mathbin{:}\;\Conid{Rel}\;\Conid{ℤ₀}\;\Varid{zero}{}\<[E]%
\\
\>[B]{}(\Varid{x+},\Varid{x-})\;\Varid{∼}\;(\Varid{y+},\Varid{y-})\;\mathrel{=}\;(\Varid{x+}\;\Conid{ℕ+}\;\Varid{y-})\;\Varid{≡}\;(\Varid{y+}\;\Conid{ℕ+}\;\Varid{x-}){}\<[E]%
\ColumnHook
\end{hscode}\resethooks

With this equivalence relation, the set of all pairs of natural
numbers are divided into equivalence classes. For each equivalence class
we can choose a representative, namely choosing the normal form within \ensuremath{\Conid{ℤ₀}}.
We call the function used to choose normal forms as normalisation. The one below is one of the canonical normalisation within the \ensuremath{\Conid{ℤ₀}}.

\begin{hscode}\SaveRestoreHook
\column{B}{@{}>{\hspre}l<{\hspost}@{}}%
\column{23}{@{}>{\hspre}l<{\hspost}@{}}%
\column{E}{@{}>{\hspre}l<{\hspost}@{}}%
\>[B]{}[\mskip1.5mu \anonymous \mskip1.5mu]\;{}\<[23]%
\>[23]{}\mathbin{:}\;\Conid{ℤ₀}\;\Varid{→}\;\Conid{ℤ₀}{}\<[E]%
\\
\>[B]{}[\mskip1.5mu \Varid{m},\Varid{0}\mskip1.5mu]\;{}\<[23]%
\>[23]{}\mathrel{=}\;\Varid{m},\Varid{0}{}\<[E]%
\\
\>[B]{}[\mskip1.5mu \Varid{0},\Conid{ℕ.suc}\;\Varid{n}\mskip1.5mu]\;{}\<[23]%
\>[23]{}\mathrel{=}\;\Varid{0},\Conid{ℕ.suc}\;\Varid{n}{}\<[E]%
\\
\>[B]{}[\mskip1.5mu \Conid{ℕ.suc}\;\Varid{m},\Conid{ℕ.suc}\;\Varid{n}\mskip1.5mu]\;\mathrel{=}\;[\mskip1.5mu \Varid{m},\Varid{n}\mskip1.5mu]{}\<[E]%
\ColumnHook
\end{hscode}\resethooks

For example,  \ensuremath{(\Varid{3},\Varid{2})} can be normalised to \ensuremath{(\Varid{2},\Varid{1})}, then to \ensuremath{(\Varid{1},\Varid{0})}. 

To prove it is normalisation we should prove the result is still in the same equivalence classes. And the normal form should be unique.

\begin{hscode}\SaveRestoreHook
\column{B}{@{}>{\hspre}l<{\hspost}@{}}%
\column{E}{@{}>{\hspre}l<{\hspost}@{}}%
\>[B]{}\Varid{normal-ok}\;\mathbin{:}\;\Varid{∀}\;\Varid{a}\;\Varid{→}\;[\mskip1.5mu \Varid{a}\mskip1.5mu]\;\Varid{∼}\;\Varid{a}{}\<[E]%
\\[\blanklineskip]%
\>[B]{}\Varid{normal-unique}\;\mathbin{:}\;\Varid{∀}\;\Varid{a}\;\Varid{b}\;\Varid{→}\;\Varid{a}\;\Varid{∼}\;\Varid{b}\;\Varid{→}\;[\mskip1.5mu \Varid{a}\mskip1.5mu]\;\Varid{≡}\;[\mskip1.5mu \Varid{b}\mskip1.5mu]{}\<[E]%
\ColumnHook
\end{hscode}\resethooks

As soon as we have the normalisation, we can use another
more general way to define equivalence relation, namely just identify
their normal form.

\begin{hscode}\SaveRestoreHook
\column{B}{@{}>{\hspre}l<{\hspost}@{}}%
\column{E}{@{}>{\hspre}l<{\hspost}@{}}%
\>[B]{}\Varid{\char95 ∼\char95 }\;\mathbin{:}\;\Conid{Rel}\;\Conid{ℤ₀}\;\Varid{zero}{}\<[E]%
\\
\>[B]{}\Varid{x}\;\Varid{∼}\;\Varid{y}\;\mathrel{=}\;[\mskip1.5mu \Varid{x}\mskip1.5mu]\;\Varid{≡}\;[\mskip1.5mu \Varid{y}\mskip1.5mu]{}\<[E]%
\ColumnHook
\end{hscode}\resethooks

The \ensuremath{[\mskip1.5mu \anonymous \mskip1.5mu]} is an endomap in the set \ensuremath{\Conid{ℤ₀}}, and the resulting
subset is actually isomorphic to the set of integers. 
However since we
do not distinguish the types of the original form and the normal form,
we lose the information that it has been normalised. Therefore we can
define the type of the result to be the set of integers.


\begin{hscode}\SaveRestoreHook
\column{B}{@{}>{\hspre}l<{\hspost}@{}}%
\column{23}{@{}>{\hspre}l<{\hspost}@{}}%
\column{E}{@{}>{\hspre}l<{\hspost}@{}}%
\>[B]{}[\mskip1.5mu \anonymous \mskip1.5mu]\;{}\<[23]%
\>[23]{}\mathbin{:}\;\Conid{ℤ₀}\;\Varid{→}\;\Conid{ℤ}{}\<[E]%
\\
\>[B]{}[\mskip1.5mu \Varid{m},\Varid{0}\mskip1.5mu]\;{}\<[23]%
\>[23]{}\mathrel{=}\;\Varid{+}\;\Varid{m}{}\<[E]%
\\
\>[B]{}[\mskip1.5mu \Varid{0},\Conid{ℕ.suc}\;\Varid{n}\mskip1.5mu]\;{}\<[23]%
\>[23]{}\mathrel{=}\;\Varid{-suc}\;\Varid{n}{}\<[E]%
\\
\>[B]{}[\mskip1.5mu \Conid{ℕ.suc}\;\Varid{m},\Conid{ℕ.suc}\;\Varid{n}\mskip1.5mu]\;\mathrel{=}\;[\mskip1.5mu \Varid{m},\Varid{n}\mskip1.5mu]{}\<[E]%
\ColumnHook
\end{hscode}\resethooks


Then this is a retraction function for the normalisation function and
we call it denormalisation function.

\begin{hscode}\SaveRestoreHook
\column{B}{@{}>{\hspre}l<{\hspost}@{}}%
\column{12}{@{}>{\hspre}l<{\hspost}@{}}%
\column{E}{@{}>{\hspre}l<{\hspost}@{}}%
\>[B]{}\Varid{⌜\char95 ⌝}\;{}\<[12]%
\>[12]{}\mathbin{:}\;\Conid{ℤ}\;\Varid{→}\;\Conid{ℤ₀}{}\<[E]%
\\
\>[B]{}\Varid{⌜}\;\Varid{+}\;\Varid{n}\;\Varid{⌝}\;{}\<[12]%
\>[12]{}\mathrel{=}\;\Varid{n},\Varid{0}{}\<[E]%
\\
\>[B]{}\Varid{⌜}\;\Varid{-suc}\;\Varid{n}\;\Varid{⌝}\;\mathrel{=}\;\Varid{0},\Conid{ℕ.suc}\;\Varid{n}{}\<[E]%
\ColumnHook
\end{hscode}\resethooks


Firstly we need to prove \ensuremath{\Varid{∼}} is actually an equivalence
relation.

\emph{Reflexivity}

\begin{hscode}\SaveRestoreHook
\column{B}{@{}>{\hspre}l<{\hspost}@{}}%
\column{E}{@{}>{\hspre}l<{\hspost}@{}}%
\>[B]{}\Varid{zrefl}\;\mathbin{:}\;\Conid{Reflexive}\;\Varid{\char95 ∼\char95 }{}\<[E]%
\\
\>[B]{}\Varid{zrefl}\;\{\mskip1.5mu \Varid{x+},\Varid{x-}\mskip1.5mu\}\;\mathrel{=}\;\Varid{refl}{}\<[E]%
\ColumnHook
\end{hscode}\resethooks

\emph{Symmetry}

\begin{hscode}\SaveRestoreHook
\column{B}{@{}>{\hspre}l<{\hspost}@{}}%
\column{E}{@{}>{\hspre}l<{\hspost}@{}}%
\>[B]{}\Varid{zsym}\;\mathbin{:}\;\Conid{Symmetric}\;\Varid{\char95 ∼\char95 }{}\<[E]%
\\
\>[B]{}\Varid{zsym}\;\{\mskip1.5mu \Varid{x+},\Varid{x-}\mskip1.5mu\}\;\{\mskip1.5mu \Varid{y+},\Varid{y-}\mskip1.5mu\}\;\mathrel{=}\;\Varid{sym}{}\<[E]%
\ColumnHook
\end{hscode}\resethooks


\emph{Transitivity}

\begin{hscode}\SaveRestoreHook
\column{B}{@{}>{\hspre}l<{\hspost}@{}}%
\column{3}{@{}>{\hspre}l<{\hspost}@{}}%
\column{29}{@{}>{\hspre}l<{\hspost}@{}}%
\column{E}{@{}>{\hspre}l<{\hspost}@{}}%
\>[B]{}\Varid{\char95 >∼<\char95 }\;\mathbin{:}\;\Conid{Transitive}\;\Varid{\char95 ∼\char95 }{}\<[E]%
\\
\>[B]{}\Varid{\char95 >∼<\char95 }\;\{\mskip1.5mu \Varid{x+},\Varid{x-}\mskip1.5mu\}\;\{\mskip1.5mu \Varid{y+},\Varid{y-}\mskip1.5mu\}\;\{\mskip1.5mu \Varid{z+},\Varid{z-}\mskip1.5mu\}\;\Varid{x=y}\;\Varid{y=z}\;\mathrel{=}\;{}\<[E]%
\\
\>[B]{}\hsindent{3}{}\<[3]%
\>[3]{}\Varid{cancel-+-left}\;(\Varid{y+}\;\Conid{ℕ+}\;\Varid{y-})\;{}\<[29]%
\>[29]{}(\Conid{ℕ.exchange₁}\;\Varid{y+}\;\Varid{y-}\;\Varid{x+}\;\Varid{z-}\;\Varid{>≡<}{}\<[E]%
\\
\>[B]{}\hsindent{3}{}\<[3]%
\>[3]{}(\Varid{y=z}\;\Varid{+=}\;\Varid{x=y})\;\Varid{>≡<}\;\Conid{ℕ.exchange₂}\;\Varid{z+}\;\Varid{y-}\;\Varid{y+}\;\Varid{x-}){}\<[E]%
\ColumnHook
\end{hscode}\resethooks

\emph{ ∼ isEquivalence relation}

\begin{hscode}\SaveRestoreHook
\column{B}{@{}>{\hspre}l<{\hspost}@{}}%
\column{3}{@{}>{\hspre}l<{\hspost}@{}}%
\column{11}{@{}>{\hspre}l<{\hspost}@{}}%
\column{E}{@{}>{\hspre}l<{\hspost}@{}}%
\>[B]{}\Varid{\char95 ∼\char95 isEquivalence}\;\mathbin{:}\;\Conid{IsEquivalence}\;\Varid{\char95 ∼\char95 }{}\<[E]%
\\
\>[B]{}\Varid{\char95 ∼\char95 isEquivalence}\;\mathrel{=}\;\Keyword{record}{}\<[E]%
\\
\>[B]{}\hsindent{3}{}\<[3]%
\>[3]{}\{\mskip1.5mu \Varid{refl}\;{}\<[11]%
\>[11]{}\mathrel{=}\;\Varid{zrefl}{}\<[E]%
\\
\>[B]{}\hsindent{3}{}\<[3]%
\>[3]{};\Varid{sym}\;{}\<[11]%
\>[11]{}\mathrel{=}\;\Varid{zsym}{}\<[E]%
\\
\>[B]{}\hsindent{3}{}\<[3]%
\>[3]{};\Varid{trans}\;\mathrel{=}\;\Varid{\char95 >∼<\char95 }{}\<[E]%
\\
\>[B]{}\hsindent{3}{}\<[3]%
\>[3]{}\mskip1.5mu\}{}\<[E]%
\ColumnHook
\end{hscode}\resethooks


Now we can prove that the \ensuremath{\Conid{ℤ₀}} and its equivalence relation \ensuremath{\Varid{∼}}
form a setoid. 

\emph{ (\ensuremath{\Conid{ℤ₀}}, ∼) is a setoid}

\begin{hscode}\SaveRestoreHook
\column{B}{@{}>{\hspre}l<{\hspost}@{}}%
\column{3}{@{}>{\hspre}l<{\hspost}@{}}%
\column{19}{@{}>{\hspre}l<{\hspost}@{}}%
\column{E}{@{}>{\hspre}l<{\hspost}@{}}%
\>[B]{}\Conid{ℤ-Setoid}\;\mathbin{:}\;\Conid{Setoid}\;\anonymous \;\anonymous {}\<[E]%
\\
\>[B]{}\Conid{ℤ-Setoid}\;\mathrel{=}\;\Keyword{record}{}\<[E]%
\\
\>[B]{}\hsindent{3}{}\<[3]%
\>[3]{}\{\mskip1.5mu \Conid{Carrier}\;{}\<[19]%
\>[19]{}\mathrel{=}\;\Conid{ℤ₀}{}\<[E]%
\\
\>[B]{}\hsindent{3}{}\<[3]%
\>[3]{};\Varid{\char95 ≈\char95 }\;{}\<[19]%
\>[19]{}\mathrel{=}\;\Varid{\char95 ∼\char95 }{}\<[E]%
\\
\>[B]{}\hsindent{3}{}\<[3]%
\>[3]{};\Varid{isEquivalence}\;\mathrel{=}\;\Varid{\char95 ∼\char95 isEquivalence}{}\<[E]%
\\
\>[B]{}\hsindent{3}{}\<[3]%
\>[3]{}\mskip1.5mu\}{}\<[E]%
\ColumnHook
\end{hscode}\resethooks

\section{Rational numbers}

The quotient definition of rational number is more natural to
understand and the normalisation is also commonly used in regular
mathematics. We just use one integer and one natural number to
represent a rational number. The reason is because it is hard to
exclude the invalid denominator if we use integers, so I choose the
natural numbers to represent positive natural number which are one bigger.

\begin{hscode}\SaveRestoreHook
\column{B}{@{}>{\hspre}l<{\hspost}@{}}%
\column{3}{@{}>{\hspre}l<{\hspost}@{}}%
\column{E}{@{}>{\hspre}l<{\hspost}@{}}%
\>[B]{}\Keyword{data}\;\Conid{ℚ₀}\;\mathbin{:}\;\Conid{Set}\;\Keyword{where}{}\<[E]%
\\
\>[B]{}\hsindent{3}{}\<[3]%
\>[3]{}\Varid{\char95 /suc\char95 }\;\mathbin{:}\;(\Varid{n}\;\mathbin{:}\;\Conid{ℤ})\;\Varid{→}\;(\Varid{d}\;\mathbin{:}\;\Conid{ℕ})\;\Varid{→}\;\Conid{ℚ₀}{}\<[E]%
\ColumnHook
\end{hscode}\resethooks

and this is the equivalence relation for it

\begin{hscode}\SaveRestoreHook
\column{B}{@{}>{\hspre}l<{\hspost}@{}}%
\column{7}{@{}>{\hspre}l<{\hspost}@{}}%
\column{28}{@{}>{\hspre}l<{\hspost}@{}}%
\column{E}{@{}>{\hspre}l<{\hspost}@{}}%
\>[B]{}\Varid{\char95 ∼\char95 }\;{}\<[7]%
\>[7]{}\mathbin{:}\;\Conid{Rel}\;\Conid{ℚ₀}\;\Varid{zero}{}\<[E]%
\\
\>[B]{}\Varid{n1}\;\Varid{/suc}\;\Varid{d1}\;\Varid{∼}\;\Varid{n2}\;\Varid{/suc}\;\Varid{d2}\;\mathrel{=}\;{}\<[28]%
\>[28]{}\Varid{n1}\;\Conid{ℤ*ℕ}\;\Varid{suc}\;\Varid{d2}\;\Varid{≡}\;\Varid{n2}\;\Conid{ℤ*ℕ}\;\Varid{suc}\;\Varid{d1}{}\<[E]%
\ColumnHook
\end{hscode}\resethooks

\emph{Reflexivity}

\begin{hscode}\SaveRestoreHook
\column{B}{@{}>{\hspre}l<{\hspost}@{}}%
\column{E}{@{}>{\hspre}l<{\hspost}@{}}%
\>[B]{}\Varid{qrefl}\;\mathbin{:}\;\Conid{Reflexive}\;\Varid{\char95 ∼\char95 }{}\<[E]%
\\
\>[B]{}\Varid{qrefl}\;\{\mskip1.5mu \Varid{n}\;\Varid{/suc}\;\Varid{d}\mskip1.5mu\}\;\mathrel{=}\;\Varid{refl}{}\<[E]%
\ColumnHook
\end{hscode}\resethooks

\emph{symmetry}

\begin{hscode}\SaveRestoreHook
\column{B}{@{}>{\hspre}l<{\hspost}@{}}%
\column{E}{@{}>{\hspre}l<{\hspost}@{}}%
\>[B]{}\Varid{qsym}\;\mathbin{:}\;\Conid{Symmetric}\;\Varid{\char95 ∼\char95 }{}\<[E]%
\\
\>[B]{}\Varid{qsym}\;\{\mskip1.5mu \Varid{a}\;\Varid{/suc}\;\Varid{ad}\mskip1.5mu\}\;\{\mskip1.5mu \Varid{b}\;\Varid{/suc}\;\Varid{bd}\mskip1.5mu\}\;\mathrel{=}\;\Varid{sym}{}\<[E]%
\ColumnHook
\end{hscode}\resethooks

\emph{transitivity}

\begin{hscode}\SaveRestoreHook
\column{B}{@{}>{\hspre}l<{\hspost}@{}}%
\column{3}{@{}>{\hspre}l<{\hspost}@{}}%
\column{E}{@{}>{\hspre}l<{\hspost}@{}}%
\>[B]{}\Varid{qtrans}\;\mathbin{:}\;\Conid{Transitive}\;\Varid{\char95 ∼\char95 }{}\<[E]%
\\
\>[B]{}\Varid{qtrans}\;\{\mskip1.5mu \Varid{a}\;\Varid{/suc}\;\Varid{ad}\mskip1.5mu\}\;\{\mskip1.5mu \Varid{b}\;\Varid{/suc}\;\Varid{bd}\mskip1.5mu\}\;\{\mskip1.5mu \Varid{c}\;\Varid{/suc}\;\Varid{cd}\mskip1.5mu\}\;\Varid{a=b}\;\Varid{b=c}\;\Keyword{with}\;\Conid{ℤ.0?}\;\Varid{b}{}\<[E]%
\\
\>[B]{}\Varid{qtrans}\;\{\mskip1.5mu \Varid{a}\;\Varid{/suc}\;\Varid{ad}\mskip1.5mu\}\;\{\mskip1.5mu \mathbin{\circ}\;(\Varid{+}\;\Varid{0})\;\Varid{/suc}\;\Varid{bd}\mskip1.5mu\}\;\{\mskip1.5mu \Varid{c}\;\Varid{/suc}\;\Varid{cd}\mskip1.5mu\}\;\Varid{a=b}\;\Varid{b=c}\;\mid \;\Varid{yes}\;\Varid{refl}\;\mathrel{=}\;{}\<[E]%
\\
\>[B]{}\hsindent{3}{}\<[3]%
\>[3]{}\Conid{ℤ.solve0'}\;(\Varid{+}\;\Varid{suc}\;\Varid{bd})\;\{\mskip1.5mu \Varid{a}\mskip1.5mu\}\;(\Varid{λ}\;())\;\Varid{a=b}\;\Varid{0}\;\mathord{\sim}\;{}\<[E]%
\\
\>[B]{}\hsindent{3}{}\<[3]%
\>[3]{}\Conid{ℤ.solve0'}\;(\Varid{+}\;\Varid{suc}\;\Varid{bd})\;\{\mskip1.5mu \Varid{c}\mskip1.5mu\}\;(\Varid{λ}\;())\;\Varid{⟨}\;\Varid{b=c}\;\Varid{⟩}{}\<[E]%
\\
\>[B]{}\Varid{qtrans}\;\{\mskip1.5mu \Varid{a}\;\Varid{/suc}\;\Varid{ad}\mskip1.5mu\}\;\{\mskip1.5mu \Varid{b}\;\Varid{/suc}\;\Varid{bd}\mskip1.5mu\}\;\{\mskip1.5mu \Varid{c}\;\Varid{/suc}\;\Varid{cd}\mskip1.5mu\}\;\Varid{a=b}\;\Varid{b=c}\;\mid \;\Varid{no}\;\Varid{¬p}\;\mathrel{=}\;{}\<[E]%
\\
\>[B]{}\hsindent{3}{}\<[3]%
\>[3]{}\Conid{ℤ.l-integrity}\;(\Varid{b}\;\Conid{ℤ*}\;(\Varid{+}\;\Varid{suc}\;\Varid{bd}))\;(\Conid{ℤ.nz*}\;\Varid{b}\;(\Varid{+}\;\Varid{suc}\;\Varid{bd})\;\Varid{¬p}\;(\Varid{λ}\;()))\;({}\<[E]%
\\
\>[B]{}\hsindent{3}{}\<[3]%
\>[3]{}\Conid{ℤ.*-exchange₁}\;\Varid{b}\;(\Varid{+}\;\Varid{suc}\;\Varid{bd})\;\Varid{a}\;(\Varid{+}\;\Varid{suc}\;\Varid{cd})\;\Varid{>≡<}{}\<[E]%
\\
\>[B]{}\hsindent{3}{}\<[3]%
\>[3]{}(\Conid{ℤ.*-cong}\;\Varid{b=c}\;\Varid{a=b})\;\Varid{>≡<}{}\<[E]%
\\
\>[B]{}\hsindent{3}{}\<[3]%
\>[3]{}\Conid{ℤ.*-exchange₂}\;\Varid{c}\;(\Varid{+}\;\Varid{suc}\;\Varid{bd})\;\Varid{b}\;(\Varid{+}\;\Varid{suc}\;\Varid{ad})){}\<[E]%
\ColumnHook
\end{hscode}\resethooks


\emph{ ∼ isEquivalence relation}

\begin{hscode}\SaveRestoreHook
\column{B}{@{}>{\hspre}l<{\hspost}@{}}%
\column{3}{@{}>{\hspre}l<{\hspost}@{}}%
\column{10}{@{}>{\hspre}l<{\hspost}@{}}%
\column{E}{@{}>{\hspre}l<{\hspost}@{}}%
\>[B]{}\Varid{isEquivalenceℚ₀}\;\mathbin{:}\;\Conid{IsEquivalence}\;\Varid{\char95 ∼\char95 }{}\<[E]%
\\
\>[B]{}\Varid{isEquivalenceℚ₀}\;\mathrel{=}\;\Keyword{record}{}\<[E]%
\\
\>[B]{}\hsindent{3}{}\<[3]%
\>[3]{}\{\mskip1.5mu \Varid{refl}\;\mathrel{=}\;\Varid{qrefl}{}\<[E]%
\\
\>[B]{}\hsindent{3}{}\<[3]%
\>[3]{};\Varid{sym}\;{}\<[10]%
\>[10]{}\mathrel{=}\;\Varid{qsym}{}\<[E]%
\\
\>[B]{}\hsindent{3}{}\<[3]%
\>[3]{};\Varid{trans}\;\mathrel{=}\;\Varid{qtrans}{}\<[E]%
\\
\>[B]{}\hsindent{3}{}\<[3]%
\>[3]{}\mskip1.5mu\}{}\<[E]%
\ColumnHook
\end{hscode}\resethooks

Then it is natural to form the setoid

\emph{ (\ensuremath{\Conid{ℚ₀}}, ∼) is a setoid}

\begin{hscode}\SaveRestoreHook
\column{B}{@{}>{\hspre}l<{\hspost}@{}}%
\column{3}{@{}>{\hspre}l<{\hspost}@{}}%
\column{5}{@{}>{\hspre}l<{\hspost}@{}}%
\column{6}{@{}>{\hspre}l<{\hspost}@{}}%
\column{15}{@{}>{\hspre}l<{\hspost}@{}}%
\column{E}{@{}>{\hspre}l<{\hspost}@{}}%
\>[B]{}\Conid{ℚ₀setoid}\;\mathbin{:}\;\Conid{Setoid}\;\anonymous \;\anonymous {}\<[E]%
\\
\>[B]{}\Conid{ℚ₀setoid}\;\mathrel{=}\;\Keyword{record}\;\{\mskip1.5mu {}\<[E]%
\\
\>[B]{}\hsindent{5}{}\<[5]%
\>[5]{}\Conid{Carrier}\;{}\<[15]%
\>[15]{}\mathrel{=}\;\Conid{ℚ₀}{}\<[E]%
\\
\>[B]{}\hsindent{3}{}\<[3]%
\>[3]{};{}\<[6]%
\>[6]{}\Varid{\char95 ≈\char95 }\;{}\<[15]%
\>[15]{}\mathrel{=}\;\Varid{\char95 ∼\char95 }{}\<[E]%
\\
\>[B]{}\hsindent{3}{}\<[3]%
\>[3]{};\Varid{isEquivalence}\;\mathrel{=}\;\Varid{isEquivalenceℚ₀}{}\<[E]%
\\
\>[B]{}\hsindent{3}{}\<[3]%
\>[3]{}\mskip1.5mu\}{}\<[E]%
\ColumnHook
\end{hscode}\resethooks

However these definition are just setoid and to form a quotient type,
we need more structure. For definable quotient types, we need a representative of each
equivalence class, we may have a set which is isomorphic to the set of
equivalence classes, namely the normal form of the quotient type. Moreover, If we abstract the structure, we can
prove some general properites for definable quotient types.

\section{The general structure of definable quotient types}

I will use the interfaces written by Thomas Amberree in this part. We need
to first establish the quotient signature.

\begin{hscode}\SaveRestoreHook
\column{B}{@{}>{\hspre}l<{\hspost}@{}}%
\column{14}{@{}>{\hspre}l<{\hspost}@{}}%
\column{17}{@{}>{\hspre}l<{\hspost}@{}}%
\column{23}{@{}>{\hspre}l<{\hspost}@{}}%
\column{E}{@{}>{\hspre}l<{\hspost}@{}}%
\>[B]{}\Keyword{record}\;\Conid{QuSig}\;(\Conid{S}\;\mathbin{:}\;\Conid{Setoid}\;\Varid{zero}\;\Varid{zero})\;\mathbin{:}\;\Conid{Set₁}\;\Keyword{where}{}\<[E]%
\\
\>[B]{}\hsindent{14}{}\<[14]%
\>[14]{}\Keyword{field}{}\<[E]%
\\
\>[14]{}\hsindent{3}{}\<[17]%
\>[17]{}\Conid{Q}\;{}\<[23]%
\>[23]{}\mathbin{:}\;\Conid{Set}{}\<[E]%
\\
\>[14]{}\hsindent{3}{}\<[17]%
\>[17]{}[\mskip1.5mu \anonymous \mskip1.5mu]\;{}\<[23]%
\>[23]{}\mathbin{:}\;\Conid{Carrier}\;\Conid{S}\;\Varid{→}\;\Conid{Q}{}\<[E]%
\\
\>[14]{}\hsindent{3}{}\<[17]%
\>[17]{}\Varid{sound}\;\mathbin{:}\;\Varid{∀}\;\{\mskip1.5mu \Varid{a}\;\Varid{b}\;\mathbin{:}\;\Conid{Carrier}\;\Conid{S}\mskip1.5mu\}\;\Varid{→}\;(\Varid{\char95 ≈\char95 }\;\Conid{S}\;\Varid{a}\;\Varid{b})\;\Varid{→}\;[\mskip1.5mu \Varid{a}\mskip1.5mu]\;\Varid{≡}\;[\mskip1.5mu \Varid{b}\mskip1.5mu]{}\<[E]%
\ColumnHook
\end{hscode}\resethooks

In this type signature, for certain setoid we have a type represent
the set of the normal form, a normalisation function, and the proof
that two elements in the same equivalence class normalised to the
same form. With soundness, we can say normalisation is a function if we treat S as the set of equivalence classes.

However, there is no surjective requirements for the map in this signature. It
means that the set of equivalence classes are not isomorphic to the set \ensuremath{\Conid{Q}}.


Actually, we can use the same type for \ensuremath{\Conid{Carrier}\;\Conid{S}} and \ensuremath{\Conid{Q}}. For example,
for Setoid \ensuremath{\Conid{ℤ₀},\Varid{∼}}, we can build a quotient siganature by giving \ensuremath{\Conid{ℤ₀}}
and the endomap normalisation function.


Now, using the quotient signature if we can prove that any function of
type \ensuremath{\Conid{S}\;\Varid{→}\;\Conid{B}} respects the equivalence relation, then we can lift it to be
a function of type \ensuremath{\Conid{Q}\;\Varid{→}\;\Conid{B}}. Of course we need to prove that it is
lift function. With the lift function we have the first definition of quotient.

\begin{hscode}\SaveRestoreHook
\column{B}{@{}>{\hspre}l<{\hspost}@{}}%
\column{15}{@{}>{\hspre}l<{\hspost}@{}}%
\column{17}{@{}>{\hspre}l<{\hspost}@{}}%
\column{23}{@{}>{\hspre}l<{\hspost}@{}}%
\column{25}{@{}>{\hspre}l<{\hspost}@{}}%
\column{30}{@{}>{\hspre}l<{\hspost}@{}}%
\column{31}{@{}>{\hspre}l<{\hspost}@{}}%
\column{42}{@{}>{\hspre}l<{\hspost}@{}}%
\column{64}{@{}>{\hspre}l<{\hspost}@{}}%
\column{75}{@{}>{\hspre}l<{\hspost}@{}}%
\column{82}{@{}>{\hspre}l<{\hspost}@{}}%
\column{85}{@{}>{\hspre}l<{\hspost}@{}}%
\column{E}{@{}>{\hspre}l<{\hspost}@{}}%
\>[B]{}\Keyword{record}\;\Conid{Qu}\;\{\mskip1.5mu \Conid{S}\;\mathbin{:}\;\Conid{Setoid}\;\Varid{zero}\;\Varid{zero}\mskip1.5mu\}\;(\Conid{QS}\;\mathbin{:}\;\Conid{QuSig}\;\Conid{S})\;\mathbin{:}\;\Conid{Set₁}\;\Keyword{where}{}\<[E]%
\\
\>[B]{}\hsindent{15}{}\<[15]%
\>[15]{}\Keyword{private}\;\Conid{S}\;{}\<[30]%
\>[30]{}\mathrel{=}\;\Conid{Carrier}\;\Conid{S}{}\<[E]%
\\
\>[15]{}\hsindent{8}{}\<[23]%
\>[23]{}\Varid{\char95 ∼\char95 }\;{}\<[31]%
\>[31]{}\mathrel{=}\;\Varid{\char95 ≈\char95 }\;\Conid{S}{}\<[E]%
\\
\>[15]{}\hsindent{8}{}\<[23]%
\>[23]{}\Conid{Q}\;{}\<[30]%
\>[30]{}\mathrel{=}\;\Conid{Q}\;\Conid{QS}{}\<[E]%
\\
\>[15]{}\hsindent{8}{}\<[23]%
\>[23]{}[\mskip1.5mu \anonymous \mskip1.5mu]\;{}\<[30]%
\>[30]{}\mathrel{=}\;[\mskip1.5mu \anonymous \mskip1.5mu]\;\Conid{QS}{}\<[E]%
\\
\>[15]{}\hsindent{8}{}\<[23]%
\>[23]{}\Varid{sound}\;{}\<[30]%
\>[30]{}\mathbin{:}\;\Varid{∀}\;\{\mskip1.5mu \Varid{a}\;\Varid{b}\;\mathbin{:}\;\Conid{S₀}\mskip1.5mu\}\;\Varid{→}\;(\Varid{a}\;\Varid{∼}\;\Varid{b})\;\Varid{→}\;[\mskip1.5mu \Varid{a}\mskip1.5mu]\;\Varid{≡}\;[\mskip1.5mu \Varid{b}\mskip1.5mu]{}\<[E]%
\\
\>[15]{}\hsindent{8}{}\<[23]%
\>[23]{}\Varid{sound}\;{}\<[30]%
\>[30]{}\mathrel{=}\;\Varid{sound}\;\Conid{QS}{}\<[E]%
\\
\>[B]{}\hsindent{15}{}\<[15]%
\>[15]{}\Keyword{field}{}\<[E]%
\\
\>[15]{}\hsindent{2}{}\<[17]%
\>[17]{}\Varid{lift}\;{}\<[25]%
\>[25]{}\mathbin{:}\;\{\mskip1.5mu \Conid{B}\;\mathbin{:}\;\Conid{Q}\;\Varid{→}\;\Conid{Set}\mskip1.5mu\}\;{}\<[E]%
\\
\>[25]{}\Varid{→}\;(\Varid{f}\;\mathbin{:}\;(\Varid{a}\;\mathbin{:}\;\Conid{S})\;\Varid{→}\;(\Conid{B}\;[\mskip1.5mu \Varid{a}\mskip1.5mu]))\;{}\<[E]%
\\
\>[25]{}\Varid{→}\;((\Varid{a}\;\Varid{a'}\;\mathbin{:}\;\Conid{S})\;\Varid{→}\;(\Varid{p}\;\mathbin{:}\;\Varid{a}\;\Varid{∼}\;\Varid{a'})\;\Varid{→}\;\Varid{subst}\;\Conid{B}\;(\Varid{sound}\;\Varid{p})\;{}\<[75]%
\>[75]{}(\Varid{f}\;\Varid{a})\;{}\<[82]%
\>[82]{}\Varid{≡}\;{}\<[85]%
\>[85]{}\Varid{f}\;\Varid{a'})\;{}\<[E]%
\\
\>[25]{}\Varid{→}\;(\Varid{c}\;\mathbin{:}\;\Conid{Q})\;\Varid{→}\;\Conid{B}\;\Varid{c}{}\<[E]%
\\
\>[15]{}\hsindent{2}{}\<[17]%
\>[17]{}\Varid{liftok}\;{}\<[25]%
\>[25]{}\mathbin{:}\;\Varid{∀}\;\{\mskip1.5mu \Conid{B}\;\Varid{a}\;\Varid{f}\;\Varid{q}\mskip1.5mu\}\;{}\<[42]%
\>[42]{}\Varid{→}\;\Varid{lift}\;\{\mskip1.5mu \Conid{B}\mskip1.5mu\}\;\Varid{f}\;\Varid{q}\;[\mskip1.5mu \Varid{a}\mskip1.5mu]\;{}\<[64]%
\>[64]{}\Varid{≡}\;\Varid{f}\;\Varid{a}{}\<[E]%
\\
\>[15]{}\hsindent{2}{}\<[17]%
\>[17]{}\Varid{liftIrr}\;\mathbin{:}\;\Varid{∀}\;\{\mskip1.5mu \Conid{B}\;\Varid{a}\;\Varid{f}\;\Varid{q}\;\Varid{q'}\mskip1.5mu\}\;\Varid{→}\;\Varid{lift}\;\{\mskip1.5mu \Conid{B}\mskip1.5mu\}\;\Varid{f}\;\Varid{q}\;[\mskip1.5mu \Varid{a}\mskip1.5mu]\;{}\<[64]%
\>[64]{}\Varid{≡}\;\Varid{lift}{}\<[E]%
\\
\>[15]{}\hsindent{2}{}\<[17]%
\>[17]{}\{\mskip1.5mu \Conid{B}\mskip1.5mu\}\;\Varid{f}\;\Varid{q'}\;[\mskip1.5mu \Varid{a}\mskip1.5mu]{}\<[E]%
\ColumnHook
\end{hscode}\resethooks

In my opinion the proof irrelevance of lift operations are
unecessary since \ensuremath{\Varid{lifeok}} implies it.


However, there can be more than one equivalence classes normalised to the same form. Therefore the normal form do
not fully contain the information of quotient type. If we can prove the completeness \nzl{I don't think this is completeness, it is only injective},
 namely two elements normalised to the same form must be in the same equivalence class. they belong to the same
equivalence class. Hence the normalisation is injective from the set of equivalence classes to the set of normal forms.

\begin{hscode}\SaveRestoreHook
\column{B}{@{}>{\hspre}l<{\hspost}@{}}%
\column{8}{@{}>{\hspre}l<{\hspost}@{}}%
\column{10}{@{}>{\hspre}l<{\hspost}@{}}%
\column{16}{@{}>{\hspre}l<{\hspost}@{}}%
\column{23}{@{}>{\hspre}l<{\hspost}@{}}%
\column{24}{@{}>{\hspre}l<{\hspost}@{}}%
\column{E}{@{}>{\hspre}l<{\hspost}@{}}%
\>[B]{}\Keyword{record}\;\Conid{QuE}\;\{\mskip1.5mu \Conid{S}\;\mathbin{:}\;\Conid{Setoid}\;\Varid{zero}\;\Varid{zero}\mskip1.5mu\}\;\{\mskip1.5mu \Conid{QS}\;\mathbin{:}\;\Conid{QuSig}\;\Conid{S}\mskip1.5mu\}\;(\Conid{QU}\;\mathbin{:}\;\Conid{Qu}\;\Conid{QS})\;\mathbin{:}\;\Conid{Set₁}\;\Keyword{where}{}\<[E]%
\\
\>[B]{}\hsindent{8}{}\<[8]%
\>[8]{}\Keyword{private}\;\Conid{S}\;{}\<[23]%
\>[23]{}\mathrel{=}\;\Conid{Carrier}\;\Conid{S}{}\<[E]%
\\
\>[8]{}\hsindent{8}{}\<[16]%
\>[16]{}\Varid{\char95 ∼\char95 }\;{}\<[24]%
\>[24]{}\mathrel{=}\;\Varid{\char95 ≈\char95 }\;\Conid{S}{}\<[E]%
\\
\>[8]{}\hsindent{8}{}\<[16]%
\>[16]{}\Conid{Q}\;{}\<[23]%
\>[23]{}\mathrel{=}\;\Conid{Q}\;\Conid{QS}{}\<[E]%
\\
\>[8]{}\hsindent{8}{}\<[16]%
\>[16]{}[\mskip1.5mu \anonymous \mskip1.5mu]\;{}\<[23]%
\>[23]{}\mathrel{=}\;[\mskip1.5mu \anonymous \mskip1.5mu]\;\Conid{QS}{}\<[E]%
\\
\>[8]{}\hsindent{8}{}\<[16]%
\>[16]{}\Varid{sound}\;{}\<[23]%
\>[23]{}\mathbin{:}\;\Varid{∀}\;\{\mskip1.5mu \Varid{a}\;\Varid{b}\;\mathbin{:}\;\Conid{S}\mskip1.5mu\}\;\Varid{→}\;(\Varid{a}\;\Varid{∼}\;\Varid{b})\;\Varid{→}\;[\mskip1.5mu \Varid{a}\mskip1.5mu]\;\Varid{≡}\;[\mskip1.5mu \Varid{b}\mskip1.5mu]{}\<[E]%
\\
\>[8]{}\hsindent{8}{}\<[16]%
\>[16]{}\Varid{sound}\;{}\<[23]%
\>[23]{}\mathrel{=}\;\Varid{sound}\;\Conid{QS}{}\<[E]%
\\
\>[B]{}\hsindent{8}{}\<[8]%
\>[8]{}\Keyword{field}{}\<[E]%
\\
\>[8]{}\hsindent{2}{}\<[10]%
\>[10]{}\Varid{complete}\;\mathbin{:}\;\Varid{∀}\;\{\mskip1.5mu \Varid{a}\;\Varid{b}\;\mathbin{:}\;\Conid{S}\mskip1.5mu\}\;\Varid{→}\;[\mskip1.5mu \Varid{a}\mskip1.5mu]\;\Varid{≡}\;[\mskip1.5mu \Varid{b}\mskip1.5mu]\;\Varid{→}\;\Varid{a}\;\Varid{∼}\;\Varid{b}{}\<[E]%
\ColumnHook
\end{hscode}\resethooks

Even if we prove the normalisation to be injective, we still not require it to be surjective. Then \ensuremath{\Conid{Q}} may have some redundance. Therefore
we need more efficient quotient type. 

In \ensuremath{\Conid{Nf}} we have a embedding function used to choose a representative for
each equivalence class. The proof of stability shows that \ensuremath{\Varid{emb}} is a section of
normalisation function. Since all elements in \ensuremath{\Conid{Q}} can be the result of the normalisation, it must be surjective. The proof \ensuremath{\Varid{compl}} shows that the
representative is in the same equivalence class hence we can prove the
completeness as well. In this definition of quotient, the
set of all equivalence classes are in fact isomorphic to the set \ensuremath{\Conid{Q}}.

\begin{hscode}\SaveRestoreHook
\column{B}{@{}>{\hspre}l<{\hspost}@{}}%
\column{8}{@{}>{\hspre}l<{\hspost}@{}}%
\column{10}{@{}>{\hspre}l<{\hspost}@{}}%
\column{16}{@{}>{\hspre}l<{\hspost}@{}}%
\column{17}{@{}>{\hspre}l<{\hspost}@{}}%
\column{19}{@{}>{\hspre}l<{\hspost}@{}}%
\column{22}{@{}>{\hspre}l<{\hspost}@{}}%
\column{23}{@{}>{\hspre}l<{\hspost}@{}}%
\column{36}{@{}>{\hspre}l<{\hspost}@{}}%
\column{E}{@{}>{\hspre}l<{\hspost}@{}}%
\>[B]{}\Keyword{record}\;\Conid{Nf}\;\{\mskip1.5mu \Conid{S}\;\mathbin{:}\;\Conid{Setoid}\;\Varid{zero}\;\Varid{zero}\mskip1.5mu\}\;(\Conid{QS}\;\mathbin{:}\;\Conid{QuSig}\;\Conid{S})\;\mathbin{:}\;\Conid{Set₁}\;\Keyword{where}{}\<[E]%
\\
\>[B]{}\hsindent{8}{}\<[8]%
\>[8]{}\Keyword{private}\;\Conid{S}\;{}\<[19]%
\>[19]{}\mathrel{=}\;\Conid{Carrier}\;\Conid{S}{}\<[E]%
\\
\>[8]{}\hsindent{8}{}\<[16]%
\>[16]{}\Varid{\char95 ∼\char95 }\;{}\<[22]%
\>[22]{}\mathrel{=}\;\Varid{\char95 ≈\char95 }\;\Conid{S}{}\<[E]%
\\
\>[8]{}\hsindent{8}{}\<[16]%
\>[16]{}\Conid{Q}\;{}\<[22]%
\>[22]{}\mathrel{=}\;\Conid{Q}\;\Conid{QS}{}\<[E]%
\\
\>[8]{}\hsindent{8}{}\<[16]%
\>[16]{}[\mskip1.5mu \anonymous \mskip1.5mu]\;{}\<[23]%
\>[23]{}\mathrel{=}\;[\mskip1.5mu \anonymous \mskip1.5mu]\;\Conid{QS}{}\<[E]%
\\
\>[B]{}\hsindent{8}{}\<[8]%
\>[8]{}\Keyword{field}{}\<[E]%
\\
\>[8]{}\hsindent{2}{}\<[10]%
\>[10]{}\Varid{emb}\;{}\<[17]%
\>[17]{}\mathbin{:}\;\Conid{Q}\;\Varid{→}\;\Conid{S}{}\<[E]%
\\
\>[8]{}\hsindent{2}{}\<[10]%
\>[10]{}\Varid{compl}\;{}\<[17]%
\>[17]{}\mathbin{:}\;\Varid{∀}\;\Varid{a}\;\Varid{→}\;\Varid{emb}\;[\mskip1.5mu \Varid{a}\mskip1.5mu]\;\Varid{∼}\;\Varid{a}{}\<[E]%
\\
\>[8]{}\hsindent{2}{}\<[10]%
\>[10]{}\Varid{stable}\;\mathbin{:}\;\Varid{∀}\;\Varid{x}\;\Varid{→}\;[\mskip1.5mu \Varid{emb}\;\Varid{x}\mskip1.5mu]\;{}\<[36]%
\>[36]{}\Varid{≡}\;\Varid{x}{}\<[E]%
\ColumnHook
\end{hscode}\resethooks

We can easily establish the function transforming the \ensuremath{\Conid{Nf}} to \ensuremath{\Conid{QuE}}, since completeness can be derived from compl.

\begin{hscode}\SaveRestoreHook
\column{B}{@{}>{\hspre}l<{\hspost}@{}}%
\column{5}{@{}>{\hspre}l<{\hspost}@{}}%
\column{7}{@{}>{\hspre}l<{\hspost}@{}}%
\column{19}{@{}>{\hspre}l<{\hspost}@{}}%
\column{21}{@{}>{\hspre}l<{\hspost}@{}}%
\column{27}{@{}>{\hspre}l<{\hspost}@{}}%
\column{35}{@{}>{\hspre}l<{\hspost}@{}}%
\column{42}{@{}>{\hspre}l<{\hspost}@{}}%
\column{43}{@{}>{\hspre}l<{\hspost}@{}}%
\column{51}{@{}>{\hspre}l<{\hspost}@{}}%
\column{E}{@{}>{\hspre}l<{\hspost}@{}}%
\>[B]{}\Varid{nf2quE}\;\mathbin{:}\;\{\mskip1.5mu \Conid{S}\;\mathbin{:}\;\Conid{Setoid}\;\Varid{zero}\;\Varid{zero}\mskip1.5mu\}\;\Varid{→}\;\{\mskip1.5mu \Conid{QS}\;\mathbin{:}\;\Conid{QuSig}\;\Conid{S}\mskip1.5mu\}\;\Varid{→}\;\{\mskip1.5mu \Conid{QU}\;\mathbin{:}\;\Conid{Qu}\;\Conid{QS}\mskip1.5mu\}\;\Varid{→}\;(\Conid{Nf}\;\Conid{QS})\;\Varid{→}\;(\Conid{QuE}\;\Conid{QU}){}\<[E]%
\\
\>[B]{}\Varid{nf2quE}\;\{\mskip1.5mu \Conid{S}\mskip1.5mu\}\;\{\mskip1.5mu \Conid{QS}\mskip1.5mu\}\;\{\mskip1.5mu \Conid{QU}\mskip1.5mu\}\;\Varid{nf}\;\mathrel{=}{}\<[E]%
\\
\>[B]{}\hsindent{5}{}\<[5]%
\>[5]{}\Keyword{record}\;\{\mskip1.5mu {}\<[E]%
\\
\>[5]{}\hsindent{2}{}\<[7]%
\>[7]{}\Varid{complete}\;\mathrel{=}\;{}\<[19]%
\>[19]{}\Varid{λ}\;\{\mskip1.5mu \Varid{a}\mskip1.5mu\}\;\{\mskip1.5mu \Varid{b}\mskip1.5mu\}\;[\mskip1.5mu \Varid{a}\mskip1.5mu]\;\Varid{≡}\;[\mskip1.5mu \Varid{b}\mskip1.5mu]\;\Varid{→}\;{}\<[E]%
\\
\>[19]{}\hsindent{2}{}\<[21]%
\>[21]{}\Varid{⟨}\;\Varid{compl}\;\Varid{a}\;\Varid{⟩}\;\Varid{▶}\;\Varid{subst}\;(\Varid{λ}\;\Varid{x}\;\Varid{→}\;\Varid{x}\;\Varid{∼}\;\Varid{b})\;(\Varid{emb}\;\Varid{⋆}\;\Varid{⟨}\;[\mskip1.5mu \Varid{a}\mskip1.5mu]\;\Varid{≡}\;[\mskip1.5mu \Varid{b}\mskip1.5mu]\;\Varid{⟩})\;(\Varid{compl}\;\Varid{b}){}\<[E]%
\\
\>[5]{}\hsindent{2}{}\<[7]%
\>[7]{}\mskip1.5mu\}{}\<[E]%
\\
\>[7]{}\hsindent{20}{}\<[27]%
\>[27]{}\Keyword{where}{}\<[E]%
\\
\>[7]{}\hsindent{20}{}\<[27]%
\>[27]{}\Keyword{private}\;\Conid{S}\;{}\<[42]%
\>[42]{}\mathrel{=}\;\Conid{Carrier}\;\Conid{S}{}\<[E]%
\\
\>[27]{}\hsindent{8}{}\<[35]%
\>[35]{}\Varid{\char95 ∼\char95 }\;{}\<[43]%
\>[43]{}\mathrel{=}\;\Varid{\char95 ≈\char95 }\;\Conid{S}{}\<[E]%
\\
\>[27]{}\hsindent{8}{}\<[35]%
\>[35]{}\Conid{Q}\;{}\<[42]%
\>[42]{}\mathrel{=}\;\Conid{Q}\;\Conid{QS}{}\<[E]%
\\
\>[27]{}\hsindent{8}{}\<[35]%
\>[35]{}[\mskip1.5mu \anonymous \mskip1.5mu]\;{}\<[42]%
\>[42]{}\mathrel{=}\;[\mskip1.5mu \anonymous \mskip1.5mu]\;\Conid{QS}{}\<[E]%
\\
\>[27]{}\hsindent{8}{}\<[35]%
\>[35]{}\Varid{emb}\;{}\<[42]%
\>[42]{}\mathrel{=}\;\Varid{emb}\;\Varid{nf}{}\<[E]%
\\
\>[27]{}\hsindent{8}{}\<[35]%
\>[35]{}\Varid{compl}\;{}\<[42]%
\>[42]{}\mathrel{=}\;\Varid{compl}\;{}\<[51]%
\>[51]{}\Varid{nf}{}\<[E]%
\\
\>[27]{}\hsindent{8}{}\<[35]%
\>[35]{}\Varid{⟨\char95 ⟩}\;{}\<[42]%
\>[42]{}\mathbin{:}\;\Conid{Symmetric}\;\Varid{\char95 ∼\char95 }{}\<[E]%
\\
\>[27]{}\hsindent{8}{}\<[35]%
\>[35]{}\Varid{⟨\char95 ⟩}\;{}\<[42]%
\>[42]{}\mathrel{=}\;\Varid{symmetric}\;\Conid{S}{}\<[E]%
\\
\>[27]{}\hsindent{8}{}\<[35]%
\>[35]{}\Varid{\char95 ▶\char95 }\;{}\<[42]%
\>[42]{}\mathbin{:}\;\Conid{Transitive}\;\Varid{\char95 ∼\char95 }{}\<[E]%
\\
\>[27]{}\hsindent{8}{}\<[35]%
\>[35]{}\Varid{\char95 ▶\char95 }\;{}\<[42]%
\>[42]{}\mathrel{=}\;\Varid{transitive}\;\Conid{S}{}\<[E]%
\ColumnHook
\end{hscode}\resethooks

We can also define non-dependent lift version of quotients. We need to prove quotient
induction when we have uniqueness of proof for certain proposition
dependent on \ensuremath{\Conid{Q}}.

\nzl{Why we need qind?}

\begin{hscode}\SaveRestoreHook
\column{B}{@{}>{\hspre}l<{\hspost}@{}}%
\column{15}{@{}>{\hspre}l<{\hspost}@{}}%
\column{17}{@{}>{\hspre}l<{\hspost}@{}}%
\column{23}{@{}>{\hspre}l<{\hspost}@{}}%
\column{24}{@{}>{\hspre}l<{\hspost}@{}}%
\column{25}{@{}>{\hspre}l<{\hspost}@{}}%
\column{26}{@{}>{\hspre}l<{\hspost}@{}}%
\column{27}{@{}>{\hspre}l<{\hspost}@{}}%
\column{30}{@{}>{\hspre}l<{\hspost}@{}}%
\column{31}{@{}>{\hspre}l<{\hspost}@{}}%
\column{43}{@{}>{\hspre}l<{\hspost}@{}}%
\column{53}{@{}>{\hspre}l<{\hspost}@{}}%
\column{60}{@{}>{\hspre}l<{\hspost}@{}}%
\column{63}{@{}>{\hspre}l<{\hspost}@{}}%
\column{66}{@{}>{\hspre}l<{\hspost}@{}}%
\column{E}{@{}>{\hspre}l<{\hspost}@{}}%
\>[B]{}\Keyword{record}\;\Conid{QuH}\;\{\mskip1.5mu \Conid{S}\;\mathbin{:}\;\Conid{Setoid}\;\Varid{zero}\;\Varid{zero}\mskip1.5mu\}\;(\Conid{QS}\;\mathbin{:}\;\Conid{QuSig}\;\Conid{S})\;\mathbin{:}\;\Conid{Set₁}\;\Keyword{where}{}\<[E]%
\\
\>[B]{}\hsindent{15}{}\<[15]%
\>[15]{}\Keyword{private}\;\Conid{S}\;{}\<[27]%
\>[27]{}\mathrel{=}\;\Conid{Carrier}\;\Conid{S}{}\<[E]%
\\
\>[15]{}\hsindent{8}{}\<[23]%
\>[23]{}\Varid{\char95 ∼\char95 }\;{}\<[30]%
\>[30]{}\mathrel{=}\;\Varid{\char95 ≈\char95 }\;\Conid{S}{}\<[E]%
\\
\>[15]{}\hsindent{8}{}\<[23]%
\>[23]{}\Conid{Q}\;{}\<[30]%
\>[30]{}\mathrel{=}\;\Conid{Q}\;\Conid{QS}{}\<[E]%
\\
\>[15]{}\hsindent{8}{}\<[23]%
\>[23]{}[\mskip1.5mu \anonymous \mskip1.5mu]\;{}\<[31]%
\>[31]{}\mathrel{=}\;[\mskip1.5mu \anonymous \mskip1.5mu]\;\Conid{QS}{}\<[E]%
\\
\>[15]{}\hsindent{8}{}\<[23]%
\>[23]{}\Varid{sound}\;{}\<[30]%
\>[30]{}\mathbin{:}\;\Varid{∀}\;\{\mskip1.5mu \Varid{a}\;\Varid{b}\;\mathbin{:}\;\Conid{S}\mskip1.5mu\}\;\Varid{→}\;(\Varid{a}\;\Varid{∼}\;\Varid{b})\;\Varid{→}\;[\mskip1.5mu \Varid{a}\mskip1.5mu]\;\Varid{≡}\;[\mskip1.5mu \Varid{b}\mskip1.5mu]{}\<[E]%
\\
\>[15]{}\hsindent{8}{}\<[23]%
\>[23]{}\Varid{sound}\;{}\<[30]%
\>[30]{}\mathrel{=}\;\Varid{sound}\;\Conid{QS}{}\<[E]%
\\
\>[B]{}\hsindent{15}{}\<[15]%
\>[15]{}\Keyword{field}{}\<[E]%
\\
\>[15]{}\hsindent{2}{}\<[17]%
\>[17]{}\Varid{liftH}\;{}\<[26]%
\>[26]{}\mathbin{:}\;\{\mskip1.5mu \Conid{B}\;\mathbin{:}\;\Conid{Set}\mskip1.5mu\}\;{}\<[E]%
\\
\>[17]{}\hsindent{8}{}\<[25]%
\>[25]{}\Varid{→}\;(\Varid{f}\;\mathbin{:}\;\Conid{S}\;\Varid{→}\;\Conid{B})\;{}\<[E]%
\\
\>[17]{}\hsindent{8}{}\<[25]%
\>[25]{}\Varid{→}\;((\Varid{a}\;\Varid{a'}\;\mathbin{:}\;\Conid{S})\;\Varid{→}\;(\Varid{a}\;\Varid{∼}\;\Varid{a'})\;\Varid{→}\;{}\<[53]%
\>[53]{}(\Varid{f}\;\Varid{a})\;{}\<[60]%
\>[60]{}\Varid{≡}\;{}\<[63]%
\>[63]{}\Varid{f}\;\Varid{a'})\;{}\<[E]%
\\
\>[17]{}\hsindent{8}{}\<[25]%
\>[25]{}\Varid{→}\;\Conid{Q}\;\Varid{→}\;\Conid{B}{}\<[E]%
\\
\>[15]{}\hsindent{2}{}\<[17]%
\>[17]{}\Varid{liftHok}\;{}\<[26]%
\>[26]{}\mathbin{:}\;\Varid{∀}\;\{\mskip1.5mu \Conid{B}\;\Varid{a}\;\Varid{f}\;\Varid{q}\mskip1.5mu\}\;{}\<[43]%
\>[43]{}\Varid{→}\;\Varid{liftH}\;\{\mskip1.5mu \Conid{B}\mskip1.5mu\}\;\Varid{f}\;\Varid{q}\;[\mskip1.5mu \Varid{a}\mskip1.5mu]\;{}\<[66]%
\>[66]{}\Varid{≡}\;\Varid{f}\;\Varid{a}{}\<[E]%
\\
\>[15]{}\hsindent{2}{}\<[17]%
\>[17]{}\mbox{\onelinecomment  quotient induction}{}\<[E]%
\\
\>[15]{}\hsindent{2}{}\<[17]%
\>[17]{}\Varid{qind}\;\mathbin{:}\;(\Conid{P}\;\mathbin{:}\;\Conid{Q}\;\Varid{→}\;\Conid{Set})\;{}\<[E]%
\\
\>[17]{}\hsindent{7}{}\<[24]%
\>[24]{}\Varid{→}\;(\Varid{∀}\;\{\mskip1.5mu \Varid{x}\mskip1.5mu\}\;\Varid{→}\;(\Varid{p}\;\Varid{p'}\;\mathbin{:}\;\Conid{P}\;\Varid{x})\;\Varid{→}\;\Varid{p}\;\Varid{≡}\;\Varid{p'})\;{}\<[E]%
\\
\>[17]{}\hsindent{7}{}\<[24]%
\>[24]{}\Varid{→}\;(\Varid{∀}\;\{\mskip1.5mu \Varid{a}\mskip1.5mu\}\;\Varid{→}\;\Conid{P}\;[\mskip1.5mu \Varid{a}\mskip1.5mu])\;{}\<[E]%
\\
\>[17]{}\hsindent{7}{}\<[24]%
\>[24]{}\Varid{→}\;(\Varid{∀}\;\{\mskip1.5mu \Varid{x}\mskip1.5mu\}\;\Varid{→}\;\Conid{P}\;\Varid{x}){}\<[E]%
\ColumnHook
\end{hscode}\resethooks

If we have the normal form definition, we can lift the function easily.


\begin{hscode}\SaveRestoreHook
\column{B}{@{}>{\hspre}l<{\hspost}@{}}%
\column{9}{@{}>{\hspre}l<{\hspost}@{}}%
\column{17}{@{}>{\hspre}l<{\hspost}@{}}%
\column{20}{@{}>{\hspre}l<{\hspost}@{}}%
\column{24}{@{}>{\hspre}l<{\hspost}@{}}%
\column{25}{@{}>{\hspre}l<{\hspost}@{}}%
\column{33}{@{}>{\hspre}l<{\hspost}@{}}%
\column{54}{@{}>{\hspre}l<{\hspost}@{}}%
\column{E}{@{}>{\hspre}l<{\hspost}@{}}%
\>[B]{}\Varid{nf2qu}\;\mathbin{:}\;\{\mskip1.5mu \Conid{S}\;\mathbin{:}\;\Conid{Setoid}\;\Varid{zero}\;\Varid{zero}\mskip1.5mu\}\;\Varid{→}\;\{\mskip1.5mu \Conid{QS}\;\mathbin{:}\;\Conid{QuSig}\;\Conid{S}\mskip1.5mu\}\;\Varid{→}\;(\Conid{Nf}\;\Conid{QS})\;\Varid{→}\;(\Conid{Qu}\;\Conid{QS}){}\<[E]%
\\
\>[B]{}\Varid{nf2qu}\;\{\mskip1.5mu \Conid{S}\mskip1.5mu\}\;\{\mskip1.5mu \Conid{QS}\mskip1.5mu\}\;\Varid{nf}\;\mathrel{=}{}\<[E]%
\\
\>[B]{}\hsindent{9}{}\<[9]%
\>[9]{}\Keyword{record}\;\{\mskip1.5mu {}\<[E]%
\\
\>[B]{}\hsindent{9}{}\<[9]%
\>[9]{}\Varid{lift}\;{}\<[17]%
\>[17]{}\mathrel{=}\;{}\<[20]%
\>[20]{}\Varid{λ}\;\{\mskip1.5mu \Conid{B}\mskip1.5mu\}\;\Varid{f}\;\Varid{q}\;\Varid{a⁻}\;\Varid{→}\;\Varid{subst}\;\Conid{B}\;(\Varid{stable₀}\;\Varid{a⁻})\;(\Varid{f}\;(\Varid{emb₀}\;\Varid{a⁻}));{}\<[E]%
\\
\>[B]{}\hsindent{9}{}\<[9]%
\>[9]{}\Varid{liftok}\;{}\<[17]%
\>[17]{}\mathrel{=}\;\Varid{λ}\;\{\mskip1.5mu \Conid{B}\mskip1.5mu\}\;\{\mskip1.5mu \Varid{a}\mskip1.5mu\}\;\{\mskip1.5mu \Varid{f}\mskip1.5mu\}\;\{\mskip1.5mu \Varid{q}\mskip1.5mu\}\;\Varid{→}{}\<[E]%
\\
\>[B]{}\hsindent{9}{}\<[9]%
\>[9]{}\Varid{substIrr}\;\Conid{B}\;(\Varid{stable}\;[\mskip1.5mu \Varid{a}\mskip1.5mu])\;(\Varid{sound}\;(\Varid{compl}\;\Varid{a}))\;{}\<[54]%
\>[54]{}(\Varid{f}\;(\Varid{emb}\;[\mskip1.5mu \Varid{a}\mskip1.5mu]))\;\Varid{▶}\;\Varid{q}\;\anonymous \;\anonymous \;(\Varid{compl}\;\Varid{a});{}\<[E]%
\\
\>[B]{}\hsindent{9}{}\<[9]%
\>[9]{}\Varid{liftIrr}\;\mathrel{=}\;\Varid{refl}{}\<[E]%
\\
\>[B]{}\hsindent{9}{}\<[9]%
\>[9]{}\mskip1.5mu\}{}\<[E]%
\\
\>[B]{}\hsindent{9}{}\<[9]%
\>[9]{}\Keyword{where}\;{}\<[17]%
\>[17]{}\Conid{S}\;{}\<[24]%
\>[24]{}\mathrel{=}\;\Conid{Carrier}\;\Conid{S}{}\<[E]%
\\
\>[17]{}\Varid{\char95 ∼\char95 }\;{}\<[25]%
\>[25]{}\mathrel{=}\;\Varid{\char95 ≈\char95 }\;\Conid{S}{}\<[E]%
\\
\>[17]{}[\mskip1.5mu \anonymous \mskip1.5mu]\;{}\<[24]%
\>[24]{}\mathrel{=}\;[\mskip1.5mu \anonymous \mskip1.5mu]\;\Conid{QS}{}\<[E]%
\\
\>[17]{}\Varid{sound}\;{}\<[24]%
\>[24]{}\mathbin{:}\;\Varid{∀}\;\{\mskip1.5mu \Varid{a}\;\Varid{b}\;\mathbin{:}\;\Conid{S}\mskip1.5mu\}\;\Varid{→}\;\Varid{a}\;\Varid{∼}\;\Varid{b}\;\Varid{→}\;[\mskip1.5mu \Varid{a}\mskip1.5mu]\;\Varid{≡}\;[\mskip1.5mu \Varid{b}\mskip1.5mu]{}\<[E]%
\\
\>[17]{}\Varid{sound}\;{}\<[24]%
\>[24]{}\mathrel{=}\;\Varid{sound}\;{}\<[33]%
\>[33]{}\Conid{QS}{}\<[E]%
\\
\>[17]{}\Varid{compl}\;{}\<[24]%
\>[24]{}\mathrel{=}\;\Varid{compl}\;{}\<[33]%
\>[33]{}\Varid{nf}{}\<[E]%
\\
\>[17]{}\Varid{stable}\;\mathrel{=}\;\Varid{stable}\;\Varid{nf}{}\<[E]%
\\
\>[17]{}\Varid{emb}\;{}\<[24]%
\>[24]{}\mathrel{=}\;\Varid{emb}\;{}\<[33]%
\>[33]{}\Varid{nf}{}\<[E]%
\ColumnHook
\end{hscode}\resethooks

\section{The properties of definable quotient types}

Not only the predicate can be lifted, but also the operators can be
lifted.

\begin{hscode}\SaveRestoreHook
\column{B}{@{}>{\hspre}l<{\hspost}@{}}%
\column{3}{@{}>{\hspre}l<{\hspost}@{}}%
\column{5}{@{}>{\hspre}l<{\hspost}@{}}%
\column{6}{@{}>{\hspre}l<{\hspost}@{}}%
\column{11}{@{}>{\hspre}l<{\hspost}@{}}%
\column{12}{@{}>{\hspre}l<{\hspost}@{}}%
\column{13}{@{}>{\hspre}l<{\hspost}@{}}%
\column{E}{@{}>{\hspre}l<{\hspost}@{}}%
\>[B]{}\Conid{Op}\;\mathbin{:}\;\Conid{ℕ}\;\Varid{→}\;\Conid{Set}\;\Varid{→}\;\Conid{Set}{}\<[E]%
\\
\>[B]{}\Conid{Op}\;\Varid{0}\;\mathrel{=}\;\Varid{λ}\;\Varid{t}\;\Varid{→}\;\Varid{t}{}\<[E]%
\\
\>[B]{}\Conid{Op}\;(\Varid{suc}\;\Varid{n})\;\mathrel{=}\;\Varid{λ}\;\Varid{t}\;\Varid{→}\;(\Varid{t}\;\Varid{→}\;\Conid{Op}\;\Varid{n}\;\Varid{t}){}\<[E]%
\\[\blanklineskip]%
\>[B]{}\Keyword{record}\;\Conid{SetoidOp}\;(\Conid{St}\;\mathbin{:}\;\Conid{Setoid}\;\Varid{zero}\;\Varid{zero})\;(\Varid{n}\;\mathbin{:}\;\Conid{ℕ})\;\mathbin{:}\;\Conid{Set₁}\;\Keyword{where}{}\<[E]%
\\
\>[B]{}\hsindent{3}{}\<[3]%
\>[3]{}\Varid{constructor}\;\Varid{§\char95 §}{}\<[E]%
\\
\>[B]{}\hsindent{3}{}\<[3]%
\>[3]{}\Keyword{private}{}\<[E]%
\\
\>[3]{}\hsindent{2}{}\<[5]%
\>[5]{}\Conid{S}\;\mathrel{=}\;\Conid{Setoid.Carrier}\;\Conid{St}{}\<[E]%
\\
\>[B]{}\hsindent{3}{}\<[3]%
\>[3]{}\Keyword{field}{}\<[E]%
\\
\>[3]{}\hsindent{2}{}\<[5]%
\>[5]{}\Varid{op}\;\mathbin{:}\;\Conid{Op}\;\Varid{n}\;\Conid{S}{}\<[E]%
\\[\blanklineskip]%
\>[B]{}\Keyword{record}\;\Conid{QuotientOp}\;\{\mskip1.5mu \Conid{St}\;\mathbin{:}\;\Conid{Setoid}\;\Varid{zero}\;\Varid{zero}\mskip1.5mu\}\;{}\<[E]%
\\
\>[B]{}\hsindent{12}{}\<[12]%
\>[12]{}\{\mskip1.5mu \Conid{Qs}\;\mathbin{:}\;\Conid{QuSig}\;\Conid{St}\mskip1.5mu\}\;(\Varid{nf}\;\mathbin{:}\;\Conid{Nf}\;\Conid{Qs})\;(\Varid{n}\;\mathbin{:}\;\Conid{ℕ})\;\mathbin{:}\;\Conid{Set₁}\;\Keyword{where}{}\<[E]%
\\
\>[B]{}\hsindent{3}{}\<[3]%
\>[3]{}\Varid{constructor}\;\Varid{§\char95 §}{}\<[E]%
\\
\>[B]{}\hsindent{3}{}\<[3]%
\>[3]{}\Keyword{private}{}\<[E]%
\\
\>[3]{}\hsindent{3}{}\<[6]%
\>[6]{}\Conid{Q}\;\mathrel{=}\;\Conid{QuSig.Q}\;\Conid{Qs}{}\<[E]%
\\
\>[B]{}\hsindent{3}{}\<[3]%
\>[3]{}\Keyword{field}{}\<[E]%
\\
\>[3]{}\hsindent{3}{}\<[6]%
\>[6]{}\Varid{op}\;\mathbin{:}\;\Conid{Op}\;\Varid{n}\;\Conid{Q}{}\<[E]%
\\[\blanklineskip]%
\>[B]{}\Varid{auxf}\;\mathbin{:}\;\{\mskip1.5mu \Conid{S}\;\Conid{Q}\;\mathbin{:}\;\Conid{Set}\mskip1.5mu\}\;(\Varid{n}\;\mathbin{:}\;\Conid{ℕ})\;([\mskip1.5mu \anonymous \mskip1.5mu]\;\mathbin{:}\;\Conid{S}\;\Varid{→}\;\Conid{Q})\;(\Varid{emb}\;\mathbin{:}\;\Conid{Q}\;\Varid{→}\;\Conid{S})\;\Varid{→}\;{}\<[E]%
\\
\>[B]{}\hsindent{11}{}\<[11]%
\>[11]{}\Conid{Op}\;\Varid{n}\;\Conid{S}\;\Varid{→}\;\Conid{Op}\;\Varid{n}\;\Conid{Q}{}\<[E]%
\\
\>[B]{}\Varid{auxf}\;\Varid{zero}\;[\mskip1.5mu \anonymous \mskip1.5mu]\;\Varid{emb}\;\Varid{op}\;\mathrel{=}\;[\mskip1.5mu \Varid{op}\mskip1.5mu]{}\<[E]%
\\
\>[B]{}\Varid{auxf}\;(\Varid{suc}\;\Varid{n})\;[\mskip1.5mu \anonymous \mskip1.5mu]\;\Varid{emb}\;\Varid{op}\;\mathrel{=}\;\Varid{λ}\;\Varid{x}\;\Varid{→}\;\Varid{auxf}\;\Varid{n}\;[\mskip1.5mu \anonymous \mskip1.5mu]\;\Varid{emb}\;(\Varid{op}\;(\Varid{emb}\;\Varid{x})){}\<[E]%
\\[\blanklineskip]%
\>[B]{}\Varid{liftop}\;\mathbin{:}\;\{\mskip1.5mu \Conid{S}\;\mathbin{:}\;\Conid{Setoid}\;\Varid{zero}\;\Varid{zero}\mskip1.5mu\}\;(\Varid{n}\;\mathbin{:}\;\Conid{ℕ})\;(\Conid{Qs}\;\mathbin{:}\;\Conid{QuSig}\;\Conid{S})\;{}\<[E]%
\\
\>[B]{}\hsindent{13}{}\<[13]%
\>[13]{}(\Conid{So}\;\mathbin{:}\;\Conid{SetoidOp}\;\Conid{S}\;\Varid{n})\;(\Varid{nf}\;\mathbin{:}\;\Conid{Nf}\;\Conid{Qs})\;\Varid{→}\;\Conid{QuotientOp}\;\Varid{nf}\;\Varid{n}{}\<[E]%
\\
\>[B]{}\Varid{liftop}\;\Varid{n}\;(\Conid{Q},[\mskip1.5mu \anonymous \mskip1.5mu],\Varid{sound})\;\Varid{§}\;\Varid{op}\;\Varid{§}\;(\Varid{emb},\Varid{compl},\Varid{stable})\;\mathrel{=}\;\Varid{§}\;\Varid{auxf}\;\Varid{n}{}\<[E]%
\\
\>[B]{}[\mskip1.5mu \anonymous \mskip1.5mu]\;\Varid{emb}\;\Varid{op}\;\Varid{§}{}\<[E]%
\ColumnHook
\end{hscode}\resethooks

We can lift operators of any order within the normal form definition
of quotient type.
According to this, lift the general properties are also possible.



\section{The undefinable quotient types}
All of integers, rational numbers and the congurence class of modulo
have definable normal form or canonical form. However real numbers is not belonging to
this group. It does not have normal forms. We can use cauchy sequences or signed digits to represent real
numbers. They are obvious quotient sets isomorphic to the true real numbers, but we cannot use the interface
introduced above. We can only postulate it in Agda. Moreover the equivalence relation is undecidable.
For these kinds of quotient types, for which we do not have normal form, we call them axiomatised quotient
types.

This could be a cauchy sequence used to represent real numbers (Simplified for readability),


\begin{hscode}\SaveRestoreHook
\column{B}{@{}>{\hspre}l<{\hspost}@{}}%
\column{3}{@{}>{\hspre}l<{\hspost}@{}}%
\column{5}{@{}>{\hspre}l<{\hspost}@{}}%
\column{E}{@{}>{\hspre}l<{\hspost}@{}}%
\>[B]{}\Keyword{record}\;\Varid{cauℝ}\;\mathbin{:}\;\Conid{Set}\;\Keyword{where}{}\<[E]%
\\
\>[B]{}\hsindent{3}{}\<[3]%
\>[3]{}\Keyword{field}{}\<[E]%
\\
\>[3]{}\hsindent{2}{}\<[5]%
\>[5]{}\Varid{f}\;\mathbin{:}\;\Conid{ℕ}\;\Varid{→}\;\Conid{ℚ}{}\<[E]%
\\
\>[3]{}\hsindent{2}{}\<[5]%
\>[5]{}\Varid{p}\;\mathbin{:}\;(\Varid{n}\;\mathbin{:}\;\Conid{ℕ})\;\Varid{→}\;\Varid{∀}\;(\Varid{m}\;\mathbin{:}\;\Conid{ℕ})\;\Varid{→}\;(\Varid{n}\;\Varid{<}\;\Varid{m})\;\Varid{→}\;\Varid{∣}\;(\Varid{f}\;\Varid{m})\;\Varid{-}\;(\Varid{f}\;\Varid{n})\;\Varid{∣}\;\Varid{<}\;(\Varid{1}\;\mathbin{/}\;\Varid{2}\;\mathbin{\uparrow}\;\Varid{n}){}\<[E]%
\ColumnHook
\end{hscode}\resethooks

It contains a function to generate the sequence of numbers and a proposition that the
sequence converges by bounded rate.

For example we can embedding rational numbers easily,


\begin{hscode}\SaveRestoreHook
\column{B}{@{}>{\hspre}l<{\hspost}@{}}%
\column{E}{@{}>{\hspre}l<{\hspost}@{}}%
\>[B]{}\Varid{emb}\;\mathbin{:}\;\Conid{ℚ₀}\;\Varid{→}\;\Conid{ℝ}{}\<[E]%
\\
\>[B]{}\Varid{emb}\;\Varid{q}\;\mathrel{=}\;\Varid{f:}\;(\Varid{λ}\;\anonymous \;\Varid{→}\;\Varid{q})\;\Varid{p:}\;\Varid{λ}\;\Varid{n}\;\Varid{m}\;\Varid{n<m}\;\Varid{→}\;(\Varid{s≤s}\;\Varid{z≤n})\;\Varid{resp}\;(\Varid{abscanc}\;\Varid{q}){}\<[E]%
\ColumnHook
\end{hscode}\resethooks

But for irrational numbers we have to use different ways to generate the sequences. For square root, we can use Taylor series.

\section{Conclusion}
Here, we only talk about definable quotient types within Agda. The
quotient is a setoid and the elements in an euqivalence classes are
not definitionally equal. However, if we axiomatize the type form of
quotient type and let the Agda automatically normalise the carrier,
then the definitional equality between different elements in same
equivalence class will be present.



\bibliography{quotients}{}
\bibliographystyle{plain}

\end{document}
