\documentclass[a4paper,12pt]{article}
\def\textmu{}
%% ODER: format ==         = "\mathrel{==}"
%% ODER: format /=         = "\neq "
%
%
\makeatletter
\@ifundefined{lhs2tex.lhs2tex.sty.read}%
  {\@namedef{lhs2tex.lhs2tex.sty.read}{}%
   \newcommand\SkipToFmtEnd{}%
   \newcommand\EndFmtInput{}%
   \long\def\SkipToFmtEnd#1\EndFmtInput{}%
  }\SkipToFmtEnd

\newcommand\ReadOnlyOnce[1]{\@ifundefined{#1}{\@namedef{#1}{}}\SkipToFmtEnd}
\usepackage{amstext}
\usepackage{amssymb}
\usepackage{stmaryrd}
\DeclareFontFamily{OT1}{cmtex}{}
\DeclareFontShape{OT1}{cmtex}{m}{n}
  {<5><6><7><8>cmtex8
   <9>cmtex9
   <10><10.95><12><14.4><17.28><20.74><24.88>cmtex10}{}
\DeclareFontShape{OT1}{cmtex}{m}{it}
  {<-> ssub * cmtt/m/it}{}
\newcommand{\texfamily}{\fontfamily{cmtex}\selectfont}
\DeclareFontShape{OT1}{cmtt}{bx}{n}
  {<5><6><7><8>cmtt8
   <9>cmbtt9
   <10><10.95><12><14.4><17.28><20.74><24.88>cmbtt10}{}
\DeclareFontShape{OT1}{cmtex}{bx}{n}
  {<-> ssub * cmtt/bx/n}{}
\newcommand{\tex}[1]{\text{\texfamily#1}}	% NEU

\newcommand{\Sp}{\hskip.33334em\relax}


\newcommand{\Conid}[1]{\mathit{#1}}
\newcommand{\Varid}[1]{\mathit{#1}}
\newcommand{\anonymous}{\kern0.06em \vbox{\hrule\@width.5em}}
\newcommand{\plus}{\mathbin{+\!\!\!+}}
\newcommand{\bind}{\mathbin{>\!\!\!>\mkern-6.7mu=}}
\newcommand{\rbind}{\mathbin{=\mkern-6.7mu<\!\!\!<}}% suggested by Neil Mitchell
\newcommand{\sequ}{\mathbin{>\!\!\!>}}
\renewcommand{\leq}{\leqslant}
\renewcommand{\geq}{\geqslant}
\usepackage{polytable}

%mathindent has to be defined
\@ifundefined{mathindent}%
  {\newdimen\mathindent\mathindent\leftmargini}%
  {}%

\def\resethooks{%
  \global\let\SaveRestoreHook\empty
  \global\let\ColumnHook\empty}
\newcommand*{\savecolumns}[1][default]%
  {\g@addto@macro\SaveRestoreHook{\savecolumns[#1]}}
\newcommand*{\restorecolumns}[1][default]%
  {\g@addto@macro\SaveRestoreHook{\restorecolumns[#1]}}
\newcommand*{\aligncolumn}[2]%
  {\g@addto@macro\ColumnHook{\column{#1}{#2}}}

\resethooks

\newcommand{\onelinecommentchars}{\quad-{}- }
\newcommand{\commentbeginchars}{\enskip\{-}
\newcommand{\commentendchars}{-\}\enskip}

\newcommand{\visiblecomments}{%
  \let\onelinecomment=\onelinecommentchars
  \let\commentbegin=\commentbeginchars
  \let\commentend=\commentendchars}

\newcommand{\invisiblecomments}{%
  \let\onelinecomment=\empty
  \let\commentbegin=\empty
  \let\commentend=\empty}

\visiblecomments

\newlength{\blanklineskip}
\setlength{\blanklineskip}{0.66084ex}

\newcommand{\hsindent}[1]{\quad}% default is fixed indentation
\let\hspre\empty
\let\hspost\empty
\newcommand{\NB}{\textbf{NB}}
\newcommand{\Todo}[1]{$\langle$\textbf{To do:}~#1$\rangle$}

\EndFmtInput
\makeatother
%
%
%
%
%
%
% This package provides two environments suitable to take the place
% of hscode, called "plainhscode" and "arrayhscode". 
%
% The plain environment surrounds each code block by vertical space,
% and it uses \abovedisplayskip and \belowdisplayskip to get spacing
% similar to formulas. Note that if these dimensions are changed,
% the spacing around displayed math formulas changes as well.
% All code is indented using \leftskip.
%
% Changed 19.08.2004 to reflect changes in colorcode. Should work with
% CodeGroup.sty.
%
\ReadOnlyOnce{polycode.fmt}%
\makeatletter

\newcommand{\hsnewpar}[1]%
  {{\parskip=0pt\parindent=0pt\par\vskip #1\noindent}}

% can be used, for instance, to redefine the code size, by setting the
% command to \small or something alike
\newcommand{\hscodestyle}{}

% The command \sethscode can be used to switch the code formatting
% behaviour by mapping the hscode environment in the subst directive
% to a new LaTeX environment.

\newcommand{\sethscode}[1]%
  {\expandafter\let\expandafter\hscode\csname #1\endcsname
   \expandafter\let\expandafter\endhscode\csname end#1\endcsname}

% "compatibility" mode restores the non-polycode.fmt layout.

\newenvironment{compathscode}%
  {\par\noindent
   \advance\leftskip\mathindent
   \hscodestyle
   \let\\=\@normalcr
   \let\hspre\(\let\hspost\)%
   \pboxed}%
  {\endpboxed\)%
   \par\noindent
   \ignorespacesafterend}

\newcommand{\compaths}{\sethscode{compathscode}}

% "plain" mode is the proposed default.
% It should now work with \centering.
% This required some changes. The old version
% is still available for reference as oldplainhscode.

\newenvironment{plainhscode}%
  {\hsnewpar\abovedisplayskip
   \advance\leftskip\mathindent
   \hscodestyle
   \let\hspre\(\let\hspost\)%
   \pboxed}%
  {\endpboxed%
   \hsnewpar\belowdisplayskip
   \ignorespacesafterend}

\newenvironment{oldplainhscode}%
  {\hsnewpar\abovedisplayskip
   \advance\leftskip\mathindent
   \hscodestyle
   \let\\=\@normalcr
   \(\pboxed}%
  {\endpboxed\)%
   \hsnewpar\belowdisplayskip
   \ignorespacesafterend}

% Here, we make plainhscode the default environment.

\newcommand{\plainhs}{\sethscode{plainhscode}}
\newcommand{\oldplainhs}{\sethscode{oldplainhscode}}
\plainhs

% The arrayhscode is like plain, but makes use of polytable's
% parray environment which disallows page breaks in code blocks.

\newenvironment{arrayhscode}%
  {\hsnewpar\abovedisplayskip
   \advance\leftskip\mathindent
   \hscodestyle
   \let\\=\@normalcr
   \(\parray}%
  {\endparray\)%
   \hsnewpar\belowdisplayskip
   \ignorespacesafterend}

\newcommand{\arrayhs}{\sethscode{arrayhscode}}

% The mathhscode environment also makes use of polytable's parray 
% environment. It is supposed to be used only inside math mode 
% (I used it to typeset the type rules in my thesis).

\newenvironment{mathhscode}%
  {\parray}{\endparray}

\newcommand{\mathhs}{\sethscode{mathhscode}}

% texths is similar to mathhs, but works in text mode.

\newenvironment{texthscode}%
  {\(\parray}{\endparray\)}

\newcommand{\texths}{\sethscode{texthscode}}

% The framed environment places code in a framed box.

\def\codeframewidth{\arrayrulewidth}
\RequirePackage{calc}

\newenvironment{framedhscode}%
  {\parskip=\abovedisplayskip\par\noindent
   \hscodestyle
   \arrayrulewidth=\codeframewidth
   \tabular{@{}|p{\linewidth-2\arraycolsep-2\arrayrulewidth-2pt}|@{}}%
   \hline\framedhslinecorrect\\{-1.5ex}%
   \let\endoflinesave=\\
   \let\\=\@normalcr
   \(\pboxed}%
  {\endpboxed\)%
   \framedhslinecorrect\endoflinesave{.5ex}\hline
   \endtabular
   \parskip=\belowdisplayskip\par\noindent
   \ignorespacesafterend}

\newcommand{\framedhslinecorrect}[2]%
  {#1[#2]}

\newcommand{\framedhs}{\sethscode{framedhscode}}

% The inlinehscode environment is an experimental environment
% that can be used to typeset displayed code inline.

\newenvironment{inlinehscode}%
  {\(\def\column##1##2{}%
   \let\>\undefined\let\<\undefined\let\\\undefined
   \newcommand\>[1][]{}\newcommand\<[1][]{}\newcommand\\[1][]{}%
   \def\fromto##1##2##3{##3}%
   \def\nextline{}}{\) }%

\newcommand{\inlinehs}{\sethscode{inlinehscode}}

% The joincode environment is a separate environment that
% can be used to surround and thereby connect multiple code
% blocks.

\newenvironment{joincode}%
  {\let\orighscode=\hscode
   \let\origendhscode=\endhscode
   \def\endhscode{\def\hscode{\endgroup\def\@currenvir{hscode}\\}\begingroup}
   %\let\SaveRestoreHook=\empty
   %\let\ColumnHook=\empty
   %\let\resethooks=\empty
   \orighscode\def\hscode{\endgroup\def\@currenvir{hscode}}}%
  {\origendhscode
   \global\let\hscode=\orighscode
   \global\let\endhscode=\origendhscode}%

\makeatother
\EndFmtInput
%
%
\ReadOnlyOnce{agda.fmt}%


\RequirePackage[T1]{fontenc}
\RequirePackage[utf8x]{inputenc}
\RequirePackage{ucs}
\RequirePackage{amsfonts}

\providecommand\mathbbm{\mathbb}

% TODO: Define more of these ...
\DeclareUnicodeCharacter{737}{\textsuperscript{l}}
\DeclareUnicodeCharacter{8718}{\ensuremath{\blacksquare}}
\DeclareUnicodeCharacter{8759}{::}
\DeclareUnicodeCharacter{9669}{\ensuremath{\triangleleft}}
\DeclareUnicodeCharacter{8799}{\ensuremath{\stackrel{\scriptscriptstyle ?}{=}}}
\DeclareUnicodeCharacter{10214}{\ensuremath{\llbracket}}
\DeclareUnicodeCharacter{10215}{\ensuremath{\rrbracket}}

% TODO: This is in general not a good idea.
\providecommand\textepsilon{$\epsilon$}
\providecommand\textmu{$\mu$}


%Actually, varsyms should not occur in Agda output.

% TODO: Make this configurable. IMHO, italics doesn't work well
% for Agda code.

\renewcommand\Varid[1]{\mathord{\textsf{#1}}}
\let\Conid\Varid
\newcommand\Keyword[1]{\textsf{\textbf{#1}}}
\EndFmtInput


\usepackage[utf8x]{inputenc}
\usepackage{ucs}
\usepackage{cite}

%\DeclareUnicodeCharacter{"2237}{\ensuremath{::}}
\DeclareUnicodeCharacter{"03BB}{\ensuremath{\lambda}}
\DeclareUnicodeCharacter{"03A3}{\ensuremath{\Sigma}}

\usepackage{color}
\newcommand{\txa}[1]{\textcolor{red}{\textbf{Thorsten:~}#1}}

\author{Li Nuo}
\title{Two presentations of equality}

\begin{document}

\maketitle

\section{Background}

Intentional equality can be defined in two ways: either defined as an inductive
relation or as a parameterized inductive predicate:

\begin{description}
\item[As a binary relation]

\begin{hscode}\SaveRestoreHook
\column{B}{@{}>{\hspre}l<{\hspost}@{}}%
\column{3}{@{}>{\hspre}l<{\hspost}@{}}%
\column{E}{@{}>{\hspre}l<{\hspost}@{}}%
\>[B]{}\Keyword{data}\;\Conid{Id}\;(\Conid{A}\;\mathbin{:}\;\Conid{Set})\;\mathbin{:}\;\Conid{A}\;\Varid{→}\;\Conid{A}\;\Varid{→}\;\Conid{Set}\;\Keyword{where}{}\<[E]%
\\
\>[B]{}\hsindent{3}{}\<[3]%
\>[3]{}\Varid{refl}\;\mathbin{:}\;(\Varid{a}\;\mathbin{:}\;\Conid{A})\;\Varid{→}\;\Conid{Id}\;\Conid{A}\;\Varid{a}\;\Varid{a}{}\<[E]%
\ColumnHook
\end{hscode}\resethooks

This one was first
proposed by Per Martin-Löf as intensional equality rather than
propositional equality\cite{Nord}.
There is exactly one member in set \ensuremath{\Conid{Id}\;\Conid{A}\;\Varid{a}\;\Varid{a}} namely\ensuremath{\Varid{refl}\;\Varid{a}} where
\ensuremath{\Varid{a}\;\mathbin{:}\;\Conid{A}}.

\item[As a predicate]

\begin{hscode}\SaveRestoreHook
\column{B}{@{}>{\hspre}l<{\hspost}@{}}%
\column{3}{@{}>{\hspre}l<{\hspost}@{}}%
\column{E}{@{}>{\hspre}l<{\hspost}@{}}%
\>[B]{}\Keyword{data}\;\Conid{Id'}\;(\Conid{A}\;\mathbin{:}\;\Conid{Set})\;(\Varid{a}\;\mathbin{:}\;\Conid{A})\;\mathbin{:}\;\Conid{A}\;\Varid{→}\;\Conid{Set}\;\Keyword{where}{}\<[E]%
\\
\>[B]{}\hsindent{3}{}\<[3]%
\>[3]{}\Varid{refl}\;\mathbin{:}\;\Conid{Id'}\;\Conid{A}\;\Varid{a}\;\Varid{a}{}\<[E]%
\ColumnHook
\end{hscode}\resethooks
This version is used in the Agda standard library. Only with dependent
type feature it can be defined. \ensuremath{\Conid{Id}\;\Conid{A}\;\Varid{a}} is a predicate of whether some \ensuremath{\Varid{x}\;\mathbin{:}\;\Conid{A}} is the same as
\ensuremath{\Varid{a}} in the type declaration. The difference here is we cannot show
what is the \ensuremath{\Varid{a}} in the constant \ensuremath{\Varid{refl}} which is unique
for each \ensuremath{\Conid{Id}\;\Conid{A}\;\Varid{a}}.
This one was proposed by Christine Paulin-Mohring \cite{coq}.
\txa{proposed by Christine Paulin-Mohring http://coq.inria.fr/refman/Reference-Manual005.html}
\end{description}

For each of them, we have a corresponding elimination rule. It was
called \ensuremath{\Varid{idpeel}} \cite{Nord} but we rename it as \ensuremath{\Conid{J}} here. It is defined as

\begin{description}
\item[As a binary relation]

\begin{hscode}\SaveRestoreHook
\column{B}{@{}>{\hspre}l<{\hspost}@{}}%
\column{5}{@{}>{\hspre}l<{\hspost}@{}}%
\column{E}{@{}>{\hspre}l<{\hspost}@{}}%
\>[B]{}\Conid{J}\;\mathbin{:}\;(\Conid{A}\;\mathbin{:}\;\Conid{Set})\;(\Conid{P}\;\mathbin{:}\;(\Varid{a}\;\Varid{b}\;\mathbin{:}\;\Conid{A})\;\Varid{→}\;\Conid{Id}\;\Conid{A}\;\Varid{a}\;\Varid{b}\;\Varid{→}\;\Conid{Set})\;{}\<[E]%
\\
\>[B]{}\hsindent{5}{}\<[5]%
\>[5]{}\Varid{→}\;(\Varid{m}\;\mathbin{:}\;(\Varid{a}\;\mathbin{:}\;\Conid{A})\;\Varid{→}\;\Conid{P}\;\Varid{a}\;\Varid{a}\;(\Varid{refl}\;\Varid{a}))\;{}\<[E]%
\\
\>[B]{}\hsindent{5}{}\<[5]%
\>[5]{}\Varid{→}\;(\Varid{a}\;\Varid{b}\;\mathbin{:}\;\Conid{A})\;(\Varid{p}\;\mathbin{:}\;\Conid{Id}\;\Conid{A}\;\Varid{a}\;\Varid{b})\;\Varid{→}\;\Conid{P}\;\Varid{a}\;\Varid{b}\;\Varid{p}{}\<[E]%
\\
\>[B]{}\Conid{J}\;\Conid{A}\;\Conid{P}\;\Varid{m}\;\Varid{.b}\;\Varid{b}\;(\Varid{refl}\;\Varid{.b})\;\mathrel{=}\;\Varid{m}\;\Varid{b}{}\<[E]%
\ColumnHook
\end{hscode}\resethooks
We need to give constants \ensuremath{\Conid{P}} and \ensuremath{\Varid{m}} to eliminate the defined equality.

\ensuremath{\Varid{m}} can be seen as an introduction rule for \ensuremath{\Conid{P}}. For all \ensuremath{\Varid{a}}, \ensuremath{(\Varid{a},\Varid{a},\Varid{refl}\;\Varid{a})} is
inhabited in \ensuremath{\Conid{P}}. And the result is a more general
property, For all \ensuremath{\Varid{a}} \ensuremath{\Varid{b}}, \ensuremath{(\Varid{a},\Varid{b},\Varid{x}\;\mathbin{:}\;\Conid{Id}\;\Conid{A}\;\Varid{a}\;\Varid{b})} is inhabited in \ensuremath{\Conid{P}}.


\ensuremath{\Conid{J}} actually maps \[ \ensuremath{\Varid{∀}\;(\Varid{a}\;\mathbin{:}\;\Conid{A})\;\Varid{→}\;\Conid{P}\;\Varid{a}\;\Varid{a}\;(\Varid{refl}\;\Varid{a})}
\Rightarrow \ensuremath{\Varid{∀}\;(\Varid{a}\;\Varid{b}\;\mathbin{:}\;\Conid{A})\;(\Varid{p}\;\mathbin{:}\;\Conid{Id}\;\Conid{A}\;\Varid{a}\;\Varid{b})\;\Varid{→}\;\Conid{P}\;\Varid{a}\;\Varid{b}\;\Varid{p}} \].

\item[As a predicate]

\begin{hscode}\SaveRestoreHook
\column{B}{@{}>{\hspre}l<{\hspost}@{}}%
\column{3}{@{}>{\hspre}l<{\hspost}@{}}%
\column{E}{@{}>{\hspre}l<{\hspost}@{}}%
\>[B]{}\Conid{J'}\;\mathbin{:}\;(\Conid{A}\;\mathbin{:}\;\Conid{Set})\;(\Varid{a}\;\mathbin{:}\;\Conid{A})\;{}\<[E]%
\\
\>[B]{}\hsindent{3}{}\<[3]%
\>[3]{}\Varid{→}\;(\Conid{P}\;\mathbin{:}\;(\Varid{b}\;\mathbin{:}\;\Conid{A})\;\Varid{→}\;\Conid{Id'}\;\Conid{A}\;\Varid{a}\;\Varid{b}\;\Varid{→}\;\Conid{Set})\;{}\<[E]%
\\
\>[B]{}\hsindent{3}{}\<[3]%
\>[3]{}\Varid{→}\;(\Varid{m}\;\mathbin{:}\;\Conid{P}\;\Varid{a}\;\Varid{refl})\;{}\<[E]%
\\
\>[B]{}\hsindent{3}{}\<[3]%
\>[3]{}\Varid{→}\;(\Varid{b}\;\mathbin{:}\;\Conid{A})\;(\Varid{p}\;\mathbin{:}\;\Conid{Id'}\;\Conid{A}\;\Varid{a}\;\Varid{b})\;\Varid{→}\;\Conid{P}\;\Varid{b}\;\Varid{p}{}\<[E]%
\\
\>[B]{}\Conid{J'}\;\Conid{A}\;\Varid{.b}\;\Conid{P}\;\Varid{m}\;\Varid{b}\;\Varid{refl}\;\mathrel{=}\;\Varid{m}{}\<[E]%
\ColumnHook
\end{hscode}\resethooks
The \ensuremath{\Conid{P}} and \ensuremath{\Varid{m}} are now bounded by the same \ensuremath{\Varid{a}} as the the identity
predicate. \ensuremath{\Conid{P}} and \ensuremath{\Varid{m}} here can be viewed as \ensuremath{\Conid{P}\;[\mskip1.5mu \Varid{a}\mskip1.5mu]}
and \ensuremath{\Varid{m}\;[\mskip1.5mu \Varid{a}\mskip1.5mu]}. Therefore, the elimination rule is all about eliminate
the predicate \ensuremath{\Conid{Id'}\;\Conid{A}\;\Varid{a}} rather than the binary equivalence relation
\ensuremath{\Conid{Id'}\;\Conid{A}}.
 
\ensuremath{\Conid{J'}} actually maps  \[\ensuremath{\Conid{P}\;\Varid{a}\;\Varid{refl}} \Rightarrow \ensuremath{(\Varid{b}\;\mathbin{:}\;\Conid{A})\;(\Varid{p}\;\mathbin{:}\;\Conid{Id'}\;\Conid{A}\;\Varid{a}\;\Varid{b})\;\Varid{→}\;\Conid{P}\;\Varid{b}\;\Varid{p}}\].
\ensuremath{\Varid{m}}! can be seen as the only object in \ensuremath{\Conid{P}} and the result is used to unify
elements equal to a (a constant) to get the unique object.
\end{description}

\section{The Problem}
Now the problem is: how to implement \ensuremath{\Conid{J}} using only \ensuremath{\Conid{J'}} (also we use the
equality \ensuremath{\Conid{Id'}}) and vice versa? We will still use corresponding equality for each
elimination rule, otherwise it cannot eliminate the identity.

\section{Solution}

From \ensuremath{\Conid{J'}} to \ensuremath{\Conid{J}} is quite simple.
\txa{Which is the first direction?}
When we eliminate the predicate identity, we only need to create the
special cases of P and m for J'.
\begin{hscode}\SaveRestoreHook
\column{B}{@{}>{\hspre}l<{\hspost}@{}}%
\column{5}{@{}>{\hspre}l<{\hspost}@{}}%
\column{E}{@{}>{\hspre}l<{\hspost}@{}}%
\>[B]{}\Conid{JId'}\;\mathbin{:}\;(\Conid{A}\;\mathbin{:}\;\Conid{Set})\;(\Conid{P}\;\mathbin{:}\;(\Varid{a}\;\Varid{b}\;\mathbin{:}\;\Conid{A})\;\Varid{→}\;\Conid{Id'}\;\Conid{A}\;\Varid{a}\;\Varid{b}\;\Varid{→}\;\Conid{Set})\;{}\<[E]%
\\
\>[B]{}\hsindent{5}{}\<[5]%
\>[5]{}\Varid{→}\;((\Varid{a}\;\mathbin{:}\;\Conid{A})\;\Varid{→}\;\Conid{P}\;\Varid{a}\;\Varid{a}\;\Varid{refl})\;{}\<[E]%
\\
\>[B]{}\hsindent{5}{}\<[5]%
\>[5]{}\Varid{→}\;(\Varid{a}\;\Varid{b}\;\mathbin{:}\;\Conid{A})\;(\Varid{p}\;\mathbin{:}\;\Conid{Id'}\;\Conid{A}\;\Varid{a}\;\Varid{b})\;\Varid{→}\;\Conid{P}\;\Varid{a}\;\Varid{b}\;\Varid{p}{}\<[E]%
\\
\>[B]{}\Conid{JId'}\;\Conid{A}\;\Conid{P}\;\Varid{m}\;\Varid{a}\;\mathrel{=}\;\Conid{J'}\;\Conid{A}\;\Varid{a}\;(\Conid{P}\;\Varid{a})\;(\Varid{m}\;\Varid{a}){}\<[E]%
\ColumnHook
\end{hscode}\resethooks

\txa{Check that \ensuremath{\Conid{JId'}\;\Conid{A}\;\Conid{P}\;\Varid{m}\;\Varid{.b}\;\Varid{b}\;(\Varid{refl}\;\Varid{.b})\;\mathrel{=}\;\Varid{m}\;\Varid{b}} holds definitionally.}

The other direction is more tricky.
We first define \ensuremath{\Varid{subst}} from \ensuremath{\Conid{J}}

\begin{hscode}\SaveRestoreHook
\column{B}{@{}>{\hspre}l<{\hspost}@{}}%
\column{9}{@{}>{\hspre}l<{\hspost}@{}}%
\column{E}{@{}>{\hspre}l<{\hspost}@{}}%
\>[B]{}\Varid{subst}\;\mathbin{:}\;(\Conid{A}\;\mathbin{:}\;\Conid{Set})\;(\Varid{a}\;\Varid{b}\;\mathbin{:}\;\Conid{A})\;(\Varid{p}\;\mathbin{:}\;\Conid{Id}\;\Conid{A}\;\Varid{a}\;\Varid{b})\;{}\<[E]%
\\
\>[B]{}\hsindent{9}{}\<[9]%
\>[9]{}(\Conid{B}\;\mathbin{:}\;\Conid{A}\;\Varid{→}\;\Conid{Set})\;\Varid{→}\;\Conid{B}\;\Varid{a}\;\Varid{→}\;\Conid{B}\;\Varid{b}{}\<[E]%
\\
\>[B]{}\Varid{subst}\;\Conid{A}\;\Varid{a}\;\Varid{b}\;\Varid{p}\;\Conid{B}\;\mathrel{=}\;\Conid{J}\;\Conid{A}\;(\Varid{λ}\;\Varid{a'}\;\Varid{b'}\;\anonymous \;\Varid{→}\;\Conid{B}\;\Varid{a'}\;\Varid{→}\;\Conid{B}\;\Varid{b'})\;(\Varid{λ}\;\anonymous \;\Varid{→}\;\Varid{id})\;\Varid{a}\;\Varid{b}\;\Varid{p}{}\<[E]%
\ColumnHook
\end{hscode}\resethooks

Then to prove \ensuremath{\Conid{J'}} from \ensuremath{\Conid{J}} and \ensuremath{\Conid{Id}},
\begin{hscode}\SaveRestoreHook
\column{B}{@{}>{\hspre}l<{\hspost}@{}}%
\column{3}{@{}>{\hspre}l<{\hspost}@{}}%
\column{E}{@{}>{\hspre}l<{\hspost}@{}}%
\>[B]{}\Conid{Q}\;\mathbin{:}\;(\Conid{A}\;\mathbin{:}\;\Conid{Set})\;(\Varid{a}\;\mathbin{:}\;\Conid{A})\;\Varid{→}\;\Conid{Set}{}\<[E]%
\\
\>[B]{}\Conid{Q}\;\Conid{A}\;\Varid{a}\;\mathrel{=}\;\Conid{Σ}\;\Conid{A}\;(\Varid{λ}\;\Varid{b}\;\Varid{→}\;\Conid{Id}\;\Conid{A}\;\Varid{a}\;\Varid{b}){}\<[E]%
\\[\blanklineskip]%
\>[B]{}\Conid{J'Id}\;\mathbin{:}\;(\Conid{A}\;\mathbin{:}\;\Conid{Set})\;(\Varid{a}\;\mathbin{:}\;\Conid{A})\;\Varid{→}\;(\Conid{P}\;\mathbin{:}\;(\Varid{b}\;\mathbin{:}\;\Conid{A})\;\Varid{→}\;\Conid{Id}\;\Conid{A}\;\Varid{a}\;\Varid{b}\;\Varid{→}\;\Conid{Set})\;{}\<[E]%
\\
\>[B]{}\hsindent{3}{}\<[3]%
\>[3]{}\Varid{→}\;\Conid{P}\;\Varid{a}\;(\Varid{refl}\;\Varid{a})\;{}\<[E]%
\\
\>[B]{}\hsindent{3}{}\<[3]%
\>[3]{}\Varid{→}\;(\Varid{b}\;\mathbin{:}\;\Conid{A})\;(\Varid{p}\;\mathbin{:}\;\Conid{Id}\;\Conid{A}\;\Varid{a}\;\Varid{b})\;\Varid{→}\;\Conid{P}\;\Varid{b}\;\Varid{p}{}\<[E]%
\\
\>[B]{}\Conid{J'Id}\;\Conid{A}\;\Varid{a}\;\Conid{P}\;\Varid{m}\;\Varid{b}\;\Varid{p}\;\mathrel{=}\;\Varid{subst}\;(\Conid{Q}\;\Conid{A}\;\Varid{a})\;(\Varid{a},\Varid{refl}\;\Varid{a})\;(\Varid{b},\Varid{p})\;{}\<[E]%
\\
\>[B]{}\hsindent{3}{}\<[3]%
\>[3]{}(\Conid{J}\;\Conid{A}\;(\Varid{λ}\;\Varid{a'}\;\Varid{b'}\;\Varid{x}\;\Varid{→}\;\Conid{Id}\;(\Conid{Q}\;\Conid{A}\;\Varid{a'})\;(\Varid{a'},\Varid{refl}\;\Varid{a'})\;(\Varid{b'},\Varid{x})){}\<[E]%
\\
\>[B]{}\hsindent{3}{}\<[3]%
\>[3]{}(\Varid{λ}\;\Varid{a'}\;\Varid{→}\;\Varid{refl}\;(\Varid{a'},\Varid{refl}\;\Varid{a'}))\;\Varid{a}\;\Varid{b}\;\Varid{p})\;(\Varid{uncurry}\;\Conid{P})\;\Varid{m}{}\<[E]%
\ColumnHook
\end{hscode}\resethooks
We can not just use \ensuremath{\Conid{J}} to eliminate the identity because \ensuremath{\Conid{J}} requires
more general \ensuremath{\Conid{P}} and \ensuremath{\Varid{m}}.
We need to formalise the result \ensuremath{\Conid{P}\;\Varid{b}\;\Varid{p}} from \ensuremath{\Conid{P}\;\Varid{a}\;(\Varid{refl}\;\Varid{a})}. We cannot
substitute \ensuremath{\Varid{a}} or \ensuremath{\Varid{refl}\;\Varid{a}} separately because the second argument is
dependent on the first argument. So when we substitute we should reveal
the dependent relation. 
\txa{Or : Instead we are going to substitute them simultanously using a dependent product.}

We could use dependent productr to do this work. In this way, we can
substitute them simultaneously. The problem now becomes substitute in
\begin{hscode}\SaveRestoreHook
\column{B}{@{}>{\hspre}l<{\hspost}@{}}%
\column{E}{@{}>{\hspre}l<{\hspost}@{}}%
\>[B]{}\Conid{P}\;((\Varid{λ}\;\Varid{a}\;\mathbin{:}\;\Conid{A}\;\Varid{p}\;\mathbin{:}\;\Conid{Id}\;\Conid{A}\;\Varid{a}\;\Varid{b}\;\Varid{→}\;(\Varid{a},\Varid{p}))\;\Varid{a}\;(\Varid{refl}\;\Varid{a})){}\<[E]%
\ColumnHook
\end{hscode}\resethooks
 to \begin{hscode}\SaveRestoreHook
\column{B}{@{}>{\hspre}l<{\hspost}@{}}%
\column{E}{@{}>{\hspre}l<{\hspost}@{}}%
\>[B]{}\Conid{P}\;((\Varid{λ}\;\Varid{a}\;\mathbin{:}\;\Conid{A}\;\Varid{p}\;\mathbin{:}\;\Conid{Id}\;\Conid{A}\;\Varid{a}\;\Varid{b}\;\Varid{→}\;(\Varid{a},\Varid{p}))\;\Varid{b}\;\Varid{p}){}\<[E]%
\ColumnHook
\end{hscode}\resethooks

From \ensuremath{\Conid{J}}, we have \ensuremath{\Conid{Id}\;(\Conid{Q}\;\Varid{a})\;(\Varid{a},\Varid{refl}\;\Varid{a})\;(\Varid{b},\Varid{x}\;\mathbin{:}\;\Conid{Id}\;\Varid{a}\;\Varid{b})} so that we can
prove \ensuremath{\Conid{P'}\;(\Varid{b},\Varid{p})} from \ensuremath{\Conid{P'}\;(\Varid{a},\Varid{refl}\;\Varid{a})} using subst. Because \ensuremath{\Conid{P'}\;(\Varid{b},\Varid{p})} is namely \ensuremath{\Conid{P}\;\Varid{b}\;\Varid{p}}, we have proved.

\txa{Check that \ensuremath{\Conid{J'Id}\;\Conid{A}\;\Varid{b}\;\Conid{P}\;\Varid{m}\;\Varid{b}\;\Varid{refl}\;\mathrel{=}\;\Varid{m}} holds definitionally!}

\txa{Add some references. For Id refer to the Nordstroem et al book, Thomas Streicher habil, Palmgren}

\txa{Compare with the construction of the isomorphism.}


\bibliography{equality1}{}
\bibliographystyle{plain}

\end{document}
