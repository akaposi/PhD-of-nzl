\abstract{\addtocontents{toc}{\vspace{1em}} % Add a gap in the Contents, for aesthetics


Martin-L\"{o}f's intuitionistic type theory (Type Theory) is a formal system serves as not only a foundation of constructive mathematics but also a dependently typed programming language. Dependent types are types depend on values of other types. It builds on Curry-Howard isomorphism which relates computer programs with mathematical proofs such that we can do computer-aided formal reasoning and write certified programs in programming languages like Agda, Epigram etc. There are two kinds of variants of Type Theory which are differentiated on the interpretation of propositional equality induced by the identity types. In \ett, propositional equality is identified with definitional equality which makes type checking undecidable. In \itt, propositional equality does not imply definitional equality and the type checking is decidable. Because of the good computational properties, intensional variants are more popular.

This thesis is about quotient types which are one of missing extensional concepts in current implementations of Type Theory which are usually intensional. A quotient type is a new type whose equality is redefined by an given equivalence relation. However in \itt, there is no mechanism to do that. In this thesis, we first propose to define an algebraic structures that consists of a "quotient type" inductively defined, a normalisation mapping from base type to quotient type and essential properties. The simulation helps us exploit benefits from quotients such as lifting functions. However it is only a compromise approach since it does not apply to all quotients. It is impossible to define computational normalisation for some quotients in Type Theory, e.g.\ real numbers. Thus quotient types are not only helpful but also essential for  formalisation of mathematics and programs construction. Then we investigate some models of Type Theory where types are interpreted as more than just sets, but structured objects like setoids, groupoids or \wog. In these models types have categorical structures, and identity types are internalised as morphisms between terms as objects so that redefining equality is possible. We implement a setoid model in Agda following Altenkirch's approach and show that quotient types are possible to define within this model which is decidable. We also investigate \hott which is a new interpretation of \mltt into homotopy theory where types are interpreted as \wog. Finally we present our syntactic encoding of \wog in Agda which can be further developed into a model where quotient types are also available.

}
