\abstract{\addtocontents{toc}{\vspace{1em}} % Add a gap in the Contents, for aesthetics


Martin-L\"{o}f's intuitionistic type theory (Type Theory) is a formal
system that serves not only as a foundation of constructive
mathematics but also as a dependently typed programming
language. Dependent types are types that depend on values of other
types. Type Theory is based on the Curry-Howard isomorphism which
relates computer programs with mathematical proofs so that we can do
computer-aided formal reasoning and write certified programs in
programming languages like Agda, Epigram etc. Martin L\"{o}f proposed
two variants of Type Theory which are differentiated by the treatment
of equality. In \itt, propositional equality defined by identity types
does not imply definitional equality, and type checking is
decidable. In \ett, propositional equality is identified with
definitional equality which makes type checking undecidable. Because
of the good computational properties, \itt is more popular, however it
lacks some important extensional concepts such as functional
extensionality and quotient types.

This thesis is about quotient types. A quotient type is a new type
whose equality is redefined by a given equivalence relation. However,
in the usual formulation of \itt, there is no type former to create a
quotient. We also lose canonicity if we add quotient types into
Intensional Type Theory as axioms. In this thesis, we first
investigate the expected syntax of quotient types and explain it with
categorical notions. For quotients which can be represented as a
setoid as well as defined as a set without a quotient type former, we
propose to define an algebraic structure of quotients
called \emph{definable quotients}. It relates the setoid
interpretation and the set definition via a normalisation function
which returns a normal form (canonical choice) for each equivalence
class. It can be seen as a simulation of quotient types and it helps
theorem proving because we can benefit from both
representations. However this approach cannot be used for all
quotients. It seems that we cannot define a normalisation function for
some quotients in Type Theory, e.g.\ Cauchy reals and finite
multisets. Quotient types are indeed essential for formalisation of
mathematics and reasoning of programs. Then we consider some models of
Type Theory where types are interpreted as structured objects such as
setoids, groupoids or \wog. In these models equalities are
internalised into types which means that it is possible to redefine
equalities. We present an implementation of
Altenkirch's \cite{alti:lics99} setoid model and show that quotient
types can be defined within this model. We also describe a new
extension of \mltt called \hott where types are interpreted
as \wog. It can be seen as a generalisation of the groupoid model
which makes extensional concepts including quotient types
available. We also introduce a syntactic encoding of \wog which can be
seen as a first step towards building a \wog model in \itt. All of
these implementations were performed in the dependently typed
programming language Agda which is based on intensional \mltt.

}
