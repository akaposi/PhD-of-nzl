\chapter{Background}
\label{tt}

\todo{Before 15th-Nov-2013}

\section{Type Theory}


To introduce type theory, it is helpful to start from set theory first. Set theory is a language to describe most definitions of mathematical objects, in another word, it serves as a foundation system for mathematics. 
In 1870s, George Cantor and Richard Dedekind firstly studied set theory. However, in the 1900s, Bertrand Russell discovered the famous paradox in naive set
theory. It tells people that if we can does not distinguish between small sets like natural numbers and real numbers with "larger" sets like the set of all sets, then it will lead to contradiction. To avoid this paradox, he
proposed the theories of types \cite{rus:1903} as an alternative to
naïve set theory. The objects are assigned to types in a hierarchical structure such that "larger" sets and small sets reside in different levels. The set of all
sets are no longer on the same level as its elements such that the
paradox disappears. 

The elementary notion in type theory is \emph{type} which plays a similar role to set in set theory, but they have some fundamental differences. Every object in type theory comes with its unique type, while an object in set theory can appear in multiple sets and we can talk about an object without knowing which sets it belongs to.
To explain the difference, we use the the number $\mathsf{2}$ as an example. In set theory, 2 is not an element of only one specific set, it belongs to the set of natural numbers $\N$ and also the set of integers $\Z$. While in type
theory, it is impossible to avoid the type of object $\mathsf{2}$. Usually the term $\AgdaInductiveConstructor{suc} \, \AgdaSymbol{(}\AgdaInductiveConstructor{suc} \, \AgdaInductiveConstructor{zero}\AgdaSymbol{)}$ stands for $\mathsf{2}$ of type $\N$ and we have a different term of type $\Z$ constructed by constructors of $\Z$ which is a different object to the one of $\N$. 


\todo{one page explanation of why type theory is useful}

Since Russell's type theory, there are a variety of type theories have been developed by famous mathematicians and computer scientists, for example Gödel and Tarski's which is used in Gödel' 1931 paper \cite{gdl:1931}. There are two families of famous type theories building the bridges between mathematics and computer science, \emph{lambda calculus} and \emph{\mltt}. 

% lambda calculus
\subsection{Lambda Calculus}

Alonzo Church introduced lambda calculus in 1930s. He first introduced
an untyped lambda calculus which was proved to be inconsistent due to
the Kleene-Rosser paradox. Then the typed one which is also called
Church's Theory of types or simply typed lambda calculus is
introduced as a foundation of mathematics. An important change is that functions become primitive objects which means functions are types defined inductively using the $\rightarrow$ type former. It is widely applied to
various fields especially computer science. Some languages are extentions of lambda calculus, for example Haskell. Haskell belongs to one of the variants of lambda calculus called System F, although it has evovled into System FC recently. There are also other refinements of lambda calculus which is illustrated by the $\lambda$-cube \cite{barendregt1991introduction}.


\subsection{Per Martin-L\"{o}f's Type Theories}

In 1970s, Per Martin-L\"{o}f \cite{per:71,per:82}  developed his profound intuitionistic type theory. His 1971's formulation which is impredicative was proved to be inconsistent with the Girard's paradox \cite{hurkens1995simplification}. The impredicativty means that for a universe $\mathsf{U}$, there is an axiom $\mathsf{U} \in \mathsf{U}$. The later version is predicative and is more widely used.

Like set theory, it can also serve as also a foundation of constructive mathematics \cite{martin1984intuitionistic}. Different to set theory whose axioms are based on first-order logic or intuitionistic logic, \mltt provides a means of implementing intuitionistic logic. This is achieved by the Curry-Howard
isomorphism:

\emph{``propositions can be interpreted as types and their
  proofs are inhabitants of that types''} 

\todo{more explanation}

Different to simply lambda calculus, dependent types are introduced.

\begin{definition}\label{dpty}
\textit{Dependent type}. Dependent types are types that depends on values of other types \cite{dtw}. 
\end{definition}

With dependent types, the quantifiers like $\forall$ and $\exists$ can be encoded.
The Curry-Howard isomorphism is then extended to predicate logic. 
A predicate on $X$ can be written as a dependent type $P x$ where $x : X$. 

%definitional equality -- intensional equality
There are also two variants of \mltt based on the treatment of equality.
The notion of equality is one of the most profound topic in type theory.
we have two kinds of equality, one is definitional equality, the other is propositional equality.

\begin{definition}
\textit{definitional equality} definitional equality is a judgement-level equality, which holds when two objects have the same normal forms\cite{nor:90}.
\end{definition}

% Objects are definitional equal if they normalise to the same form. 
% Usually types like $a \equiv b$ stands for a definitional
% equality between $a$ and $b$. It is some primitive jundgements which are part of the
% meta-theory rather than construction like types or terms . Definitional
% equality can be judged and decided by type-checker. 


With dependent types, it is possible to write a type to encode the equality of objects.

\begin{definition}
\textit{Propositional equality} Propositional equality is a type which represents a propostion that two objects of the same type are equal.
\end{definition}

Intuitively if two objects are definitionally equal, they must be propositionally equal.

\begin{equation*}
\infer[Id-intro]{a = b}{a \equiv b}
\end{equation*}

But how about the other way around? Are two propositional equal objects definitional equal?


In \itt, the answer is no. Propositional equality (also called intensional equality  \cite{nor:90}) is different to definitional equality. 
The definitional equality is always decidable hence type checking that depends on definitional equality is
decidable as well~\cite{alti:lics99}. Therefore \itt has better computational behaviors.
Types like $\mathsf{a = b}$ which stands for a
propostion that $\mathsf{a}$ equals $\mathsf{b}$ are propositional equalities. They are some types which we need to prove
or disprove by construction. Each of them has an unique element $\mathsf{refl}$ which only exists if $\mathsf{a}$ and $\mathsf{b}$ are
definitionally equal in all cases. However it is not enough for other extensional equalities, for example the equality of functions.


In \ett, the propositional equality is extensional and is undistinguished with definitional equality, in other words, two propositional equal objects are judgementally equal. This is called reflecition rule.

\begin{equation}
\label{reflection}
\infer[Reflection]{a \equiv b}{a = b}
\end{equation}

Objects of different normal forms, for example point-wise equal functions or different proofs of the same proposition, may be definitionally equal. This is called functional extensionality.

\begin{equation}
\label{fun-ext}
\infer[functional-extensionality]{f = g}{f g : A \rightarrow B, \forall a : A, f \, a = g \, a}
\end{equation}

It is provable in \ett.

\begin{lemma}
\label{functional extensionality is available in ett}
We can prove \ref{fun-ext} in \ett where we have \ref{reflection}.
\end{lemma}
 
\begin{proof}
Suppose $\Gamma \vdash f \,a = g \,a$, with reflection rule we have $\Gamma \vdash f \,a \equiv g \,a$.
Then using $\xi$-rule, we know that $\Gamma \vdash \lambda a . f \,a \equiv \lambda a . g \,a$.
From $\eta$-equivalence, we know that $\Gamma \vdash f \equiv g$. We can conclude that $\Gamma \vdash f = g$.
\end{proof}

In \itt, this is not provable. If we add it as an axiom, we will lose the canonicity since we can construct a natural number with this extensionality and substitution for propositional equality. It means that the definitional equality is not decidable and type checking becomes undecidable as well. The type checker will not always terminate.


Usually we have to make a choice of them, either the better adpation of extensional equality or better computational behaviors. Agda chooses the intensional one while NuPRL chooses the extensional one. However
Altenkirch and McBride introduced a variant of \ett called
\emph{Observational Type Theory}  \cite{alt:06} in which definitional equality is
decidable and propositional equality is extensional.


\mltt can be encoded as programming languages in
which the evaluation of a well-typed program always terminates \cite{nor:90}.
There are various implementations based on different variants of it, such as
NuPRL, LEGO, Coq, Epigram and Agda. Since we usually discuss in \itt, we will use Agda throughout this thesis.

\section{Agda}



Agda is a dependently typed functional programming language which is designed based on intensional version
of \mltt{} \cite{agdawiki:main}. 

As we have seen, \mltt{} is based on the Curry-Howard
isomorphism: types are identified with propositions and terms (or programs) are identified with proofs. It turns Agda is into a proof assistant like Coq, which allows users to do mathematical reasoning. We can also reason about Agda programs inside itself.
Usually to prove the correctness of programs, we need to state some theorems of programming languages on the meta-level, but in Agda we can prove and use these theorems alongwith writing programs.

There are more features of Agda as follows:

\begin{itemize}
\item \textit{Dependent type}. 
As mentioned in \ref{dpty}, dependent types are types that depends on values of other types \cite{dtw}. They enable us to write more expresive types as program speficication or propositions in order to reduce bugs. In Haskell and other Hindley-Milner style languages, types and values are clearly distinct \cite{tutorial}, In Agda, we can define types depending on values which means the border between types and values is vague. To illustrate what this means, the most common example is $\mathsf{Vector A n}$ where we can length-explicit lists called vectors. It is a data type which represents a vector containing elements of type \textbf{A} and depends on a natural number \textbf{n} which is the length of the list. We can specify types with more constraints such that the we can express what programs we can better and leave the checking work to the type chekcer. For instance, to use the length-explicit vector, we will not encounter exceptions like out of bounds in Java, since it is impossible to define such functions before compiling.

\item \textit{Functional programming language}. As the name indicates that, functional programming languages emphasizes the application of functions rather than changing data in the imperative style like C{}\verb!++! and Java. The base of functional programming is lambda calculus. The key motivation to develop functional programmming language is to eliminating the side effects which means we can ensure the result will be the same no matter how many times we input the same data. There are several generations of functional programming languages, for example Lisp, Erlang, Haskell etc. Most of the applications of them are currently in the academic fields, however as the functional programming developed, more applications will be explored.

% \item \textit{Per Martin-Löf Type Theory}. It has different names like Intuitionistic type theory or Constructive type theory and is developed by Per Martin-Löf in 1980s. It associated functional programs with proofs of mathematical propositions written as dependent types. That means we can now represent propositions we want to prove as types in Agda by dependent types and Curry-Howard isomorphism \cite{aboa}. Then we only need to construct a program of the corresponding type to prove that propostion. For example:

% As Nordström et al. \cite{nps} pointed out that we could express both specifications and programs at the same time when using the type theory to construct proofs using programs.

\item \textit{A proof assistant} Based on the Curry-Howard isomorphism, we have predicate logic available in Agda and we can prove mathematical theorems and theorems about the programs encoded in Agda itself. 

The general approach to do theorem proving in Agda is as follows: First we give the name of the proposition and encode it as the type. Then we can gradually refine the goal to formalise a type-correct program namely the proof. As long as we have the proof, it can be used as a lemma in other proofs or programs. Usually, there are no tactics like in Coq (it may be implemented in the future). But with the gradually refinement mechanism, the process of building proofs is very similar to conceiving proofs in regular mathematics.

\end{itemize}

As a functional programming languages, Agda has some nice features for theorem proving,

\begin{itemize}

\item \textit{Pattern matching}. The mechanism for dependently typed pattern matching is very powerful \cite{alti:pisigma-new}. We could prove propositions case by case. In fact it is similar to the approach to prove propositions case by case in regular mathematics. Pattern match is a more intuitive way to use the eliminators of types.

\item \textit{Inductive \& Recursive definition}. In Agda, types are often defined inductively, for example, natural numbers is defined as

\begin{code}\>\<%
\\
\>\AgdaKeyword{data} \AgdaDatatype{ℕ} \AgdaSymbol{:} \AgdaPrimitiveType{Set} \AgdaKeyword{where}\<%
\\
\>[0]\AgdaIndent{2}{}\<[2]%
\>[2]\AgdaInductiveConstructor{zero} \AgdaSymbol{:} \AgdaDatatype{ℕ}\<%
\\
\>[0]\AgdaIndent{2}{}\<[2]%
\>[2]\AgdaInductiveConstructor{suc} \<[7]%
\>[7]\AgdaSymbol{:} \AgdaSymbol{(}\AgdaBound{n} \AgdaSymbol{:} \AgdaDatatype{ℕ}\AgdaSymbol{)} \AgdaSymbol{→} \AgdaDatatype{ℕ}\<%
\\
\>\<\end{code}

The function for inductive types are usually written in recursive style, for example, the double function for natural numbers,

\begin{code}\>\<%
\\
\>\AgdaFunction{double} \AgdaSymbol{:} \AgdaDatatype{ℕ} \AgdaSymbol{→} \AgdaDatatype{ℕ}\<%
\\
\>\AgdaFunction{double} \AgdaInductiveConstructor{zero} \AgdaSymbol{=} \AgdaInductiveConstructor{zero}\<%
\\
\>\AgdaFunction{double} \AgdaSymbol{(}\AgdaInductiveConstructor{suc} \AgdaBound{n}\AgdaSymbol{)} \AgdaSymbol{=} \AgdaInductiveConstructor{suc} \AgdaSymbol{(}\AgdaInductiveConstructor{suc} \AgdaSymbol{(}\AgdaFunction{double} \AgdaBound{n}\AgdaSymbol{))}\<%
\\
%
\\
\>\<\end{code}

The availability of recursive definition enables programmers to prove propositions in the same manner of mathematical induction. 

\item \textit{Construction of functions}. One of the advantage of using a functional programming language as a theorem prover is the construction of functions which makes the proving more flexible.

In functional programming languages, complicated programs are commonly built gradually using aunxiliary functions and frequently used functions in the library.

Described as a proof assistant, complicated theorems are commonly proved gradually using lemmas and other theorems we have proved.

This decreases the difficulty of interpreting proofs in mathematics into Agda.

\item \textit{Lazy evaluation}. Lazy evaluation could eliminate unecessary operation because Agda is lazy to delay a computation until we need its result. It is often used to handle infinite data structures. \cite{wiki:Lazy_evaluation}

\end{itemize}

Agda also has some special functions in its interactive emacs interface beyond simple functional programming languages.

\begin{itemize}
\item \textit{Type Checker}. Type checker is an essential part of Agda. You can use to to type check a file without compiling it. It is the type checker that detect type mismatch problem and for theorem proving, it means the proof is incorrect. It interactively shows the goals, assumptions and variables when buiding a proof. 

The coverage checker makes sure that the patterns cover all possible cases \cite{aboa}. 

The termination checker will warn possiblily non-terminated error. The missing cases error will be reported by type checker. The suspected non-terminated definition can not be used by other ones. All programs must terminate in Agda so that it will not crash \cite{tutorial}.  The type checker then ensures that the proof is complete and not been proved by itself. 

In Agda, type signatures are essential due to the presence of type checker.
 
\item \textit{Interactive interface}. It has a Emacs-based interface for interactively writing and verifying proofs.  With type checker we can refine our proofs step by step \cite{aboa}. It also has some convenient functions and emacs means the potential to be extended.

\item \textit{Unicode support}. In Haskell and Coq, unicode support is not an essential part. However in Agda, to be a better theorem prover, it reads unicode symbols like: $\beta$, $\forall$ and $\exists$ and supports mixfix operators like: $+$ and $-$, which are very common for mathematics. It provides more meaningful names for types and lemmas and more flexible way to define operators. This also improve the readablity of the Agda proofs. For example, the commutativity of plus for natural numbers can be encoded as follows

\begin{code}
%
\\
\>\AgdaFunction{comm} \AgdaSymbol{:} \AgdaSymbol{∀} \AgdaSymbol{(}\AgdaBound{a} \AgdaBound{b} \AgdaSymbol{:} \AgdaDatatype{ℕ}\AgdaSymbol{)} \AgdaSymbol{→} \AgdaBound{a} \AgdaFunction{+} \AgdaBound{b} \AgdaDatatype{≡} \AgdaBound{b} \AgdaFunction{+} \AgdaBound{a}\<%
\\
\end{code}

We can use symbols we are familiar in regular mathematics.



Secondly we could use symbols to replace some common-used properties to simply the proofs a lot. The following code was simplied using several symbols,


Finally, we could use some other languages characters to define functions such as Chinese characters.

\item \textit{Code navigation}. As long as a program is loaded, it provides shortcut keys to move to the original definitions of certain object and move back. In real life programming it alleviates a great deal of work of programmers to look up the library.

\item \textit{Implicit arguments}. Sometime it is unnessary to write an argument since it can be inferred from other arguments by the type checker. It can simplify the application of functions and make the programs more concise. For example, to define a polymorphic function $\mathsf{id}$,

\begin{code}\>\<%
\\
\>\AgdaFunction{id} \AgdaSymbol{:} \AgdaSymbol{\{}\AgdaBound{A} \AgdaSymbol{:} \AgdaPrimitiveType{Set}\AgdaSymbol{\}} \AgdaSymbol{→} \AgdaBound{A} \AgdaSymbol{→} \AgdaBound{A}\<%
\\
\>\AgdaFunction{id} \AgdaBound{a} \AgdaSymbol{=} \AgdaBound{a}\<%
\\
\end{code}

Whenever we give an argument $\mathsf{a}$,  its type $\mathsf{A}$ must be inferable.

\item \textit{Module system}. The mechanism of parametrised modules makes it possible to define generic operations and prove a whole set of generic properties.

\item \textit{Coinduction}. We can define coinductive types like streams in Agda which are typically infinite data structures. Coinductive occurences must be labelled with $\infty$ and coninductive types do not need to terminate but has to be productive. It is often used in conjunction with lazy evaluation. \cite{wiki:Coinduction}
	
\end{itemize} 

With these helpful features, Agda is a very powerful proof assisstant. It does not magically prove theorems for people, but it really helps mathematicians and computer scientists to do formalised reasoning with verification by high-performance computers. 


\subsection{basic syntax}
\todo{to help the reader understand the Agda code in text}

To understand the code in this thesis, I will introduce some basic types and syntax of Agda.

First of all, like other languages, we use "$\AgdaSymbol{=}$" for function definition rather than propositional equality type former which is  "$\AgdaDatatype{≡}$". This is inconsistent with our conventional choices of symbols in articles, but it follows the conventions in Haskell and other programming languages that "$\AgdaSymbol{=}$" is used for definition.

Different to Haskell, we use single colon $\AgdaSymbol{:}$ for typing judgement, for example $\AgdaFunction{a} \AgdaSymbol{:} \AgdaDatatype{A}$ means that $\AgdaFunction{a}$ is of type $\AgdaDatatype{A}$. 

We have universe levels parameters in a lot of definitions which makes code looks unnecessarily cumbersome. We will follow the \textbf{typical ambiguity} in this thesis which says that we write $\AgdaBound{A} \AgdaSymbol{:} \AgdaPrimitiveType{Set}$ for $\AgdaBound{A} \AgdaSymbol{:} \AgdaPrimitiveType{Set} \AgdaBound{a}$ and $\AgdaPrimitiveType{Set} \AgdaSymbol{:} \AgdaPrimitiveType{Set}$ which stands for $\AgdaPrimitiveType{Set}\AgdaBound{i} \AgdaSymbol{:} \AgdaPrimitiveType{Set}\AgdaBound{(i+1)}$.

The underscore marks the spaces for the explicit arguments in non-prefix operators.

\subsection{Identity Type}

Identity type is the type introduced by Martin-L\"{o}f to encode the propositional equality for definitionally equal terms \cite{nor:90}. For any two terms of $\mathsf{a\,b : A}$, we have the type $\mathsf{Id (A , a , b)}$ which is inhabitted when $\mathsf{a}$ and $\mathsf{b}$ are definitionally equal. Here we use an alternative equivalent version named after Paulin-Mohring which is parameterized with the left side of the identity. %This also includes the identity type for arbitrary universe level.

\begin{code}%
\\
\>\AgdaKeyword{data} \AgdaDatatype{\_≡\_} \AgdaSymbol{\{}\AgdaBound{A} \AgdaSymbol{:} \AgdaPrimitiveType{Set}\AgdaSymbol{\}} \AgdaSymbol{(}\AgdaBound{x} \AgdaSymbol{:} \AgdaBound{A}\AgdaSymbol{)} \AgdaSymbol{:} \AgdaBound{A} \AgdaSymbol{→} \AgdaPrimitiveType{Set} \AgdaKeyword{where}\<%
\\
\>[0]\AgdaIndent{2}{}\<[2]%
\>[2]\AgdaInductiveConstructor{refl} \AgdaSymbol{:} \AgdaBound{x} \AgdaDatatype{≡} \AgdaBound{x}\<%
\\
%
\end{code}

In Agda eliminators are not automatically derived for the types defined. Instead we have pattern matching generally which is sometimes stronger than eliminators.
As long as we pattern match on a variable of an identity type with the unique inhabitant $\AgdaInductiveConstructor{refl}$, all occurences of both variables become the same.
It is stronger and it provides the eliminator J.

\begin{code}
%
\\
\>\AgdaFunction{J} \AgdaSymbol{:} \AgdaSymbol{(}\AgdaBound{A} \AgdaSymbol{:} \AgdaPrimitiveType{Set}\AgdaSymbol{)(}\AgdaBound{a} \AgdaSymbol{:} \AgdaBound{A}\AgdaSymbol{)} \AgdaSymbol{→} \AgdaSymbol{(}\AgdaBound{P} \AgdaSymbol{:} \AgdaSymbol{(}\AgdaBound{b} \AgdaSymbol{:} \AgdaBound{A}\AgdaSymbol{)} \AgdaSymbol{→} \AgdaBound{a} \AgdaDatatype{≡} \AgdaBound{b} \AgdaSymbol{→} \AgdaPrimitiveType{Set}\AgdaSymbol{)}\<%
\\
\>[0]\AgdaIndent{2}{}\<[2]%
\>[2]\AgdaSymbol{→} \AgdaBound{P} \AgdaBound{a} \AgdaInductiveConstructor{refl}\<%
\\
\>[0]\AgdaIndent{2}{}\<[2]%
\>[2]\AgdaSymbol{→} \AgdaSymbol{(}\AgdaBound{b} \AgdaSymbol{:} \AgdaBound{A}\AgdaSymbol{)(}\AgdaBound{p} \AgdaSymbol{:} \AgdaBound{a} \AgdaDatatype{≡} \AgdaBound{b}\AgdaSymbol{)} \AgdaSymbol{→} \AgdaBound{P} \AgdaBound{b} \AgdaBound{p}\<%
\\
\>\AgdaFunction{J} \AgdaBound{A} \AgdaSymbol{.}\AgdaBound{b} \AgdaBound{P} \AgdaBound{m} \AgdaBound{b} \AgdaInductiveConstructor{refl} \AgdaSymbol{=} \AgdaBound{m}\<%
\\
%
\end{code}

\subsection{Extensionality}

In regular mathematics, equality does not only exists between intensionally equal terms. Pointwise equal functions are usually identified and it is called functional extensionality as we mentioned \ref{fun-ext}. Therefore the identity type in \itt is not powerful enough, we need extensionality in \itt.

As Martin Hofmann summarises in \cite{hof:phd}, there are several related extensional concepts, \emph{Functional extensionality}, \emph{Uniqueness of identity}, \emph{Proof-irrelevance}, \emph{Subset types}, \emph{Propositional extensionality}, \emph{Quotient types}. These notions are not available in currect setting of \itt{}, but they are worth interpreting to help both Mathematics and programs constructions. However, it only makes sense if the type-checking decidability and terms canonicity are not sacrificed.

Quotient types is one of the most interesting extensional concepts in type theory. It generally enables us to redefine equality on a type with a given equivalence relation so that we have more tools to formalise mathematical objects, like the real numbers.


\section{Homotopy Type Theory}

\hott is a variant of intensional \mltt{} which is a new branch developed between theoretical computer science
and mathematics. Vladimir Voevodsky found a surprising connection between homotopy theory and type theory \cite{voe:06}. He proposed the univalence axiom, which identifies isomorphic structures, as a univalent foundation for mathematics. 

\begin{definition}
\emph{univalence axiom}. for any two types $X$ and $Y$, 
$$X = Y \simeq X \simeq Y$$ holds.
\end{definition}

In \hott, we interpret the notions in type theory with homotopy-theoretical terms. A type is regarded as a \emph{space} and a term of this type is a \emph{point} of this space. Functions between types are \emph{continous maps} and identity types are usually considered as \emph{paths}. Identity types of identity types are \emph{homotopies}. Although these notions are originally defined with topological bases, we only employ them as homotopical notions on a higher level. 
It does not only help us model type theory with a focus on the equality, but also provides mathematicians type theoretical tools to study homotopy theory.


 which is the universe, and identity types are just the paths. 

I will not explain this topic in detail here, \todo{I will!!}


A well-written text book on Homotopy Type Theory which is written by many brilliant mathematicians and computer
scientists is available now \cite{hott}. 








It is possible to
define the quotient types in Homotopy Type Theory but to implement the
Homotopy Type Theory in \itt{}, it is still a difficult problem. We
work on defining semi-simplicial types and \wog{} to solve it. 





\section{Some conventions}

\todo{What conventions we will follow in this thesis and some commonly
used symbols and function}

The universe of small types is encoded as $\Set_{0}$ or $\Set$ rather than $\Type$, even though it is not a set in set-theoretical sense.




\section{literature review}


In \cite{cab}, Mendler et al. have firstly considered building new types from a
given type using a quotient operator $//$. Their work is based on an
implementation of \ett, NuPRL. In NuPRL, every type
comes with its own equality relation, so the quotient operator can be
seen as a way of redefining equality in a type. But it is not all
about building new types. They also discuss problems that arise from
defining functions on the new type which can be illustrated using a simple example. 

Assume the base type is $A$ and the new equivalence relation is $E$, the new
type can be formed as $A//E$. 

When we want to define a function $f \,\colon\, A//E \to Bool$,  $f\,a \not= f\,b$ may
exists for $a, b \,\colon A$ such that $E\,a\,b$. This will lead to
inconsistency since $E\,a\,b$ implies $a$ converts to $b$ in \ett{}, hence
the left hand side $f\,a$ can be converted to $f\,b$, namely we get $f\,b \not= f\,b$
which is contradicted with the equality reflection rule. 

Therefore a function is said to be well-defined \cite{cab} on the new type only
if it respects the equivalence relation $E$, namely

$$\forall \, a\,b\,\colon A, E\,a\,b \to f\,a = f\,b$$

We call this \emph{soundness} property in \cite{aan}.

 After the introduction of quotient types, Mendler further investigates
 this topic from a categorical perspective in ~\cite{men:90}. He uses
 the correspondence between quotient types in \mltt{} and coequalizers
 in a category of types to define a notion called \emph{squash types},
 which is further discussed by Nogin \cite{nog:02}.

To add quotient types to \mltt{}, Hofmann proposes  three models for
quotient types in his PhD thesis \cite{hof:phd}. The first one is a setoid model for
quotient types. In this model all types are attached with partial
equivalence relations, namely all types are setoids rather than
sets. Types without a specific equivalence relation can be seen as
setoids with the basic intensional equality. This is similar to
\ett{} in some sense. The second one is groupoid model which solves some problems
but it is not definable in \itt{}. He also proposes a third model to
combine the advantages of the first two models, but it also has some
disadvantages. Later in \cite{hof:95:sm} he gives a simple model in which we have type dependency only at the propositional level, he also shows that extensional Type Theory is conservative over \itt  extended with quotient types and a universe \cite{hof:95:con}.

Nogin \cite{nog:02} considers a modular approach to axiomatizing the
same quotient types in NuPRL as well. Despite the ease of constructing new types
from base types, he also discusses some
problems about quotient types. For example, since the equality is
extensional, we cannot recover the
witness of the equality.  He suggests including more axioms to
conceptualise quotients. He decomposes the formalisation of quotient type
into several smaller primitives such that they can be handled much
simpler.

Homeier \cite{hom} axiomatises quotient types in Higher Order Logic
(HOL), which is also a theorem prover. He creates a tool package to
construct quotient types as a conservative extension of HOL such that
users are able to define new types in HOL. Next he defines the
normalisation functions and proves several properties of
these. Finally he discussed the issues when quotienting on the
aggregate types such as lists and pairs.


Courtieu \cite{cou:01} shows an extension of Calculus of Inductive Constructions
with \emph{Normalised Types} which are similar to quotient types, but equivalence relations are replaced by normalisation functions. 
However not all quotient types have normal forms. Normalised types are
proper subsets of quotient types, because we can easily recover a quotient
type from a normalised type as below
%$$ \[ (A, Q, \class\dotph \colon A \to Q) \rightarrow(A, \lambda \,a \,b \to \class a = \class b)\]$$


Barthe and Geuvers \cite{bar:96} also propose a new notion called
\emph{congruence types}, which is also a special class of quotient
types, in which the base type are inductively defined and with a set
of reduction rules called the term-rewriting system. The idea behind
it is the $\beta$-equivalence is replaced by a set of
$\beta$-conversion rules. Congruence types can be treated as an
alternative to the pattern matching introduced in \cite{coq:92}. The main
purpose of introducing congruence types is to solve problems in
term rewriting systems rather than to implement quotient types.


Barthe and Capretta \cite{bar:03} compare different ways to setoids in type theory.
The setoid is classified as partial setoid or total setoid depending
on whether the equality relation is reflexive or not. They also
consider obtain quotients with different kinds of setoids, especially
the ones from partial setoids are difficult to define because the lack
of reflexivity.

Abbott, Altenkirch et al. \cite{abb:04} provides the basis for
programming with quotient datatypes polymorphically based on their
works on containers which are datatypes whose instances are
collections of objects, such as arrays, trees and so on. Generalising
the notion of container, they define quotient containers as the
containers quotiented by a collection of isomorphisms on the positions
within the containers.

Voevodsky \cite{voe:hset} implements quotients in Coq based on a set
of axioms of Homotopy Type Theory. It is based on the groupoid model
for \itt{} where isomorphisms are equalities. He firstly implement
equivalence class and use it to implement quotients which is an
analogy to the construction of quotient sets in set theory. 
