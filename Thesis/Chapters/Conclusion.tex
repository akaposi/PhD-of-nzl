\chapter{Conclusion and Future Work}

We presented the evolution of type theories focusing on \mltt (Type
Theory) and discussed different variants.  We compared two versions of
Type Theory: \ett (ETT) and \itt (ITT). ITT has decidable type
checking but lacks some extensional concepts such as functional
extensionality and quotient types. On the other hand, ETT has equality
reflection which provides these extensional concepts but makes type
checking undecidable due to the identification of propositional and
definitional equalities.


The notion of quotient types is one of the important extensional
concepts which facilitates mathematical and programming
constructions. An interesting question is whether ITT could be
extended with quotient types. We presented a definition of quotient
types in a type theory with a proof-irrelevant universe, and we showed
that simply adding the rules of quotient types to \itt as axioms
results in the loss of the $\N$-canonicity property. We also clarified
the correspondence with coequalizers in $\Set$ and a left adjoint
functor in category theory.


We discussed the definability of a normalisation function for a given
quotient represented as a setoid. For quotients which can be defined
inductively with a normalisation function e.g.\ the set of integers
and the set of rational numbers, we proposed an algebraic structure to
bridge the gap between the setoid representations and the set
definitions. We showed that the application of a definable quotient
structure can improve the constructions by keeping good properties of
both representations. As definable quotients can be seen as a
simulation of quotient types, we expect similar benefits from using
quotient types.

An interesting future project is the further development of the
implementation of numbers in Agda using the definable quotient
structure. It could be extended to other definable quotients
implementable in our algebraic quotient structures. This would make
the Agda standard library more convenient to use for mathematical
applications. Another possibility is the extension of Agda with
normalised types \cite{cou:01}, that is, building a special case of
quotient types with respect to a normalisation function in the sense
of \Cref{def:nor}.

Although a quotient type former is not necessary for definable
quotients, it seems indispensable for some other quotients which don't
have a definable normalisation function. With the assumption that
Brouwer's continuity holds in the meta-theory, we proved that there is
no definable normalisation function for Cauchy reals
$\qset{\R_{0}}$. Other examples include the partiality monad and
finite multisets.  In the future, we would like to investigate the
definability of quotients in general, and in particular, we would like
to find out whether the non-existence of a normalisation function for
a quotient implies that it is not definable as a set in general.


A way of introducing quotient types in \itt without losing good
computational properties is building models where types are
interpreted as sets with an internally defined equality, such as
setoids, groupoids or \wog. We have developed an implementation of
Altenkirch's setoid model in Agda, and explained our construction of
quotient types inside it.


There are more open research questions regarding the setoid model, for
example the verification of certain properties or the definition of a
type of propositions for which we can write the type of equivalence
relations using $\Pi$-types. A simplification would be the usage of
heterogeneous equality as discussed in \Cref{models}. One could also
consider the usage of h-propositions instead of a universe of
propositions in the metatheory. However $\Pi$-closure of
h-propositions needs functional extensionality. It would be
interesting to compare this approach with the one we have
presented. It is also worthwhile to extend the setoid model with
examples of quotients like the set of real numbers and finite
multisets which are not definable via normalisation.  Other
extensional concepts and coinductive types can also be considered in
the setoid model.


We also investigated another extension of \mltt -- \hott. In \hott,
types are interpreted as \wog which are generalizations of
groupoids. We discussed quotients in \hott. With univalence, quotients
can be defined impredicatively. We can also define quotients using
higher inductive types (HITs), and in fact HITs can be seen as
"generalized quotient types".  Therefore a computational
interpretation of \hott can also be seen as a way of adding quotient
types to \itt.

We showed a syntactic construction of \wog in Agda as a first step
towards building a weak $\omega$-groupoid model of Type Theory. We
defined the type theory \tig which describes the coherence conditions
of a \wogs required for a globular set. Inside this theory, we showed
how to reconstruct some coherences laws, for example the groupoid laws
using suspensions and replacement techniques. Here we also used
heterogeneous equality for terms to ease implementation.

There are further interesting questions regarding our syntactic
framework. For instance, we would like to investiage the relation
between the \tig and a type theory with equality types and the $\J$
eliminator which is called $\mathcal{T}_{eq}$. One direction is to
simulate the $\J$ eliminator syntactically in \tig as we mentioned
before, the other direction is to derive $\J$ using $\mathsf{coh}$ if
we can prove that the $\mathcal{T}_{eq}$ is a weak $\omega$-groupoid.
The syntax could be simplified by adopting categories with
families. An alternative way may be the usage of higher inductive
types to formalize the syntax of type theory.

When attempting to prove that $\AgdaFunction{Idω}$ is a weak
$\omega$-groupoid, we encountered the problem that the base set in a
globular set is an h-set which is incompatible with
$\AgdaFunction{Idω}$. Altenkirch suggests \cite{CoherenceProblem} a
solution using a universe with extensional equality, and Agda's
propositional equality as strict equality so that we can define
$\AgdaFunction{Idω}$ as a globular set in this universe. Finally,
modelling Type Theory with weak $\omega$-groupoids and thus
eliminating the univalence axiom would be the most challenging task to
do in the future.

It would also be interesting to consider quotient \emph{types} in
\hott.  The notion of quotient types we considered in this thesis
refers to the quotients with a \emph{propositional} equivalence
relation. However in a type theory with higher dimensions, like \hott,
the notion of quotient types can be more general and we would like to
consider non-propositional quotients, for example, the quotient of a
set by a groupoid.
