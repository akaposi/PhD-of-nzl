\chapter{Definable Quotients}
\label{dq}

In \itt~\cite{nor:90}, quotient
types are unavailable and setoids are sometimes used to represent quotients~\cite{bar:03}
.
However, since setoids is considered to be an alternative encoding of sets,
it is necessary to redefine the same operations again on setoids e.g.\
function types, equality types, lists and ``setoids'' defined on setoids.
Moreover, setoids are not safe in the sense
that any consumer of a setoid may access the underlying
representation. 
In some practical cases the quotient sets are
themselves definable without using quotient types.
The relation between the setoids and these quotient sets are
implementable as algebraic structures via a
normalisation function.
In this chapter we introduce one of the main result in this thesis --
\emph{definable quotients} along with examples e.g. the set
of integers and the set of rational numbers. We will also study the
application of definable quotients to illustrate some advantages of
using them.


\section{?}

Given a setoid, a corresponding quotient set is not necessarily formed by quotient
types. An example is the set of integers. It can be seen as a quotient set on
the setoid given by pairs of natural numbers
$\Z_0=\N\times\N$, where the equivalence relation identifies pairs
representing the same difference, that is  $(a,b)\sim(c,d)$ if{f}
$a+d=c+b$.
However a setoid is unnecessary, if we interpret it as a natural
number with a sign and a set can be defined inductively.
We can now define operations like
addition and multiplication and show algebraic properties,
such as verifying that the structure is a ring. However, this is
quite complicated and uses many unnecessary case
distinctions. E.g. the proving of distributivity within this setting
is not satisfactory since too many cases has to be proved from
scratch. We found it is often easier to define the operations on the
setoid and the required algebraic properties are direct consequences
of the semiring structure of the natural numbers.
Another common example is the set of rational numbers. It makes more
sense to treat a rational number as an equivalence class of fractions
which can be reduced to the same one. The benefits of being a quotient
is more clear. The coprimarity has to been kept and it complicates
defining operations and proving properties.
In these cases, the property of being a quotient is not revealed in the
definition.

In such a situation, both the set interpretation and the setoid one
have some nice features.
Hence we propose to use both the setoid
and the associated set, but to use the setoid structure to define
operations on the quotient set and to reason about it.
In the present paper we introduce the formal framework to define the
the quotients via normalisation function in Type Theory.
It can be seen a ``manual construction'' of quotient types, in other words,
instead of creating a type immediately given a setoid, we
prove another given type \emph{is} the quotient.
The quotient structure provides conversions between base types and
''definable quotient types''. For example functions which respect
the equivalence relation can be lifted. It can alleviate repetitive work on
definable quotient types, because the base types usually have a
list of available functions and properties to use. In some cases the
base types are simpler to manipulate, which give us a shortcut to
handle the quotient types. 



. 

\section{Definable quotient structure}

Quotient types are considered to be meta-theoretical extension of Type
Theory, but the definable quotient structure is just defined in Type
Theory without adding any axioms. We are going to give the definition
which has been implemented in Agda and can be found in the appendix.

We start from the definition of \emph{setoid} which provides the setoid
interpretation of the "quotient''.


\begin{definition}
Setoid.
\noindent A setoid is formed by
\begin{enumerate}
\item a set $A$,
\item a relation $\sim : A \to A \to \Prop$, and
\item it is an equivalence, i.e. it is reflexive, symmetric and transitive.
\end{enumerate}
\end{definition}

Given a setoid and a function respects this equivalence (note: not necessary to be
a normalisation function) we obtain a \emph{prequotient}.

\begin{definition}
Prequotient.
\noindent
Given a setoid $(A,\sim)$,  a \emph{prequotient} $(Q,[\_],\sound)$ over that setoid consists in
\begin{enumerate}
\item \label{enum:Q} a set $Q$,
\item \label{enum:box}a function $[\_]: A \rightarrow Q$,
\item \label{enum:sound} a proof \emph{sound} that  the function $[\_]$ is compatible with the relation $\sim$,
that is \[\sound\colon (a,b : A) \rightarrow a\sim b \rightarrow [a] = [b],\]
\end{enumerate}
\end{definition}

Roughly speaking, \ref{enum:Q} corresponds to the formation rule,
\ref{enum:box} and \ref{enum:sound} correspond to the introduction
rule. The function $[\_]$ is intended to be the \emph{normalisation}
function with respect to the equivalence relation, however it is not
enough to determine it now. It is observable that given any non-empty set $Q$ with any
constant function fit in this definition. 

To complete a \emph{quotient}, we also need the elimination
rule and computation rule.

\begin{enumerate}
\setcounter{enumi}{3}
\item \label{enum:elim}
for any $B: Q\rightarrow\Set$, an eliminator
 \begin{align*}
 \qelim_B\,\,:\,\,\,&(f\colon (a:A) \rightarrow B\,\class a) \\
        {\rightarrow}\, &((p:a\sim b) \rightarrow f\,a \simeq_{\sound\,p}f\,b)\\
        {\rightarrow}\, &((q:Q) \rightarrow B\,q)
 \end{align*}
such that $\qelimbeta : \qelim_B\,f \,p\,\class a\equiv f\,a$.
\end{enumerate}
This eliminator is also called dependent lifting function

An eliminator is also called lifting function since it
lifts a function well-defined on the setoid to a
function defined on the quotient set. 
This is a dependent version because the target type in the function
may be dependent on the quotient set.
 
There is also an equivalent definition given by Martin Hofmann
which has a non-dependent eliminator 

\[\lift : (f : A \to B) \to (\forall a,b \to a\sim b \to f~a
\equiv f~b) \to (Q \to B)\]

combined with an induction principle.
Suppose $B$ is a predicate, 
\[\qind \colon((a: A)\rightarrow B \,\class a)\rightarrow ((q : Q)\rightarrow B\,q)\]

There is one more essential property required. The quotient is
\emph{exact} (or effective) if exactly one equivalence class corresponds to
an element of $Q$.

$exact :(\forall a,b : A) \rightarrow  \class a \equiv \class b \rightarrow a \sim b$


This is indeed provable if we have univalence axiom for
propositions. The proof can be found in \ref{PUEF}.



An alternative characterisation of an exact quotient is called
\emph{definable quotient}.

\begin{definition} Definable quotient.
Given a setoid $(A,\sim)$, a definable quotient is a prequotient $(Q, \class{\dotph}, \sound)$ with 
\begin{align*}
\emb &: Q \rightarrow A\\
\complete &: (a : A) \rightarrow \emb {\class a} \sim a\\
\stable &: (q:Q) \rightarrow \class{\emb\,q} \equiv q.
\end{align*}
\end{definition}

Intuitively speaking, $Q$ is the set of the equivalence classes and
the embedding function ``emb'' selects one
representative for each equivalent class. 
In fact, the embedding function can be seen as the lifted version of
the real normalisation function with respect to the equivalence relation $\sim$,

\begin{align*}
\text{emb}_0 &: A \rightarrow A\\
\text{embsound} &: (a : A) \to emb a \sim a\\
\text{oneForEach} &: (a,b : A) \to a \sim b \to \emb a = \emb b
\end{align*}

It can also be obtained by composing the ``normalisation'' function $\class\dotph$ with the
embedding function

Categorically speaking, it forms a \emph{split coequalizer}.

A definable quotient proves is an \emph{exact} quotient which is proved in \cite{aan}. In fact,
We can prove that it is equivalent to the exact quotient \footnote{The proof
can be found in Appendix}. 
This is an alternative and more
flexible way to eliminate the quotient type $Q$ by embedding
function.







\section{The set of integers}

Usually we can describe an integer is a natural number with a positive
or negative sign in front: 
\begin{code}
\\
\>\AgdaKeyword{data} \AgdaDatatype{ℤ} \AgdaSymbol{:} \AgdaPrimitiveType{Set} \AgdaKeyword{where}\<%
\\
\>[0]\AgdaIndent{2}{}\<[2]%
\>[2]\AgdaInductiveConstructor{+\_} \AgdaSymbol{:} \AgdaDatatype{ℕ} \AgdaSymbol{→} \AgdaDatatype{ℤ}\<%
\\
\>[0]\AgdaIndent{2}{}\<[2]%
\>[2]\AgdaInductiveConstructor{-\_} \AgdaSymbol{:} \AgdaDatatype{ℕ} \AgdaSymbol{→} \AgdaDatatype{ℤ}\<%
\\
\end{code}

However this gives two different constructor for 0 which is
considered harmful. We lose canonicity and it will lead to unnecessary troubles.

A better representation provides a separate symbol for 0 but $+suc
0$ for 1, $-suc 0$ for $- 1$. This does not sacrifice canonicity,
although the symbols does not so natural.

\begin{code}
\\
\>\AgdaKeyword{data} \AgdaDatatype{ℤ} \AgdaSymbol{:} \AgdaPrimitiveType{Set} \AgdaKeyword{where}\<%
\\
\>[0]\AgdaIndent{2}{}\<[2]%
\>[2]\AgdaInductiveConstructor{+suc\_} \AgdaSymbol{:} \AgdaDatatype{ℕ} \AgdaSymbol{→} \AgdaDatatype{ℤ}\<%
\\
\>[0]\AgdaIndent{2}{}\<[2]%
\>[2]\AgdaInductiveConstructor{zero} \<[8]%
\>[8]\AgdaSymbol{:} \AgdaDatatype{ℤ}\<%
\\
\>[0]\AgdaIndent{2}{}\<[2]%
\>[2]\AgdaInductiveConstructor{-suc\_} \AgdaSymbol{:} \AgdaDatatype{ℕ} \AgdaSymbol{→} \AgdaDatatype{ℤ}\<%
\\
\>\<\end{code}

Take into account the embedding of natural numbers into integers, it
makes sense to combine the positive integers with zeros. Actually it is
better pragmatically. In principle, definition with less
cases is preferable because pattern matching grows with exponentially
with the number of cases.

\begin{code}
\\
\>\AgdaKeyword{data} \AgdaDatatype{ℤ} \AgdaSymbol{:} \AgdaPrimitiveType{Set} \AgdaKeyword{where}\<%
\\
\>[0]\AgdaIndent{2}{}\<[2]%
\>[2]\AgdaInductiveConstructor{+\_} \<[8]%
\>[8]\AgdaSymbol{:} \AgdaDatatype{ℕ} \AgdaSymbol{→} \AgdaDatatype{ℤ}\<%
\\
\>[0]\AgdaIndent{2}{}\<[2]%
\>[2]\AgdaInductiveConstructor{-suc\_} \AgdaSymbol{:} \AgdaDatatype{ℕ} \AgdaSymbol{→} \AgdaDatatype{ℤ}\<%
\\
\>\<\end{code}

Although it is not symmetric, it is better than the others, so it is
our choice.

\subsection{The setoid of Integers}

The introduction of negative numbers is believed to denote the results
of subtraction of a larger number from a smaller one. Generalising this interpretation,
integers is used to represent the results of subtraction of any
natural number from another. 
Therefore integers can be represented by pairs of natural numbers

$$\Z_0=\N \times \N$$

for example, from the equation $1 - 4 = - 3$, we learn that $(1,4)$ gives rise to $- 3$.

Also different pairs can result in the same integer.
For any two pairs of natural
numbers $(n_1, n_2)$ and $(n_3, n_4)$, we know they represent the same
integer if

$$ n_1 - n_2 = n_3 - n_4$$

but technically this does not work because the ``subtraction'' defined for natural
numbers only returns zero if the pair is for negative number. We only
need to modify the equation a bit,

$$ n_1 + n_4 = n_3 + n_2$$

This gives rise to an equivalence relation (reflexive, symmetric and
transitive needs to prove separately). Combined with the carrier
$\Z_0$, it forms a setoid interpretation of integers.

\begin{code}
\\
\>\AgdaFunction{ℤ-Setoid} \AgdaSymbol{:} \AgdaRecord{Setoid}\<%
\\
\>\AgdaFunction{ℤ-Setoid} \AgdaSymbol{=} \AgdaKeyword{record}\<%
\\
\>[2]\AgdaIndent{3}{}\<[3]%
\>[3]\AgdaSymbol{\{} \AgdaField{Carrier} \<[19]%
\>[19]\AgdaSymbol{=} \AgdaFunction{ℤ₀}\<%
\\
\>[2]\AgdaIndent{3}{}\<[3]%
\>[3]\AgdaSymbol{;} \AgdaField{\_≈\_} \<[19]%
\>[19]\AgdaSymbol{=} \AgdaPostulate{\_∼\_}\<%
\\
\>[2]\AgdaIndent{3}{}\<[3]%
\>[3]\AgdaSymbol{;} \AgdaField{isEquivalence} \AgdaSymbol{=} \AgdaPostulate{\_∼\_isEquivalence}\<%
\\
\>[2]\AgdaIndent{3}{}\<[3]%
\>[3]\AgdaSymbol{\}}\<%
\\
\end{code}

\subsection{The definable quotient of integers}

The basic ingredients for the definable quotient of integers have been
given. One essential component of the quotient structure which relates
the base type and quotient type is a normalisation
function which can be recursively defined as follows:
 
\begin{code}
\>\AgdaFunction{[\_]} \<[18]%
\>[18]\AgdaSymbol{:} \AgdaFunction{ℤ₀} \AgdaSymbol{→} \AgdaDatatype{ℤ}\<%
\\
\>\AgdaFunction{[} \AgdaBound{m} \AgdaInductiveConstructor{,} \AgdaInductiveConstructor{0} \AgdaFunction{]} \<[18]%
\>[18]\AgdaSymbol{=} \AgdaInductiveConstructor{+} \AgdaBound{m}\<%
\\
\>\AgdaFunction{[} \AgdaInductiveConstructor{0} \AgdaInductiveConstructor{,} \AgdaInductiveConstructor{suc} \AgdaBound{n} \AgdaFunction{]} \<[18]%
\>[18]\AgdaSymbol{=} \AgdaInductiveConstructor{-suc} \AgdaBound{n}\<%
\\
\>\AgdaFunction{[} \AgdaInductiveConstructor{suc} \AgdaBound{m} \AgdaInductiveConstructor{,} \AgdaInductiveConstructor{suc} \AgdaBound{n} \AgdaFunction{]} \AgdaSymbol{=} \AgdaFunction{[} \AgdaBound{m} \AgdaInductiveConstructor{,} \AgdaBound{n} \AgdaFunction{]}\<%
\end{code}

In this case, we choose to define the embedding function before
proving the soundness property.


It is plausible to find a representative in each equivalence classes. 
In fact the definition of $\AgdaFunction{[\_]}$ already gives the
answer in the first two cases.


\begin{code}
\\
\>\AgdaFunction{⌜\_⌝} \<[11]%
\>[11]\AgdaSymbol{:} \AgdaDatatype{ℤ} \AgdaSymbol{→} \AgdaFunction{ℤ₀}\<%
\\
\>\AgdaFunction{⌜} \AgdaInductiveConstructor{+} \AgdaBound{n} \AgdaFunction{⌝} \<[11]%
\>[11]\AgdaSymbol{=} \AgdaBound{n} \AgdaInductiveConstructor{,} \AgdaNumber{0}\<%
\\
\>\AgdaFunction{⌜} \AgdaInductiveConstructor{-suc} \AgdaBound{n} \AgdaFunction{⌝} \AgdaSymbol{=} \AgdaNumber{0} \AgdaInductiveConstructor{,} \AgdaInductiveConstructor{suc} \AgdaBound{n}\<%
\\
\end{code}

To complete the definition of definable quotient, there are several
properties to prove including sound, complete, stable and exact
\footnote{see Appendix}.

Because we have shown that exact quotient is equivalent to definable
quotient, the lifting functions (dependent and non-dependent) are also derivable.

\subsection{The application of definable quotients}

Usually the definable quotient structure is useful when the base type (or carrier) is easier to handle, in other
words, it is more convenient to use the setoid interpretation in
defining operations and proving properties.
In the case of integers, Since there is one case for he setoid
representation, it generates less cases. Although for simple
operations it does not simply too much, it makes a big different in
proving complicated theorems like distributivity.

We will introduce the way we apply definable quotient in practical use.

{\textbf{\large{Operations}}}

With two functions converting the two encodings, it is enough to lift
all functions. It is possible to lift operations uniformly by mixing the
normalisation and embedding functions. For example, given an unary
operator 

\begin{code}
\\
\>\AgdaFunction{lift₁} \AgdaSymbol{:} \AgdaSymbol{(}\AgdaBound{op} \AgdaSymbol{:} \AgdaFunction{ℤ₀} \AgdaSymbol{→} \AgdaFunction{ℤ₀}\AgdaSymbol{)} \AgdaSymbol{→} \AgdaDatatype{ℤ} \AgdaSymbol{→} \AgdaDatatype{ℤ}\<%
\\
\>\AgdaFunction{lift₁} \AgdaBound{op} \AgdaSymbol{=} \AgdaFunction{[\_]} \AgdaFunction{∘} \AgdaBound{op} \AgdaFunction{∘} \AgdaFunction{⌜\_⌝}\<%
\\
\end{code}

This can be generalised to n-ary operators.

As long as we implement an operation of integers, it is safe to lift
the setoid version. However it is clear that not all operations on the
base types should be lifted and they do not make sense. Hence it is
better to verify if the operation is well-defined on the setoid:

$$a \sim b → op~a \sim op~b$$



Most of the operations for setoid integers can be concluded from
expression rewriting equations. For example to define addition, 
we only need to transform the expression  
$(a_1 - b_1) + (a_2 - b_2) = (a_1 + a_2) - (b_1) - (b_2)$
so that we only do valid
operations on natural numbers ($+$ or $*$) and minus is going to be
replaced by pairing operation.

The addition of integers can be defined in one line

\begin{code}
\\
\>\AgdaFunction{\_+\_} \AgdaSymbol{:} \AgdaFunction{ℤ₀} \AgdaSymbol{→} \AgdaFunction{ℤ₀} \AgdaSymbol{→} \AgdaFunction{ℤ₀}\<%
\\
\>\AgdaSymbol{(}\AgdaBound{x+} \AgdaInductiveConstructor{,} \AgdaBound{x-}\AgdaSymbol{)} \AgdaFunction{+} \AgdaSymbol{(}\AgdaBound{y+} \AgdaInductiveConstructor{,} \AgdaBound{y-}\AgdaSymbol{)} \AgdaSymbol{=} \AgdaSymbol{(}\AgdaBound{x+} \AgdaFunction{ℕ+} \AgdaBound{y+}\AgdaSymbol{)} \AgdaInductiveConstructor{,} \AgdaSymbol{(}\AgdaBound{x-} \AgdaFunction{ℕ+} \AgdaBound{y-}\AgdaSymbol{)}\<%
\\
\end{code}

We can easily verify it well-defined but it is not necessary because
it has been implied in the mathematical reasoning before implementing it.



\textbf{\large{Properties}}

Following axioms and theorems we learnt about integers, we can prove
the plus, multiplication is commutative and associative, minus is
inverse of plus etc. All these properties about plus, minus and
multiplication forms a commutative ring of integers \footnote{can cite
somewhere else http://www.millersville.edu/~bikenaga/number-theory/ring-of-integers/ring-of-integers.html}.
Like for natural numbers, axioms in classic mathematics are theorems
to prove in constructive mathematics. The only axioms for integers are
contained in the definition.

One of the important motivations of using setoid integers is that
the setoid definition reduces the complexity as we have shown in the
previous part, but it
will be much more evident when proving properties of the ring of integers.

In practice, for the set definition of integers, most of basic operations and simple theorems are not unbearably
complicated to deal with. However, as we mentioned before, the number of
cases grows exponentially when case splitting is unavoidable.
Although it is
possible to prove lemmas which covers several cases and reduce number
of cases, it often does not reduce complexity in total.
 An important case which is extremely difficult to conduct in practice is the proving of distributivity.

\paragraph{An efficient utilization of quotient structure: the
  proving of distributivity}

At first when we attempted to define the ring of integers, we were
stuck in proving the the distributivity of $*$ over $+$. As an example
we will show the attempt of left
distributivity $ x \times (y + z) = x \times y + x \times z$.

To simplify it, we define the multiplication in an arithmetic way
instead of pattern matching.

\begin{code}
\\
\>\AgdaFunction{\_ℤ*\_} \AgdaSymbol{:} \AgdaDatatype{ℤ} \AgdaSymbol{→} \AgdaDatatype{ℤ} \AgdaSymbol{→} \AgdaDatatype{ℤ}\<%
\\
\>\AgdaBound{i} \AgdaFunction{ℤ*} \AgdaBound{j} \AgdaSymbol{=}
\AgdaFunction{sign} \AgdaBound{i} \AgdaFunction{S*}
\AgdaFunction{sign} \AgdaBound{j} \AgdaFunction{◃} \AgdaFunction{∣}
\AgdaBound{i} \AgdaFunction{∣} \AgdaPrimitive{ℕ*} \AgdaFunction{∣}
\AgdaBound{j} \AgdaFunction{∣}\<%
\\
\end{code}

Of course it is not rational to split cases into $2 * 2 * 2$. 
The first idea is to apply the left distributivity law of natural
numbers when they are non-negative. To utilize the it more, 
it can be generalised to the cases when all three variables have the same
signs. Moreover when only $y$ and $z$ have the
same symbol, it is still plausible. 
The following is a partial definition (Note: $\AgdaFunction{DistributesOverˡ}$ means
that the distributivity of the first operators over the second one)  and three lemmas are given after it

\begin{code}
\\
\>\AgdaFunction{distˡ} \<[7]%
\>[7]\AgdaSymbol{:} \AgdaFunction{\_ℤ*\_} \AgdaFunction{DistributesOverˡ} \AgdaFunction{\_ℤ+\_}\<%
\\
\>\AgdaFunction{distˡ} \AgdaBound{x} \AgdaBound{y} \AgdaBound{z} \AgdaKeyword{with} \AgdaFunction{sign} \AgdaBound{y} \AgdaFunction{S≟} \AgdaFunction{sign} \AgdaBound{z}\<%
\\
\>\AgdaFunction{distˡ} \AgdaBound{x} \AgdaBound{y} \AgdaBound{z} \<[15]%
\>[15]\AgdaSymbol{|} \AgdaInductiveConstructor{yes} \AgdaBound{p} \<[23]%
\>[23]\<%
\\
\>[4]\AgdaIndent{6}{}\<[6]%
\>[6]\AgdaKeyword{rewrite} \AgdaBound{p} \<[16]%
\>[16]\<%
\\
\>[6]\AgdaIndent{12}{}\<[12]%
\>[12]\AgdaSymbol{|} \AgdaFunction{lem1} \AgdaBound{y} \AgdaBound{z} \AgdaBound{p} \<[25]%
\>[25]\<%
\\
\>[6]\AgdaIndent{12}{}\<[12]%
\>[12]\AgdaSymbol{|} \AgdaFunction{lem2} \AgdaBound{y} \AgdaBound{z} \AgdaBound{p} \AgdaSymbol{=} \<[27]%
\>[27]\<%
\\
\>[0]\AgdaIndent{6}{}\<[6]%
\>[6]\AgdaFunction{trans} \AgdaSymbol{(}\AgdaFunction{cong} \AgdaSymbol{(λ} \AgdaBound{n} \AgdaSymbol{→} \AgdaFunction{sign} \AgdaBound{x} \AgdaFunction{S*} \AgdaFunction{sign} \AgdaBound{z} \AgdaFunction{◃} \AgdaBound{n}\AgdaSymbol{)} \<[47]%
\>[47]\<%
\\
\>[0]\AgdaIndent{6}{}\<[6]%
\>[6]\AgdaSymbol{(}\AgdaFunction{ℕdistˡ} \AgdaFunction{∣} \AgdaBound{x} \AgdaFunction{∣} \AgdaFunction{∣} \AgdaBound{y} \AgdaFunction{∣} \AgdaSymbol{(}\AgdaFunction{∣} \AgdaBound{z} \AgdaFunction{∣}\AgdaSymbol{)))} \<[36]%
\>[36]\<%
\\
\>[0]\AgdaIndent{6}{}\<[6]%
\>[6]\AgdaSymbol{(}\AgdaFunction{lem3} \AgdaSymbol{(}\AgdaFunction{∣} \AgdaBound{x} \AgdaFunction{∣} \AgdaPrimitive{ℕ*} \AgdaFunction{∣} \AgdaBound{y} \AgdaFunction{∣}\AgdaSymbol{)} \AgdaSymbol{(}\AgdaFunction{∣} \AgdaBound{x} \AgdaFunction{∣} \AgdaPrimitive{ℕ*} \AgdaFunction{∣} \AgdaBound{z} \AgdaFunction{∣}\AgdaSymbol{)} \AgdaSymbol{\_)}\<%
\\
\>\AgdaFunction{distˡ} \AgdaBound{x} \AgdaBound{y} \AgdaBound{z} \AgdaSymbol{|} \AgdaInductiveConstructor{no} \AgdaBound{¬p} \AgdaSymbol{=} \AgdaBound{...}\<%
\\
\end{code}
 
To prove these simpler cases we need three lemmas,

\begin{code}
\\
\>\AgdaFunction{lem1} \AgdaSymbol{:} \AgdaSymbol{∀} \AgdaBound{x} \AgdaBound{y} \AgdaSymbol{→} \AgdaFunction{sign} \AgdaBound{x} \AgdaDatatype{≡} \AgdaFunction{sign} \AgdaBound{y} \AgdaSymbol{→} \AgdaFunction{∣} \AgdaBound{x} \AgdaFunction{ℤ+} \AgdaBound{y} \AgdaFunction{∣} \AgdaDatatype{≡} \AgdaFunction{∣} \AgdaBound{x} \AgdaFunction{∣} \AgdaPrimitive{ℕ+} \AgdaFunction{∣} \AgdaBound{y} \AgdaFunction{∣} \<[61]%
\>[61]\<%
\\
\>\AgdaFunction{lem1} \AgdaSymbol{(}\AgdaInductiveConstructor{-suc} \AgdaBound{x}\AgdaSymbol{)} \AgdaSymbol{(}\AgdaInductiveConstructor{-suc} \AgdaBound{y}\AgdaSymbol{)} \AgdaBound{e} \AgdaSymbol{=} \AgdaFunction{cong} \AgdaInductiveConstructor{suc} \AgdaSymbol{(}\AgdaFunction{sym} \AgdaSymbol{(}\AgdaFunction{m+1+n≡1+m+n} \AgdaBound{x} \AgdaBound{y}\AgdaSymbol{))}\<%
\\
\>\AgdaFunction{lem1} \AgdaSymbol{(}\AgdaInductiveConstructor{-suc} \AgdaBound{x}\AgdaSymbol{)} \AgdaSymbol{(}\AgdaInductiveConstructor{+} \AgdaBound{y}\AgdaSymbol{)} \AgdaSymbol{()}\<%
\\
\>\AgdaFunction{lem1} \AgdaSymbol{(}\AgdaInductiveConstructor{+} \AgdaBound{x}\AgdaSymbol{)} \AgdaSymbol{(}\AgdaInductiveConstructor{-suc} \AgdaBound{y}\AgdaSymbol{)} \AgdaSymbol{()}\<%
\\
\>\AgdaFunction{lem1} \AgdaSymbol{(}\AgdaInductiveConstructor{+} \AgdaBound{x}\AgdaSymbol{)} \AgdaSymbol{(}\AgdaInductiveConstructor{+} \AgdaBound{y}\AgdaSymbol{)} \AgdaBound{e} \AgdaSymbol{=} \AgdaInductiveConstructor{refl}\<%
\\
%
\\
\>\AgdaFunction{lem2} \AgdaSymbol{:} \AgdaSymbol{∀} \AgdaBound{x} \AgdaBound{y} \AgdaSymbol{→} \AgdaFunction{sign} \AgdaBound{x} \AgdaDatatype{≡} \AgdaFunction{sign} \AgdaBound{y} \AgdaSymbol{→} \AgdaFunction{sign} \AgdaSymbol{(}\AgdaBound{x} \AgdaFunction{ℤ+} \AgdaBound{y}\AgdaSymbol{)} \AgdaDatatype{≡} \AgdaFunction{sign} \AgdaBound{y}\<%
\\
\>\AgdaFunction{lem2} \AgdaSymbol{(}\AgdaInductiveConstructor{-suc} \AgdaBound{x}\AgdaSymbol{)} \AgdaSymbol{(} \AgdaInductiveConstructor{-suc} \AgdaBound{y}\AgdaSymbol{)} \AgdaBound{e} \AgdaSymbol{=} \AgdaInductiveConstructor{refl}\<%
\\
\>\AgdaFunction{lem2} \AgdaSymbol{(}\AgdaInductiveConstructor{-suc} \AgdaBound{x}\AgdaSymbol{)} \AgdaSymbol{(}\AgdaInductiveConstructor{+} \AgdaBound{y}\AgdaSymbol{)} \AgdaSymbol{()}\<%
\\
\>\AgdaFunction{lem2} \AgdaSymbol{(}\AgdaInductiveConstructor{+} \AgdaBound{x}\AgdaSymbol{)} \AgdaSymbol{(}\AgdaInductiveConstructor{-suc} \AgdaBound{y}\AgdaSymbol{)} \AgdaSymbol{()}\<%
\\
\>\AgdaFunction{lem2} \AgdaSymbol{(}\AgdaInductiveConstructor{+} \AgdaBound{x}\AgdaSymbol{)} \AgdaSymbol{(}\AgdaInductiveConstructor{+} \AgdaBound{y}\AgdaSymbol{)} \AgdaBound{e} \AgdaSymbol{=} \AgdaInductiveConstructor{refl}\<%
\\
%
\\
%
\\
\>\AgdaFunction{lem3} \AgdaSymbol{:} \AgdaSymbol{∀} \AgdaBound{x} \AgdaBound{y} \AgdaBound{s} \AgdaSymbol{→} \AgdaBound{s} \AgdaFunction{◃} \AgdaSymbol{(}\AgdaBound{x} \AgdaPrimitive{ℕ+} \AgdaBound{y}\AgdaSymbol{)} \AgdaDatatype{≡} \AgdaSymbol{(}\AgdaBound{s} \AgdaFunction{◃} \AgdaBound{x}\AgdaSymbol{)} \AgdaFunction{ℤ+} \AgdaSymbol{(}\AgdaBound{s} \AgdaFunction{◃} \AgdaBound{y}\AgdaSymbol{)}\<%
\\
\>\AgdaFunction{lem3} \AgdaInductiveConstructor{0} \AgdaInductiveConstructor{0} \AgdaBound{s} \AgdaSymbol{=} \AgdaInductiveConstructor{refl}\<%
\\
\>\AgdaFunction{lem3} \AgdaInductiveConstructor{0} \AgdaSymbol{(}\AgdaInductiveConstructor{suc} \AgdaBound{y}\AgdaSymbol{)} \AgdaBound{s} \AgdaSymbol{=} \AgdaFunction{sym} \AgdaSymbol{(}\AgdaFunction{ℤ-id-l} \AgdaSymbol{\_)}\<%
\\
\>\AgdaFunction{lem3} \AgdaSymbol{(}\AgdaInductiveConstructor{suc} \AgdaBound{x}\AgdaSymbol{)} \AgdaBound{y} \AgdaBound{s} \AgdaSymbol{=} \AgdaFunction{trans} \AgdaSymbol{(}\AgdaFunction{h} \AgdaBound{s} \AgdaSymbol{(}\AgdaBound{x} \AgdaPrimitive{ℕ+} \AgdaBound{y}\AgdaSymbol{))} \AgdaSymbol{(}\<%
\\
\>[4]\AgdaIndent{19}{}\<[19]%
\>[19]\AgdaFunction{trans} \AgdaSymbol{(}\AgdaFunction{cong} \AgdaSymbol{(λ} \AgdaBound{n} \AgdaSymbol{→} \AgdaSymbol{(}\AgdaBound{s} \AgdaFunction{◃} \AgdaInductiveConstructor{suc} \AgdaNumber{0}\AgdaSymbol{)} \AgdaFunction{ℤ+} \AgdaBound{n}\AgdaSymbol{)} \AgdaSymbol{(}\AgdaFunction{lem3} \AgdaBound{x} \AgdaBound{y} \AgdaBound{s}\AgdaSymbol{))} \AgdaSymbol{(}\<%
\\
\>[4]\AgdaIndent{19}{}\<[19]%
\>[19]\AgdaFunction{trans} \AgdaSymbol{(}\AgdaFunction{sym} \AgdaSymbol{(}\AgdaPostulate{ℤ-+-assoc} \AgdaSymbol{(}\AgdaBound{s} \AgdaFunction{◃} \AgdaInductiveConstructor{suc} \AgdaNumber{0}\AgdaSymbol{)} \AgdaSymbol{(}\AgdaBound{s} \AgdaFunction{◃} \AgdaBound{x}\AgdaSymbol{)} \AgdaSymbol{(}\AgdaBound{s} \AgdaFunction{◃} \AgdaBound{y}\AgdaSymbol{)))} \AgdaSymbol{(}\<%
\\
\>[4]\AgdaIndent{19}{}\<[19]%
\>[19]\AgdaFunction{cong} \AgdaSymbol{(λ} \AgdaBound{n} \AgdaSymbol{→} \AgdaBound{n} \AgdaFunction{ℤ+} \AgdaSymbol{(}\AgdaBound{s} \AgdaFunction{◃} \AgdaBound{y}\AgdaSymbol{))} \AgdaSymbol{(}\AgdaFunction{sym} \AgdaSymbol{(}\AgdaFunction{h} \AgdaBound{s} \AgdaBound{x}\AgdaSymbol{)))))}\<%
\\
\>[13]\AgdaIndent{2}{}\<[2]%
\>[2]\AgdaKeyword{where}\<%
\\
\>[0]\AgdaIndent{4}{}\<[4]%
\>[4]\AgdaFunction{h} \AgdaSymbol{:} \AgdaSymbol{∀} \AgdaBound{s} \AgdaBound{y} \AgdaSymbol{→} \AgdaBound{s} \AgdaFunction{◃} \AgdaInductiveConstructor{suc} \AgdaBound{y} \AgdaDatatype{≡} \AgdaSymbol{(}\AgdaBound{s} \AgdaFunction{◃} \AgdaSymbol{(}\AgdaInductiveConstructor{suc} \AgdaNumber{0}\AgdaSymbol{))} \AgdaFunction{ℤ+} \AgdaSymbol{(}\AgdaBound{s} \AgdaFunction{◃} \AgdaBound{y}\AgdaSymbol{)}\<%
\\
\>[0]\AgdaIndent{4}{}\<[4]%
\>[4]\AgdaFunction{h} \AgdaBound{s} \AgdaInductiveConstructor{0} \AgdaSymbol{=} \AgdaFunction{sym} \AgdaSymbol{(}\AgdaPostulate{ℤ-id-r} \AgdaSymbol{\_)}\<%
\\
\>[0]\AgdaIndent{4}{}\<[4]%
\>[4]\AgdaFunction{h} \AgdaInductiveConstructor{Sign.-} \AgdaSymbol{(}\AgdaInductiveConstructor{suc} \AgdaBound{y}\AgdaSymbol{)} \AgdaSymbol{=} \AgdaInductiveConstructor{refl}\<%
\\
\>[0]\AgdaIndent{4}{}\<[4]%
\>[4]\AgdaFunction{h} \AgdaInductiveConstructor{Sign.+} \AgdaSymbol{(}\AgdaInductiveConstructor{suc} \AgdaBound{y}\AgdaSymbol{)} \AgdaSymbol{=} \AgdaInductiveConstructor{refl}\<%
\\
\end{code}

In the second category of cases, if $y$ and $z$ have different
signs, it is impossible to apply the left distributivity law for
natural numbers. Intuitively speaking, there is no rule to turn expressions like $x *
(y  - z)$ into expression which only contains natural numbers,
therefore we have to prove it from scratch.
Although it is possible to prove finally, it is not the best solution
we want. 

It is much simpler to prove the distributivity for setoid
representations. In fact with the help of ring solver, it can be
proved automatically.
All these definitions of operators only involves operators for natural
numbers which forms a commutative semiring for natural numbers.
This implies that these expressions of setoid integers which only
involves plus, minus and multiplication, can be turned into equation of natural numbers.
The theorems are also simply equations of natural numbers which are
automatically solvable.


\begin{remark}
A \emph{ring solver} is an automatic equation checker for rings e.g.\ the
ring of integers. It is implemented based on the theory described in
"Proving Equalities in a Commutative Ring Done Right in Coq" by
Grégoire and Mahboubi \cite{gregoire2005proving}.
\end{remark}


Therefore to prove the distributivity, the
simplest way is to use semiring
solver for natural numbers. 


\begin{code}
\\
\>\AgdaFunction{distˡ} \AgdaSymbol{:} \<[9]%
\>[9]\AgdaFunction{\_*\_} \AgdaFunction{DistributesOverˡ} \AgdaFunction{\_+\_}\<%
\\
\>\AgdaFunction{distˡ} \AgdaSymbol{(}\AgdaBound{a} \AgdaInductiveConstructor{,} \AgdaBound{b}\AgdaSymbol{)} \AgdaSymbol{(}\AgdaBound{c} \AgdaInductiveConstructor{,} \AgdaBound{d}\AgdaSymbol{)} \AgdaSymbol{(}\AgdaBound{e} \AgdaInductiveConstructor{,} \AgdaBound{f}\AgdaSymbol{)} \AgdaSymbol{=} \AgdaFunction{solve} \AgdaNumber{6} \<[40]%
\>[40]\<%
\\
\>[0]\AgdaIndent{2}{}\<[2]%
\>[2]\AgdaSymbol{(λ} \AgdaBound{a} \AgdaBound{b} \AgdaBound{c} \AgdaBound{d} \AgdaBound{e} \AgdaBound{f} \AgdaSymbol{→} \AgdaBound{a} \AgdaFunction{:*} \AgdaSymbol{(}\AgdaBound{c} \AgdaFunction{:+} \AgdaBound{e}\AgdaSymbol{)} \AgdaFunction{:+} \AgdaBound{b} \AgdaFunction{:*} \AgdaSymbol{(}\AgdaBound{d} \AgdaFunction{:+} \AgdaBound{f}\AgdaSymbol{)} \AgdaFunction{:+}\<%
\\
\>[2]\AgdaIndent{6}{}\<[6]%
\>[6]\AgdaSymbol{(}\AgdaBound{a} \AgdaFunction{:*} \AgdaBound{d} \AgdaFunction{:+} \AgdaBound{b} \AgdaFunction{:*} \AgdaBound{c} \AgdaFunction{:+} \AgdaSymbol{(}\AgdaBound{a} \AgdaFunction{:*} \AgdaBound{f} \AgdaFunction{:+} \AgdaBound{b} \AgdaFunction{:*} \AgdaBound{e}\AgdaSymbol{))}\<%
\\
\>[2]\AgdaIndent{6}{}\<[6]%
\>[6]\AgdaFunction{:=}\<%
\\
\>[2]\AgdaIndent{6}{}\<[6]%
\>[6]\AgdaBound{a} \AgdaFunction{:*} \AgdaBound{c} \AgdaFunction{:+} \AgdaBound{b} \AgdaFunction{:*} \AgdaBound{d} \AgdaFunction{:+} \AgdaSymbol{(}\AgdaBound{a} \AgdaFunction{:*} \AgdaBound{e} \AgdaFunction{:+} \AgdaBound{b} \AgdaFunction{:*} \AgdaBound{f}\AgdaSymbol{)} \AgdaFunction{:+}\<%
\\
\>[2]\AgdaIndent{6}{}\<[6]%
\>[6]\AgdaSymbol{(}\AgdaBound{a} \AgdaFunction{:*} \AgdaSymbol{(}\AgdaBound{d} \AgdaFunction{:+} \AgdaBound{f}\AgdaSymbol{)} \AgdaFunction{:+} \AgdaBound{b} \AgdaFunction{:*} \AgdaSymbol{(}\AgdaBound{c} \AgdaFunction{:+} \AgdaBound{e}\AgdaSymbol{)))} \AgdaInductiveConstructor{refl} \AgdaBound{a} \AgdaBound{b} \AgdaBound{c} \AgdaBound{d} \AgdaBound{e} \AgdaBound{f}\<%
\\
\end{code}

It is not the simplest way to use ring solver since we have to feed
the type (i.e.\ the equation) to the solver which can be automatically
figured out by "reflection". It helps us quote the type of the goal so
that we can define a function that automatically do it without
explicitly providing the equations. There is already some work
done by van der Walt \cite{van2012reflection}. It can be seen as an
analogy of the "ring" tactic from Coq.


Use the ring solver for natural numbers we can prove all theorems
required for ring solver for integers. However the process of proof
terms generation and type checking take very long time in practice. It may heavily slow the type
checking although the optimization of Agda already shows a big
improvement in this technical efficiency issue. As
these theorems are going to be used quite often, it is reasonable to
manually construct the proof terms to improve efficiency of library
code, sacrificing some conveniences.

Luckily it is still much simpler than the ones for the set of integers $\Z$.
First there is only one case of integer and as we know the equations is
indeed equations of natural numbers which can be proved using only the
properties in the commutative semiring of natural numbers. There is no
need to prove some properties for $\Z$ from scratch like in the
proving of distributivity.

\begin{code}
\\
\>\AgdaFunction{dist-lemˡ} \AgdaSymbol{:} \AgdaSymbol{∀} \AgdaBound{a} \AgdaBound{b} \AgdaBound{c} \AgdaBound{d} \AgdaBound{e} \AgdaBound{f} \AgdaSymbol{→} \<[28]%
\>[28]\<%
\\
\>[0]\AgdaIndent{2}{}\<[2]%
\>[2]\AgdaBound{a} \AgdaPrimitive{ℕ*} \AgdaSymbol{(}\AgdaBound{c} \AgdaPrimitive{ℕ+} \AgdaBound{e}\AgdaSymbol{)} \AgdaPrimitive{ℕ+} \AgdaBound{b} \AgdaPrimitive{ℕ*} \AgdaSymbol{(}\AgdaBound{d} \AgdaPrimitive{ℕ+} \AgdaBound{f}\AgdaSymbol{)} \AgdaDatatype{≡}\<%
\\
\>[0]\AgdaIndent{2}{}\<[2]%
\>[2]\AgdaSymbol{(}\AgdaBound{a} \AgdaPrimitive{ℕ*} \AgdaBound{c} \AgdaPrimitive{ℕ+} \AgdaBound{b} \AgdaPrimitive{ℕ*} \AgdaBound{d}\AgdaSymbol{)} \AgdaPrimitive{ℕ+} \AgdaSymbol{(}\AgdaBound{a} \AgdaPrimitive{ℕ*} \AgdaBound{e} \AgdaPrimitive{ℕ+} \AgdaBound{b} \AgdaPrimitive{ℕ*} \AgdaBound{f}\AgdaSymbol{)} \<[43]%
\>[43]\<%
\\
\>\AgdaFunction{dist-lemˡ} \AgdaBound{a} \AgdaBound{b} \AgdaBound{c} \AgdaBound{d} \AgdaBound{e} \AgdaBound{f} \AgdaSymbol{=} \AgdaFunction{trans}\<%
\\
\>[0]\AgdaIndent{2}{}\<[2]%
\>[2]\AgdaSymbol{(}\AgdaFunction{cong₂} \AgdaPrimitive{\_ℕ+\_} \AgdaSymbol{(}\AgdaFunction{ℕdistˡ} \AgdaBound{a} \AgdaBound{c} \AgdaBound{e}\AgdaSymbol{)} \AgdaSymbol{(}\AgdaFunction{ℕdistˡ} \AgdaBound{b} \AgdaBound{d} \AgdaBound{f}\AgdaSymbol{))}\<%
\\
\>[0]\AgdaIndent{2}{}\<[2]%
\>[2]\AgdaSymbol{(}\AgdaPostulate{swap23} \AgdaSymbol{(}\AgdaBound{a} \AgdaPrimitive{ℕ*} \AgdaBound{c}\AgdaSymbol{)} \AgdaSymbol{(}\AgdaBound{a} \AgdaPrimitive{ℕ*} \AgdaBound{e}\AgdaSymbol{)} \AgdaSymbol{(}\AgdaBound{b} \AgdaPrimitive{ℕ*} \AgdaBound{d}\AgdaSymbol{)} \AgdaSymbol{(}\AgdaBound{b} \AgdaPrimitive{ℕ*} \AgdaBound{f}\AgdaSymbol{))}\<%
\\
%
\\
\>\AgdaFunction{distˡ} \AgdaSymbol{:} \AgdaFunction{\_ℤ₀*\_} \AgdaFunction{DistributesOverˡ} \AgdaFunction{\_ℤ₀+\_}\<%
\\
\>\AgdaFunction{distˡ} \AgdaSymbol{(}\AgdaBound{a} \AgdaInductiveConstructor{,} \AgdaBound{b}\AgdaSymbol{)} \AgdaSymbol{(}\AgdaBound{c} \AgdaInductiveConstructor{,} \AgdaBound{d}\AgdaSymbol{)} \AgdaSymbol{(}\AgdaBound{e} \AgdaInductiveConstructor{,} \AgdaBound{f}\AgdaSymbol{)} \AgdaSymbol{=} \<[32]%
\>[32]\<%
\\
\>[0]\AgdaIndent{2}{}\<[2]%
\>[2]\AgdaFunction{cong₂} \AgdaPrimitive{\_ℕ+\_} \AgdaSymbol{(}\AgdaFunction{dist-lemˡ} \AgdaBound{a} \AgdaBound{b} \AgdaBound{c} \AgdaBound{d} \AgdaBound{e} \AgdaBound{f}\AgdaSymbol{)}\<%
\\
\>[0]\AgdaIndent{2}{}\<[2]%
\>[2]\AgdaSymbol{(}\AgdaFunction{sym} \AgdaSymbol{(}\AgdaFunction{dist-lemˡ} \AgdaBound{a} \AgdaBound{b} \AgdaBound{d} \AgdaBound{c} \AgdaBound{f} \AgdaBound{e}\AgdaSymbol{))}\<%
\\
\end{code}

It only needs one special lemma which can be proved by applying distributivity
laws for natural numbers. The $\AgdaFunction{swap23}$ is a commonly
used equation rewriting lemma

$(m + n) + (p + q) = (m + p) + (n + q)$

After all, the application of quotient structure in the integer case
provides us a general approach to define functions and prove theorems
when the base types are simpler to deal with.
Adopting this approach, both the appearance -- the set representation of integers
and the underlying -- the convenience of manipulating setoid
representation of integers, are maintained.
In the case of rational numbers, the base type is also simpler to deal with.


\section{Rational numbers}

In the Agda standard library, the set of rational numbers is defined
as fractions whose numerator and denominator are coprime. It ensures
the canonicity of representations, but it complicates the manipulation
of rational numbers since the coprimarity has to be kept for each
intermediate step of calculations. From experience of calculating
fractions in arithmetic, we usually do not reduce fractions until we are going to
show a simplified answer, and all the calculation can be carried out correctly
without reducing. Actually from  a discussion on Agda mailing list
\footnote{http://comments.gmane.org/gmane.comp.lang.agda/6372}, a better way to manipulate the
rational numbers is required. A flexibility way to choose from unreduced fractions or reduced ones in
operations can be fulfilled by a definable quotient of
rational numbers. The unreduced fractions are mainly used for underlying calculations and
property proving because it ignores a lot of reducing steps and
coprimarity checking, and the reduced fractions are accessible when
displaying rational numbers. 
It simplifies defining functions and proving properties related to rational
numbers. Moreover, we believe the efficiency of programs involving rational
numbers calculation is improved due to the removal of reducing , even
though some people claim that the unreduced numbers are too
large to make it efficient.


\subsection{The setoid of rational numbers: unreduced fractions}

The setoid representation of rational numbers -- unreduced fractions
are more common in mathematics.
Usually a fraction denoted as $\frac{m}{n}$ consists of an integers
$m$ called \emph{numerator} and a non-zero integer $n$ called
\emph{denominator}.
In type theory, since the data types have different
complexity, we have a decision to make in the choice of best
definition like what we did for integers.
Alternatively keeping the numerator integer, the denominator can be a
positive natural number such that the sign of rational number is kept
in the numerator solely and there is no need to add a restriction to
exclude $0$ which is not so easy for the set of integer.

$$\Q_0 = \Z \times \N$$

Technically there is no need to define a new type for positive
natural numbers $\N^{+}$, so we use $\N$ which we can name it as
\emph{denominator-1} or implement it as follows

\begin{code}
\\
\>\AgdaKeyword{data} \AgdaDatatype{ℚ₀} \AgdaSymbol{:} \AgdaPrimitiveType{Set} \AgdaKeyword{where}\<%
\\
\>[-1]\AgdaIndent{2}{}\<[2]%
\>[2]\AgdaInductiveConstructor{\_/suc\_} \AgdaSymbol{:} \AgdaSymbol{(}\AgdaBound{n} \AgdaSymbol{:} \AgdaDatatype{ℤ}\AgdaSymbol{)} \AgdaSymbol{→} \AgdaSymbol{(}\AgdaBound{d} \AgdaSymbol{:} \AgdaDatatype{ℕ}\AgdaSymbol{)} \AgdaSymbol{→} \AgdaDatatype{ℚ₀}\<%
\\
\end{code}

such that $2 /suc 2$ stands for $\frac{2}{3}$.


\begin{remark}
It is also reasonable to define it with a pair of natural numbers with
a sign defined separately. However this also complicates calculation
in some way.
\end{remark}



The rational number represented by a fraction $\frac{m}{n}$ is indeed the result of
division $m \div n$. Therefore two different fractions can represent
the same rational numbers.
In mathematics, to judge the equality of two fractions, it is easier to conduct the following conversion,

$$ \frac{a}{b} = \frac{c}{d} \iff a \times d = c \times b $$


Hence the equivalence relation can be defined as,

\begin{code}
\\
\>\AgdaFunction{\_*\_} \AgdaSymbol{:} \AgdaDatatype{ℤ} \AgdaSymbol{→} \AgdaDatatype{ℕ} \AgdaSymbol{→} \AgdaDatatype{ℤ}\<%
\\
\>\AgdaSymbol{(}\AgdaInductiveConstructor{+} \AgdaBound{x}\AgdaSymbol{)} \AgdaFunction{*} \AgdaBound{d} \AgdaSymbol{=} \AgdaInductiveConstructor{+} \AgdaSymbol{(}\AgdaBound{x} \AgdaPrimitive{ℕ*} \AgdaBound{d}\AgdaSymbol{)}\<%
\\
\>\AgdaSymbol{(}\AgdaInductiveConstructor{-suc} \AgdaBound{x}\AgdaSymbol{)} \AgdaFunction{*} \AgdaInductiveConstructor{0} \AgdaSymbol{=} \AgdaInductiveConstructor{+} \AgdaNumber{0}\<%
\\
\>\AgdaSymbol{(}\AgdaInductiveConstructor{-suc} \AgdaBound{x}\AgdaSymbol{)} \AgdaFunction{*} \AgdaInductiveConstructor{suc} \AgdaBound{d} \AgdaSymbol{=} \AgdaInductiveConstructor{-suc} \AgdaSymbol{(}\AgdaBound{x} \AgdaPrimitive{ℕ+} \AgdaInductiveConstructor{suc} \AgdaBound{x} \AgdaPrimitive{ℕ*} \AgdaBound{d}\AgdaSymbol{)}\<%
\\
%
\\
\>\AgdaFunction{\_∼\_} \<[35]%
\>[35]\AgdaSymbol{:} \AgdaFunction{Rel} \AgdaDatatype{ℚ₀} \AgdaSymbol{\_}\<%
\\
\>\AgdaSymbol{(}\AgdaBound{n1} \AgdaInductiveConstructor{/suc} \AgdaBound{d1}\AgdaSymbol{)} \AgdaFunction{∼} \AgdaSymbol{(}\AgdaBound{n2} \AgdaInductiveConstructor{/suc} \AgdaBound{d2}\AgdaSymbol{)} \AgdaSymbol{=} \<[31]%
\>[31]\AgdaBound{n1} \AgdaFunction{*} \AgdaInductiveConstructor{suc} \AgdaBound{d2} \AgdaDatatype{≡} \AgdaBound{n2} \AgdaFunction{*} \AgdaInductiveConstructor{suc} \AgdaBound{d1}\<%
\\
\end{code}


\subsection{The set definition of rational numbers: reduced fractions}

The reduced fractions are canonical representations of rational
numbers. It is a subset of fractions, so we only need to add a
restriction that the numerator and denominator is coprime,

% The normal form of rational numbers, namely the quotient type in this quotient is the set of irreducible fractions. We only need to add a restriction that the numerator and denominator is coprime,

$$\Q = \Sigma ((n , d) : \Z \times \N)). \mathsf{coprime} ~n ~(d +1)$$

It is implemented as follows (also available in standard library),

\begin{code}
\\
\>\AgdaKeyword{record} \AgdaRecord{ℚ} \AgdaSymbol{:} \AgdaPrimitiveType{Set} \AgdaKeyword{where}\<%
\\
\>[0]\AgdaIndent{2}{}\<[2]%
\>[2]\AgdaKeyword{field}\<%
\\
\>[2]\AgdaIndent{4}{}\<[4]%
\>[4]\AgdaField{numerator} \<[18]%
\>[18]\AgdaSymbol{:} \AgdaDatatype{ℤ}\<%
\\
\>[2]\AgdaIndent{4}{}\<[4]%
\>[4]\AgdaField{denominator-1} \AgdaSymbol{:} \AgdaDatatype{ℕ}\<%
\\
\>[2]\AgdaIndent{4}{}\<[4]%
\>[4]\AgdaField{isCoprime} \<[18]%
\>[18]\AgdaSymbol{:} \AgdaFunction{True} \AgdaSymbol{(}\AgdaFunction{C.coprime?} \AgdaFunction{∣} \AgdaBound{numerator} \AgdaFunction{∣} \AgdaSymbol{(}\AgdaInductiveConstructor{suc} \AgdaBound{denominator-1}\AgdaSymbol{))}\<%
\\
%
\\
\>[0]\AgdaIndent{2}{}\<[2]%
\>[2]\AgdaFunction{denominator} \AgdaSymbol{:} \AgdaDatatype{ℤ}\<%
\\
\>[0]\AgdaIndent{2}{}\<[2]%
\>[2]\AgdaFunction{denominator} \AgdaSymbol{=} \AgdaInductiveConstructor{+} \AgdaInductiveConstructor{suc} \AgdaFunction{denominator-1}\<%
\\
%
\\
\>[0]\AgdaIndent{2}{}\<[2]%
\>[2]\AgdaFunction{coprime} \AgdaSymbol{:} \AgdaFunction{Coprime} \AgdaFunction{numerator} \AgdaFunction{denominator}\<%
\\
\>[0]\AgdaIndent{2}{}\<[2]%
\>[2]\AgdaFunction{coprime} \AgdaSymbol{=} \AgdaFunction{toWitness} \AgdaFunction{isCoprime}\<%
\\
\end{code}

\subsection{The definable quotient of rational numbers}

From the definition, we already have a setoid which contains unreduced
fractions $\Q_0$ as
base type and an equivalence relation on it. Compare with the set
definition, it is clear that normalisation from $\Q_0 \to \Q$ is just
reducing functions.

To implement reducing process, we can utilize the library code about
\emph{great common divisor (gcd)}. The function $\AgdaFunction{gcd}$
calculates the divisor and a data type $\AgdaDatatype{GCD}$ containing
information from calculating the
greatest common divisor which can be converted to another data type $\AgdaDatatype{GCD′}$
\footnote{http://www.cse.chalmers.se/~nad/repos/lib/src/Data/Nat/Coprimality.agda}
which contains the three essential results we need, i.e.\ the new numerator, new denominator and the
proof term that they are coprime. Combing the library functions and
types with a self-defined function
$\AgdaFunction{GCD′→ℚ}$\footnote{see \autoref{rational-gcd}} which
constructs the reduced fraction we need, the normalisation function is
implemented as follows,

\begin{code}
\\
\>\AgdaFunction{[\_]} \AgdaSymbol{:} \AgdaDatatype{ℚ₀} \AgdaSymbol{→} \AgdaRecord{ℚ}\<%
\\
\>\AgdaFunction{[} \AgdaSymbol{(}\AgdaInductiveConstructor{+} \AgdaInductiveConstructor{0}\AgdaSymbol{)} \AgdaInductiveConstructor{/suc} \AgdaBound{d} \AgdaFunction{]} \AgdaSymbol{=} \AgdaInductiveConstructor{ℤ.+\_} \AgdaNumber{0} \AgdaFunction{÷} \AgdaNumber{1}\<%
\\
\>\AgdaFunction{[} \AgdaSymbol{(}\AgdaInductiveConstructor{+} \AgdaSymbol{(}\AgdaInductiveConstructor{suc} \AgdaBound{n}\AgdaSymbol{))} \AgdaInductiveConstructor{/suc} \AgdaBound{d} \AgdaFunction{]} \AgdaKeyword{with} \AgdaFunction{gcd} \AgdaSymbol{(}\AgdaInductiveConstructor{suc} \AgdaBound{n}\AgdaSymbol{)} \AgdaSymbol{(}\AgdaInductiveConstructor{suc} \AgdaBound{d}\AgdaSymbol{)}\<%
\\
\>\AgdaFunction{[} \AgdaSymbol{(}\AgdaInductiveConstructor{+} \AgdaInductiveConstructor{suc} \AgdaBound{n}\AgdaSymbol{)} \AgdaInductiveConstructor{/suc} \AgdaBound{d} \AgdaFunction{]} \AgdaSymbol{|} \AgdaBound{di} \AgdaInductiveConstructor{,} \AgdaBound{g} \AgdaSymbol{=} \AgdaFunction{GCD′→ℚ} \AgdaSymbol{(}\AgdaInductiveConstructor{suc} \AgdaBound{n}\AgdaSymbol{)} \AgdaSymbol{(}\AgdaInductiveConstructor{suc} \AgdaBound{d}\AgdaSymbol{)} \AgdaBound{di} \AgdaSymbol{(λ} \AgdaSymbol{())} \AgdaSymbol{(}\AgdaFunction{C.gcd-gcd′} \AgdaBound{g}\AgdaSymbol{)}\<%
\\
\>\AgdaFunction{[} \AgdaSymbol{(}\AgdaInductiveConstructor{-suc} \AgdaBound{n}\AgdaSymbol{)} \AgdaInductiveConstructor{/suc} \AgdaBound{d} \AgdaFunction{]} \AgdaKeyword{with} \AgdaFunction{gcd} \AgdaSymbol{(}\AgdaInductiveConstructor{suc} \AgdaBound{n}\AgdaSymbol{)} \AgdaSymbol{(}\AgdaInductiveConstructor{suc} \AgdaBound{d}\AgdaSymbol{)}\<%
\\
\>\AgdaSymbol{...} \AgdaSymbol{|} \AgdaBound{di} \AgdaInductiveConstructor{,} \AgdaBound{g} \AgdaSymbol{=} \AgdaFunction{-} \AgdaFunction{GCD′→ℚ} \AgdaSymbol{(}\AgdaInductiveConstructor{suc} \AgdaBound{n}\AgdaSymbol{)} \AgdaSymbol{(}\AgdaInductiveConstructor{suc} \AgdaBound{d}\AgdaSymbol{)} \AgdaBound{di} \AgdaSymbol{(λ} \AgdaSymbol{())} \AgdaSymbol{(}\AgdaFunction{C.gcd-gcd′} \AgdaBound{g}\AgdaSymbol{)}\<%
\\
\end{code}


and the embedding function is trivial, because we only need to forget
the coprime proof in the normal form,

\begin{code}
\\
\>\AgdaFunction{⌜\_⌝} \AgdaSymbol{:} \AgdaRecord{ℚ} \AgdaSymbol{→} \AgdaDatatype{ℚ₀}\<%
\\
\>\AgdaFunction{⌜} \AgdaBound{x} \AgdaFunction{⌝} \AgdaSymbol{=} \AgdaSymbol{(}\AgdaFunction{ℤcon} \AgdaSymbol{(}\AgdaFunction{ℚ.numerator} \AgdaBound{x}\AgdaSymbol{))} \AgdaInductiveConstructor{/suc} \AgdaSymbol{(}\AgdaFunction{ℚ.denominator-1} \AgdaBound{x}\AgdaSymbol{)}\<%
\\
\end{code}

Similar to the case of integers, complete the definable quotient, we
also need to prove the sound, complete, stable and exact
properties. There is no need to show the proof of them since the
normalisation function is implemented using an approach which has been
tested in mathematics.
Also we can apply the definable quotient structures in helping form
the field of rational numbers. After embedding the natural numbers
into integers, any theorems using only the operations in the field
$\Q$, are turned out to be propositions of integers which can be
solved using ring solver or theorems proven for integers.



\section{Summary}

In this chapter, we have shown that quotients does not have to be
constructed using a type former if it can be defined. We present some
quotient structures and prove some equivalences (conversions) among
the them. In fact they provide us conversions between base types and
quotient types such that we can usually benefit from the ease from
base types while still keeping the canonicity and a better display from the
quotient types. To show the application of definable quotients, we
used two examples, the set of integers and the set of rational
numbers. Some concrete cases are also presented to show how to
uniformly lift operations and theorems from base types, and how can we
benefit from the utilization of definable quotient structures.

