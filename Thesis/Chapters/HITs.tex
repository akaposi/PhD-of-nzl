\chapter{Homotopy Type Theory and higher inductive types}\label{HITs}


\hott (HoTT) is a new "\emph{hot}" field developed as a
branch between theoretical computer science
and mathematics. Indeed it arises from a new interpretation of intensional \mltt
into homotopy theory where identity types are interpreted as paths and
the identity types of "paths" as homotopies and extended with the Vladimir Voevodsky's \emph{univalence axiom} which identifies isomorphic
structures (formally speaking, it identifies isomorphism with
equality). 
\hott does not only help us model type theory with a focus on the equality, but also provides mathematicians type theoretical tools to study homotopy theory.

In category theory, it is very common to state some notions "... up to
isomorphism". Isomorphism is the only equivalent relation instead of
instead of "equality". This seems to be the inspiration for the
univalent axiom which can be seen as a formalisation of this vague notion.

In \hott, since paths and higher paths are an essential part of the
structure of types,  it is appropriate to interpret types as higher groupoids
which is a "multi-sorted object" which contains points for terms,
paths for identity types and higher paths for higher level iterated identity
types\cite{hott-online}. 

To accomodate this interpretation, the usual schema to define types --
\emph{inductive
types} lacks in expressive power, then a more
general schema for defining types is invented \textemdash \emph{higher
inductive types}. Briefly speaking, in a definition of a higher inductive type, there can
be constructors not only for points but also for paths.

With higher inductive types, it is much easier to define quotient
types. However the implementation of \hott in \itt is still an open problem. We
work on defining semi-simplicial types and \wog to solve this problem. There
is also the very new model using cubical sets proposed by Bezem,
Coquand and Huber in \cite{bezem2013model}.

In this chapter, we introduce some basic notions in \hott like
univalent axioms, homotopy types,
higher inductive types, and discuss quotient types in \hott and the
models of \hott. It serves as an essential prerequiste for next chapter.
For further reference, a well-written text book on Homotopy Type
Theory elaborated by many brilliant mathematicians and computer scientists is available \cite{hott}. 








\section{Basic Homotopy Type theoretical notions}

Homotopy type theory can be seen as an \itt extended with the
following axiom:

\subsection{Univalent axiom}

Vladimir Voevodsky proposed to add the following \emph{univalence} to
type theory, as the fundamental axiom of the \hott.
 
\begin{definition}
\emph{univalence axiom}. for any two types $X$ and $Y$, the function 
$$f : X = Y \rightarrow X \simeq Y$$ is an equivalence.
\end{definition}


The univalent axiom implies functional extensionality and
propositional equality(?).

\subsection{Homotopy types}

The \hott first provides an interpretation of type theory into
homotopy theory. Notice we mainly focus on homotopical notions rather
than lying down the topological basis first.

\begin{itemize}
\item Types are interpreted as spaces. $a : A$ can be stated as $a$ is
  a point of space $A$.
\item Terms are continous functions, for example, $f : A \rightarrow B$ is a
  continous function between spaces and it is equivalent to say $a$ is
  a point of space or $a : 1 \rightarrow A$ is a continous function.
\item Identity types are path spaces, 
\item Identity terms are 
\item Identity types of identity types are called homotopies (if we
  represent a path as a continous function $p : [0,1] \rightarrow X$).
\item There are also 3-homotopies and 4-homotopies and even higher
  levels which forms an infinite structure called  \og (or $\infty$-groupoids).
\end{itemize}

From the interpretation, it is natural to define the types as some
generalisation of setoids or groupoids.

It is all about adding more structures on types. 

A setoid is a set with
an equivalence relation. Categorically, terms are objects and in setoid models, there is at most one morphism
between terms which stands for the equivalence relation.
The reflexivity is the identity morphism, the symmetry provides an
inverse of each morphism and the transitivity is just the composition
of morphism. One important deduction is that for any $p : a \sim b, p
\dot p^{-1} = id$. However it is not true for groupoid models.

In groupoid models, Instead of the definitional equality $p : a \sim
b, p . p^{-1} = id$, we have a 2-dimensional isomorphism $G3 : p . p^{-1}
\sim id$ (It is one of the groupoid laws \cite{MR1686862}) which
stands for the propositional equality.
Setoids are just a special cases of groupoids. The morphism is not
unique in each homset, i.e. the identity types are
non-trivial.

The second level is non-trivial in a groupoid, what if the third level
and higher levels are non-trivial? It we turn the equalities, namely
groupoid laws of each level, into explicit morphisms (which are called
equivalence), we get \og.

By building the infinite tower of identity types, every type has a \wog \cite{van2011types}.
Similar to the interpretation of types as setoids, \wog is more
general choice. It has more expressive power than setoid models.


\section{Higher inductive types}

Inductive types are types which can be freely generated by finite
constructors. Constructors are functions of arbitrary number of
arguments which can be the type itself being defined. Examples like
natural numbers can be defined as,

\begin{itemize}
\item $0 : \N$
\item $suc : \N \rightarrow \N$
\end{itemize}

% In \itt, the uniqueness of identity types is not
% accepted in general, but derivable for types whose propositional
% equality is decidable. The homotopy interpretation fits
% nicely by provides higher levels structures which are weaker
% equivalence relation (compared to strict equality) between identity types.

In homotopy type theories, inductive types are not enough because we
should take into account the \emph{higher} structures of types, namely the iterated identity types on each
levels. Therefore, we need a new and more general approach to describe types: the higher
inductive types: a type does not only has constructors for terms but
also constructors for paths. One common example is a circle
$\base:\Sn^1$ (1-sphere) can be \emph{inductively} defined with two constructors,

\begin{itemize}
\item A point $\base:\Sn^1$, and
\item A path $\lloop : {\id[\Sn^1]\base\base}$.
\end{itemize}

Of course, this also implies that the eliminator should include the
scheme to elimate the paths as well. It means that if we want to
define some function $f : \base:\Sn^1 \rightarrow B$, assume $f(base)=b$ we
have to map $\lloop$ to
an identity path $l : b = b$ as well $ap_f(\lloop)=l$.

\section{Quotient types with help of higher inductive types}

By utilizing higher inductive types, it is simpler to encode the
equivalence relation into the definition of certain types. Therefore
it is possible to define quotient types in this model.

Assume $A$ is a set and $\_\sim\_ : A \rightarrow A \rightarrow \Prop$
($\Prop$ stands for mere propositions in \hott) which is an
equivalence relation. The quotient set $\qset{A}$
can be defined as a higher inductive type:

\begin{itemize}
\item $[\_] : A \rightarrow \qset{A}$
\item $eqv : (a,b : A) \rightarrow A \sim B \rightarrow  [a] =  [b]$

which is a set as well, so

\item $isSet : (x.y:\qset{A}) \rightarrow (p1,p2 : x = y) \rightarrow p1 = p2$

\end{itemize}

Notice that the first one is the constructor of objects, the second
one builds the morphism and the third one states that all path spaces
are contractible so that the resulting types are h-sets. It is not a
generic quotient construction of any types but specially for sets, so
we also it set-quotient. Since it is quotient types defined with the
help of HITs we also call it QITs (quotient inductive types).


\textbf{An example of Quotient Inductive types}
We can prove that QITs are more powerful than quotient types by an
example: \emph{unordered trees}. 

\todo{cite http://stackoverflow.com/questions/12525429/difference-between-ordered-and-unordered-rooted-trees}

\begin{definition}
An unordered tree is a tree where there is no ordering on subtrees.
\end{definition}

Here we considered $\omega$-ary trees. Firsrtly we define the ordered version $\mathsf{Tree}$ as:

\begin{itemize}
\item A leaf $l: \mathsf{Tree}$, or
\item A span which is indexed by $\mathbb{N}$, $sp : (\mathbb{N} \rightarrow \mathsf{Tree}) \rightarrow \mathsf{Tree}$,
\end{itemize}

Then we ought to define the equivalence relation so that we can encode
the unordered trees as quotient types.

\begin{itemize}
\item Leafs are trivially equal, $l \sim l$,
\item a leaf is not equivalent to a span and vice versa, and
\item two spans $sp~f$ and $sp~g$ are equivalent if $f$ is a
  permutation of $g$.
 $(\exists p: \mathbb{N} \rightarrow \mathbb{N}, p~\text{is bijective} \rightarrow
\forall n: \mathbb{N},  f~n \sim g~(p~n)) \rightarrow sp~f \sim sp~g$,
\end{itemize}

What would be the quotient type? $ \mathsf{Tree^{\sim}} =
\qset{\mathsf{Tree}}$ 

$\overline{sp} : (\mathbb{N} \rightarrow \mathsf{Tree^{\sim}}) \rightarrow \mathsf{Tree^{\sim}}$

such that for any $f : \mathbb{N} \rightarrow \mathsf{Tree}$

$\overline{sp}~ ([\_] \circ f) = [ sp~f ] $

it seems that it is impossible to define $\overline{sp}$ because we
cannot eliminate infinite number of $\mathsf{Tree^{\sim}}$.
However it is possible for binary trees.

$\overline{sp} : (\textbf{2} \rightarrow \mathsf{BTree}^{\sim}) \rightarrow
\mathsf{BTree}^{\sim}$ which is ismorphic to

$\overline{sp} : \mathsf{BTree}^{\sim} \rightarrow \mathsf{BTree}^{\sim} \rightarrow \mathsf{BTree}^{\sim}$

$\overline{sp}~a~b = lift~(lift~sp~a)~b$


For k-ary tree, we can follow the same approach to lift the the span
function for finite number of times. However, it is impossible to lift
it infinitely for $\omega$-ary trees (where the subtree is indexed by
the set of $\mathbb{N}$).

It means that we are unable to define this permutation tree using
quotient types because the definition and equivalence has to be
defined mutually (?).

However quotient inductive types are more powerful. It provides us the
way to define trees and paths simultaneously.

\begin{itemize}
\item A leaf $l: \mathsf{Tree}$, 
\item A set of subtrees $st : (\mathbb{N} \rightarrow \mathsf{Tree}) \rightarrow \mathsf{Tree}$,
  and
\item a set of pathes relates two permutated trees:

$l_{eq} : l  =_{\mathsf{Tree}} l $

$st_{eq} : \forall f~g : (\mathbb{N} \rightarrow \mathsf{Tree}) \rightarrow
f \sim_{p} g \rightarrow  st~f = st~g$
\end{itemize}

It is possible to define a binary permutation trees because the
equivalence relation is decidable. The lifting of functions can be
nested as we will show later \todo{}.
However a problem arises when defing a infinite permutation trees
whose child nodes can be inifinitely many on each level.

We use a symbol $Tree^{\sim}~A$ for the permutation trees.

To define a function on permutation trees, it is necessary to define
them as

$f : Tree~A \rightarrow B$ which respects the equivalence relation
$a~b : Tree~A, f a = f b$

\section{Quotient inductive inductive types}

Altenkirch points out that we could apply higher inductive types in a
specific way. We can construct a \emph{set} by using a a mere
relation. So in this sense, each non-trivial morphism is the unique
one. This looks like a setoid, but indeed there is no difference
between sets and setoids categorically because a skeleton of "setoids"
is just a discrete category which is equivalent to sets (?review).


\section{models of \hott}

\subsection{Kan simplicial sets}

To interpret types as weak $\omega$-groupoids, one main problems is
the complexity of the definition of weak $\omega$-groupoid. The
coherence conditions are very difficult to specified.

However it is feasible and simpler to interpret types as Kan
simplicial sets.

\begin{definition}
A simplicial set $X$ is a functor from $\Delta^{op}$ to $\Set$ where
$\Delta$ is the simplex category.
\end{definition}

$\Delta^{op}$ is a category whose objects are non-empty totally ordered
finite sets. The morphisms are order-preserving functions. 
Face maps and degeneracy maps are the most important morphisms in this
category. \todo{more explanation of face maps and degeneracy maps} 

A simplexes is a generalisation of a triangle to arbitrary
dimensions. 3-dimensional simplex is tetrahedron and k+1-simplex can
be obtained by adding one point to k-simplex which does not lie in the
dimension where the k-simplex is.

A simplicial complex is a collection of simplexes. Topologically speaking, it
is constructed by gluing n-dimensional simplexes together. 
\todo{show an example graph}

A simplicial set, therefore, can be illustrate by the same graph where
the set of points is given by $X_0$, the set of lines is $X_1$ and so
on. The graph looks very similar to n-groupoids. In fact simplical
sets can interpret types in a similar way (?).


\subsection{semi-simplicial sets}

\todo{cite https://uf-ias-2012.wikispaces.com/Semi-simplicial+types}

However the simplicial set model is not constructive as Coquand showed
that there is something classical in the simplical model.
The distinction of semi-simplicial set is there is only face maps
but no degeneracy maps. We can denote a semi-simplicial set as a
functor $X : \Delta_{inj} \rightarrow |Set$. The morphisms in $\Delta_{inj}$ are not only order preserving
but also injective.
A “iterated dependency” approach is believed to solve the coherence
issues.

%Klaus and me were trying to implement semi-simplical set in Agda. 


\subsection{Cubical sets}

Someone can deduce what is a cubical set from the name literaly. It is
also a functor $S : \Box^{op} \rightarrow \Set$ (or a presheaf on the
cube category $ \Box^{op}$).

\section{Summary}

In this chapter we introduce the basic notions in \hott, discuss the
various implementations of \hott. In the next chapter we will focus on
thesyntactic implementation of \wog following
Brunerie's approach. We attempt to formalise the groupoid model of
\hott in intensional type theory, specially in Agda.

