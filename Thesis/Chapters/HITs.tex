\chapter{Homotopy Type Theory and higher inductive types}\label{HITs}


\hott (HoTT) is a new "\emph{hot}" field developed as a
branch between theoretical computer science
and mathematics. Indeed it arises from a new interpretation of intensional \mltt
into homotopy theory where identity types are interpreted as paths and
the identity types of "paths" as homotopies. 

It is extended with Vladimir Voevodsky's \emph{univalence axiom} which identifies isomorphic
structures. 
Formally speaking, it identifies isomorphism with
equality.

In category theory, we usually use
isomorphism instead of "equality". The univalent axiom
seems like a formalisation of this vague notion.

In \hott, we interpret types as higher groupoids
which is a "multi-sorted object" which contains points for terms,
paths for identity types and higher paths for higher level iterated identity
types\cite{hott-online}. To accomodate this interpretation, inductive
types are not enough, then a more
general schema for defining types is invented \textemdash \emph{higher
inductive types}. Briefly speaking, in a definition of a higher inductive type, there can
be constructors not only for points but also for paths.
With higher inductive types, it is much easier to define quotient
types. However the implementation of \hott in \itt{} is still an open problem. We
work on defining semi-simplicial types and \wog{} to solve it. There
is also the very new model using cubical sets proposed by Bezem,
Coquand and Huber in \cite{bezem2013model}.

In this chapter, we will discuss some basic notions in \hott like
higher inductive types, quotient types in \hott and a discussion of
the implementation of \hott in \itt. It can be seen as an extended
introduction to \hott and serves as an essential prerequiste for next chapter.


\section{Basic Homotopy Type theoretical notions}

\subsection{Univalent axiom}

Vladimir Voevodsky proposed to add the following \emph{univalence} to
type theory, as the fundamental axiom of the \hott.
 
\begin{definition}
\emph{univalence axiom}. for any two types $X$ and $Y$, 
$$X = Y \simeq X \simeq Y$$ holds.
\end{definition}


\subsection{Homotopy types}

The \hott first provides an interpretation of type theory into
homotopy theory. Notice we mainly focus on homotopical notions rather
than lying down the topological basis first.

\begin{itemize}
\item Types are interpreted as spaces. $a : A$ can be stated as $a$ is
  a point of space $A$.
\item Terms are continous functions, for example, $f : A \rightarrow B$ is a
  continous function between spaces and it is equivalent to say $a$ is
  a point of space or $a : 1 \rightarrow A$ is a continous function.
\item Identity types are path spaces, 
\item Identity terms are 
\item Identity types of identity types are called homotopies (if we
  represent a path as a continous function $p : [0,1] \rightarrow X$).
\item There are also 3-homotopies and 4-homotopies and even higher
  levels which forms an infinite structure called  \og (or $\infty$-groupoids).
\end{itemize}

From the interpretation, it is natural to define the types as some
generalisation of setoids or groupoids.

It is all about adding more structures on types. 

A setoid is a set with
an equivalence relation. Categorically, terms are objects and in setoid models, there is at most one morphism
between terms which stands for the equivalence relation.
The reflexivity is the identity morphism, the symmetry provides an
inverse of each morphism and the transitivity is just the composition
of morphism. One important deduction is that for any $p : a \sim b, p
\dot p^{-1} = id$. However it is not true for groupoid models.

In groupoid models, Instead of the definitional equality $p : a \sim
b, p . p^{-1} = id$, we have a 2-dimensional isomorphism $G3 : p . p^{-1}
\sim id$ (It is one of the groupoid laws \cite{MR1686862}) which
stands for the propositional equality.
Setoids are just a special cases of groupoids. The morphism is not
unique in each homset, i.e. the identity types are
non-trivial.

The second level is non-trivial in a groupoid, what if the third level
and higher levels are non-trivial? It we turn the equalities, namely
groupoid laws of each level, into explicit morphisms (which are called
equivalence), we get \og.

By building the infinite tower of identity types, every type has a \wog \cite{van2011types}.
Similar to the interpretation of types as setoids, \wog is more
general choice. It has more expressive power than setoid models.


\section{Higher inductive types}

Inductive types are types which can be freely generated by finite
constructors. Constructors are functions of arbitrary number of
arguments which can be the type itself being defined. Examples like
natural numbers can be defined as,

\begin{itemize}
\item $0 : \N$
\item $suc : \N \rightarrow \N$
\end{itemize}

% In \itt, the uniqueness of identity types is not
% accepted in general, but derivable for types whose propositional
% equality is decidable. The homotopy interpretation fits
% nicely by provides higher levels structures which are weaker
% equivalence relation (compared to strict equality) between identity types.

In homotopy type theories, inductive types are not enough because we
should take into account the \emph{higher} structures of types, namely the iterated identity types on each
levels. Therefore, we need a new and more general approach to describe types: the higher
inductive types: a type does not only has constructors for terms but
also constructors for paths. One common example is a circle
$\base:\Sn^1$ (1-sphere) can be \emph{inductively} defined with two constructors,

\begin{itemize}
\item A point $\base:\Sn^1$, and
\item A path $\lloop : {\id[\Sn^1]\base\base}$.
\end{itemize}

Of course, this also implies that the eliminator should include the
scheme to elimate the paths as well. It means that if we want to
define some function $f : \base:\Sn^1 \rightarrow B$, assume $f(base)=b$ we
have to map $\lloop$ to
an identity path $l : b = b$ as well $ap_f(\lloop)=l$.

\section{Quotient types with help of higher inductive types}

By utilizing higher inductive types, it is simpler to encode the
equivalence relation into the definition of certain types. Therefore
it is possible to define quotient types in this model.

Assume $A$ is a set and $\_\sim\_ : A \rightarrow A \rightarrow \Prop$
($\Prop$ stands for mere propositions in \hott). The quotient set $\qset{A}$
can be defined as a higher inductive type:

\begin{itemize}
\item $[\_] : A \rightarrow \qset{A}$
\item $eqv : (a,b : A) \rightarrow A \sim B \rightarrow  [a] =  [b]$

which is a set as well, so

\item $isSet : (x.y:\qset{A}) \rightarrow (p1,p2 : x = y) \rightarrow p1 = p2$

\end{itemize}

Notice that the first one is the constructor of objects, the second
one builds the morphism and the third one states that all path spaces
are contractible so that the resulting types are h-sets. It is not a
generic quotient construction of any types but specially for sets, so we call it set-quotient. 



\section{Quotient inductive inductive types}

Altenkirch points out that we could apply higher inductive types in a
specific way. We can construct a \emph{set} by using a a mere
relation. So in this sense, each non-trivial morphism is the unique
one. This looks like a setoid, but indeed there is no difference
between sets and setoids categorically because a skeleton of "setoids"
is just a discrete category which is equivalent to sets (?review).

\todo{discuss with Thorsten}



\section{Various implementations of \hott}

\subsection{Simplicial sets}


\subsection{Cubical sets}


\subsection{Brunerie's syntactic approach}

\section{Summary}

In this chapter we introduce the basic notions in \hott, discuss the
various implementations of \hott. In the next chapter we will focus on
thesyntactic implementation of \wog following
Brunerie's approach. We attempt to formalise the groupoid model of
\hott in intensional type theory, specially in Agda.



\todo{Maybe I should combine HIT chapter with \wog model chapter?}