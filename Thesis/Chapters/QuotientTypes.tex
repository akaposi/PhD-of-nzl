\chapter{Quotient Types}
\label{qt}

Quotient types is one of the important extensional constructions in
type theory. Generally speaking, given a setoid $A$ and an
equivalence relation $\sim$ on it, a quotient type denoted as
$\qset{A}$ is a type generated by redefining equality of $A$ by
$\sim$. In \itt, we usually use setoids to represent quotients but it
is not satisfactory. It is a very important topic to extend \itt with
quotient types.

From a programmming persepective, there are more tools to define data
types. 
In the chapter, we first introduce the quotients in Mathematics and
then quotient types in type theory with categorically interpretations.


\section{Quotients in Mathematics}

% "M"athematics must be capital

Quotient is one of the basic notions in Mathematics. In arithematics, quotient
is the result of division, e.g.\ $2$ is the quotient of $8 \div 4$.

Moving on to abstract algebra, basic operations in arithmatics usually
have similar but essentially different meaning. For instance, the
product is extended to the cartesian product in set theory. 
Similarly quotient is also extended to other branches, such as set
theory, group theory, topology etc.

\begin{remark} \textbf{Example : quotient sets}

In set theory, we divide a set by an equivalence relation on it ,

\begin{definition}
\textbf{Equivalence relation}.
An equivalence relation is a binary
relation which is refleixive, symmetric and transitive.
\end{definition}

Intuitively, given any equivalence relation, a set can be partitioned into
some equivalence classes,

\begin{definition}
\textbf{Equivalence class}.
The equivalence class of an element $a$ is a set whose elements are
all equivalent to $a$
\begin{equation}
[ a ] = \{x : A \;| \; a \sim x \}
\end{equation}
\end{definition}

The collection of these equivalence classes is called
the quotient set.

\begin{definition}
\textbf{Quotient set}.
Given a set A equipped with an equivalence relation $\sim$, a quotient
set is denoted as $\qset{A}$,
\begin{equation}
\qset{A} = \{ [ a ] \;|\; a : A \}
\end{equation}
\end{definition}
\end{remark} 

Similar constructions also work for some other algebraic structures
e.g.\ spaces, groups, categories.


\section{Quotient types in type theory}

In \ett like NuPRL, it is possible to redefine
equality of some types, hence we can define extensional quotient
types. However since the equality is extensional, we cannot recover the witness of
the equality.

in  \itt like Agda, quotient types are unavailable. Alternatively \emph{setoids}
are usually used to represent quotients. 

\begin{definition}
\textbf{Setoid}. A setoid $(A,\sim)\,\colon\Set_1$ \footnote{Setoid
  could be universe polymorphic.}  contains a set $A$ and an
equivalence relation ${\,\sim\,}\colon A \to A \to \Prop$.
\end{definition}

Just like the quotient $4$
can be represented as the pair $(8,2)$, a quotient set $\qset{A}$ can
be represented by $(A, \sim)$.

In Agda, we define a setoid as

\begin{code}
\\
\>\AgdaKeyword{record} \AgdaRecord{Setoid} \AgdaSymbol{:} \AgdaPrimitiveType{Set₁} \AgdaKeyword{where}\<%
\\
\>[0]\AgdaIndent{2}{}\<[2]%
\>[2]\AgdaKeyword{field}\<%
\\
\>[2]\AgdaIndent{4}{}\<[4]%
\>[4]\AgdaField{Carrier} \<[22]%
\>[22]\AgdaSymbol{:} \AgdaPrimitiveType{Set}\<%
\\
\>[2]\AgdaIndent{4}{}\<[4]%
\>[4]\AgdaField{\_≈\_} \<[23]%
\>[23]\AgdaSymbol{:} \AgdaBound{Carrier} \AgdaSymbol{→} \AgdaBound{Carrier} \AgdaSymbol{→} \AgdaPrimitiveType{Set}\<%
\\
\>[2]\AgdaIndent{4}{}\<[4]%
\>[4]\AgdaField{isEquivalence} \AgdaSymbol{:} \AgdaRecord{IsEquivalence} \AgdaBound{\_≈\_}\<%
\\
\end{code}


However since we have a new representation
of sets in type theory, it is essential to redefine everything for sets on
setoids. For example, functions on setoids, equalities on setoids,
products on setoids etc. Moreover we have to define ``setoids''
on setoids and higher setoids.


Not only the manipulation of equivalence complicates
these definitions, from a programming perspective, setoids are also
unsafe because we have access to the underlying
 sets \cite{aan}. We can perform operations which does not respect
 equivalence relation and the result does not make any sense.

Therefore, setoids are not the best solution for quotients in \itt.
Ideally the sort of the type representing a quotient derived from a setoid
should also be $\Set$. It is
also the case in the other mathematical theories, the
base object and the quotient object are of the same sort. 


\section{Impredicative encoding of quotient types}

One of the option is to define quotient sets in set theoretical way.
In set theory, we have the subset relation such that we can construct equivalence
classes and then quotient set. 
However in \itt, the subset types are also unvailable and equivalence
classes cannot be implemented.

Vladimir Voevodsky introduces an impredicative definition of quotients
\footnote{\url{http://www.cse.chalmers.se/~coquand/cirm.pdf}} which has been encoded in Coq
\cite{voe:hset}. It requires resizing rules and propositional
univalence as prerequisites.

Given a setoid $(A,\sim)$, a quotient type $\qset{A}$ is defined as the product of a predicate
and a proof that it gives rise to an equivalence class and it is
non-empty. Resizing rules \footnote{\url{http://www.math.ias.edu/~vladimir/Site3/Univalent_Foundations_files/2011_Bergen.pdf}} are essential here, otherwise
$\qset{A}$ is in the universe of $Set_1$ which is not expected.

$\qset{A} = \Sigma P : A \to \Prop, \text{EqClass} ~P  \times (\exists
x : A, P ~ x)$

where the equivalence class proof is encoded as

$\text{EqClass} ~ P = \forall x, y : A, P ~x \to (P~ y \iff a \sim b)$

Notice that it is impossible to extract any element of $A$ from an
equivalence $P$ satifying $P$.  An element of $\qset{A}$ can be produced by

$[\_] : A \to \qset{A}$
$[ a ] = (\lambda b \to a \sim b, ... , (a , \text{refl}))$

where $\text{refl}$ is the reflexivity and the missing part can be
proved simply by symmetry and transitivity. 

To make it a valid definition, the following universal properties has
to been fulfilled:

$\forall x y : A, x \sim y \to [ x ] = [ y ]$

which can only be proved if we have propositional univalence i.e.\
propositional extensionality.

\begin{definition}
\textbf{propositional extensionality} (propositional univalence).

$\forall P, Q : \Prop, (P \iff Q) \iff P = Q.$
\end{definition}

It is also one of the extensional concepts which is not available in \itt.

Moreover, the proof part in this definition
is indeed encoded using
resizing rules \footnote{\url{http://phdsinlogic2014.wp.hum.uu.nl/files/2014/04/thierry-coquand1.pdf}} as follows,

$|| A || = (P : \Prop)(A \to P) \to P$

As we know the resizing rules makes the definition impredicative and
the propositional extensionality is unavailable in \itt. It is a valid
definition in the context of \hott which is also an extension of
intensional \mltt.








\subsection{Intensional quotient set}

In type theory, given a setoid $(A,\sim)$ a quotient type is a new
type formalised as follows,

\infrule[Q-\bf{Form}]
{A : \Set  \andalso \sim : A \to A
  \to \bf{Prop}  \andalso \sim \text{is an equivalence
    relation \footnote{i.e.\ it is reflexive, transitive and symmetric}}}
{\qset{A} : \Set}


\infrule[Q-\bf{Intro}]{a : A}{[ a ] : \qset{A}}


\infrule[Q-\bf{Ax}]
{a , b :  A  \andalso p : a \sim b}
{\text{Qax} ~ p : [a]=[b]}

where $=$ stands for propositional equality.

According to Hofmann's\cite{hof:95:sm} definition, we can define the eliminator


\infrule[Q-\bf{elim}]
{B :  \Set      \andalso f : A \to B             \andalso        q : \qset{A}\\
f-resp : (a, b : A)  \to (p : a \sim b)  \to f ~a =  f ~ b}
{\hat{f} ~ q : B}

with an induction principle

\infrule[Q-\bf{ind}]
{P : \qset{A} \to \Prop \andalso ~ h : (a : A) \to P ~ [ a ] \andalso ~ q :  q : \qset{A}}
{\text{Qind} ~h ~q :P~q}

An equivalent depedent eliminator is

\infrule[Q-\bf{d-elim}]
{B : \qset{A} \to  \Set \andalso f : (a : A) \to B\, [ a ] \andalso q :
\qset{A} \\
f-resp : (a, b : A)  \to (p : a \sim b)  \to f ~a
\stackrel{p}{=}  f ~b}
{\hat{f} ~q : B~ q}

Here, $\stackrel{p}{=}$ stands for equality under substitution on types.

A quotient is \emph{exact} (or effective) if we have

\infrule[Q-\bf{exact}]
{a , b :  A  \andalso p : [a] = [b] }
{\text{Qexact}~{p} : a \sim b}



\section{Categorical interpretation}


Categorically speaking, a quotient is a coequalizer. First, let's
review the definition of coequalizer.

\begin{definition}
\textbf{Coequalizer}.
Given two objects $X$ and $Y$ and two parallel morphisms $f, g : \morph{X}{Y}$ , a coequalizer is an object Q with a morphism $q : \morph{Y}{Q}$ such that $q \circ f = q \circ g$. It has to be universal as well. Any pair (Q' , q') $q' \circ f = q' \circ g$ has a unique factorisation u such that $q' = u \circ q$
\begin{displaymath}
    \xymatrix{X \ar@<0.5ex>[rr]^f \ar@<-0.5ex>[rr]_g && Y \ar[rr]^q
      \ar[ddrr]_{q'} && Q
      \ar@{.>}[dd]^u \\ \\
& &&& Q' }
\end{displaymath}
\end{definition}

A quotient is the coequalizer when we have two projections $\pi_1$ and
$\pi_2$ from the relation $R = \{(a_1,a_2) : A \times A ~|~ a_1 \sim a_2\}$
\begin{displaymath}
    \xymatrix{R \ar@<0.5ex>[rr]^{\pi_1} \ar@<-0.5ex>[rr]_{\pi_2} && A \ar[rr]^{ [\_]}
      \ar[ddrr]_{q'} && Q
      \ar@{.>}[dd]^u \\ \\
& &&& Q' }
\end{displaymath}


\subsection{Adjunction between {\textbf{Sets}} and \textbf{Setoids}}

From a higher point of view, Quotient is a functor defined as $\bf{Q}
(B , \sim) = \qset{B}$, which is
left-adjoint to $\nabla$, the trivial embedding functor from
\textbf{Sets} to \textbf{Setoids} $\nabla A = (A , =)$.

$$ \bf{Q}  \dashv \nabla$$

Recalling the definition of adjunction.

\begin{definition}
\text{Adjunction}.
Given two categories $A$ $B$, a functor $F : A \to B$ is left adjoint
to $G : B \to A$ if we have a natural isomorphism
$\Omega : hom_{B}(F ~\_ , \_) \to hom_{A}(\_, G ~\_)$
\end{definition}

The isomorphism

\begin{equation*}
\begin{aligned}
\qset{B} & \to A \\
\midrule
\midrule
(B , \sim) & \to (A , =)
\end{aligned}
\end{equation*}

the adjunction above can be given by following isomorphism,

$\Omega ~ f = f \circ [\_]$ and 

$\Omega^{-1} ~ g = \hat{g}$
 $f : \qset{B} \to A$,

which are proved by definitions of coequalizer:

$\Omega^{-1} (\Omega~f) = \dlift{f \circ [\_]} = f$ because
$\dlift{f \circ [\_]} \circ  [\_] =  f \circ [\_]$ and $  [\_] $ is
an epimorphism by definition.

$\Omega (\Omega^{-1}~g) = \hat{g} \circ [\_] = g$ by definition of lifting.


\subsection{Functional extensionality and quotient types}

As we have mentioned before, in \itt{} propositional equality $Id(A,a,b)$ is inhabited
if and only if $a$ and $b$ are definitionally equal terms. The Agda
definition could be written as

% \begin{code}

% data Id (A : Set) : A → A → Set where
%   refl : (a : A) → Id A a a

% \end{code}

However the equality of functions are not only judged  by
definitions. Functions are
usually viewed extensionally as black boxes. If two functions pointwise
generate the same outputs for the same inputs, they are equivalent
even though their definitions may differ. This is called
functional extensionality which is not inhabited \cite{alti:lics99} in original
\itt{} and can be expressed as following,

given two types $A$ and $B$, and two functions $f,\,g\,\colon A \to B$,

\[Ext = \forall\, x\colon A, f x = g x \to f = g\]

The problem seems easy to solve by just adding a constant $ext : Ext$
to \itt{} as following codes in Agda

% \begin{code}

% postulate
%   ext : {A : Set}{B : A → Set}{f g : (x : A) → B x}
%         → ((x : A) → Id (B x) (f x) (g x)) 
%         → Id ((x : A) → B x) f g

% \end{code}

However, postulating something could lead to inconsistence. If we
postulate $Ext$, then theory is no longer adequate, which means it is possible to define irreducible terms. 
It can be easily verified in Agda through formalising a non-canonical
term for a natural number by an eliminator of intensional equality. 

Using the eliminator |J| \footnote{It is originally
  used by Martin-L\"{o}f \cite{nor:90} and a good explanation could be
found in \cite{ngk:11}}  of the |Id A a b| :

% \begin{code}

% J : (A : Set)(P : (a b : A) → Id A a b → Set)
%     → ((a : A) → P a a (refl a))
%     → (a b : A)(p : Id A a b) → P a b p
% J A P m .b b (refl .b) = m b

% \end{code}
we can construct an irreducible term of natural number as
% \begin{code}

% irr : ℕ
% irr = J (ℕ → ℕ) (λ f g P → ℕ) (λ f → 0) (λ x → x) (λ x → x) (Ext refl)

% \end{code}

With this term, we can construct irreducible terms of any type $A$ by a
mapping $f : \N \to A$. This will destroy some good features of \itt
since it could leads to nonterminating programs.

Altenkirch investigates this issue and gives a solution in
\cite{alti:lics99}. He proposes an extension of \itt by a universe of
propositions $\Prop$ in which all proofs of same propositions are
definitionally equal, namely the theory is proof irrelevant. At the same time,
a setoid model where types are interpreted by a type and an equivalence relation acts as the metatheory and $\eta$-rules for
$\Pi$-types and $\Sigma$-types hold in the metatheory. The extended type
theory generated from the metatheory is decidable and adequate, $Ext$ is
inhabited and it permits large elimination (defining a dependent type by recursion). Within this type theory,
introduction of quotient types is straightforward. 
The set of functions are naturally quotient types, the hidden information is the
definition of the functions and the equivalence relation is the
functional extensionality.
% extension

There are more problems concerning quotient types and most
of them are related to equality. One of the main problems is how to lift the functions for
base types to the ones for quotient types. Only functions respecting the
equivalence relation can be lifted. Even in \ett, the implementation
of quotient types does not stop at replacing equality of the types. We
will discuss these in next section.



\section{Example of Quotients}

The introduction of quotient types is very helpful. Many types can be
defined using quotient types, some of them can only be defined with quotient types, such as real
numbers (the reason will be covered in \todo{here}).




quotient groups, quotient space, partiality monad.


\section{literature review}



In \cite{cab}, Mendler et al. have firstly considered building new types from a
given type using a quotient operator $//$. Their work is done in an
implementation of \ett, NuPRL. 

In NuPRL, every type comes with its own equality relation, so the quotient operator can be
seen as a way of redefining equality in a type. But it is not all
about building new types. They also discuss problems that arise from
defining functions on the new type which can be illustrated using a simple example. 

Assume the base type is $A$ and the new equivalence relation is $E$, the new
type can be formed as $A//E$. 

When we want to define a function $f \,\colon\, A//E \to Bool$,  $f\,a \not= f\,b$ may
exists for $a, b \,\colon A$ such that $E\,a\,b$. This will lead to
inconsistency since $E\,a\,b$ implies $a$ converts to $b$ in \ett{}, hence
the left hand side $f\,a$ can be converted to $f\,b$, namely we get $f\,b \not= f\,b$
which is contradicted with the equality reflection rule. 

Therefore a function is said to be well-defined \cite{cab} on the new type only
if it respects the equivalence relation $E$, namely

$$\forall \, a\,b\,\colon A, E\,a\,b \to f\,a = f\,b$$

We call this \emph{soundness} property in \cite{aan}.

 After the introduction of quotient types, Mendler further investigates
 this topic from a categorical perspective in ~\cite{men:90}. He uses
 the correspondence between quotient types in \mltt{} and coequalizers
 in a category of types to define a notion called \emph{squash types},
 which is further discussed by Nogin \cite{nog:02}.

To add quotient types to \mltt{}, Hofmann proposes  three models for
quotient types in his PhD thesis \cite{hof:phd}. The first one is a setoid model for
quotient types. In this model all types are attached with partial
equivalence relations, namely all types are setoids rather than
sets. Types without a specific equivalence relation can be seen as
setoids with the basic intensional equality. This is similar to
\ett{} in some sense. The second one is groupoid model which solves some problems
but it is not definable in \itt{}. He also proposes a third model to
combine the advantages of the first two models, but it also has some
disadvantages. Later in \cite{hof:95:sm} he gives a simple model in which we have type dependency only at the propositional level, he also shows that extensional Type Theory is conservative over \itt  extended with quotient types and a universe \cite{hof:95:con}.

Nogin \cite{nog:02} considers a modular approach to axiomatizing the
same quotient types in NuPRL as well. Despite the ease of constructing new types
from base types, he also discusses some
problems about quotient types. For example, since the equality is
extensional, we cannot recover the
witness of the equality.  He suggests including more axioms to
conceptualise quotients. He decomposes the formalisation of quotient type
into several smaller primitives such that they can be handled much
simpler.

Homeier \cite{hom} axiomatises quotient types in Higher Order Logic
(HOL), which is also a theorem prover. He creates a tool package to
construct quotient types as a conservative extension of HOL such that
users are able to define new types in HOL. Next he defines the
normalisation functions and proves several properties of
these. Finally he discussed the issues when quotienting on the
aggregate types such as lists and pairs.


Courtieu \cite{cou:01} shows an extension of Calculus of Inductive Constructions
with \emph{Normalised Types} which are similar to quotient types, but equivalence relations are replaced by normalisation functions. 
However not all quotient types have normal forms. Normalised types are
proper subsets of quotient types, because we can easily recover a quotient
type from a normalised type as below
%$$ \[ (A, Q, \class\dotph \colon A \to Q) \to (A, \lambda \,a \,b \to \class a = \class b)\]$$


Barthe and Geuvers \cite{bar:96} also propose a new notion called
\emph{congruence types}, which is also a special class of quotient
types, in which the base type are inductively defined and with a set
of reduction rules called the term-rewriting system. The idea behind
it is the $\beta$-equivalence is replaced by a set of
$\beta$-conversion rules. Congruence types can be treated as an
alternative to the pattern matching introduced in \cite{coq:92}. The main
purpose of introducing congruence types is to solve problems in
term rewriting systems rather than to implement quotient types.


Barthe and Capretta \cite{bar:03} compare different ways to setoids in type theory.
The setoid is classified as partial setoid or total setoid depending
on whether the equality relation is reflexive or not. They also
consider obtain quotients with different kinds of setoids, especially
the ones from partial setoids are difficult to define because the lack
of reflexivity.

Abbott, Altenkirch et al. \cite{abb:04} provides the basis for
programming with quotient datatypes polymorphically based on their
works on containers which are datatypes whose instances are
collections of objects, such as arrays, trees and so on. Generalising
the notion of container, they define quotient containers as the
containers quotiented by a collection of isomorphisms on the positions
within the containers.

Voevodsky \cite{voe:hset} implements quotients in Coq based on a set
of axioms of Homotopy Type Theory. It is based on the groupoid model
for \itt{} where isomorphisms are equalities. He firstly implement
equivalence class and use it to implement quotients which is an
analogy to the construction of quotient sets in set theory. 







\subsection{Propositional extensionality implies effective quotients}

\todo{question of where shall we apply pe}
with the proposional extensionality we can prove that
\begin{theorem}\label{thm-p-e}
Given $(P , prf) : \qset{A}$, all proofs of $\exists x : A, P (x)$ are equal
\end{theorem}

\begin{proof}
Given any two proofs of $\exists x : A, P (x)$ written as $(x , px)$ and $(y
, py)$, apply the $EqClass(P)$ to $(x, y, px, py)$ we know that $x
\sim y$. Hence the truncation of 
\end{proof}

%Given a element $a : A$, the equivalence class is

%$ [ x ] = \Sigma y : A, \exists P : X \to \Prop.


%$$[ x ] = \exists P : X \to \Prop, \forall a : X, a \sim x \iff P \, a$$


%proposition

%propositional extensionality (univalence)

%impredicative encoding 

%trunction in HoTT

%impredicative encoding 


Moreover we have another suprising theorem:

\begin{theorem} \label{PUEF}
if we have \emph{propositional univalence}, all quotients are effective.
\end{theorem}
\begin{proof}
Suppose we have a set $A$ with an equivalence relation $\sim : A \to A
\to \Prop$.

Given $a : A$, and a predicate $P : A \to \Prop$ defined as 
$$P~x := a \sim x$$

We need to show that this predicate preserves the equivalence
relation, $x \sim y \to P~x = P~y$, namely it is well-defined
\textit{w.r.t} $\sim$,

Suppose we have $x \sim y$,

by symmtry transitivity, we have

$a \sim x \iff a \sim y$ which is definitionally equal to

$P~x \iff P~y$, and by propositional univalence,

$P~x = P~y$

Satisfying this condition, we can lift $P$ as

$$\hat{P}~[ x ] \equiv  P~x$$

Suppose we have a premise $[ a ] = [ b ]$, it is true that

$$[ a ] = [ b ]$$

and then

$$P~a = \hat{P} [ a ] = \hat{P} [ b ] = P~b$$

which is just

$$a \sim a = a \sim b$$

with propositional univalence, we know that they are logically equivalent

$$a \sim a \iff a \sim b$$

since $a \sim a$ is the reflexivity which is true trivially,

$$ a \sim b$$ is also true.

Therefore we have a proof that $[ a ] = [ b ] \to a \sim b$ which means
that the quotient is effective.
\end{proof}

Alternatively, we can prove it as follows:

\begin{proof}

Firstly we prove the equivalence relation is well-defined on the
quotient types, namely it respects the equivalence relation:

Suppose we have $a \sim b$ and $c \sim d$, we can deduce $a \sim c \iff
b \sim d$. Then applying the propositional univalence axiom, we know
that $a \sim c = b \sim d$, hence the equivalence relation is
well-defined.

Because it is well-defined, we can lift it such that

$[ a ] ~\hat{\sim}~ [ b ] \equiv a \sim b$


From reflexivity of the equivalence relation, $\forall x : A, x \sim x$, 
we know that $\forall x : A, [x]~\hat{\sim}~[x]$.

Assume $[a]=[b]$, using J-eliminator in $[a]~\hat{\sim}~[a]$
(reflexivity), $[a]~\hat{\sim}~[b]$ which is definitionally equal to $a \sim b$, Hence the quotient is
effective.
\end{proof}

\begin{corollary}
If we generalise the theorem \ref{thm-p-e}, assume we quotient a set with a groupoid relation, $\infixeqv: A \to A \to
\Set$, the soundness $[\_]^{=}$ is an equivalence (or isomorphism),
namely it has an inverse \emph{effecitive}.
\end{corollary}

Note that the additional structure we attach is not propositional but
groupoid, therefore the quotient types have more higher structures, or
we can say $[\_]$ is functorial.

As usual, we have soundness

$[\_]^{=} : a \sim b \to [a] = [b]$

on the next level, we need

$[\_]^{id}: [ \text{refl}^{\sim} ]^{=} = \text{refl}$ and

$[\_]^{comp}: [ p ; q ]^{=} =  [ p ]^{=} ; [ q ]^{=} $


\begin{proof}

First, let's prove that $\infixeqv$ is well-defined (functorial).

Given $p : a \sim a'$ and $q : b \sim b'$,  we can constuct an
equivalence $p \sim_1 q : a \sim b \to a' \sim b'$ which is defined as
$(p \sim_1 q)(r) := p^{-1} ; r ; q$ (and vice versa)

\begin{displaymath}
    \xymatrix{a \ar[r]^r  \ar[d]_p & b \ar[d]^q \\
      a' \ar@{.>}[r] & b'\\ }
\end{displaymath}

From univalence axiom we can deduce that $a \sim b = a' \sim b'$.

On the next level, 

$(\trefl^{\sim} \sim_1 \trefl^{\sim})(r) \equiv \trefl^{\sim-1} ; r ; \trefl^{\sim} = r$ and

$((p;p') \sim_1 (q;q'))(r) \equiv (p;p')^{-1} ; r ; (q;q') =
p'^{-1};p^{-1};r;q;q' = p'^{-1}(p^{-1};r;q);q' \equiv (p' \sim_1
q')((p~q)(r))$

hence the equivalence relation is functorial and we can lift it as
well. Following condition must be satisfied,

$[a]~\hat{\sim}~[b] \equiv a \sim b$

By definition, $\trefl : a \sim a$ is also of type $[a]~\hat{\sim}~[a]$,
but to make it more clear we call it $\trefl_{\hat{\sim}}$.

The inverse of soundness, $[a]=[b] \to
[a]~\hat{\sim}~[b]$ can be defined as

$[\_]^{=-1} := \lambda p \to \transport(p)(\trefl_{\hat{\sim}})$

To prove it is indeed an inverse of soundness,

$ [[p]^{=}]^{=-1} \equiv \transport ([p]^{=})(\trefl_{\hat{\sim}}) $

\end{proof}
