\chapter{Quotient Types}
\label{QuotientTypes} % For referencing the chapter elsewhere, use \ref{QuotientTypes} 

\todo{Before 1st-Dec-2013}

Quotient is a very common notion is mathematics. Usually the first
quotient we learn is the result of division. $8 \div 4$ (or $8 / 4$)
gives the result 2 which is called quotient.

At first most mathematicians are only interested in numbers. As long
as they start working on other mathematical objects, some more
abstract structures, many notions are extended. As a simple case, the
product of numbers is extended to the product of vectors and the
product of sets. 

Similarly, quotient is also extended to other objects, for example the
quotient in set theory.

%----------------------------------------------------------------------------------------

\section{Quotients in mathematics}

\subsection{Quotient sets}
In set theory, we have a similar operation which turns some set
into another set but the divisor is not the same kind of object as the
dividend. We use equivalence relation to divide a set,

\begin{definition}

\textbf{Equivalence relation} An equivalence relation is a binary
relation which is refleixive, symmetric and transitive.

\end{definition}

Intuitively, given any equivalence relation a set is partitioned into
some cells, so that the elements equivalent to each other are in the
same cell. The cells are called equivalence classes.

\begin{definition}

\textbf{Equivalence class} 

\begin{equation}
[ a ] = \{x : A \;| \; a \sim x \}
\end{equation}

\end{definition}

The set of these equivalent classes is called
the quotient set.

\begin{definition}

\textbf{Quotient set} Given a set A equipped with an equivalence relation $\sim$, a quotient
set is denoted as $\qset{A}$ which contains the set of equivalence
classes.

\begin{equation}
\qset{A} = \{ [ a ] \;|\; a : A \}
\end{equation}

\end{definition}

Not only in set theory, the quotient of some algebraic structures is
a common notions in other branches of mathematics. The notion of
equivalence relation is extended to spaces, groups, categories and so
does the quotient derived using the same construction. 

Generally
speaking, it describes the collection of equivalent classes of some equivalent relation on sets, spaces or other abstract structures. In type theory, following similar procedure, quotient type is also a conceivable notion.


\paragraph{Quotient types}

%The basic notion in Martin-L¨of’s type theory is the notion of type. A type
%is explained by saying what an object of the type is and what it means for
%two objects of the type to be identical. This means that we can make the
%judgement -- http://www.cse.chalmers.se/research/group/logic/TypesSS05/Extra/nordstrom.pdf

In type theory, quotient type can be formalised as following:


\begin{equation*}
\infer[Q-\bf{Form}]{\qset{A}}{A & \sim\, : A \rightarrow A \rightarrow \bf{Prop}}
\end{equation*}

\begin{equation*}
\infer[Q-\bf{Intro}]{[ a ] : \qset{A}}{a : A}
\end{equation*}

\begin{equation*}
\infer[Q-\bf{elim}]{\hat{f} : (q : Q) \rightarrow B \, q}{
\begin{array}{l}
B : \qset{A} \rightarrow  \bf{Set}\\
f : (a : A) \rightarrow B\, [ a ]\\
(a, b : A)  \rightarrow (p : a \sim b)  \rightarrow f a  \, \stackrel{p}{=} \, f b\\
\end{array}}
\end{equation*}
 
In general, quotient types are unavailable in \itt.
Quotients are everywhere, for example, rational numbers, real numbers, multi-sets.
The introduction of quotient types is very helpful. Some of them can be defined 
more effectively, such as the
set of integers.
Some of them can only be defined with quotient types, such as real
numbers (the reason will be covered in \todo).

\section{Categorical intuition}

Categorically speaking, a quotient is a coequalizer.

\begin{definition}

Given two objects $X$ and $Y$ and two parallel morphisms $f, g : \morph{X}{Y}$ , a coequalizer is an object Q with a morphism $q : \morph{Y}{Q}$ such that $q \circ f = q \circ g$. It has to be universal as well. Any pair (Q' , q') $q' \circ f = q' \circ g$ has a unique factorisation u such that $q' = u \circ q$
\end{definition}

\begin{displaymath}
    \xymatrix{X \ar@<0.5ex>[rr]^f \ar@<-0.5ex>[rr]_g && Y \ar[rr]^q
      \ar[ddrr]_{q'} && Q
      \ar@{.>}[dd]^u \\ \\
& &&& Q' }
\end{displaymath}


\begin{displaymath}
    \xymatrix{A \times A \ar@<0.5ex>[rr]^{\pi_1} \ar@<-0.5ex>[rr]_{\pi_2} && A \ar[rr]^{[\cdot]}
      \ar[ddrr]_{q'} && Q
      \ar@{.>}[dd]^u \\ \\
& &&& Q' }
\end{displaymath}



\subsection{Adjuction between {\textbf{Sets}} and \textbf{Setoids}}

From a higher point of view, Quotient is a Functor which is left-adjoint to $\nabla$ which is the trivial embedding functor from \textbf{Sets} to \textbf{Setoids}.

\begin{definition}

$\nabla A = (A , \equiv)$, $Q (B , \sim) = \qset{B}$

\end{definition}

We have following isomorphism for  the adjunction of the Quotient
Functor and $\nabla$ functor.

\begin{equation*}
\begin{aligned}
\qset{B} & \rightarrow A\\
\midrule
\midrule
(B , \sim) & \rightarrow \nabla A\\
\end{aligned}
\end{equation*}


\section{Example of Quotients}


Because different sets may contain the same elements, we
have the subset relation such that we can construct equivalence
classes then quotient set. 
In type theory we have to give constructors for any type before we can construct elements, which is different to the situation in set theory that
elements exist before we construct quotient sets. Therefore this approach to construct quotients in set
theory has some problems in type theory. In fact, Voevodsky constructs
quotients using this approach in Homotopy Type Theory using Coq
\cite{voe:hset} but here we
mainly discuss how to reinterpret quotient sets in the current settings of\itt(e.g. Agda).



quotient groups, quotient space, partiality monad.