\chapter{Real numbers and other undefinable quotients}
\label{rl}

If you mean something specific, always write "the"

In the previous chapter, only definable quotient types are
investigated. But some other
types are undefinable in \itt{} without quotients. In this chapter, the real numbers,
the multisets and
the partiality monad and also the proofs that
the undefinablity of them will be disccussed in detail.


\section{Real numbers}

We have several choices to represent real numbers. We choose Cauchy
sequences of rational numbers to represent real numbers \cite{bis:85}.

$$\R_{0} = \set{s : \N\to\Q \mid \forall\varepsilon
  :\Q,\varepsilon>0\to\exists m:\N, \forall i:\N, i>m\to \vert  s_i -
  s_m \vert  <\varepsilon}$$

It is implementable in Type Theory, but there is a problem of the
choice of type of the second part of the cauchy sequence, namely the property
that it is a cauchy sequence. Do we distinguish the same sequences
with different proofs? Logically speaking we should not. It means that
we need to truncate it to proposition but we will lose the important
tool to guess what the real number is. To avoid this problem, we could
use an alternative equivalent definition which is a subset of $\R_{0}$:

$$\R_0' = \set{f: \N\to\Q \mid \forall k
  :\N,\forall m , n > k, \to \vert  f_m -
  f_n \vert  < \frac{1}{k}}$$

With the definition, the condition part is propositional and we can
guess the number by applying any number k to the sequence and we know
the interval where it should be located.

Different cauchy sequences can represent the same number. Therefore an equivalence relation\footnote{
The Agda version is in Appendix} is expected. In mathematics two Cauchy sequences $\R_0$ are said to be
equal if their pointwise difference converges to zero,

$$r \sim s = \forall\varepsilon :\Q,\varepsilon>0\to\exists m:\N,
\forall i:\N, i>m\to \vert  r_i - s_i \vert <\varepsilon$$

\subsection{Non-normalizability of Cauchy Sequences}

To prove that it is impossible to give a full definable quotient
structure of real numbers with the setoid of cauchy sequences, we
could show it by proving that it is impossible to define a
normalisation function for the cauchy sequences.

\begin{definition}\label{def:nor}
We say that a quotient structure $\qset{A}$ is definable via a
normalisation, if we have a normalisation function
 \begin{equation}
  nf : A \to A
 \end{equation}
 with the property that it respects $\sim$
\begin{equation}
 p : \prd{c_1,c_2 : A} c_1 \sim c_2 \to nf(c_1) = nf(c_2).
\end{equation}
such that
 \begin{equation}
 q : \prd{c : A}  {nf(c)} \sim c.
 \end{equation}
\end{definition}

It is equivalent to say that we have a definable quotient structure in
the sense of \cite{aan}, because we can form the set of equivalence
classes as

\begin{equation*}
Q \defeq \Sigma_{c : \R_0} nf(c) = c
\end{equation*}

where the second part is propositional, and the "normalisation''
function can be defined as

\begin{equation*}
[c] \defeq nf(c) , p(q)
\end{equation*}

and the embedding function is just the first projection. The properties can be verified
easily.

In the other way around, the true normalisation function is just

\begin{equation*}
n \defeq emb \circ [\_]
\end{equation*}

and the properties hold as well.


We have made an attempt in the original version of our \cite{aan}
draft, but there is something important problem pointed out by Martin Escardo. Laterly, Nicolai
Kraus suggests to fix the proof by proving it as a meta-theoretical
property. We will show an adaption of his proof here.

\textbf{Some preliminaries}

In fact the proof is mainly conducted using topological tools. The
following definitions are helpful for someone who are not so familiar
with topological concepts.


\begin{definition}
Metric space. In mathematics, a metric space is a set where a notion
of distance (called a metric) between elements of the set is defined.

A metric space is an ordered pair $(M , d)$ where M is a set and d is a metric on M:
\begin{enumerate}
\item M is a set,
\item and $d : M \times M \rightarrow \R$ s.t.
\item $d (x , y) = 0 \iff x = y$
\item $d(x,y)=d(y,x)$
\item $d(x,y)+d(y,z) \ge d(x,z)$ 
\end{enumerate}
\end{definition}

We usually define a standard topological structure for discrete types.

\begin{itemize}

\item $(\bool , h)$ where 
$h(m,n) =
\left\{
	\begin{array}{ll}
		0  & \mbox{if } m = n \\
		1 & \mbox{if } m \neq n
	\end{array}
\right.
$

\item $(\N , d)$ where 
$d(m,n) =
\left\{
	\begin{array}{ll}
		0  & \mbox{if } m = n \\
		1 & \mbox{if } m \neq n
	\end{array}
\right.
$

\item $(\Q , e)$ where 
$e(m,n) =
\left\{
	\begin{array}{ll}
		0  & \mbox{if } m = n \\
		1 & \mbox{if } m \neq n
	\end{array}
\right.
$

\end{itemize}

We use a slightly different definition of cauchy sequences of rational
numbers here:

\begin{definition}
 We call a function $f : \PN \to \Q$\footnote{we
   use $\PN$ instead of $\N$ because $n$ must be positive in $\frac 1 n$} a \emph{Cauchy Sequence} if it satisfies
 \begin{equation}
  \iscauchy (f) \defeq \fa{n}{\PN} \fa{m} {\PN} m > n \to |f(n) - f(m)| < \frac 1 n. \label{eq:prop-property-short}
 \end{equation}
 The type of Cauchy Sequences is thus
 \begin{equation*}
  \R_0 \defeq \sm{f : \PN \to \Q} \iscauchy (f).
 \end{equation*}
\end{definition}

And the standard metric space for the sequences $\PN \rightarrow \Q $
is defined by the distance function

\begin{equation}
g(f_1,f_2) = 2^{-\mathsf{inf}\{k \in \N \; | \; f_1(k) \not= f_2(k)\}}
\end{equation}

Among all these standard metric spaces, It is a folklore that all
definable functions are continous.

\begin{theorem}\label{defcon}
All definable functions are continous.
\end{theorem}

Let us introduce the following auxiliary definition:
\begin{definition}
 For a sequence $f : \N \to \Q$, we say that $f$ is \emph{Cauchy with
   factor $k$}, written as $\iscauchy_k$, for some $k \in \Q$, $k > 0$, if
 \begin{equation}
  \iscauchy_k(f) \; \defeq \; \fapairs{n}{m}{\N} m > n \to |f(n) - f(m)| < \frac 1 {k \cdot n}. \label{eq:cauchy-aux}
 \end{equation}
 The usual Cauchy condition $\iscauchy$ is therefore ``Cauchy with factor $1$''.
\end{definition}

\begin{remark} If we claim a function $f$ is defined on $\R_0$ that
  respects $\sim$, it means that we have a proof
\begin{equation}
 p : \prd{c_1,c_2 : \R_0} c_1 \sim c_2 \to f(c_1) = f(c_2).
\end{equation}

\end{remark}

Now we have enough tools to prove the following proposition.

\begin{proposition} \label{prop:main}
 \textbf{"$\R_0 / \sim$" is connected.} It means that any continuous function $f$ 
 \begin{equation}
  f : \R_0 \to \bool
 \end{equation} that respects $\sim$ is constant. We prove that it is impossible to find $c_1, c_2:
\R_0$ such that $f(c_1) \not= f(c_2)$ meta-theoretically, instead of
deriving a proof term of this in type theory.

The definability of function implies that it is a continous function \ref{defcon}
between the standard metric spaces for $\R_0$ and $\bool$\footnote{The
  metric of $\R$ comes from the first component. Technically, if
  $\R_0$ is defined by \ref{eq:usedbytxa}, this would not make it a
  metric space (as the distance between non-equal elements could be
  $0$); however, this would not matter, and for our definition, there
  is no problem anyway.}.

\end{proposition}



\begin{proof}

The general idea is to interpret our definition in classical mathematics,
assume we have a non-constant function and deduce a
contradiction.
 
Consider the ``naive'' set model (with ``classical standard mathematics'' as meta-theory). This clearly works if we are in a minimalisitic type theory with $\Pi$, $\Sigma$, $\mathsf{W}$, $=$, $\mathbb N$; however, if we restrict ourselves to the types in the lowest universe of homotopy type theory (which is enough), it also works for HoTT.
 We use $\tometa \cdot$ as an interpreation function; 
 for example, we write $\R$ for the field of real numbers which can be defined as $\tometa {\R_0} / \tometa{\sim}$. 
 By abuse of notation, we write $\tometa {\R_0}$ for the set of Cauchy sequences in the model that fulfill the Cauchy condition, without the actual proof thereof. This is justified as this property is propositional. 
 
 For readability, we use the symbol $=$ for equality in the theory as well in the model, and we do not use the semantic brackets for natural numbers such as $2$ or $4$.
 In the model, we use $\clim \cdot : \tometa {\R_0} \to \R$ as the
 function that gives us the limit of a Cauchy sequence (Not all
 functions in the ``naive'' set model have to be continuous).
 Thus, for $r : \R_0$, we write $\climeta r \in \R$ for the real number it represents.

Assume $f, p$ are given. We prove that $\tometa f : \tometa {\R_0} \to \tometa \bool$ is constant in the model, which implies the statement of Proposition \ref{prop:main}.

Thus, assume $\tometa f$ is non-constant,  there are $c_1, c_2 : \tometa {\R_0}$ with $\tometa f(c_1) \not= \tometa f(c_2)$. 

 Define
 \begin{align}
  m_1 & \defeq \mathsf{sup} \{\clim d \in \R \; | \; d \in \tometa {\R_0}, \clim d \leq \mathsf{max}(\clim{c_1}, \clim{c_2}), \tometa f(d) = \tometa \btrue \}\\
  m_2 & \defeq \mathsf{sup} \{\clim d \in \R \; | \; d \in \tometa {\R_0}, \clim d \leq \mathsf{max}(\clim{c_1}, \clim{c_2}), \tometa f(d) = \tometa \bfalse \}
 \end{align}
 (note that one of these two necessarily has to be $\clim{c_1}$ or $\clim{c_2}$, whichever is bigger).
 Set $m \defeq \mathsf{min}(m_1,m_2)$, and we can observe that in
 \emph{every} neighborhood $U$ of $m$, given any $t$,
we can always find another point $x \in U$ such that $x = \clim e$
(for some $e$) with $\tometa f(e) \not= \tometa f(t)$.
 
 Let $c \in \tometa {\R_0}$ be a Cauchy sequence such that $\clim c$ is equal to $m$. 
 We may assume that $c$ satisfies the condition
 $\tometa{\iscauchy_5}$.\footnote{the factor 5 is chosen due to the
   need of a later proof.} 
%  The slight difference ($\frac 1 n$ is replaced by $\frac 1 {4n}$) will be important.
%  Without loss of generality, assume $\tometa{f}(c) = \tometa \btrue$. % not needed.
 
 As $f$ (and thereby $\tometa f$) is continuous (remember the metric
 spaces), there exists $n_0 \in \tometa {\N}$ such that for any Cauchy
 sequence $c' \in \tometa {\R_0}$, if the first $n_0$ sequence
 elements of $c'$ coincide with those of $c$ (namely the distance
 $g(c, c') = 2^{-\mathsf{inf}\{k \in \N \; | \; c(k) \not= c'(k)\}}
 \leq 2^{-{n_0}}$ ), then $\tometa f(c') = \tometa f(c)$. 
 Write $U \subset \tometa {\R_0}$ for the set of Cauchy sequences which fulfill this property, and $\clim U \defeq \{\clim d \; | \; d \in U\}$ for the set of reals that $U$ corresponds to.
 
 We claim that $\clim U$ is a neighborhood of $m$. 
 More precisely, we prove: 
 The interval $I \defeq (m - \frac 1 {2n_0} , m+ \frac 1 {2n_0})$ is contained in $\clim U$.
 Let $x \in \R$ be in $I$. There is a sequence $t : \tometa {\N \to \Q}$ such that $\tometa {\iscauchy_{5n_0}}(t)$ and $\clim t = x$. 
 Let us now ``concatenate'' the first $n_0$ elements of the sequence $c$ with $t$, that is, define
 \begin{align}
  &g : \tometa{\N \to \Q} \\
  &g (n) = \begin{cases}
            c(n) & \text{if $n \leq n_0$} \\
            t(n-n_0) & \text{else}.
           \end{cases}
 \end{align}

 Observe that $g$ is also a Cauchy sequence, i.e. $\tometa \iscauchy(g)$: The only thing that needs to be checked is whether the two ``parts'' of $g$ work well together. Let $0 < n \leq n_0$ and $m > n_0$ be two natural numbers. We need to show that
 \begin{equation}
  |g(n) - g(m)| < \frac 1 n.
 \end{equation}
 Calculate
 \begin{align}
  & \quad |g(n) - g(m)| \\
  = & \quad|c(n) - t(m - n_0)| \\
  = & \quad |c(n) - \clim c + \clim c - \clim t + \clim t - t(m - n_0)| \\
  \leq & \quad | c(n) - \clim c | + |\clim c - \clim t| + |\clim t - t(m - n_0)| \\
  \leq &  \quad  \frac{1}{5n}  + \frac{1}{2n_0} + \frac{1}{5n_0 \cdot (m-n_0)} \\
  \leq &  \quad  \frac{1}{5n}  + \frac{1}{2n} + \frac{1}{5n_0} \\
  <  & \quad \frac 1 n.
 \end{align}

From the continuity property of $f$ and the definition of
$g$ we know that $\tometa f (g) = \tometa f(c)$. Clearly, $\clim g =
\clim t = x \in I$. Therefore \emph{all}  $s \in \tometa {\R_0}$ with $\clim
s \in I$, we can use the "concatenation'' approach to find a $g$
satisfies $s \, \tometa \sim \,  g$ (namely $\clim s = \clim g$), and by the condition that $f$ (and
thereby $\tometa f$) repects $\sim$, we can conclude that $\tometa f
(s) = \tometa f (g) = \tometa f(c)$.
 
%  It is easy to see that $g$ is a Cauchy sequence, i.e. $\tometa \iscauchy(g)$,\footnote{Of course, this is why we needed $\iscauchy_5$ instead of $\iscauchy$ earlier. The $5$ is somewhat arbitrary (todo: think about what is actually needed).} with $\clim g = x$. Further, the first $n_0$ elements of $g$ coincide with those of $c$, and thus, $\tometa f(g) = \tometa f(c)$.
 
However, as we have seen, in \emph{every} neighborhood of $m$, and thus in particular in $(m - \frac 1 {2n_0} , m+ \frac 1 {2n_0})$, there is an $x$ such that $x = \clim e$ (for some $e$) with $\tometa f(e) \not= \tometa f(c)$, in contradiction to the just established statement.
\end{proof}

The proposition that "$\R_0 / \sim$ is connected'' implies the
following corollary:

\begin{corollary}\label{dis:con}
Any continous function from $\R_0$ to any discrete type that respects $\sim$ is constant.
\end{corollary}



\begin{theorem}
Any continuous endofunction $f$ on $\R_0$ that respects $\sim$ which means
 \begin{equation}
  p : \prd{c_1,c_2 : \R_0} c_1 \sim c_2 \to f(c_1) = f(c_2).
 \end{equation}
 is constant (in the sense of corollary \ref{dis:con}).
\end{theorem}
\begin{proof}

Assume we have $f,p$ as required. 

To prove $f$ is constant, it is enough to show that the sequence part is
constant because the proof part is propositional. Again, by slight abuse of notation, we write $\tometa f : \tometa{\R_0} \to \tometa {\R_0}$, omitting the proof part of $f$.
 

Given a postive natural number $n : \tometa \PN$, we have a projection
function $\pi_{n} : \tometa{ \R_0} \to \tometa \Q$. Define a function
$h_n : \tometa {\R_0} \to \tometa \Q$ as

 \begin{equation*}
 h_n(c) \defeq \pi_{n} \circ f
 \end{equation*}

By corollary \ref{dis:con} we know that $h_n$ is constant, hence $f$
is constant everywhere, it is enough to show that $f$ is constant.


% By Proposition \ref{prop:main} we know it is a constant function and
% we can observe that $h (\mathsf 0) = true$, can we infer that $\forall
% c : \tometa {\R_0}, \tometa f c = \tometa f (\mathsf 0)$


%  We only need to show that $\pi_1 \circ f$ (the actual sequence) is constant as the proof of being a Cauchy sequence is propositional.\footnote{Even if $\iscauchy$ is not a propositional predicate (as in \ref{eq:usedbytxa}), it will still be true that $m$ is constant. This is simply because $\sim$ is defined only in terms of the actual sequence part.} Again, by slight abuse of notation, we write $\tometa f : \tometa{\R_0} \to \tometa {\R_0}$, omitting the proof part of $f$.
 
%  Given $c : \tometa {\R_0}$, we want to show $\tometa f (c) = \tometa f (\mathsf 0)$, where $\mathsf 0$ is the sequence that is constantly $0$. 
%  To do so, it is enough to show that, for a given $k : \tometa \N$, we have $\tometa{f}(c)(k) = \tometa{f}(\mathsf 0)(k)$. 
%  If this was not true, we would have a function $\tometa \R \to \tometa \bool$, defined by 
%  \begin{equation*}
%   \lambda c . \mathsf{isEquval }\left(\tometa f (c)(k)\right) \left(\tometa f (\mathsf 0)(k)\right)
%  \end{equation*}
% that is not constant, contradicting Proposition \ref{prop:main}.
\end{proof}


\begin{corollary}
 There is no definable normalisation function on $\R_0$ in the sense
 of \ref{def:nor}
\end{corollary}

\begin{corollary}
 $\R_0 / \sim$ is not definable in the sense of \cite{aan}.
\end{corollary}


However, it doesn't imply that we cannot define the set of real numbers in minimalisitic type theory with $\Pi, \Sigma, \mathsf{W}, =,
\N$. The meaning of definability of real numbers is still not clear enough. To make it more precise, we define
it as whether there is a type $A$ in $\mathsf{TT}$ (minimalisitic type
theory) such that its embedding $\tometa A$ in $\mathsf{TT} + \mathsf{Q}$ (type theory
extended with quotient types) is isomorphic to $\tometa \R_0
/ \tometa \sim$ (where it is a valid type). We conjecture that it is still not
definable.

\begin{proof}
Assume the set of real numbers is definable, we have a type $A$ and
its embedding in $\mathsf{TT} + \mathsf{Q}$  is $\tometa A$. It also
gives us a normalisation function and a representative function
between $\tometa \R_0$ and $\tometa A$.

\todo{P(T) -> Connected (T) -> Contractible(T), $\R/ \sim$ is connected but
  not contractible?}

\end{proof}

\subsection{Cauchy completeness of the Cauchy reals}

\todo{expand Cauchy completeness of Cauchy reals: it should rely on
  the axiom of countable choice}

Whether our definition of Cauchy sequence is Cauchy complete? In other
words, is there a representative Cauchy sequence as a limit for every
equivalence class? The answer is no.

In the HoTT book \cite{hott}, an alternative definition is used
instead which is called cauchy approximation. Because every
approximation has been proven to have a limit, it is Cauchy complete.
The definition uses the higher inductive types which will be discussed
in later Chapter \ref{HITs}.

\section{Multisets}

\begin{definition}

Multiset. A multiset (or bag) is a set without the constraint that there is no repetitive elements.

\end{definition}

A set is just a special case of multiset when the
\emph{multiplicity}(the number of the occurences) of every element is one.
Multisets (or bags) are believed to be used in ancient times, but it
is only 
studied by mathematicans from 20th century.

In set theory, a multiset is defined as a pair of a set $A$ and a
occurences counting function $m : A \rightarrow \N$.

However in type theory, the set-theoretical \emph{"set"} is not a
primitive notion and we need to define multisets in a type-theoretic
way. 
In type theoretical language, a multiset can be seen as a unordered
list. It can be encoded as lists which identifies
permutations. To make it simpler we use length-explict list
\emph{Vector} ("Vector A 3" stands for a list of type A of length 3).


The equivalence relation required is as follows:

Given two lists of type $A$ and has length n, $p \, q : Vector\,A\,n$,
they are equivalent if we have an isomorphism between them.

$$ p \sim q = \Sigma \phi : Fin\,n \rightarrow Fin\,n, \phi\,is\,a\,
bijection \wedge \forall x, p_x = q_{\phi(x)}$$


\todo{decidable order -> of course definable, A-> N definable, give a
  explicit proof}

\todo{definable -> split quotient because definable is too general}


\section{Partiality monad}

The partiality monad is a coninductive type which is available in the
standard library of Agda. It is used to encode delayed computation.

\begin{code}
\\
\>\AgdaKeyword{data} \AgdaDatatype{Delay} \AgdaSymbol{(}\AgdaBound{A} \AgdaSymbol{:} \AgdaPrimitiveType{Set}\AgdaSymbol{)} \AgdaSymbol{:} \AgdaPrimitiveType{Set} \AgdaKeyword{where}\<%
\\
\>[0]\AgdaIndent{2}{}\<[2]%
\>[2]\AgdaInductiveConstructor{now} \AgdaSymbol{:} \AgdaBound{A} \AgdaSymbol{→} \AgdaDatatype{Delay} \AgdaBound{A}\<%
\\
\>[0]\AgdaIndent{2}{}\<[2]%
\>[2]\AgdaInductiveConstructor{later} \AgdaSymbol{:} \AgdaDatatype{∞} \AgdaSymbol{(}\AgdaDatatype{Delay} \AgdaBound{A}\AgdaSymbol{)} \AgdaSymbol{→} \AgdaDatatype{Delay} \AgdaBound{A}\<%
\\
\end{code}

A non-terminating program can be defined in a coinductive way.

\begin{code}
\\
\>\AgdaFunction{never} \AgdaSymbol{:} \AgdaSymbol{\{}\AgdaBound{A} \AgdaSymbol{:} \AgdaPrimitiveType{Set}\AgdaSymbol{\}} \AgdaSymbol{→} \AgdaDatatype{Delay} \AgdaBound{A}\<%
\\
\>\AgdaFunction{never} \AgdaSymbol{=} \AgdaInductiveConstructor{later} \AgdaSymbol{(}\AgdaCoinductiveConstructor{♯} \AgdaFunction{never}\AgdaSymbol{)}\<%
\\
\end{code}

We have two equality for the Delay type: strong bisimilarity and weak
bisimilarity. 
Two computation are strongly bisimilar if they are the
same after the same number of steps delay (there can be infinite
steps).

\begin{code}
\\
\>\AgdaKeyword{data} \AgdaDatatype{\_∼\_} \AgdaSymbol{\{}\AgdaBound{A} \AgdaSymbol{:} \AgdaPrimitiveType{Set}\AgdaSymbol{\}} \AgdaSymbol{:} \AgdaDatatype{Delay} \AgdaBound{A} \AgdaSymbol{→} \AgdaDatatype{Delay} \AgdaBound{A} \AgdaSymbol{→} \AgdaPrimitiveType{Set} \AgdaKeyword{where}\<%
\\
\>[2]\AgdaIndent{4}{}\<[4]%
\>[4]\AgdaInductiveConstructor{now} \<[11]%
\>[11]\AgdaSymbol{:} \AgdaSymbol{∀} \AgdaSymbol{\{}\AgdaBound{x}\AgdaSymbol{\}} \AgdaSymbol{→} \AgdaSymbol{(}\AgdaInductiveConstructor{now} \AgdaBound{x}\AgdaSymbol{)} \AgdaDatatype{∼} \AgdaSymbol{(}\AgdaInductiveConstructor{now} \AgdaBound{x}\AgdaSymbol{)}\<%
\\
\>[2]\AgdaIndent{4}{}\<[4]%
\>[4]\AgdaInductiveConstructor{later} \<[11]%
\>[11]\AgdaSymbol{:} \AgdaSymbol{∀} \AgdaSymbol{\{}\AgdaBound{x} \AgdaBound{y}\AgdaSymbol{\}} \AgdaSymbol{(}\AgdaBound{x∼y} \AgdaSymbol{:} \AgdaDatatype{∞} \AgdaSymbol{((}\AgdaFunction{♭} \AgdaBound{x}\AgdaSymbol{)} \AgdaDatatype{∼} \AgdaSymbol{(}\AgdaFunction{♭} \AgdaBound{y}\AgdaSymbol{)))} \AgdaSymbol{→} \AgdaSymbol{(}\AgdaInductiveConstructor{later} \AgdaBound{x}\AgdaSymbol{)} \AgdaDatatype{∼} \AgdaSymbol{(}\AgdaInductiveConstructor{later} \AgdaBound{y}\AgdaSymbol{)}\<%
\\
\end{code}

We inductively define an operator which states that "x terminates with
y" if we write $\AgdaBound{x}\AgdaSymbol{))} \AgdaDatatype{↓} \AgdaBound{y}$.

\begin{code}
\\
\>\AgdaKeyword{infix} \AgdaNumber{4} \_↓\_\<%
\\
%
\\
\>\AgdaKeyword{data} \AgdaDatatype{\_↓\_} \AgdaSymbol{\{}\AgdaBound{A} \AgdaSymbol{:} \AgdaPrimitiveType{Set}\AgdaSymbol{\}} \AgdaSymbol{:} \AgdaDatatype{Delay} \AgdaBound{A} \AgdaSymbol{→} \AgdaBound{A} \AgdaSymbol{→} \AgdaPrimitiveType{Set} \AgdaKeyword{where}\<%
\\
\>[0]\AgdaIndent{2}{}\<[2]%
\>[2]\AgdaInductiveConstructor{nowT} \<[9]%
\>[9]\AgdaSymbol{:} \AgdaSymbol{∀\{}\AgdaBound{a}\AgdaSymbol{\}} \AgdaSymbol{→} \AgdaSymbol{(}\AgdaInductiveConstructor{now} \AgdaBound{a}\AgdaSymbol{)} \AgdaDatatype{↓} \AgdaBound{a}\<%
\\
\>[0]\AgdaIndent{2}{}\<[2]%
\>[2]\AgdaInductiveConstructor{laterT} \AgdaSymbol{:} \AgdaSymbol{∀\{}\AgdaBound{d} \AgdaBound{a}\AgdaSymbol{\}} \AgdaSymbol{→} \AgdaBound{d} \AgdaDatatype{↓} \AgdaBound{a} \AgdaSymbol{→} \AgdaSymbol{(}\AgdaInductiveConstructor{later} \AgdaSymbol{(}\AgdaCoinductiveConstructor{♯} \AgdaBound{d}\AgdaSymbol{))} \AgdaDatatype{↓} \AgdaBound{a}\<%
\\
\end{code}

And two computation are weakly bisimilar if they terminates with the
same value.

\begin{code}
\\
\>\AgdaKeyword{data} \AgdaDatatype{\_≈\_} \AgdaSymbol{\{}\AgdaBound{A} \AgdaSymbol{:} \AgdaPrimitiveType{Set}\AgdaSymbol{\}} \AgdaSymbol{:} \AgdaDatatype{Delay} \AgdaBound{A} \AgdaSymbol{→} \AgdaDatatype{Delay} \AgdaBound{A} \AgdaSymbol{→} \AgdaPrimitiveType{Set} \AgdaKeyword{where}\<%
\\
\>[2]\AgdaIndent{4}{}\<[4]%
\>[4]\AgdaInductiveConstructor{now} \<[11]%
\>[11]\AgdaSymbol{:} \AgdaSymbol{∀} \AgdaSymbol{\{}\AgdaBound{x} \AgdaBound{y} \AgdaBound{a}\AgdaSymbol{\}} \AgdaSymbol{→} \AgdaBound{x} \AgdaDatatype{↓} \AgdaBound{a} \AgdaSymbol{→} \AgdaBound{y} \AgdaDatatype{↓} \AgdaBound{a} \AgdaSymbol{→} \AgdaBound{x} \AgdaDatatype{≈} \AgdaBound{y}\<%
\\
\>[2]\AgdaIndent{4}{}\<[4]%
\>[4]\AgdaInductiveConstructor{later} \<[11]%
\>[11]\AgdaSymbol{:} \AgdaSymbol{∀} \AgdaSymbol{\{}\AgdaBound{x} \AgdaBound{y}\AgdaSymbol{\}} \AgdaSymbol{(}\AgdaBound{x∼y} \AgdaSymbol{:} \AgdaDatatype{∞} \AgdaSymbol{((}\AgdaFunction{♭} \AgdaBound{x}\AgdaSymbol{)} \AgdaDatatype{≈} \AgdaSymbol{(}\AgdaFunction{♭} \AgdaBound{y}\AgdaSymbol{)))} \AgdaSymbol{→} \AgdaSymbol{(}\AgdaInductiveConstructor{later} \AgdaBound{x}\AgdaSymbol{)} \AgdaDatatype{≈} \AgdaSymbol{(}\AgdaInductiveConstructor{later} \AgdaBound{y}\AgdaSymbol{)}\<%
\\
\end{code}


The quotient derived from the equivalent relation (weak bisimilarity) which represent
the set of all computations can also be a good
example of undefinable quotient.

It is equivalent to another definition inductively on either left side
or right side.


\section{Not all connected type is contractible}

Martin's example is that we can define a type which is proved
different but we cannot find a function from it to \emph{2}.

\todo{Martin Escardo's
  \url{http://www.cs.bham.ac.uk/~mhe/agda/FailureOfTotalSeparatedness.html}
should be considered and discussed here, couterexample of Connected ->
contractible(?)}

conjecture: every quotient definable in pure type theory without
quotient is split

one way to prove it is any connected type is contractible.

the question is still open.


This will implies that reals are not only non-split but also
undefianble in general.


1. why quotients simplifies reasoning. (writened up)
2. Undefinablity
3. omega-groupoids model
