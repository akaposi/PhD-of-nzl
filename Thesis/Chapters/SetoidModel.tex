\chapter{The Setoid Model}
\label{sm}
% always forgets the "the"

\todo{the Setoid Model}

Quotient types are one of the extensional concepts in Type Theory \cite{hof:phd}. There are several existing intensional models for extensional
concepts. The first one we are going to work with is Altenkirch's
setoid model. To introduce an extensional propositional equality in \itt{}, 
Altenkirch \cite{alti:lics99} proposes an intensional setoid model
with a proof-irrelevant universe of propositions \textbf{Prop}.


\begin{equation}[proof-irr]
\begin{aligned}
\Gamma \vdash P : \textbf{Prop} & & \Gamma \vdash p,q : P & \\
\midrule
& \Gamma \vdash p=q : P & \\
\end{aligned}
\end{equation}

It only contains "propositional'' sets which has at most one
inhabitant. Notice that it is not a definition of types, which means
that we cannot conclude a type is of type \textbf{Prop} if we have a
proof that all
inhabitants are definitionally equal.

The propositional universe is closed under "$\Pi$" and "$\Sigma$", namely dependent functions
and dependent products.

\begin{equation}[\Pi-Prop]
\begin{aligned}
\Gamma \vdash A : \textbf{Set} & & \Gamma,x : A \vdash P \in \textbf{Prop} & \\
\midrule
& \Gamma \vdash \Pi\, x : A.P & \\
\end{aligned}
\end{equation}



\begin{equation}[\Sigma-Prop]
\begin{aligned}
\Gamma \vdash P : \textbf{Prop} & & \Gamma,x : P \vdash Q \in \textbf{Prop} & \\
\midrule
& \Gamma \vdash \Sigma\, x : P.Q & \\
\end{aligned}
\end{equation}



 It is called a setoid model since types are interpreted as setoids.
The solution to introduce the extensional equality is an object type theory defined inside the setoid model which serves as the metatheory. He also proved that the extended type theory generated from the metatheory is decidable and adequate, functional extensionality is
inhabited and it permits large elimination (defining a dependent type by recursion). Within this type theory,
introduction of quotient types is straightforward.

This model is different to a setoid model as an E-category, for instance
the one introduced by Hofmann \cite{hofmann1995interpretation} . An E-category is a category equipped with
an equivalence relation for homsets. To distinguish them, we call this
category \textbf{E-setoids}.  All morphisms of \textbf{E-setoids}
gives rise to types and they are cartesian closed, namely it is a a locally
cartesian closed category (LCCC). Not all morphisms in our category of setoids give
rise to types and it is not an LCCC. Every LCCC can serve as a model for categories with
families but not every category with families has to be an
LCCC. 

\todo{write why this model is not lccc explicitly. refer to Nicolais's
result}



\paragraph{The category of setoids is not a LCCC}

The pullback functor.

\begin{displaymath}
    \xymatrix{X' \ar[r]^{p} \ar[d]_{f^{*}(a)} & Y' \ar[d]^a \\
      X \ar[r]^f& Y }
\end{displaymath}


Observe that $X \rightarrow 1 \cong X$, therefore the pullback of y which is
$X/1 \rightarrow X \times Y / Y$ can be seen as a pullback of X of type $X \rightarrow
X/Y$.

The left adjoint to the pullback functor $f*$ is just the post
composition of $f$ written as $f \circ\_$ or $\Sigma_f$.

\begin{displaymath}
    \xymatrix{X' \ar@{=}[r] \ar[d]_{a} & X' \ar[d]^{\Sigma_f a} \\
      X \ar[r]^f& Y }
\end{displaymath}


However this setoid model is still a model for Type
Theory just like the groupoid model which is a generalisation of it.
To develop this model of type theory in Agda, we have implemented the
categories with families of setoids. 
We build a category with families of setoids to accommodate the types theory described in
\cite{alti:lics99}  so that it is possible to define quotient types
following Martin Hofmann's Paper \cite{hof:95:sm}.  Only necessary
part for the Setoid model will be present here.


\section{An implementation of categories with Families in Agda}

Following the work in \cite{alti:99}, we first define a
proof-irrelevant universe of propositions. We name it as \textbf{hProp}
since \textbf{Prop} is a  reserved word which can't be used and
\textbf{hProp} is a notion from Homotopy Type Theory which we will introduce later.

\section{hProp}

\AgdaHide{

\begin{code}\>\<%
\\
%
\\
\>\AgdaKeyword{open} \AgdaKeyword{import} \AgdaModule{Level}\<%
\\
\>\AgdaKeyword{open} \AgdaKeyword{import} \AgdaModule{Relation.Binary.PropositionalEquality}\<%
\\
%
\\
\>\AgdaKeyword{module} \AgdaModule{hProp} \AgdaSymbol{(}\AgdaBound{ext} \AgdaSymbol{:} \AgdaFunction{Extensionality} \AgdaPrimitive{zero} \AgdaPrimitive{zero}\AgdaSymbol{)} \AgdaKeyword{where}\<%
\\
%
\\
\>\AgdaKeyword{open} \AgdaKeyword{import} \AgdaModule{Relation.Nullary}\<%
\\
\>\AgdaKeyword{open} \AgdaKeyword{import} \AgdaModule{Data.Unit}\<%
\\
\>\AgdaKeyword{open} \AgdaKeyword{import} \AgdaModule{Data.Empty}\<%
\\
\>\AgdaKeyword{open} \AgdaKeyword{import} \AgdaModule{Data.Nat}\<%
\\
\>\AgdaKeyword{open} \AgdaKeyword{import} \AgdaModule{Data.Product}\<%
\\
%
\\
\>\AgdaKeyword{infixr} \AgdaNumber{2} \_⇒\_\<%
\\
%
\\
\>\AgdaKeyword{infixr} \AgdaNumber{3} \_∧\_\<%
\\
%
\\
%
\\
\>\<\end{code}
}

A proof-irrelvant universe only contains sets with at most one inhabitant. 

\begin{code}\>\<%
\\
%
\\
\>\AgdaKeyword{record} \AgdaRecord{hProp} \AgdaSymbol{:} \AgdaPrimitiveType{Set₁} \AgdaKeyword{where}\<%
\\
\>[0]\AgdaIndent{2}{}\<[2]%
\>[2]\AgdaKeyword{constructor} \AgdaInductiveConstructor{hp}\<%
\\
\>[0]\AgdaIndent{2}{}\<[2]%
\>[2]\AgdaKeyword{field}\<%
\\
\>[2]\AgdaIndent{4}{}\<[4]%
\>[4]\AgdaField{prf} \AgdaSymbol{:} \AgdaPrimitiveType{Set}\<%
\\
\>[2]\AgdaIndent{4}{}\<[4]%
\>[4]\AgdaField{Uni} \AgdaSymbol{:} \AgdaSymbol{\{}\AgdaBound{p} \AgdaBound{q} \AgdaSymbol{:} \AgdaBound{prf}\AgdaSymbol{\}} \AgdaSymbol{→} \AgdaBound{p} \AgdaDatatype{≡} \AgdaBound{q}\<%
\\
%
\\
\>\AgdaKeyword{open} \AgdaModule{hProp} \AgdaKeyword{public} \AgdaKeyword{renaming} \AgdaSymbol{(}prf \AgdaSymbol{to} <\_>\AgdaSymbol{)}\<%
\\
%
\\
\>\<\end{code}

We can extract the proof of any propostion $A : hProp$ by using $<>$ and there is always a proof that all inhabitants of it are the same, in other words, if there is any proof of it, the proof is unique. This is not exactly the same as the $Prop$ universe in Altenkirch's approach which is judgemental. It is just a judgement whether a set behaves like a $Proposition$. The $hProp$ we define above is propositional since we can extract the proof of uniqueness.

We would like to have some basic propositions $\top$ and $\bot$. To distinguish them with the ones for non-proof irrelevant propositions which are already available in Agda library, we add a prime to all similar symbols.

\begin{code}\>\<%
\\
%
\\
\>\AgdaFunction{⊤'} \AgdaSymbol{:} \AgdaRecord{hProp}\<%
\\
\>\AgdaFunction{⊤'} \AgdaSymbol{=} \AgdaInductiveConstructor{hp} \AgdaRecord{⊤} \AgdaInductiveConstructor{refl}\<%
\\
%
\\
\>\AgdaFunction{⊥'} \AgdaSymbol{:} \AgdaRecord{hProp}\<%
\\
\>\AgdaFunction{⊥'} \AgdaSymbol{=} \AgdaInductiveConstructor{hp} \AgdaDatatype{⊥} \AgdaSymbol{(λ} \AgdaSymbol{\{}\AgdaBound{p}\AgdaSymbol{\}} \AgdaSymbol{→} \AgdaFunction{⊥-elim} \AgdaBound{p}\AgdaSymbol{)}\<%
\\
%
\\
\>\<\end{code}

We also want the universal and existential quantifier for $hProp$, namely it is closed under $\Pi$-types and $\Sigma$-types.
The universal quantifier of $hProp$ can be axiomitised but we decide to explicitly state that we 
require the functional extensionality to use this module. The reason is that functional extensionality is actually equivalent to the closure under $\Pi$-types.

\begin{code}\>\<%
\\
%
\\
\>\AgdaFunction{∀'} \AgdaSymbol{:} \AgdaSymbol{(}\AgdaBound{A} \AgdaSymbol{:} \AgdaPrimitiveType{Set}\AgdaSymbol{)(}\AgdaBound{P} \AgdaSymbol{:} \AgdaBound{A} \AgdaSymbol{→} \AgdaRecord{hProp}\AgdaSymbol{)} \AgdaSymbol{→} \AgdaRecord{hProp}\<%
\\
\>\AgdaFunction{∀'} \AgdaBound{A} \AgdaBound{P} \AgdaSymbol{=} \AgdaInductiveConstructor{hp} \AgdaSymbol{((}\AgdaBound{x} \AgdaSymbol{:} \AgdaBound{A}\AgdaSymbol{)} \AgdaSymbol{→} \AgdaFunction{<} \AgdaBound{P} \AgdaBound{x} \AgdaFunction{>}\AgdaSymbol{)} \AgdaSymbol{(}\AgdaBound{ext} \AgdaSymbol{(λ} \AgdaBound{x} \AgdaSymbol{→} \AgdaFunction{Uni} \AgdaSymbol{(}\AgdaBound{P} \AgdaBound{x}\AgdaSymbol{)))}\<%
\\
%
\\
\>\<\end{code}


\AgdaHide{
\begin{code}\>\<%
\\
%
\\
\>\AgdaFunction{sig-eq} \AgdaSymbol{:} \AgdaSymbol{\{}\AgdaBound{A} \AgdaSymbol{:} \AgdaPrimitiveType{Set}\AgdaSymbol{\}\{}\AgdaBound{B} \AgdaSymbol{:} \AgdaBound{A} \AgdaSymbol{→} \AgdaPrimitiveType{Set}\AgdaSymbol{\}\{}\AgdaBound{a} \AgdaBound{b} \AgdaSymbol{:} \AgdaBound{A}\AgdaSymbol{\}} \AgdaSymbol{→} \<[43]%
\>[43]\<%
\\
\>[4]\AgdaIndent{9}{}\<[9]%
\>[9]\AgdaSymbol{(}\AgdaBound{p} \AgdaSymbol{:} \AgdaBound{a} \AgdaDatatype{≡} \AgdaBound{b}\AgdaSymbol{)} \AgdaSymbol{→} \<[23]%
\>[23]\<%
\\
\>[4]\AgdaIndent{9}{}\<[9]%
\>[9]\AgdaSymbol{\{}\AgdaBound{c} \AgdaSymbol{:} \AgdaBound{B} \AgdaBound{a}\AgdaSymbol{\}\{}\AgdaBound{d} \AgdaSymbol{:} \AgdaBound{B} \AgdaBound{b}\AgdaSymbol{\}} \AgdaSymbol{→} \<[30]%
\>[30]\<%
\\
\>[4]\AgdaIndent{9}{}\<[9]%
\>[9]\AgdaSymbol{(}\AgdaFunction{subst} \AgdaSymbol{(λ} \AgdaBound{x} \AgdaSymbol{→} \AgdaBound{B} \AgdaBound{x}\AgdaSymbol{)} \AgdaBound{p} \AgdaBound{c} \AgdaDatatype{≡} \AgdaBound{d}\AgdaSymbol{)} \<[37]%
\>[37]\<%
\\
\>[4]\AgdaIndent{9}{}\<[9]%
\>[9]\AgdaSymbol{→} \AgdaDatatype{\_≡\_} \AgdaSymbol{\{\_\}} \AgdaSymbol{\{}\AgdaRecord{Σ} \AgdaBound{A} \AgdaBound{B}\AgdaSymbol{\}} \AgdaSymbol{(}\AgdaBound{a} \AgdaInductiveConstructor{,} \AgdaBound{c}\AgdaSymbol{)} \AgdaSymbol{(}\AgdaBound{b} \AgdaInductiveConstructor{,} \AgdaBound{d}\AgdaSymbol{)}\<%
\\
\>\AgdaFunction{sig-eq} \AgdaInductiveConstructor{refl} \AgdaInductiveConstructor{refl} \AgdaSymbol{=} \AgdaInductiveConstructor{refl}\<%
\\
%
\\
\>\<\end{code}
}

\begin{code}\>\<%
\\
%
\\
%
\\
\>\AgdaFunction{Σ'} \AgdaSymbol{:} \AgdaSymbol{(}\AgdaBound{P} \AgdaSymbol{:} \AgdaRecord{hProp}\AgdaSymbol{)(}\AgdaBound{Q} \AgdaSymbol{:} \AgdaFunction{<} \AgdaBound{P} \AgdaFunction{>} \AgdaSymbol{→} \AgdaRecord{hProp}\AgdaSymbol{)} \AgdaSymbol{→} \AgdaRecord{hProp}\<%
\\
\>\AgdaFunction{Σ'} \AgdaBound{P} \AgdaBound{Q} \AgdaSymbol{=} \AgdaInductiveConstructor{hp} \AgdaSymbol{(}\AgdaRecord{Σ} \AgdaFunction{<} \AgdaBound{P} \AgdaFunction{>} \AgdaSymbol{(λ} \AgdaBound{x} \AgdaSymbol{→} \AgdaFunction{<} \AgdaBound{Q} \AgdaBound{x} \AgdaFunction{>}\AgdaSymbol{))} \<[38]%
\>[38]\<%
\\
\>[9]\AgdaIndent{12}{}\<[12]%
\>[12]\AgdaSymbol{(λ} \AgdaSymbol{\{}\AgdaBound{p}\AgdaSymbol{\}} \AgdaSymbol{\{}\AgdaBound{q}\AgdaSymbol{\}} \AgdaSymbol{→} \<[25]%
\>[25]\<%
\\
\>[9]\AgdaIndent{12}{}\<[12]%
\>[12]\AgdaFunction{sig-eq} \AgdaSymbol{(}\AgdaFunction{Uni} \AgdaBound{P}\AgdaSymbol{)} \AgdaSymbol{(}\AgdaFunction{Uni} \AgdaSymbol{(}\AgdaBound{Q} \AgdaSymbol{(}\AgdaFunction{proj₁} \AgdaBound{q}\AgdaSymbol{))))}\<%
\\
%
\\
\>\<\end{code}

Implication and conjuction which are independent ones of them follow simply.

\begin{code}\>\<%
\\
%
\\
\>\AgdaFunction{\_⇒\_} \AgdaSymbol{:} \AgdaSymbol{(}\AgdaBound{P} \AgdaBound{Q} \AgdaSymbol{:} \AgdaRecord{hProp}\AgdaSymbol{)} \AgdaSymbol{→} \AgdaRecord{hProp}\<%
\\
\>\AgdaBound{P} \AgdaFunction{⇒} \AgdaBound{Q} \AgdaSymbol{=} \<[9]%
\>[9]\AgdaFunction{∀'} \AgdaFunction{<} \AgdaBound{P} \AgdaFunction{>} \AgdaSymbol{(λ} \AgdaBound{\_} \AgdaSymbol{→} \AgdaBound{Q}\AgdaSymbol{)}\<%
\\
%
\\
\>\AgdaFunction{\_∧\_} \AgdaSymbol{:} \AgdaSymbol{(}\AgdaBound{P} \AgdaBound{Q} \AgdaSymbol{:} \AgdaRecord{hProp}\AgdaSymbol{)} \AgdaSymbol{→} \AgdaRecord{hProp}\<%
\\
\>\AgdaBound{P} \AgdaFunction{∧} \AgdaBound{Q} \AgdaSymbol{=} \AgdaFunction{Σ'} \AgdaBound{P} \AgdaSymbol{(λ} \AgdaBound{\_} \AgdaSymbol{→} \AgdaBound{Q}\AgdaSymbol{)}\<%
\\
%
\\
\>\<\end{code}

\AgdaHide{
\begin{code}\>\<%
\\
\>\AgdaKeyword{syntax} ∀' A \AgdaSymbol{(λ} x \AgdaSymbol{→} B\AgdaSymbol{)} \AgdaSymbol{=} ∀'[ x ∶ A ] B


\AgdaKeyword{syntax} Σ' A \AgdaSymbol{(λ} x \AgdaSymbol{→} B\AgdaSymbol{)} \AgdaSymbol{=} Σ'[ x ∶ A ] B

\<\end{code}
}

As long as we have implication and conjuction, more operators on proposition can be defined, for instances negation and logical equivalence.

\begin{code}\>\<%
\\
%
\\
\>\AgdaFunction{¬} \AgdaSymbol{:} \AgdaRecord{hProp} \AgdaSymbol{→} \AgdaRecord{hProp}\<%
\\
\>\AgdaFunction{¬} \AgdaBound{P} \AgdaSymbol{=} \AgdaBound{P} \AgdaFunction{⇒} \AgdaFunction{⊥'} \<[13]%
\>[13]\<%
\\
%
\\
\>\AgdaFunction{\_↔\_} \<[6]%
\>[6]\AgdaSymbol{:} \AgdaSymbol{(}\AgdaBound{P} \AgdaBound{Q} \AgdaSymbol{:} \AgdaRecord{hProp}\AgdaSymbol{)} \AgdaSymbol{→} \AgdaRecord{hProp}\<%
\\
\>\AgdaBound{P} \AgdaFunction{↔} \AgdaBound{Q} \AgdaSymbol{=} \AgdaSymbol{(}\AgdaBound{P} \AgdaFunction{⇒} \AgdaBound{Q}\AgdaSymbol{)} \AgdaFunction{∧} \AgdaSymbol{(}\AgdaBound{Q} \AgdaFunction{⇒} \AgdaBound{P}\AgdaSymbol{)}\<%
\\
%
\\
\>\<\end{code}

\section{Category}

To define category of setoids we should define category first.


\AgdaHide{
\begin{code}\>\<%
\\
%
\\
\>\AgdaSymbol{\{-\#} \AgdaKeyword{OPTIONS} --type-in-type \AgdaSymbol{\#-\}}\<%
\\
%
\\
\>\AgdaKeyword{module} \AgdaModule{Category} \AgdaKeyword{where}\<%
\\
%
\\
\>\AgdaKeyword{open} \AgdaKeyword{import} \AgdaModule{Data.Product}\<%
\\
\>\AgdaKeyword{open} \AgdaKeyword{import} \AgdaModule{Relation.Binary.PropositionalEquality}\<%
\\
%
\\
\>\AgdaKeyword{open} \AgdaKeyword{import} \AgdaModule{Level}\<%
\\
%
\\
\>\<\end{code}
}

\AgdaHide{
\begin{code}\>\<%
\\
%
\\
\>\AgdaKeyword{record} \AgdaRecord{IsCategory}\<%
\\
\>[0]\AgdaIndent{2}{}\<[2]%
\>[2]\AgdaSymbol{(}\AgdaBound{obj} \<[12]%
\>[12]\AgdaSymbol{:} \AgdaPrimitiveType{Set}\AgdaSymbol{)}\<%
\\
%
\\
\>[0]\AgdaIndent{2}{}\<[2]%
\>[2]\AgdaSymbol{(}\AgdaBound{hom} \<[12]%
\>[12]\AgdaSymbol{:} \AgdaBound{obj} \AgdaSymbol{→} \AgdaBound{obj} \AgdaSymbol{→} \AgdaPrimitiveType{Set}\AgdaSymbol{)}\<%
\\
%
\\
\>[0]\AgdaIndent{2}{}\<[2]%
\>[2]\AgdaSymbol{(}\AgdaBound{id} \<[12]%
\>[12]\AgdaSymbol{:} \AgdaSymbol{∀} \AgdaBound{α} \AgdaSymbol{→} \AgdaBound{hom} \AgdaBound{α} \AgdaBound{α}\AgdaSymbol{)}\<%
\\
%
\\
\>[0]\AgdaIndent{2}{}\<[2]%
\>[2]\AgdaSymbol{(}\AgdaBound{[\_⇒\_]\_∘\_} \AgdaSymbol{:} \AgdaSymbol{∀} \AgdaBound{α} \AgdaSymbol{\{}\AgdaBound{β}\AgdaSymbol{\}} \AgdaBound{γ}\<%
\\
\>[2]\AgdaIndent{12}{}\<[12]%
\>[12]\AgdaSymbol{→} \AgdaBound{hom} \AgdaBound{β} \AgdaBound{γ}\<%
\\
\>[2]\AgdaIndent{12}{}\<[12]%
\>[12]\AgdaSymbol{→} \AgdaBound{hom} \AgdaBound{α} \AgdaBound{β}\<%
\\
\>[2]\AgdaIndent{12}{}\<[12]%
\>[12]\AgdaSymbol{→} \AgdaBound{hom} \AgdaBound{α} \AgdaBound{γ}\AgdaSymbol{)}\<%
\\
\>[0]\AgdaIndent{2}{}\<[2]%
\>[2]\AgdaSymbol{:} \AgdaPrimitiveType{Set}\<%
\\
\>[0]\AgdaIndent{2}{}\<[2]%
\>[2]\AgdaKeyword{where}\<%
\\
\>[2]\AgdaIndent{4}{}\<[4]%
\>[4]\AgdaKeyword{constructor} \AgdaInductiveConstructor{IsCatC}\<%
\\
\>[2]\AgdaIndent{4}{}\<[4]%
\>[4]\AgdaKeyword{field}\<%
\\
\>[4]\AgdaIndent{6}{}\<[6]%
\>[6]\AgdaField{id₁} \<[11]%
\>[11]\AgdaSymbol{:} \AgdaSymbol{∀} \AgdaBound{α} \AgdaBound{β} \AgdaSymbol{(}\AgdaBound{f} \AgdaSymbol{:} \AgdaBound{hom} \AgdaBound{α} \AgdaBound{β}\AgdaSymbol{)}\<%
\\
\>[6]\AgdaIndent{11}{}\<[11]%
\>[11]\AgdaSymbol{→} \AgdaBound{[} \AgdaBound{α} \AgdaBound{⇒} \AgdaBound{β} \AgdaBound{]} \AgdaBound{f} \AgdaBound{∘} \AgdaSymbol{(}\AgdaBound{id} \AgdaBound{α}\AgdaSymbol{)} \AgdaDatatype{≡} \AgdaBound{f}\<%
\\
%
\\
\>[0]\AgdaIndent{6}{}\<[6]%
\>[6]\AgdaField{id₂} \<[11]%
\>[11]\AgdaSymbol{:} \AgdaSymbol{∀} \AgdaBound{α} \AgdaBound{β} \AgdaSymbol{(}\AgdaBound{f} \AgdaSymbol{:} \AgdaBound{hom} \AgdaBound{α} \AgdaBound{β}\AgdaSymbol{)}\<%
\\
\>[0]\AgdaIndent{11}{}\<[11]%
\>[11]\AgdaSymbol{→} \AgdaBound{[} \AgdaBound{α} \AgdaBound{⇒} \AgdaBound{β} \AgdaBound{]} \AgdaSymbol{(}\AgdaBound{id} \AgdaBound{β}\AgdaSymbol{)} \AgdaBound{∘} \AgdaBound{f} \AgdaDatatype{≡} \AgdaBound{f}\<%
\\
%
\\
\>[0]\AgdaIndent{6}{}\<[6]%
\>[6]\AgdaField{comp} \AgdaSymbol{:} \AgdaSymbol{∀} \AgdaBound{α} \AgdaSymbol{\{}\AgdaBound{β} \AgdaBound{γ}\AgdaSymbol{\}} \AgdaBound{δ} \AgdaSymbol{(}\AgdaBound{f} \AgdaSymbol{:} \AgdaBound{hom} \AgdaBound{α} \AgdaBound{β}\AgdaSymbol{)} \AgdaSymbol{(}\AgdaBound{g} \AgdaSymbol{:} \AgdaBound{hom} \AgdaBound{β} \AgdaBound{γ}\AgdaSymbol{)} \AgdaSymbol{(}\AgdaBound{h} \AgdaSymbol{:} \AgdaBound{hom} \AgdaBound{γ} \AgdaBound{δ}\AgdaSymbol{)}\<%
\\
\>[0]\AgdaIndent{11}{}\<[11]%
\>[11]\AgdaSymbol{→} \AgdaBound{[} \AgdaBound{α} \AgdaBound{⇒} \AgdaBound{δ} \AgdaBound{]} \AgdaBound{[} \AgdaBound{β} \AgdaBound{⇒} \AgdaBound{δ} \AgdaBound{]} \AgdaBound{h} \AgdaBound{∘} \AgdaBound{g} \AgdaBound{∘} \AgdaBound{f} \AgdaDatatype{≡} \AgdaBound{[} \AgdaBound{α} \AgdaBound{⇒} \AgdaBound{δ} \AgdaBound{]} \AgdaBound{h} \AgdaBound{∘} \AgdaSymbol{(}\AgdaBound{[} \AgdaBound{α} \AgdaBound{⇒} \AgdaBound{γ} \AgdaBound{]} \AgdaBound{g} \AgdaBound{∘} \AgdaBound{f}\AgdaSymbol{)}\<%
\\
%
\\
%
\\
\>\<\end{code}
}


\begin{code}\>\<%
\\
\>\AgdaKeyword{record} \AgdaRecord{Category} \AgdaSymbol{:} \AgdaPrimitiveType{Set} \AgdaKeyword{where}\<%
\\
\>[0]\AgdaIndent{2}{}\<[2]%
\>[2]\AgdaKeyword{constructor} \AgdaInductiveConstructor{CatC}\<%
\\
\>[0]\AgdaIndent{2}{}\<[2]%
\>[2]\AgdaKeyword{field}\<%
\\
\>[2]\AgdaIndent{4}{}\<[4]%
\>[4]\AgdaField{obj} \<[15]%
\>[15]\AgdaSymbol{:} \AgdaPrimitiveType{Set}\<%
\\
%
\\
\>[2]\AgdaIndent{4}{}\<[4]%
\>[4]\AgdaField{hom} \<[15]%
\>[15]\AgdaSymbol{:} \AgdaBound{obj} \AgdaSymbol{→} \AgdaBound{obj} \AgdaSymbol{→} \AgdaPrimitiveType{Set}\<%
\\
%
\\
\>[2]\AgdaIndent{4}{}\<[4]%
\>[4]\AgdaField{id} \<[15]%
\>[15]\AgdaSymbol{:} \AgdaSymbol{∀} \AgdaBound{α}\<%
\\
\>[4]\AgdaIndent{15}{}\<[15]%
\>[15]\AgdaSymbol{→} \AgdaBound{hom} \AgdaBound{α} \AgdaBound{α}\<%
\\
%
\\
\>[0]\AgdaIndent{4}{}\<[4]%
\>[4]\AgdaField{[\_⇒\_]\_∘\_} \<[15]%
\>[15]\AgdaSymbol{:} \AgdaSymbol{∀} \AgdaBound{α} \AgdaSymbol{\{}\AgdaBound{β}\AgdaSymbol{\}} \AgdaBound{γ}\<%
\\
\>[0]\AgdaIndent{15}{}\<[15]%
\>[15]\AgdaSymbol{→} \AgdaBound{hom} \AgdaBound{β} \AgdaBound{γ}\<%
\\
\>[0]\AgdaIndent{15}{}\<[15]%
\>[15]\AgdaSymbol{→} \AgdaBound{hom} \AgdaBound{α} \AgdaBound{β}\<%
\\
\>[0]\AgdaIndent{15}{}\<[15]%
\>[15]\AgdaSymbol{→} \AgdaBound{hom} \AgdaBound{α} \AgdaBound{γ}\<%
\\
%
\\
\>[0]\AgdaIndent{4}{}\<[4]%
\>[4]\AgdaField{isCategory} \AgdaSymbol{:} \AgdaRecord{IsCategory} \AgdaBound{obj} \AgdaBound{hom} \AgdaBound{id} \AgdaBound{[\_⇒\_]\_∘\_}\<%
\\
%
\\
%
\\
\>\<\end{code}

\AgdaHide{
\begin{code}\>\<%
\\
%
\\
\>[0]\AgdaIndent{2}{}\<[2]%
\>[2]\AgdaKeyword{open} \AgdaModule{IsCategory} \AgdaKeyword{public}\<%
\\
%
\\
\>\<\end{code}
}

$isCategory$ contains all the laws for this structure to be a category, for instance the
associativity laws for composition.

\section{Category of setoids}


\AgdaHide{
\begin{code}\>\<%
\\
%
\\
\>\AgdaSymbol{\{-\#} \AgdaKeyword{OPTIONS} --type-in-type \AgdaSymbol{\#-\}}\<%
\\
%
\\
\>\AgdaKeyword{open} \AgdaKeyword{import} \AgdaModule{Level}\<%
\\
\>\AgdaKeyword{open} \AgdaKeyword{import} \AgdaModule{Relation.Binary.PropositionalEquality} \AgdaSymbol{as} \AgdaModule{PE} \AgdaKeyword{hiding} \AgdaSymbol{(}refl\AgdaSymbol{;} sym \AgdaSymbol{;} trans\AgdaSymbol{;} isEquivalence\AgdaSymbol{)}\<%
\\
%
\\
\>\AgdaKeyword{module} \AgdaModule{CategoryOfSetoid} \<[25]%
\>[25]\AgdaSymbol{(}\AgdaBound{ext} \AgdaSymbol{:} \AgdaFunction{Extensionality} \AgdaPrimitive{zero} \AgdaPrimitive{zero}\AgdaSymbol{)} \AgdaKeyword{where}\<%
\\
%
\\
\>\AgdaKeyword{open} \AgdaKeyword{import} \AgdaModule{Cats.Category}\<%
\\
\>\AgdaKeyword{open} \AgdaKeyword{import} \AgdaModule{Function}\<%
\\
\>\AgdaKeyword{open} \AgdaKeyword{import} \AgdaModule{Relation.Binary.Core} \AgdaKeyword{using} \AgdaSymbol{(}\_⇒\_\AgdaSymbol{)}\<%
\\
\>\AgdaKeyword{open} \AgdaKeyword{import} \AgdaModule{Data.Empty}\<%
\\
\>\AgdaKeyword{open} \AgdaKeyword{import} \AgdaModule{HProp} \AgdaBound{ext} \AgdaKeyword{public}\<%
\\
\>\AgdaKeyword{open} \AgdaKeyword{import} \AgdaModule{Data.Unit}\<%
\\
%
\\
\>\<\end{code}
}

\section{Category of setoids}

\begin{code}\>\<%
\\
%
\\
\>\AgdaKeyword{record} \AgdaRecord{HSetoid} \AgdaSymbol{:} \AgdaPrimitiveType{Set₁} \AgdaKeyword{where}\<%
\\
\>[0]\AgdaIndent{2}{}\<[2]%
\>[2]\AgdaKeyword{constructor} \AgdaInductiveConstructor{\_,\_,\_,\_,\_}\<%
\\
\>[0]\AgdaIndent{2}{}\<[2]%
\>[2]\AgdaKeyword{infix} \AgdaNumber{4} \_≈h\_ \_≈\_\<%
\\
\>[0]\AgdaIndent{2}{}\<[2]%
\>[2]\AgdaKeyword{field}\<%
\\
\>[2]\AgdaIndent{4}{}\<[4]%
\>[4]\AgdaField{Carrier} \AgdaSymbol{:} \AgdaPrimitiveType{Set}\<%
\\
\>[2]\AgdaIndent{4}{}\<[4]%
\>[4]\AgdaField{\_≈h\_} \<[13]%
\>[13]\AgdaSymbol{:} \AgdaBound{Carrier} \AgdaSymbol{→} \AgdaBound{Carrier} \AgdaSymbol{→} \AgdaRecord{HProp}\<%
\\
\>[2]\AgdaIndent{4}{}\<[4]%
\>[4]\AgdaField{refl} \<[12]%
\>[12]\AgdaSymbol{:} \AgdaSymbol{\{}\AgdaBound{x} \AgdaSymbol{:} \AgdaBound{Carrier}\AgdaSymbol{\}} \AgdaSymbol{→} \AgdaFunction{<} \AgdaBound{x} \AgdaBound{≈h} \AgdaBound{x} \AgdaFunction{>}\<%
\\
\>[2]\AgdaIndent{4}{}\<[4]%
\>[4]\AgdaField{sym} \<[12]%
\>[12]\AgdaSymbol{:} \AgdaSymbol{\{}\AgdaBound{x} \AgdaBound{y} \AgdaSymbol{:} \AgdaBound{Carrier}\AgdaSymbol{\}} \AgdaSymbol{→} \AgdaFunction{<} \AgdaBound{x} \AgdaBound{≈h} \AgdaBound{y} \AgdaFunction{>} \AgdaSymbol{→} \AgdaFunction{<} \AgdaBound{y} \AgdaBound{≈h} \AgdaBound{x} \AgdaFunction{>}\<%
\\
\>[2]\AgdaIndent{4}{}\<[4]%
\>[4]\AgdaField{trans} \<[12]%
\>[12]\AgdaSymbol{:} \AgdaSymbol{\{}\AgdaBound{x} \AgdaBound{y} \AgdaBound{z} \AgdaSymbol{:} \AgdaBound{Carrier}\AgdaSymbol{\}} \AgdaSymbol{→} \AgdaFunction{<} \AgdaBound{x} \AgdaBound{≈h} \AgdaBound{y} \AgdaFunction{>} \AgdaSymbol{→} \AgdaFunction{<} \AgdaBound{y} \AgdaBound{≈h} \AgdaBound{z} \AgdaFunction{>} \AgdaSymbol{→} \AgdaFunction{<} \AgdaBound{x} \AgdaBound{≈h} \AgdaBound{z} \AgdaFunction{>}\<%
\\
\>[0]\AgdaIndent{2}{}\<[2]%
\>[2]\<%
\\
\>[0]\AgdaIndent{2}{}\<[2]%
\>[2]\AgdaFunction{\_≈\_} \AgdaSymbol{:} \AgdaFunction{Carrier} \AgdaSymbol{→} \AgdaFunction{Carrier} \AgdaSymbol{→} \AgdaPrimitiveType{Set}\<%
\\
\>[0]\AgdaIndent{2}{}\<[2]%
\>[2]\AgdaBound{a} \AgdaFunction{≈} \AgdaBound{b} \AgdaSymbol{=} \AgdaFunction{<} \AgdaBound{a} \AgdaFunction{≈h} \AgdaBound{b} \AgdaFunction{>}\<%
\\
%
\\
\>[0]\AgdaIndent{2}{}\<[2]%
\>[2]\AgdaFunction{PI} \AgdaSymbol{:} \AgdaSymbol{\{}\AgdaBound{x} \AgdaBound{y} \AgdaSymbol{:} \AgdaFunction{Carrier}\AgdaSymbol{\}\{}\AgdaBound{B} \AgdaSymbol{:} \AgdaPrimitiveType{Set}\AgdaSymbol{\}(}\AgdaBound{A} \AgdaSymbol{:} \AgdaBound{x} \AgdaFunction{≈} \AgdaBound{y} \AgdaSymbol{→} \AgdaBound{B}\AgdaSymbol{)\{}\AgdaBound{p} \AgdaBound{q} \AgdaSymbol{:} \AgdaBound{x} \AgdaFunction{≈} \AgdaBound{y}\AgdaSymbol{\}} \AgdaSymbol{→} \AgdaBound{A} \AgdaBound{p} \AgdaDatatype{≡} \AgdaBound{A} \AgdaBound{q}\<%
\\
\>[0]\AgdaIndent{2}{}\<[2]%
\>[2]\AgdaFunction{PI} \AgdaSymbol{\{}\AgdaBound{x}\AgdaSymbol{\}} \AgdaSymbol{\{}\AgdaBound{y}\AgdaSymbol{\}} \AgdaBound{A} \AgdaSymbol{\{}\AgdaBound{p}\AgdaSymbol{\}} \AgdaSymbol{\{}\AgdaBound{q}\AgdaSymbol{\}} \AgdaKeyword{with} \AgdaFunction{Uni} \AgdaSymbol{(}\AgdaBound{x} \AgdaFunction{≈h} \AgdaBound{y}\AgdaSymbol{)} \AgdaSymbol{\{}\AgdaBound{p}\AgdaSymbol{\}} \AgdaSymbol{\{}\AgdaBound{q}\AgdaSymbol{\}}\<%
\\
\>[0]\AgdaIndent{2}{}\<[2]%
\>[2]\AgdaFunction{PI} \AgdaBound{A} \AgdaSymbol{|} \AgdaInductiveConstructor{PE.refl} \AgdaSymbol{=} \AgdaInductiveConstructor{PE.refl}\<%
\\
%
\\
\>[0]\AgdaIndent{2}{}\<[2]%
\>[2]\AgdaFunction{reflexive} \AgdaSymbol{:} \AgdaDatatype{\_≡\_} \AgdaFunction{⇒} \AgdaFunction{\_≈\_}\<%
\\
\>[0]\AgdaIndent{2}{}\<[2]%
\>[2]\AgdaFunction{reflexive} \AgdaInductiveConstructor{PE.refl} \AgdaSymbol{=} \AgdaFunction{refl}\<%
\\
%
\\
\>\AgdaKeyword{open} \AgdaModule{HSetoid} \AgdaKeyword{public} \AgdaKeyword{renaming} \AgdaSymbol{(}refl \AgdaSymbol{to} [\_]refl\AgdaSymbol{;} sym \AgdaSymbol{to} [\_]sym\AgdaSymbol{;} 
  \_≈\_ \AgdaSymbol{to} [\_]\_≈\_ \AgdaSymbol{;} \_≈h\_ \AgdaSymbol{to} [\_]\_≈h\_ \AgdaSymbol{;} Carrier \AgdaSymbol{to} ∣\_∣ \AgdaSymbol{;} trans \AgdaSymbol{to} [\_]trans\AgdaSymbol{)}\<%
\\
%
\\
\>\AgdaFunction{rel} \AgdaSymbol{:} \AgdaSymbol{(}\AgdaBound{A} \AgdaSymbol{:} \AgdaRecord{HSetoid}\AgdaSymbol{)} \AgdaSymbol{→} \AgdaFunction{∣} \AgdaBound{A} \AgdaFunction{∣} \AgdaSymbol{→} \AgdaFunction{∣} \AgdaBound{A} \AgdaFunction{∣} \AgdaSymbol{→} \AgdaRecord{HProp}\<%
\\
\>\AgdaFunction{rel} \AgdaBound{A} \AgdaBound{a} \AgdaBound{b} \AgdaSymbol{=} \AgdaFunction{[} \AgdaBound{A} \AgdaFunction{]} \AgdaBound{a} \AgdaFunction{≈h} \AgdaBound{b}\<%
\\
%
\\
\>\AgdaFunction{[\_]uip} \AgdaSymbol{:} \AgdaSymbol{∀(}\AgdaBound{Γ} \AgdaSymbol{:} \AgdaRecord{HSetoid}\AgdaSymbol{)\{}\AgdaBound{a} \AgdaBound{b} \AgdaSymbol{:} \AgdaFunction{∣} \AgdaBound{Γ} \AgdaFunction{∣}\AgdaSymbol{\}\{}\AgdaBound{p} \AgdaBound{q} \AgdaSymbol{:} \AgdaFunction{[} \AgdaBound{Γ} \AgdaFunction{]} \AgdaBound{a} \AgdaFunction{≈} \AgdaBound{b}\AgdaSymbol{\}} \AgdaSymbol{→} \AgdaBound{p} \AgdaDatatype{≡} \AgdaBound{q}\<%
\\
\>\AgdaFunction{[} \AgdaBound{Γ} \AgdaFunction{]uip} \AgdaSymbol{\{}\AgdaBound{a}\AgdaSymbol{\}} \AgdaSymbol{\{}\AgdaBound{b}\AgdaSymbol{\}} \AgdaSymbol{=} \AgdaFunction{Uni} \AgdaSymbol{(}\AgdaFunction{[} \AgdaBound{Γ} \AgdaFunction{]} \AgdaBound{a} \AgdaFunction{≈h} \AgdaBound{b}\AgdaSymbol{)}\<%
\\
%
\\
\>\<\end{code}

Morphisms between HSetoids (Functors)

\begin{code}\>\<%
\\
%
\\
\>\AgdaKeyword{infix} \AgdaNumber{5} \_⇉\_\<%
\\
%
\\
\>\AgdaKeyword{record} \AgdaRecord{\_⇉\_} \AgdaSymbol{(}\AgdaBound{A} \AgdaBound{B} \AgdaSymbol{:} \AgdaRecord{HSetoid}\AgdaSymbol{)} \AgdaSymbol{:} \AgdaPrimitiveType{Set} \AgdaKeyword{where}\<%
\\
\>[0]\AgdaIndent{2}{}\<[2]%
\>[2]\AgdaKeyword{constructor} \AgdaInductiveConstructor{fn:\_resp:\_}\<%
\\
\>[0]\AgdaIndent{2}{}\<[2]%
\>[2]\AgdaKeyword{field}\<%
\\
\>[2]\AgdaIndent{4}{}\<[4]%
\>[4]\AgdaField{fn} \<[9]%
\>[9]\AgdaSymbol{:} \AgdaFunction{∣} \AgdaBound{A} \AgdaFunction{∣} \AgdaSymbol{→} \AgdaFunction{∣} \AgdaBound{B} \AgdaFunction{∣}\<%
\\
\>[2]\AgdaIndent{4}{}\<[4]%
\>[4]\AgdaField{resp} \AgdaSymbol{:} \AgdaSymbol{\{}\AgdaBound{x} \AgdaBound{y} \AgdaSymbol{:} \AgdaFunction{∣} \AgdaBound{A} \AgdaFunction{∣}\AgdaSymbol{\}} \AgdaSymbol{→} \<[27]%
\>[27]\<%
\\
\>[4]\AgdaIndent{11}{}\<[11]%
\>[11]\AgdaFunction{[} \AgdaBound{A} \AgdaFunction{]} \AgdaBound{x} \AgdaFunction{≈} \AgdaBound{y} \AgdaSymbol{→} \<[25]%
\>[25]\<%
\\
\>[4]\AgdaIndent{11}{}\<[11]%
\>[11]\AgdaFunction{[} \AgdaBound{B} \AgdaFunction{]} \AgdaBound{fn} \AgdaBound{x} \AgdaFunction{≈} \AgdaBound{fn} \AgdaBound{y}\<%
\\
\>\AgdaKeyword{open} \AgdaModule{\_⇉\_} \AgdaKeyword{public} \AgdaKeyword{renaming} \AgdaSymbol{(}fn \AgdaSymbol{to} [\_]fn \AgdaSymbol{;} resp \AgdaSymbol{to} [\_]resp\AgdaSymbol{)}\<%
\\
%
\\
\>\<\end{code}

Identity

\begin{code}\>\<%
\\
%
\\
\>\AgdaFunction{idCH} \AgdaSymbol{:} \AgdaSymbol{\{}\AgdaBound{Γ} \AgdaSymbol{:} \AgdaRecord{HSetoid}\AgdaSymbol{\}} \AgdaSymbol{→} \AgdaBound{Γ} \AgdaRecord{⇉} \AgdaBound{Γ} \<[29]%
\>[29]\<%
\\
\>\AgdaFunction{idCH} \AgdaSymbol{=} \AgdaKeyword{record} \AgdaSymbol{\{} \AgdaField{fn} \AgdaSymbol{=} \AgdaFunction{id}\AgdaSymbol{;} \AgdaField{resp} \AgdaSymbol{=} \AgdaFunction{id}\AgdaSymbol{\}}\<%
\\
%
\\
\>\<\end{code}

Composition

\begin{code}\>\<%
\\
%
\\
\>\AgdaKeyword{infixl} \AgdaNumber{5} \_∘c\_\<%
\\
%
\\
\>\AgdaFunction{\_∘c\_} \AgdaSymbol{:} \AgdaSymbol{∀\{}\AgdaBound{Γ} \AgdaBound{Δ} \AgdaBound{Z}\AgdaSymbol{\}} \AgdaSymbol{→} \AgdaBound{Δ} \AgdaRecord{⇉} \AgdaBound{Z} \AgdaSymbol{→} \AgdaBound{Γ} \AgdaRecord{⇉} \AgdaBound{Δ} \AgdaSymbol{→} \AgdaBound{Γ} \AgdaRecord{⇉} \AgdaBound{Z}\<%
\\
\>\AgdaBound{yz} \AgdaFunction{∘c} \AgdaBound{xy} \AgdaSymbol{=} \AgdaKeyword{record} \<[18]%
\>[18]\<%
\\
\>[4]\AgdaIndent{11}{}\<[11]%
\>[11]\AgdaSymbol{\{} \AgdaField{fn} \AgdaSymbol{=} \AgdaFunction{[} \AgdaBound{yz} \AgdaFunction{]fn} \AgdaFunction{∘} \AgdaFunction{[} \AgdaBound{xy} \AgdaFunction{]fn}\<%
\\
\>[4]\AgdaIndent{11}{}\<[11]%
\>[11]\AgdaSymbol{;} \AgdaField{resp} \AgdaSymbol{=} \AgdaFunction{[} \AgdaBound{yz} \AgdaFunction{]resp} \AgdaFunction{∘} \AgdaFunction{[} \AgdaBound{xy} \AgdaFunction{]resp}\<%
\\
\>[4]\AgdaIndent{11}{}\<[11]%
\>[11]\AgdaSymbol{\}}\<%
\\
\>\<\end{code}

Categorical laws

\begin{code}\>\<%
\\
%
\\
\>\AgdaFunction{id₁} \AgdaSymbol{:} \AgdaSymbol{∀} \AgdaSymbol{\{}\AgdaBound{Γ} \AgdaBound{Δ}\AgdaSymbol{\}(}\AgdaBound{ch} \AgdaSymbol{:} \AgdaBound{Γ} \AgdaRecord{⇉} \AgdaBound{Δ}\AgdaSymbol{)} \AgdaSymbol{→} \AgdaBound{ch} \AgdaFunction{∘c} \AgdaFunction{idCH} \AgdaDatatype{≡} \AgdaBound{ch}\<%
\\
\>\AgdaFunction{id₁} \AgdaBound{ch} \AgdaSymbol{=} \AgdaInductiveConstructor{PE.refl}\<%
\\
%
\\
\>\AgdaFunction{id₂} \AgdaSymbol{:} \AgdaSymbol{∀} \AgdaSymbol{\{}\AgdaBound{Γ} \AgdaBound{Δ}\AgdaSymbol{\}(}\AgdaBound{ch} \AgdaSymbol{:} \AgdaBound{Γ} \AgdaRecord{⇉} \AgdaBound{Δ}\AgdaSymbol{)} \AgdaSymbol{→} \AgdaFunction{idCH} \AgdaFunction{∘c} \AgdaBound{ch} \AgdaDatatype{≡} \AgdaBound{ch}\<%
\\
\>\AgdaFunction{id₂} \AgdaBound{ch} \AgdaSymbol{=} \AgdaInductiveConstructor{PE.refl}\<%
\\
%
\\
\>\AgdaFunction{comp} \AgdaSymbol{:} \AgdaSymbol{∀} \AgdaSymbol{\{}\AgdaBound{Γ} \AgdaBound{Δ} \AgdaBound{Φ} \AgdaBound{Ψ}\AgdaSymbol{\}} \AgdaSymbol{(}\AgdaBound{f} \AgdaSymbol{:} \AgdaBound{Γ} \AgdaRecord{⇉} \AgdaBound{Δ}\AgdaSymbol{)} \AgdaSymbol{(}\AgdaBound{g} \AgdaSymbol{:} \AgdaBound{Δ} \AgdaRecord{⇉} \AgdaBound{Φ}\AgdaSymbol{)} \AgdaSymbol{(}\AgdaBound{h} \AgdaSymbol{:} \AgdaBound{Φ} \AgdaRecord{⇉} \AgdaBound{Ψ}\AgdaSymbol{)}\<%
\\
\>[0]\AgdaIndent{7}{}\<[7]%
\>[7]\AgdaSymbol{→} \AgdaBound{h} \AgdaFunction{∘c} \AgdaBound{g} \AgdaFunction{∘c} \AgdaBound{f} \AgdaDatatype{≡} \AgdaBound{h} \AgdaFunction{∘c} \AgdaSymbol{(}\AgdaBound{g} \AgdaFunction{∘c} \AgdaBound{f}\AgdaSymbol{)}\<%
\\
\>\AgdaFunction{comp} \AgdaBound{f} \AgdaBound{g} \AgdaBound{h} \AgdaSymbol{=} \AgdaInductiveConstructor{PE.refl}\<%
\\
%
\\
\>\AgdaFunction{\_f≈\_} \AgdaSymbol{:} \<[8]%
\>[8]\AgdaSymbol{∀\{}\AgdaBound{Γ} \AgdaBound{Δ} \AgdaSymbol{:} \AgdaRecord{HSetoid}\AgdaSymbol{\}} \AgdaSymbol{→} \AgdaSymbol{(}\AgdaBound{f} \AgdaBound{g} \AgdaSymbol{:} \AgdaBound{Γ} \AgdaRecord{⇉} \AgdaBound{Δ}\AgdaSymbol{)} \AgdaSymbol{→} \AgdaRecord{HProp}\<%
\\
\>\AgdaFunction{\_f≈\_} \AgdaSymbol{\{}\AgdaBound{Γ} \AgdaInductiveConstructor{,} \AgdaBound{\_≈h\_} \AgdaInductiveConstructor{,} \AgdaBound{refl} \AgdaInductiveConstructor{,} \AgdaBound{sym} \AgdaInductiveConstructor{,} \AgdaBound{trans}\AgdaSymbol{\}} \<[37]%
\>[37]\<%
\\
\>[0]\AgdaIndent{5}{}\<[5]%
\>[5]\AgdaSymbol{\{}\AgdaBound{Δ} \AgdaInductiveConstructor{,} \AgdaBound{\_≈h₁\_} \AgdaInductiveConstructor{,} \AgdaBound{refl₁} \AgdaInductiveConstructor{,} \AgdaBound{sym₁} \AgdaInductiveConstructor{,} \AgdaBound{trans₁}\AgdaSymbol{\}}\<%
\\
\>[0]\AgdaIndent{5}{}\<[5]%
\>[5]\AgdaSymbol{(}\AgdaInductiveConstructor{fn:} \AgdaBound{fn} \AgdaInductiveConstructor{resp:} \AgdaBound{fresp}\AgdaSymbol{)} \AgdaSymbol{(}\AgdaInductiveConstructor{fn:} \AgdaBound{gn} \AgdaInductiveConstructor{resp:} \AgdaBound{gresp}\AgdaSymbol{)} \<[47]%
\>[47]\<%
\\
\>[0]\AgdaIndent{2}{}\<[2]%
\>[2]\AgdaSymbol{=} \AgdaKeyword{record} \<[11]%
\>[11]\<%
\\
\>[0]\AgdaIndent{11}{}\<[11]%
\>[11]\AgdaSymbol{\{} \AgdaField{prf} \AgdaSymbol{=} \AgdaSymbol{(}\AgdaBound{g} \AgdaSymbol{:} \AgdaBound{Γ}\AgdaSymbol{)} \AgdaSymbol{→} \AgdaFunction{<} \AgdaBound{fn} \AgdaBound{g} \AgdaBound{≈h₁} \AgdaBound{gn} \AgdaBound{g} \AgdaFunction{>}\<%
\\
\>[0]\AgdaIndent{11}{}\<[11]%
\>[11]\AgdaSymbol{;} \AgdaField{Uni} \AgdaSymbol{=} \AgdaBound{ext} \AgdaSymbol{(λ} \AgdaBound{g} \AgdaSymbol{→} \AgdaFunction{Uni} \AgdaSymbol{(}\AgdaBound{fn} \AgdaBound{g} \AgdaBound{≈h₁} \AgdaBound{gn} \AgdaBound{g}\AgdaSymbol{))}\<%
\\
\>[0]\AgdaIndent{11}{}\<[11]%
\>[11]\AgdaSymbol{\}}\<%
\\
%
\\
\>\<\end{code}

Category of Setoids

\begin{code}\>\<%
\\
%
\\
%
\\
\>\AgdaFunction{Std} \AgdaSymbol{:} \AgdaRecord{Category}\<%
\\
\>\AgdaFunction{Std} \AgdaSymbol{=} \AgdaInductiveConstructor{CatC} \AgdaRecord{HSetoid} \AgdaRecord{\_⇉\_} \AgdaSymbol{(λ} \AgdaBound{\_} \AgdaSymbol{→} \AgdaFunction{idCH}\AgdaSymbol{)} \AgdaSymbol{(λ} \AgdaBound{\_} \AgdaBound{\_} \AgdaSymbol{→} \AgdaFunction{\_∘c\_}\AgdaSymbol{)} \<[51]%
\>[51]\<%
\\
\>[0]\AgdaIndent{6}{}\<[6]%
\>[6]\AgdaSymbol{(}\AgdaInductiveConstructor{IsCatC} \AgdaSymbol{(λ} \AgdaBound{α} \AgdaBound{β} \AgdaBound{f} \AgdaSymbol{→} \AgdaInductiveConstructor{PE.refl}\AgdaSymbol{)} \<[34]%
\>[34]\<%
\\
\>[0]\AgdaIndent{6}{}\<[6]%
\>[6]\AgdaSymbol{(λ} \AgdaBound{α} \AgdaBound{β} \AgdaBound{f} \AgdaSymbol{→} \AgdaInductiveConstructor{PE.refl}\AgdaSymbol{)} \<[26]%
\>[26]\<%
\\
\>[0]\AgdaIndent{6}{}\<[6]%
\>[6]\AgdaSymbol{(λ} \AgdaBound{α} \AgdaBound{δ} \AgdaBound{f} \AgdaBound{g} \AgdaBound{h} \AgdaSymbol{→} \AgdaInductiveConstructor{PE.refl}\AgdaSymbol{))}\<%
\\
%
\\
\>\<\end{code}

Terminal object

\begin{code}\>\<%
\\
%
\\
\>\AgdaFunction{●} \AgdaSymbol{:} \AgdaRecord{HSetoid}\<%
\\
\>\AgdaFunction{●} \<[4]%
\>[4]\AgdaSymbol{=} \AgdaKeyword{record} \AgdaSymbol{\{}\<%
\\
\>[0]\AgdaIndent{6}{}\<[6]%
\>[6]\AgdaField{Carrier} \AgdaSymbol{=} \AgdaRecord{⊤}\AgdaSymbol{;}\<%
\\
\>[0]\AgdaIndent{6}{}\<[6]%
\>[6]\AgdaField{\_≈h\_} \<[14]%
\>[14]\AgdaSymbol{=} \AgdaSymbol{λ} \AgdaBound{\_} \AgdaBound{\_} \AgdaSymbol{→} \AgdaFunction{⊤'}\AgdaSymbol{;}\<%
\\
\>[0]\AgdaIndent{6}{}\<[6]%
\>[6]\AgdaField{refl} \<[14]%
\>[14]\AgdaSymbol{=} \AgdaInductiveConstructor{tt}\AgdaSymbol{;}\<%
\\
\>[0]\AgdaIndent{6}{}\<[6]%
\>[6]\AgdaField{sym} \<[14]%
\>[14]\AgdaSymbol{=} \AgdaSymbol{λ} \AgdaBound{\_} \AgdaSymbol{→} \AgdaInductiveConstructor{tt}\AgdaSymbol{;}\<%
\\
\>[0]\AgdaIndent{6}{}\<[6]%
\>[6]\AgdaField{trans} \<[14]%
\>[14]\AgdaSymbol{=} \AgdaSymbol{λ} \AgdaBound{\_} \AgdaBound{\_} \AgdaSymbol{→} \AgdaInductiveConstructor{tt} \AgdaSymbol{\}}\<%
\\
%
\\
%
\\
\>\AgdaFunction{⋆} \AgdaSymbol{:} \AgdaSymbol{\{}\AgdaBound{Δ} \AgdaSymbol{:} \AgdaRecord{HSetoid}\AgdaSymbol{\}} \AgdaSymbol{→} \AgdaBound{Δ} \AgdaRecord{⇉} \AgdaFunction{●}\<%
\\
\>\AgdaFunction{⋆} \AgdaSymbol{=} \AgdaKeyword{record} \<[11]%
\>[11]\<%
\\
\>[0]\AgdaIndent{6}{}\<[6]%
\>[6]\AgdaSymbol{\{} \AgdaField{fn} \AgdaSymbol{=} \AgdaSymbol{λ} \AgdaBound{\_} \AgdaSymbol{→} \AgdaInductiveConstructor{tt}\<%
\\
\>[0]\AgdaIndent{6}{}\<[6]%
\>[6]\AgdaSymbol{;} \AgdaField{resp} \AgdaSymbol{=} \AgdaSymbol{λ} \AgdaBound{\_} \AgdaSymbol{→} \AgdaInductiveConstructor{tt} \AgdaSymbol{\}}\<%
\\
%
\\
\>\AgdaFunction{uniqueHom} \AgdaSymbol{:} \AgdaSymbol{∀} \AgdaSymbol{(}\AgdaBound{Δ} \AgdaSymbol{:} \AgdaRecord{HSetoid}\AgdaSymbol{)} \AgdaSymbol{→} \AgdaSymbol{(}\AgdaBound{f} \AgdaSymbol{:} \AgdaBound{Δ} \AgdaRecord{⇉} \AgdaFunction{●}\AgdaSymbol{)} \AgdaSymbol{→} \AgdaBound{f} \AgdaDatatype{≡} \AgdaFunction{⋆}\<%
\\
\>\AgdaFunction{uniqueHom} \AgdaBound{Δ} \AgdaBound{f} \AgdaSymbol{=} \AgdaInductiveConstructor{PE.refl}\<%
\\
%
\\
\>\<\end{code}


\section{categories with families of setoids}


A Category with families consists of a base category and a functor
\cite{clairambault2005categories}. We firstly define the categories with
families of sets in Agda  as a guidance for the one for setoids. We
would present the setoid one here since it is relevant.

\AgdaHide{
\begin{code}\>\<%
\\
%
\\
\>\AgdaKeyword{module} \AgdaModule{CwF-setoid} \AgdaKeyword{where}\<%
\\
%
\\
\>\AgdaKeyword{open} \AgdaKeyword{import} \AgdaModule{CategoryofSetoid}\<%
\\
\>\AgdaKeyword{open} \AgdaKeyword{import} \AgdaModule{Data.Product}\<%
\\
\>\AgdaKeyword{open} \AgdaKeyword{import} \AgdaModule{Function}\<%
\\
\>\AgdaKeyword{open} \AgdaKeyword{import} \AgdaModule{Data.Nat}\<%
\\
\>\AgdaKeyword{open} \AgdaKeyword{import} \AgdaModule{Data.Empty}\<%
\\
\>\AgdaKeyword{open} \AgdaKeyword{import} \AgdaModule{Data.Unit}\<%
\\
%
\\
\>\<\end{code}
}

\section{Categories with families}

Context are interpreted as setoids

\begin{code}\>\<%
\\
\>\AgdaFunction{Con} \AgdaSymbol{=} \AgdaRecord{Setoid}\<%
\\
\>\<\end{code}

Semantic Types

\begin{code}\>\<%
\\
\>\AgdaKeyword{record} \AgdaRecord{Ty} \AgdaSymbol{(}\AgdaBound{Γ} \AgdaSymbol{:} \AgdaRecord{Setoid}\AgdaSymbol{)} \AgdaSymbol{:} \AgdaPrimitiveType{Set₁} \AgdaKeyword{where}\<%
\\
\>[0]\AgdaIndent{2}{}\<[2]%
\>[2]\AgdaKeyword{field}\<%
\\
\>[2]\AgdaIndent{4}{}\<[4]%
\>[4]\AgdaField{fm} \<[11]%
\>[11]\AgdaSymbol{:} \AgdaFunction{∣} \AgdaBound{Γ} \AgdaFunction{∣} \AgdaSymbol{→} \AgdaRecord{Setoid}\<%
\\
\>[2]\AgdaIndent{4}{}\<[4]%
\>[4]\AgdaField{substT} \AgdaSymbol{:} \AgdaSymbol{\{}\AgdaBound{x} \AgdaBound{y} \AgdaSymbol{:} \AgdaFunction{∣} \AgdaBound{Γ} \AgdaFunction{∣}\AgdaSymbol{\}} \AgdaSymbol{→} \<[29]%
\>[29]\<%
\\
\>[4]\AgdaIndent{13}{}\<[13]%
\>[13]\AgdaSymbol{.(}\AgdaFunction{[} \AgdaBound{Γ} \AgdaFunction{]} \AgdaBound{x} \AgdaFunction{≈} \AgdaBound{y}\AgdaSymbol{)} \AgdaSymbol{→}\<%
\\
\>[4]\AgdaIndent{13}{}\<[13]%
\>[13]\AgdaFunction{∣} \AgdaBound{fm} \AgdaBound{x} \AgdaFunction{∣} \AgdaSymbol{→} \<[24]%
\>[24]\<%
\\
\>[4]\AgdaIndent{13}{}\<[13]%
\>[13]\AgdaFunction{∣} \AgdaBound{fm} \AgdaBound{y} \AgdaFunction{∣}\<%
\\
\>[0]\AgdaIndent{4}{}\<[4]%
\>[4]\AgdaSymbol{.}\AgdaField{subst*} \AgdaSymbol{:} \AgdaSymbol{∀\{}\AgdaBound{x} \AgdaBound{y} \AgdaSymbol{:} \AgdaFunction{∣} \AgdaBound{Γ} \AgdaFunction{∣}\AgdaSymbol{\}}\<%
\\
\>[0]\AgdaIndent{13}{}\<[13]%
\>[13]\AgdaSymbol{(}\AgdaBound{p} \AgdaSymbol{:} \AgdaSymbol{(}\AgdaFunction{[} \AgdaBound{Γ} \AgdaFunction{]} \AgdaBound{x} \AgdaFunction{≈} \AgdaBound{y}\AgdaSymbol{))}\<%
\\
\>[0]\AgdaIndent{13}{}\<[13]%
\>[13]\AgdaSymbol{\{}\AgdaBound{a} \AgdaBound{b} \AgdaSymbol{:} \AgdaFunction{∣} \AgdaBound{fm} \AgdaBound{x} \AgdaFunction{∣}\AgdaSymbol{\}} \AgdaSymbol{→}\<%
\\
\>[0]\AgdaIndent{13}{}\<[13]%
\>[13]\AgdaSymbol{.(}\AgdaFunction{[} \AgdaBound{fm} \AgdaBound{x} \AgdaFunction{]} \AgdaBound{a} \AgdaFunction{≈} \AgdaBound{b}\AgdaSymbol{)} \AgdaSymbol{→}\<%
\\
\>[0]\AgdaIndent{13}{}\<[13]%
\>[13]\AgdaSymbol{(}\AgdaFunction{[} \AgdaBound{fm} \AgdaBound{y} \AgdaFunction{]} \AgdaBound{substT} \AgdaBound{p} \AgdaBound{a} \AgdaFunction{≈} \AgdaBound{substT} \AgdaBound{p} \AgdaBound{b}\AgdaSymbol{)}\<%
\\
%
\\
\>[0]\AgdaIndent{4}{}\<[4]%
\>[4]\AgdaSymbol{.}\AgdaField{refl*} \<[12]%
\>[12]\AgdaSymbol{:} \AgdaSymbol{∀\{}\AgdaBound{x} \AgdaSymbol{:} \AgdaFunction{∣} \AgdaBound{Γ} \AgdaFunction{∣}\AgdaSymbol{\}\{}\AgdaBound{a} \AgdaSymbol{:} \AgdaFunction{∣} \AgdaBound{fm} \AgdaBound{x} \AgdaFunction{∣}\AgdaSymbol{\}} \AgdaSymbol{→} \<[43]%
\>[43]\<%
\\
\>[0]\AgdaIndent{13}{}\<[13]%
\>[13]\AgdaFunction{[} \AgdaBound{fm} \AgdaBound{x} \AgdaFunction{]} \AgdaBound{substT} \AgdaSymbol{(}\AgdaFunction{[} \AgdaBound{Γ} \AgdaFunction{]refl}\AgdaSymbol{)} \AgdaBound{a} \AgdaFunction{≈} \AgdaBound{a}\<%
\\
\>[0]\AgdaIndent{4}{}\<[4]%
\>[4]\AgdaSymbol{.}\AgdaField{trans*} \AgdaSymbol{:} \AgdaSymbol{∀\{}\AgdaBound{x} \AgdaBound{y} \AgdaBound{z} \AgdaSymbol{:} \AgdaFunction{∣} \AgdaBound{Γ} \AgdaFunction{∣}\AgdaSymbol{\}}\<%
\\
\>[0]\AgdaIndent{13}{}\<[13]%
\>[13]\AgdaSymbol{\{}\AgdaBound{p} \AgdaSymbol{:} \AgdaFunction{[} \AgdaBound{Γ} \AgdaFunction{]} \AgdaBound{x} \AgdaFunction{≈} \AgdaBound{y}\AgdaSymbol{\}}\<%
\\
\>[0]\AgdaIndent{13}{}\<[13]%
\>[13]\AgdaSymbol{\{}\AgdaBound{q} \AgdaSymbol{:} \AgdaFunction{[} \AgdaBound{Γ} \AgdaFunction{]} \AgdaBound{y} \AgdaFunction{≈} \AgdaBound{z}\AgdaSymbol{\}}\<%
\\
\>[0]\AgdaIndent{13}{}\<[13]%
\>[13]\AgdaSymbol{(}\AgdaBound{a} \AgdaSymbol{:} \AgdaFunction{∣} \AgdaBound{fm} \AgdaBound{x} \AgdaFunction{∣}\AgdaSymbol{)} \AgdaSymbol{→} \<[30]%
\>[30]\<%
\\
\>[0]\AgdaIndent{13}{}\<[13]%
\>[13]\AgdaFunction{[} \AgdaBound{fm} \AgdaBound{z} \AgdaFunction{]} \AgdaBound{substT} \AgdaBound{q} \AgdaSymbol{(}\AgdaBound{substT} \AgdaBound{p} \AgdaBound{a}\AgdaSymbol{)} \<[44]%
\>[44]\<%
\\
\>[0]\AgdaIndent{12}{}\<[12]%
\>[12]\AgdaFunction{≈} \AgdaBound{substT} \AgdaSymbol{(}\AgdaFunction{[} \AgdaBound{Γ} \AgdaFunction{]trans} \AgdaBound{p} \AgdaBound{q}\AgdaSymbol{)} \AgdaBound{a}\<%
\\
%
\\
\>[0]\AgdaIndent{2}{}\<[2]%
\>[2]\AgdaSymbol{.}\AgdaFunction{tr*} \AgdaSymbol{:} \AgdaSymbol{∀\{}\AgdaBound{x} \AgdaBound{y} \AgdaSymbol{:} \AgdaFunction{∣} \AgdaBound{Γ} \AgdaFunction{∣}\AgdaSymbol{\}}\<%
\\
\>[2]\AgdaIndent{10}{}\<[10]%
\>[10]\AgdaSymbol{\{}\AgdaBound{p} \AgdaSymbol{:} \AgdaFunction{[} \AgdaBound{Γ} \AgdaFunction{]} \AgdaBound{y} \AgdaFunction{≈} \AgdaBound{x}\AgdaSymbol{\}}\<%
\\
\>[2]\AgdaIndent{10}{}\<[10]%
\>[10]\AgdaSymbol{\{}\AgdaBound{q} \AgdaSymbol{:} \AgdaFunction{[} \AgdaBound{Γ} \AgdaFunction{]} \AgdaBound{x} \AgdaFunction{≈} \AgdaBound{y}\AgdaSymbol{\}}\<%
\\
\>[2]\AgdaIndent{10}{}\<[10]%
\>[10]\AgdaSymbol{\{}\AgdaBound{a} \AgdaSymbol{:} \AgdaFunction{∣} \AgdaFunction{fm} \AgdaBound{x} \AgdaFunction{∣}\AgdaSymbol{\}} \AgdaSymbol{→} \<[27]%
\>[27]\<%
\\
\>[2]\AgdaIndent{10}{}\<[10]%
\>[10]\AgdaFunction{[} \AgdaFunction{fm} \AgdaBound{x} \AgdaFunction{]} \AgdaFunction{substT} \AgdaBound{p} \AgdaSymbol{(}\AgdaFunction{substT} \AgdaBound{q} \AgdaBound{a}\AgdaSymbol{)} \AgdaFunction{≈} \AgdaBound{a}\<%
\\
\>[0]\AgdaIndent{2}{}\<[2]%
\>[2]\AgdaFunction{tr*} \AgdaSymbol{=} \AgdaFunction{[} \AgdaFunction{fm} \AgdaSymbol{\_} \AgdaFunction{]trans} \AgdaSymbol{(}\AgdaFunction{trans*} \AgdaSymbol{\_)} \AgdaFunction{refl*}\<%
\\
%
\\
\>[0]\AgdaIndent{2}{}\<[2]%
\>[2]\AgdaFunction{substT-inv} \AgdaSymbol{:} \AgdaSymbol{\{}\AgdaBound{x} \AgdaBound{y} \AgdaSymbol{:} \AgdaFunction{∣} \AgdaBound{Γ} \AgdaFunction{∣}\AgdaSymbol{\}} \AgdaSymbol{→} \<[31]%
\>[31]\<%
\\
\>[2]\AgdaIndent{13}{}\<[13]%
\>[13]\AgdaSymbol{.(}\AgdaFunction{[} \AgdaBound{Γ} \AgdaFunction{]} \AgdaBound{x} \AgdaFunction{≈} \AgdaBound{y}\AgdaSymbol{)} \AgdaSymbol{→}\<%
\\
\>[2]\AgdaIndent{13}{}\<[13]%
\>[13]\AgdaFunction{∣} \AgdaFunction{fm} \AgdaBound{y} \AgdaFunction{∣} \AgdaSymbol{→} \<[24]%
\>[24]\<%
\\
\>[2]\AgdaIndent{13}{}\<[13]%
\>[13]\AgdaFunction{∣} \AgdaFunction{fm} \AgdaBound{x} \AgdaFunction{∣}\<%
\\
\>[0]\AgdaIndent{2}{}\<[2]%
\>[2]\AgdaFunction{substT-inv} \AgdaBound{p} \AgdaBound{y} \AgdaSymbol{=} \AgdaFunction{substT} \AgdaSymbol{(}\AgdaFunction{[} \AgdaBound{Γ} \AgdaFunction{]sym} \AgdaBound{p}\AgdaSymbol{)} \AgdaBound{y}\<%
\\
\>\<\end{code}

\AgdaHide{
\begin{code}\>\<%
\\
\>[0]\AgdaIndent{2}{}\<[2]%
\>[2]\AgdaFunction{subst-mir1} \AgdaSymbol{:} \AgdaSymbol{∀\{}\AgdaBound{x} \AgdaBound{y} \AgdaSymbol{:} \AgdaFunction{∣} \AgdaBound{Γ} \AgdaFunction{∣}\AgdaSymbol{\}\{}\AgdaBound{P} \AgdaSymbol{:} \AgdaPrimitiveType{Set}\AgdaSymbol{\}}\<%
\\
\>[2]\AgdaIndent{14}{}\<[14]%
\>[14]\AgdaSymbol{\{}\AgdaBound{p} \AgdaSymbol{:} \AgdaFunction{[} \AgdaBound{Γ} \AgdaFunction{]} \AgdaBound{x} \AgdaFunction{≈} \AgdaBound{y}\AgdaSymbol{\}\{}\AgdaBound{q} \AgdaSymbol{:} \AgdaFunction{[} \AgdaBound{Γ} \AgdaFunction{]} \AgdaBound{y} \AgdaFunction{≈} \AgdaBound{x}\AgdaSymbol{\}}\<%
\\
\>[2]\AgdaIndent{14}{}\<[14]%
\>[14]\AgdaSymbol{\{}\AgdaBound{a} \AgdaSymbol{:} \AgdaFunction{∣} \AgdaFunction{fm} \AgdaBound{x} \AgdaFunction{∣}\AgdaSymbol{\}\{}\AgdaBound{b} \AgdaSymbol{:} \AgdaFunction{∣} \AgdaFunction{fm} \AgdaBound{y} \AgdaFunction{∣}\AgdaSymbol{\}} \AgdaSymbol{→} \<[45]%
\>[45]\<%
\\
\>[2]\AgdaIndent{14}{}\<[14]%
\>[14]\AgdaSymbol{(.(}\AgdaFunction{[} \AgdaFunction{fm} \AgdaBound{y} \AgdaFunction{]} \AgdaFunction{substT} \AgdaBound{p} \AgdaBound{a} \AgdaFunction{≈} \AgdaBound{b}\AgdaSymbol{)} \AgdaSymbol{→} \AgdaBound{P}\AgdaSymbol{)} \<[47]%
\>[47]\<%
\\
\>[-1]\AgdaIndent{13}{}\<[13]%
\>[13]\AgdaSymbol{→} \AgdaSymbol{(.(}\AgdaFunction{[} \AgdaFunction{fm} \AgdaBound{x} \AgdaFunction{]} \AgdaBound{a} \AgdaFunction{≈} \AgdaFunction{substT} \AgdaBound{q} \AgdaBound{b}\AgdaSymbol{)} \AgdaSymbol{→} \AgdaBound{P}\AgdaSymbol{)}\<%
\\
\>[0]\AgdaIndent{2}{}\<[2]%
\>[2]\AgdaFunction{subst-mir1} \AgdaBound{eq} \AgdaBound{p} \AgdaSymbol{=} \AgdaBound{eq} \AgdaSymbol{(}\AgdaFunction{[} \AgdaFunction{fm} \AgdaSymbol{\_} \AgdaFunction{]trans} \AgdaSymbol{(}\AgdaFunction{subst*} \AgdaSymbol{\_} \AgdaBound{p}\AgdaSymbol{)} \AgdaFunction{tr*}\AgdaSymbol{)}\<%
\\
%
\\
\>\AgdaKeyword{open} \AgdaModule{Ty} \AgdaKeyword{public} \<[15]%
\>[15]\<%
\\
\>[0]\AgdaIndent{2}{}\<[2]%
\>[2]\AgdaKeyword{renaming} \AgdaSymbol{(}substT \AgdaSymbol{to} [\_]subst\AgdaSymbol{;} substT-inv \AgdaSymbol{to} [\_]subst-inv
           \AgdaSymbol{;} subst* \AgdaSymbol{to} [\_]subst*\AgdaSymbol{;} fm \AgdaSymbol{to} [\_]fm \AgdaSymbol{;}
            refl* \AgdaSymbol{to} [\_]refl* \AgdaSymbol{;} trans* \AgdaSymbol{to} [\_]trans*\AgdaSymbol{;} tr* \AgdaSymbol{to} [\_]tr*\AgdaSymbol{)}\<%
\\
%
\\
\>\<\end{code}
}

Type substitution

\begin{code}\>\<%
\\
\>\AgdaFunction{\_[\_]T} \AgdaSymbol{:} \AgdaSymbol{∀} \AgdaSymbol{\{}\AgdaBound{Γ} \AgdaBound{Δ} \AgdaSymbol{:} \AgdaRecord{Setoid}\AgdaSymbol{\}} \AgdaSymbol{→} \AgdaRecord{Ty} \AgdaBound{Δ} \AgdaSymbol{→} \AgdaBound{Γ} \AgdaRecord{⇉} \AgdaBound{Δ} \AgdaSymbol{→} \AgdaRecord{Ty} \AgdaBound{Γ}\<%
\\
\>\AgdaFunction{\_[\_]T} \AgdaSymbol{\{}\AgdaBound{Γ}\AgdaSymbol{\}} \AgdaSymbol{\{}\AgdaBound{Δ}\AgdaSymbol{\}} \AgdaBound{A} \AgdaBound{f}\<%
\\
\>[2]\AgdaIndent{5}{}\<[5]%
\>[5]\AgdaSymbol{=} \AgdaKeyword{record}\<%
\\
\>[2]\AgdaIndent{5}{}\<[5]%
\>[5]\AgdaSymbol{\{} \AgdaField{fm} \<[14]%
\>[14]\AgdaSymbol{=} \AgdaSymbol{λ} \AgdaBound{x} \AgdaSymbol{→} \AgdaFunction{fm} \AgdaSymbol{(}\AgdaFunction{fn} \AgdaBound{x}\AgdaSymbol{)}\<%
\\
\>[2]\AgdaIndent{5}{}\<[5]%
\>[5]\AgdaSymbol{;} \AgdaField{substT} \AgdaSymbol{=} \AgdaSymbol{λ} \AgdaBound{p} \AgdaSymbol{→} \AgdaFunction{substT} \AgdaSymbol{\_}\<%
\\
\>[2]\AgdaIndent{5}{}\<[5]%
\>[5]\AgdaSymbol{;} \AgdaField{subst*} \AgdaSymbol{=} \AgdaSymbol{λ} \AgdaBound{p} \AgdaSymbol{→} \AgdaFunction{subst*} \AgdaSymbol{(}\AgdaFunction{resp} \AgdaBound{p}\AgdaSymbol{)}\<%
\\
\>[2]\AgdaIndent{5}{}\<[5]%
\>[5]\AgdaSymbol{;} \AgdaField{refl*} \<[14]%
\>[14]\AgdaSymbol{=} \AgdaFunction{refl*}\<%
\\
\>[2]\AgdaIndent{5}{}\<[5]%
\>[5]\AgdaSymbol{;} \AgdaField{trans*} \AgdaSymbol{=} \AgdaFunction{trans*}\<%
\\
\>[2]\AgdaIndent{5}{}\<[5]%
\>[5]\AgdaSymbol{\}}\<%
\\
\>[2]\AgdaIndent{5}{}\<[5]%
\>[5]\AgdaKeyword{where} \<[11]%
\>[11]\<%
\\
\>[5]\AgdaIndent{7}{}\<[7]%
\>[7]\AgdaKeyword{open} \AgdaModule{Ty} \AgdaBound{A}\<%
\\
\>[5]\AgdaIndent{7}{}\<[7]%
\>[7]\AgdaKeyword{open} \AgdaModule{\_⇉\_} \AgdaBound{f}\<%
\\
%
\\
%
\\
\>\<\end{code}


Semantic Terms

\begin{code}\>\<%
\\
\>\AgdaKeyword{record} \AgdaRecord{Tm} \AgdaSymbol{\{}\AgdaBound{Γ} \AgdaSymbol{:} \AgdaFunction{Con}\AgdaSymbol{\}(}\AgdaBound{A} \AgdaSymbol{:} \AgdaRecord{Ty} \AgdaBound{Γ}\AgdaSymbol{)} \AgdaSymbol{:} \AgdaPrimitiveType{Set} \AgdaKeyword{where}\<%
\\
\>[0]\AgdaIndent{2}{}\<[2]%
\>[2]\AgdaKeyword{constructor} \AgdaInductiveConstructor{tm:\_resp:\_}\<%
\\
\>[0]\AgdaIndent{2}{}\<[2]%
\>[2]\AgdaKeyword{field}\<%
\\
\>[2]\AgdaIndent{4}{}\<[4]%
\>[4]\AgdaField{tm} \<[10]%
\>[10]\AgdaSymbol{:} \AgdaSymbol{(}\AgdaBound{x} \AgdaSymbol{:} \AgdaFunction{∣} \AgdaBound{Γ} \AgdaFunction{∣}\AgdaSymbol{)} \AgdaSymbol{→} \AgdaFunction{∣} \AgdaFunction{[} \AgdaBound{A} \AgdaFunction{]fm} \AgdaBound{x} \AgdaFunction{∣}\<%
\\
\>[2]\AgdaIndent{4}{}\<[4]%
\>[4]\AgdaSymbol{.}\AgdaField{respt} \AgdaSymbol{:} \AgdaSymbol{∀} \AgdaSymbol{\{}\AgdaBound{x} \AgdaBound{y} \AgdaSymbol{:} \AgdaFunction{∣} \AgdaBound{Γ} \AgdaFunction{∣}\AgdaSymbol{\}} \AgdaSymbol{→} \<[31]%
\>[31]\<%
\\
\>[4]\AgdaIndent{10}{}\<[10]%
\>[10]\AgdaSymbol{(}\AgdaBound{p} \AgdaSymbol{:} \AgdaFunction{[} \AgdaBound{Γ} \AgdaFunction{]} \AgdaBound{x} \AgdaFunction{≈} \AgdaBound{y}\AgdaSymbol{)} \AgdaSymbol{→} \<[30]%
\>[30]\<%
\\
\>[4]\AgdaIndent{10}{}\<[10]%
\>[10]\AgdaFunction{[} \AgdaFunction{[} \AgdaBound{A} \AgdaFunction{]fm} \AgdaBound{y} \AgdaFunction{]} \AgdaFunction{[} \AgdaBound{A} \AgdaFunction{]subst} \AgdaBound{p} \AgdaSymbol{(}\AgdaBound{tm} \AgdaBound{x}\AgdaSymbol{)} \AgdaFunction{≈} \AgdaBound{tm} \AgdaBound{y}\<%
\\
%
\\
\>\AgdaKeyword{open} \AgdaModule{Tm} \AgdaKeyword{public} \AgdaKeyword{renaming} \AgdaSymbol{(}tm \AgdaSymbol{to} [\_]tm \AgdaSymbol{;} respt \AgdaSymbol{to} [\_]respt\AgdaSymbol{)}\<%
\\
\>\<\end{code}

Term substitution

\begin{code}\>\<%
\\
\>\AgdaFunction{\_[\_]m} \AgdaSymbol{:} \AgdaSymbol{∀} \AgdaSymbol{\{}\AgdaBound{Γ} \AgdaBound{Δ} \AgdaSymbol{:} \AgdaFunction{Con}\AgdaSymbol{\}\{}\AgdaBound{A} \AgdaSymbol{:} \AgdaRecord{Ty} \AgdaBound{Δ}\AgdaSymbol{\}} \AgdaSymbol{→} \AgdaRecord{Tm} \AgdaBound{A} \<[39]%
\>[39]\<%
\\
\>[0]\AgdaIndent{6}{}\<[6]%
\>[6]\AgdaSymbol{→} \AgdaSymbol{(}\AgdaBound{f} \AgdaSymbol{:} \AgdaBound{Γ} \AgdaRecord{⇉} \AgdaBound{Δ}\AgdaSymbol{)} \AgdaSymbol{→} \AgdaRecord{Tm} \AgdaSymbol{(}\AgdaBound{A} \AgdaFunction{[} \AgdaBound{f} \AgdaFunction{]T}\AgdaSymbol{)}\<%
\\
\>\AgdaFunction{\_[\_]m} \AgdaBound{t} \AgdaBound{f} \AgdaSymbol{=} \AgdaKeyword{record} \<[19]%
\>[19]\<%
\\
\>[0]\AgdaIndent{10}{}\<[10]%
\>[10]\AgdaSymbol{\{} \AgdaField{tm} \AgdaSymbol{=} \AgdaFunction{[} \AgdaBound{t} \AgdaFunction{]tm} \AgdaFunction{∘} \AgdaFunction{[} \AgdaBound{f} \AgdaFunction{]fn}\<%
\\
\>[0]\AgdaIndent{10}{}\<[10]%
\>[10]\AgdaSymbol{;} \AgdaField{respt} \AgdaSymbol{=} \AgdaFunction{[} \AgdaBound{t} \AgdaFunction{]respt} \AgdaFunction{∘} \AgdaFunction{[} \AgdaBound{f} \AgdaFunction{]resp} \<[43]%
\>[43]\<%
\\
\>[0]\AgdaIndent{10}{}\<[10]%
\>[10]\AgdaSymbol{\}}\<%
\\
\>\<\end{code}

Context comprehension

\begin{code}\>\<%
\\
\>\AgdaFunction{\_\&\_} \AgdaSymbol{:} \AgdaSymbol{(}\AgdaBound{Γ} \AgdaSymbol{:} \AgdaRecord{Setoid}\AgdaSymbol{)} \AgdaSymbol{→} \AgdaRecord{Ty} \AgdaBound{Γ} \AgdaSymbol{→} \AgdaRecord{Setoid}\<%
\\
\>\AgdaBound{Γ} \AgdaFunction{\&} \AgdaBound{A} \AgdaSymbol{=} \<[8]%
\>[8]\<%
\\
\>[0]\AgdaIndent{2}{}\<[2]%
\>[2]\AgdaKeyword{record} \AgdaSymbol{\{} \AgdaField{Carrier} \AgdaSymbol{=} \AgdaRecord{Σ[} \AgdaBound{x} \AgdaRecord{∶} \AgdaFunction{∣} \AgdaBound{Γ} \AgdaFunction{∣} \AgdaRecord{]} \AgdaFunction{∣} \AgdaFunction{fm} \AgdaBound{x} \AgdaFunction{∣} \<[45]%
\>[45]\<%
\\
\>[0]\AgdaIndent{9}{}\<[9]%
\>[9]\AgdaSymbol{;} \AgdaField{\_≈\_} \AgdaSymbol{=} \AgdaSymbol{λ\{(}\AgdaBound{x} \AgdaInductiveConstructor{,} \AgdaBound{a}\AgdaSymbol{)} \AgdaSymbol{(}\AgdaBound{y} \AgdaInductiveConstructor{,} \AgdaBound{b}\AgdaSymbol{)} \AgdaSymbol{→} \<[37]%
\>[37]\<%
\\
\>[9]\AgdaIndent{13}{}\<[13]%
\>[13]\AgdaRecord{Σ[} \AgdaBound{p} \AgdaRecord{∶} \AgdaBound{x} \AgdaFunction{≈} \AgdaBound{y} \AgdaRecord{]} \AgdaFunction{[} \AgdaFunction{fm} \AgdaBound{y} \AgdaFunction{]} \AgdaSymbol{(}\AgdaFunction{substT} \AgdaBound{p} \AgdaBound{a}\AgdaSymbol{)} \AgdaFunction{≈} \AgdaBound{b} \AgdaSymbol{\}}\<%
\\
\>[0]\AgdaIndent{9}{}\<[9]%
\>[9]\AgdaSymbol{;} \AgdaField{refl} \AgdaSymbol{=} \AgdaFunction{refl} \AgdaInductiveConstructor{,} \AgdaFunction{refl*}\<%
\\
\>[0]\AgdaIndent{9}{}\<[9]%
\>[9]\AgdaSymbol{;} \AgdaField{sym} \AgdaSymbol{=} \<[18]%
\>[18]\AgdaSymbol{λ} \AgdaSymbol{\{(}\AgdaBound{p} \AgdaInductiveConstructor{,} \AgdaBound{q}\AgdaSymbol{)} \AgdaSymbol{→} \AgdaSymbol{(}\AgdaFunction{sym} \AgdaBound{p}\AgdaSymbol{)} \AgdaInductiveConstructor{,} \<[41]%
\>[41]\<%
\\
\>[9]\AgdaIndent{13}{}\<[13]%
\>[13]\AgdaFunction{[} \AgdaFunction{fm} \AgdaSymbol{\_} \AgdaFunction{]trans} \AgdaSymbol{(}\AgdaFunction{subst*} \AgdaSymbol{\_} \AgdaSymbol{(}\AgdaFunction{[} \AgdaFunction{fm} \AgdaSymbol{\_} \AgdaFunction{]sym} \AgdaBound{q}\AgdaSymbol{))} \AgdaFunction{tr*} \AgdaSymbol{\}}\<%
\\
\>[0]\AgdaIndent{9}{}\<[9]%
\>[9]\AgdaSymbol{;} \AgdaField{trans} \AgdaSymbol{=} \AgdaSymbol{λ} \AgdaSymbol{\{(}\AgdaBound{p} \AgdaInductiveConstructor{,} \AgdaBound{q}\AgdaSymbol{)} \AgdaSymbol{(}\AgdaBound{m} \AgdaInductiveConstructor{,} \AgdaBound{n}\AgdaSymbol{)} \AgdaSymbol{→} \AgdaFunction{trans} \AgdaBound{p} \AgdaBound{m} \AgdaInductiveConstructor{,}\<%
\\
\>[0]\AgdaIndent{19}{}\<[19]%
\>[19]\AgdaFunction{[} \AgdaFunction{fm} \AgdaSymbol{\_} \AgdaFunction{]trans} \AgdaSymbol{(}\AgdaFunction{[} \AgdaFunction{fm} \AgdaSymbol{\_} \AgdaFunction{]trans} \<[48]%
\>[48]\<%
\\
\>[0]\AgdaIndent{19}{}\<[19]%
\>[19]\AgdaSymbol{(}\AgdaFunction{[} \AgdaFunction{fm} \AgdaSymbol{\_} \AgdaFunction{]sym} \AgdaSymbol{(}\AgdaFunction{trans*} \AgdaSymbol{\_))} \AgdaSymbol{(}\AgdaFunction{subst*} \AgdaSymbol{\_} \AgdaBound{q}\AgdaSymbol{))} \AgdaBound{n}\AgdaSymbol{\}}\<%
\\
\>[0]\AgdaIndent{9}{}\<[9]%
\>[9]\AgdaSymbol{\}}\<%
\\
\>[0]\AgdaIndent{4}{}\<[4]%
\>[4]\AgdaKeyword{where}\<%
\\
\>[4]\AgdaIndent{6}{}\<[6]%
\>[6]\AgdaKeyword{open} \AgdaModule{Setoid} \AgdaBound{Γ}\<%
\\
\>[4]\AgdaIndent{6}{}\<[6]%
\>[6]\AgdaKeyword{open} \AgdaModule{Ty} \AgdaBound{A}\<%
\\
%
\\
%
\\
\>\AgdaKeyword{infixl} \AgdaNumber{5} \_\&\_\<%
\\
%
\\
\>\AgdaFunction{fst\&} \AgdaSymbol{:} \AgdaSymbol{\{}\AgdaBound{Γ} \AgdaSymbol{:} \AgdaFunction{Con}\AgdaSymbol{\}\{}\AgdaBound{A} \AgdaSymbol{:} \AgdaRecord{Ty} \AgdaBound{Γ}\AgdaSymbol{\}} \AgdaSymbol{→} \AgdaBound{Γ} \AgdaFunction{\&} \AgdaBound{A} \AgdaRecord{⇉} \AgdaBound{Γ}\<%
\\
\>\AgdaFunction{fst\&} \AgdaSymbol{=} \AgdaKeyword{record} \<[14]%
\>[14]\<%
\\
\>[6]\AgdaIndent{9}{}\<[9]%
\>[9]\AgdaSymbol{\{} \AgdaField{fn} \AgdaSymbol{=} \AgdaFunction{proj₁}\<%
\\
\>[6]\AgdaIndent{9}{}\<[9]%
\>[9]\AgdaSymbol{;} \AgdaField{resp} \AgdaSymbol{=} \AgdaFunction{proj₁}\<%
\\
\>[6]\AgdaIndent{9}{}\<[9]%
\>[9]\AgdaSymbol{\}}\<%
\\
\>\<\end{code}

Pairing operation

\begin{code}\>\<%
\\
\>\AgdaFunction{\_,,\_} \AgdaSymbol{:} \AgdaSymbol{\{}\AgdaBound{Γ} \AgdaBound{Δ} \AgdaSymbol{:} \AgdaFunction{Con}\AgdaSymbol{\}\{}\AgdaBound{A} \AgdaSymbol{:} \AgdaRecord{Ty} \AgdaBound{Δ}\AgdaSymbol{\}(}\AgdaBound{f} \AgdaSymbol{:} \AgdaBound{Γ} \AgdaRecord{⇉} \AgdaBound{Δ}\AgdaSymbol{)} \<[40]%
\>[40]\<%
\\
\>[-2]\AgdaIndent{5}{}\<[5]%
\>[5]\AgdaSymbol{→} \AgdaSymbol{(}\AgdaRecord{Tm} \AgdaSymbol{(}\AgdaBound{A} \AgdaFunction{[} \AgdaBound{f} \AgdaFunction{]T}\AgdaSymbol{))} \AgdaSymbol{→} \AgdaBound{Γ} \AgdaRecord{⇉} \AgdaSymbol{(}\AgdaBound{Δ} \AgdaFunction{\&} \AgdaBound{A}\AgdaSymbol{)}\<%
\\
\>\AgdaBound{f} \AgdaFunction{,,} \AgdaBound{t} \AgdaSymbol{=} \AgdaKeyword{record} \<[16]%
\>[16]\<%
\\
\>[0]\AgdaIndent{9}{}\<[9]%
\>[9]\AgdaSymbol{\{} \AgdaField{fn} \AgdaSymbol{=} \AgdaFunction{<} \AgdaFunction{[} \AgdaBound{f} \AgdaFunction{]fn} \AgdaFunction{,} \AgdaFunction{[} \AgdaBound{t} \AgdaFunction{]tm} \AgdaFunction{>}\<%
\\
\>[0]\AgdaIndent{9}{}\<[9]%
\>[9]\AgdaSymbol{;} \AgdaField{resp} \AgdaSymbol{=} \AgdaFunction{<} \AgdaFunction{[} \AgdaBound{f} \AgdaFunction{]resp} \AgdaFunction{,} \AgdaFunction{[} \AgdaBound{t} \AgdaFunction{]respt} \AgdaFunction{>}\<%
\\
\>[0]\AgdaIndent{9}{}\<[9]%
\>[9]\AgdaSymbol{\}}\<%
\\
\>\<\end{code}

Projections

\begin{code}\>\<%
\\
\>\AgdaFunction{fst} \AgdaSymbol{:} \AgdaSymbol{\{}\AgdaBound{Γ} \AgdaBound{Δ} \AgdaSymbol{:} \AgdaFunction{Con}\AgdaSymbol{\}\{}\AgdaBound{A} \AgdaSymbol{:} \AgdaRecord{Ty} \AgdaBound{Δ}\AgdaSymbol{\}} \AgdaSymbol{→} \AgdaBound{Γ} \AgdaRecord{⇉} \AgdaSymbol{(}\AgdaBound{Δ} \AgdaFunction{\&} \AgdaBound{A}\AgdaSymbol{)} \AgdaSymbol{→} \AgdaBound{Γ} \AgdaRecord{⇉} \AgdaBound{Δ}\<%
\\
\>\AgdaFunction{fst} \AgdaBound{f} \AgdaSymbol{=} \AgdaKeyword{record} \<[15]%
\>[15]\<%
\\
\>[0]\AgdaIndent{8}{}\<[8]%
\>[8]\AgdaSymbol{\{} \AgdaField{fn} \AgdaSymbol{=} \AgdaFunction{proj₁} \AgdaFunction{∘} \AgdaFunction{[} \AgdaBound{f} \AgdaFunction{]fn}\<%
\\
\>[0]\AgdaIndent{8}{}\<[8]%
\>[8]\AgdaSymbol{;} \AgdaField{resp} \AgdaSymbol{=} \AgdaFunction{proj₁} \AgdaFunction{∘} \AgdaFunction{[} \AgdaBound{f} \AgdaFunction{]resp} \<[35]%
\>[35]\<%
\\
\>[0]\AgdaIndent{8}{}\<[8]%
\>[8]\AgdaSymbol{\}}\<%
\\
%
\\
\>\AgdaFunction{snd} \AgdaSymbol{:} \AgdaSymbol{\{}\AgdaBound{Γ} \AgdaBound{Δ} \AgdaSymbol{:} \AgdaFunction{Con}\AgdaSymbol{\}\{}\AgdaBound{A} \AgdaSymbol{:} \AgdaRecord{Ty} \AgdaBound{Δ}\AgdaSymbol{\}} \AgdaSymbol{→} \AgdaSymbol{(}\AgdaBound{f} \AgdaSymbol{:} \AgdaBound{Γ} \AgdaRecord{⇉} \AgdaSymbol{(}\AgdaBound{Δ} \AgdaFunction{\&} \AgdaBound{A}\AgdaSymbol{))}\<%
\\
\>[0]\AgdaIndent{4}{}\<[4]%
\>[4]\AgdaSymbol{→} \AgdaRecord{Tm} \AgdaSymbol{(}\AgdaBound{A} \AgdaFunction{[} \AgdaFunction{fst} \AgdaSymbol{\{}A \AgdaSymbol{=} \AgdaBound{A}\AgdaSymbol{\}} \AgdaBound{f} \AgdaFunction{]T}\AgdaSymbol{)}\<%
\\
\>\AgdaFunction{snd} \AgdaBound{f} \AgdaSymbol{=} \AgdaKeyword{record} \<[15]%
\>[15]\<%
\\
\>[0]\AgdaIndent{8}{}\<[8]%
\>[8]\AgdaSymbol{\{} \AgdaField{tm} \AgdaSymbol{=} \AgdaFunction{proj₂} \AgdaFunction{∘} \AgdaFunction{[} \AgdaBound{f} \AgdaFunction{]fn}\<%
\\
\>[0]\AgdaIndent{8}{}\<[8]%
\>[8]\AgdaSymbol{;} \AgdaField{respt} \AgdaSymbol{=} \AgdaFunction{proj₂} \AgdaFunction{∘} \AgdaFunction{[} \AgdaBound{f} \AgdaFunction{]resp} \<[36]%
\>[36]\<%
\\
\>[0]\AgdaIndent{8}{}\<[8]%
\>[8]\AgdaSymbol{\}}\<%
\\
%
\\
\>\AgdaFunction{\_\textasciicircum\_} \AgdaSymbol{:} \AgdaSymbol{\{}\AgdaBound{Γ} \AgdaBound{Δ} \AgdaSymbol{:} \AgdaFunction{Con}\AgdaSymbol{\}(}\AgdaBound{f} \AgdaSymbol{:} \AgdaBound{Γ} \AgdaRecord{⇉} \AgdaBound{Δ}\AgdaSymbol{)(}\AgdaBound{A} \AgdaSymbol{:} \AgdaRecord{Ty} \AgdaBound{Δ}\AgdaSymbol{)} \<[39]%
\>[39]\<%
\\
\>[0]\AgdaIndent{4}{}\<[4]%
\>[4]\AgdaSymbol{→} \AgdaBound{Γ} \AgdaFunction{\&} \AgdaBound{A} \AgdaFunction{[} \AgdaBound{f} \AgdaFunction{]T} \AgdaRecord{⇉} \AgdaBound{Δ} \AgdaFunction{\&} \AgdaBound{A}\<%
\\
\>\AgdaBound{f} \AgdaFunction{\textasciicircum} \AgdaBound{A} \AgdaSymbol{=} \AgdaKeyword{record} \<[15]%
\>[15]\<%
\\
\>[0]\AgdaIndent{8}{}\<[8]%
\>[8]\AgdaSymbol{\{} \AgdaField{fn} \AgdaSymbol{=} \AgdaFunction{<} \AgdaFunction{[} \AgdaBound{f} \AgdaFunction{]fn} \AgdaFunction{∘} \AgdaFunction{proj₁} \AgdaFunction{,} \AgdaFunction{proj₂} \AgdaFunction{>}\<%
\\
\>[0]\AgdaIndent{8}{}\<[8]%
\>[8]\AgdaSymbol{;} \AgdaField{resp} \AgdaSymbol{=} \AgdaFunction{<} \AgdaFunction{[} \AgdaBound{f} \AgdaFunction{]resp} \AgdaFunction{∘} \AgdaFunction{proj₁} \AgdaFunction{,} \AgdaFunction{proj₂} \AgdaFunction{>}\<%
\\
\>[0]\AgdaIndent{8}{}\<[8]%
\>[8]\AgdaSymbol{\}}\<%
\\
\>\<\end{code}

$\Pi$-types (object level)

\begin{code}\>\<%
\\
\>\AgdaFunction{Π} \AgdaSymbol{:} \AgdaSymbol{\{}\AgdaBound{Γ} \AgdaSymbol{:} \AgdaRecord{Setoid}\AgdaSymbol{\}(}\AgdaBound{A} \AgdaSymbol{:} \AgdaRecord{Ty} \AgdaBound{Γ}\AgdaSymbol{)(}\AgdaBound{B} \AgdaSymbol{:} \AgdaRecord{Ty} \AgdaSymbol{(}\AgdaBound{Γ} \AgdaFunction{\&} \AgdaBound{A}\AgdaSymbol{))} \AgdaSymbol{→} \AgdaRecord{Ty} \AgdaBound{Γ}\<%
\\
\>\AgdaFunction{Π} \AgdaSymbol{\{}\AgdaBound{Γ}\AgdaSymbol{\}} \AgdaBound{A} \AgdaBound{B} \AgdaSymbol{=} \AgdaKeyword{record} \<[19]%
\>[19]\<%
\\
\>[0]\AgdaIndent{2}{}\<[2]%
\>[2]\AgdaSymbol{\{} \AgdaField{fm} \AgdaSymbol{=} \AgdaSymbol{λ} \AgdaBound{x} \AgdaSymbol{→} \AgdaKeyword{let} \AgdaBound{Ax} \AgdaSymbol{=} \AgdaFunction{[} \AgdaBound{A} \AgdaFunction{]fm} \AgdaBound{x} \AgdaKeyword{in}\<%
\\
\>[0]\AgdaIndent{15}{}\<[15]%
\>[15]\AgdaKeyword{let} \AgdaBound{Bx} \AgdaSymbol{=} \AgdaSymbol{λ} \AgdaBound{a} \AgdaSymbol{→} \AgdaFunction{[} \AgdaBound{B} \AgdaFunction{]fm} \AgdaSymbol{(}\AgdaBound{x} \AgdaInductiveConstructor{,} \AgdaBound{a}\AgdaSymbol{)} \AgdaKeyword{in}\<%
\\
\>[0]\AgdaIndent{9}{}\<[9]%
\>[9]\AgdaKeyword{record}\<%
\\
\>[0]\AgdaIndent{9}{}\<[9]%
\>[9]\AgdaSymbol{\{} \AgdaField{Carrier} \AgdaSymbol{=} \AgdaRecord{Subset} \AgdaSymbol{((}\AgdaBound{a} \AgdaSymbol{:} \AgdaFunction{∣} \AgdaBound{Ax} \AgdaFunction{∣}\AgdaSymbol{)} \AgdaSymbol{→} \AgdaFunction{∣} \AgdaBound{Bx} \AgdaBound{a} \AgdaFunction{∣}\AgdaSymbol{)} \AgdaSymbol{(λ} \AgdaBound{fn} \AgdaSymbol{→} \<[62]%
\>[62]\<%
\\
\>[9]\AgdaIndent{21}{}\<[21]%
\>[21]\AgdaSymbol{(}\AgdaBound{a} \AgdaBound{b} \AgdaSymbol{:} \AgdaFunction{∣} \AgdaBound{Ax} \AgdaFunction{∣}\AgdaSymbol{)}\<%
\\
\>[9]\AgdaIndent{21}{}\<[21]%
\>[21]\AgdaSymbol{(}\AgdaBound{p} \AgdaSymbol{:} \AgdaFunction{[} \AgdaBound{Ax} \AgdaFunction{]} \AgdaBound{a} \AgdaFunction{≈} \AgdaBound{b}\AgdaSymbol{)} \AgdaSymbol{→}\<%
\\
\>[9]\AgdaIndent{21}{}\<[21]%
\>[21]\AgdaFunction{[} \AgdaBound{Bx} \AgdaBound{b} \AgdaFunction{]} \AgdaFunction{[} \AgdaBound{B} \AgdaFunction{]subst} \AgdaSymbol{(}\AgdaFunction{[} \AgdaBound{Γ} \AgdaFunction{]refl} \AgdaInductiveConstructor{,} \<[54]%
\>[54]\<%
\\
\>[9]\AgdaIndent{21}{}\<[21]%
\>[21]\AgdaFunction{[} \AgdaBound{Ax} \AgdaFunction{]trans} \AgdaFunction{[} \AgdaBound{A} \AgdaFunction{]refl*} \AgdaBound{p}\AgdaSymbol{)} \AgdaSymbol{(}\AgdaBound{fn} \AgdaBound{a}\AgdaSymbol{)} \AgdaFunction{≈} \AgdaBound{fn} \AgdaBound{b}\AgdaSymbol{)}\<%
\\
%
\\
%
\\
\>[0]\AgdaIndent{9}{}\<[9]%
\>[9]\AgdaSymbol{;} \AgdaField{\_≈\_} \<[18]%
\>[18]\AgdaSymbol{=} \AgdaSymbol{λ\{(}\AgdaBound{f} \AgdaInductiveConstructor{,} \AgdaSymbol{\_)} \AgdaSymbol{(}\AgdaBound{g} \AgdaInductiveConstructor{,} \AgdaSymbol{\_)} \AgdaSymbol{→} \AgdaSymbol{∀} \AgdaBound{a} \AgdaSymbol{→} \AgdaFunction{[} \AgdaBound{Bx} \AgdaBound{a} \AgdaFunction{]} \AgdaBound{f} \AgdaBound{a} \AgdaFunction{≈} \AgdaBound{g} \AgdaBound{a} \AgdaSymbol{\}}\<%
\\
\>[0]\AgdaIndent{9}{}\<[9]%
\>[9]\AgdaSymbol{;} \AgdaField{refl} \<[19]%
\>[19]\AgdaSymbol{=} \AgdaSymbol{λ} \AgdaBound{a} \AgdaSymbol{→} \AgdaFunction{[} \AgdaBound{Bx} \AgdaSymbol{\_} \AgdaFunction{]refl} \<[40]%
\>[40]\<%
\\
\>[0]\AgdaIndent{9}{}\<[9]%
\>[9]\AgdaSymbol{;} \AgdaField{sym} \<[19]%
\>[19]\AgdaSymbol{=} \AgdaSymbol{λ} \AgdaBound{f} \AgdaBound{a} \AgdaSymbol{→} \AgdaFunction{[} \AgdaBound{Bx} \AgdaSymbol{\_} \AgdaFunction{]sym} \AgdaSymbol{(}\AgdaBound{f} \AgdaBound{a}\AgdaSymbol{)}\<%
\\
\>[0]\AgdaIndent{9}{}\<[9]%
\>[9]\AgdaSymbol{;} \AgdaField{trans} \<[19]%
\>[19]\AgdaSymbol{=} \AgdaSymbol{λ} \AgdaBound{f} \AgdaBound{g} \AgdaBound{a} \AgdaSymbol{→} \AgdaFunction{[} \AgdaBound{Bx} \AgdaSymbol{\_} \AgdaFunction{]trans} \AgdaSymbol{(}\AgdaBound{f} \AgdaBound{a}\AgdaSymbol{)} \AgdaSymbol{(}\AgdaBound{g} \AgdaBound{a}\AgdaSymbol{)}\<%
\\
\>[0]\AgdaIndent{9}{}\<[9]%
\>[9]\AgdaSymbol{\}}\<%
\\
%
\\
\>[0]\AgdaIndent{2}{}\<[2]%
\>[2]\AgdaSymbol{;} \AgdaField{substT} \AgdaSymbol{=} \AgdaSymbol{λ} \AgdaSymbol{\{}\AgdaBound{x}\AgdaSymbol{\}} \AgdaSymbol{\{}\AgdaBound{y}\AgdaSymbol{\}} \AgdaBound{p} \AgdaSymbol{→} \AgdaSymbol{λ} \AgdaSymbol{\{(}\AgdaBound{f} \AgdaInductiveConstructor{,} \AgdaBound{rsp}\AgdaSymbol{)} \AgdaSymbol{→}\<%
\\
\>[0]\AgdaIndent{19}{}\<[19]%
\>[19]\AgdaKeyword{let} \AgdaBound{y2x} \AgdaSymbol{=} \AgdaSymbol{λ} \AgdaBound{a} \AgdaSymbol{→} \AgdaFunction{[} \AgdaBound{A} \AgdaFunction{]subst} \AgdaSymbol{(}\AgdaFunction{[} \AgdaBound{Γ} \AgdaFunction{]sym} \AgdaBound{p}\AgdaSymbol{)} \AgdaBound{a} \AgdaKeyword{in}\<%
\\
\>[0]\AgdaIndent{19}{}\<[19]%
\>[19]\AgdaKeyword{let} \AgdaBound{x2y} \AgdaSymbol{=} \AgdaSymbol{λ} \AgdaBound{a} \AgdaSymbol{→} \AgdaFunction{[} \AgdaBound{A} \AgdaFunction{]subst} \AgdaBound{p} \AgdaBound{a} \AgdaKeyword{in}\<%
\\
\>[0]\AgdaIndent{13}{}\<[13]%
\>[13]\AgdaSymbol{(λ} \AgdaBound{a} \AgdaSymbol{→} \AgdaFunction{[} \AgdaBound{B} \AgdaFunction{]subst} \AgdaSymbol{(}\AgdaBound{p} \AgdaInductiveConstructor{,} \AgdaFunction{[} \AgdaBound{A} \AgdaFunction{]tr*}\AgdaSymbol{)} \<[46]%
\>[46]\<%
\\
\>[0]\AgdaIndent{13}{}\<[13]%
\>[13]\AgdaSymbol{(}\AgdaBound{f} \AgdaSymbol{(}\AgdaBound{y2x} \AgdaBound{a}\AgdaSymbol{)))} \AgdaInductiveConstructor{,} \<[28]%
\>[28]\<%
\\
\>[0]\AgdaIndent{13}{}\<[13]%
\>[13]\AgdaSymbol{(λ} \AgdaBound{a} \AgdaBound{b} \AgdaBound{q} \AgdaSymbol{→}\<%
\\
\>[13]\AgdaIndent{16}{}\<[16]%
\>[16]\AgdaKeyword{let} \AgdaBound{a'} \AgdaSymbol{=} \AgdaBound{y2x} \AgdaBound{a} \AgdaKeyword{in} \<[34]%
\>[34]\<%
\\
\>[13]\AgdaIndent{16}{}\<[16]%
\>[16]\AgdaKeyword{let} \AgdaBound{b'} \AgdaSymbol{=} \AgdaBound{y2x} \AgdaBound{b} \AgdaKeyword{in}\<%
\\
\>[13]\AgdaIndent{16}{}\<[16]%
\>[16]\AgdaKeyword{let} \AgdaBound{q'} \AgdaSymbol{=} \AgdaFunction{[} \AgdaBound{A} \AgdaFunction{]subst*} \AgdaSymbol{(}\AgdaFunction{[} \AgdaBound{Γ} \AgdaFunction{]sym} \AgdaBound{p}\AgdaSymbol{)} \AgdaBound{q} \AgdaKeyword{in}\<%
\\
\>[13]\AgdaIndent{16}{}\<[16]%
\>[16]\AgdaKeyword{let} \AgdaBound{H} \AgdaSymbol{=} \AgdaBound{rsp} \AgdaBound{a'} \AgdaBound{b'} \AgdaSymbol{(}\AgdaFunction{[} \AgdaBound{A} \AgdaFunction{]subst*} \AgdaSymbol{(}\AgdaFunction{[} \AgdaBound{Γ} \AgdaFunction{]sym} \AgdaBound{p}\AgdaSymbol{)} \AgdaBound{q}\AgdaSymbol{)} \AgdaKeyword{in}\<%
\\
\>[13]\AgdaIndent{16}{}\<[16]%
\>[16]\AgdaKeyword{let} \AgdaBound{r} \AgdaSymbol{:} \AgdaFunction{[} \AgdaBound{Γ} \AgdaFunction{\&} \AgdaBound{A} \AgdaFunction{]} \AgdaSymbol{(}\AgdaBound{x} \AgdaInductiveConstructor{,} \AgdaBound{b'}\AgdaSymbol{)} \AgdaFunction{≈} \AgdaSymbol{(}\AgdaBound{y} \AgdaInductiveConstructor{,} \AgdaBound{b}\AgdaSymbol{)}
                    r \AgdaSymbol{=} \AgdaSymbol{(}\AgdaBound{p} \AgdaInductiveConstructor{,} \AgdaFunction{[} \AgdaBound{A} \AgdaFunction{]tr*}\AgdaSymbol{)} \AgdaKeyword{in}\<%
\\
\>[13]\AgdaIndent{16}{}\<[16]%
\>[16]\AgdaKeyword{let} \AgdaBound{pre} \AgdaSymbol{=} \AgdaFunction{[} \AgdaBound{B} \AgdaFunction{]subst*} \AgdaBound{r} \<[40]%
\>[40]\<%
\\
\>[16]\AgdaIndent{24}{}\<[24]%
\>[24]\AgdaSymbol{(}\AgdaBound{rsp} \AgdaBound{a'} \AgdaBound{b'} \AgdaSymbol{(}\AgdaFunction{[} \AgdaBound{A} \AgdaFunction{]subst*} \AgdaSymbol{(}\AgdaFunction{[} \AgdaBound{Γ} \AgdaFunction{]sym} \AgdaBound{p}\AgdaSymbol{)} \AgdaBound{q}\AgdaSymbol{))} \AgdaKeyword{in} \<[68]%
\>[68]\<%
\\
\>[0]\AgdaIndent{16}{}\<[16]%
\>[16]\AgdaFunction{[} \AgdaFunction{[} \AgdaBound{B} \AgdaFunction{]fm} \AgdaSymbol{(}\AgdaBound{y} \AgdaInductiveConstructor{,} \AgdaBound{b}\AgdaSymbol{)} \AgdaFunction{]trans} \<[41]%
\>[41]\<%
\\
\>[0]\AgdaIndent{16}{}\<[16]%
\>[16]\AgdaSymbol{(}\AgdaFunction{[} \AgdaBound{B} \AgdaFunction{]trans*} \AgdaSymbol{\_)} \<[32]%
\>[32]\<%
\\
\>[0]\AgdaIndent{16}{}\<[16]%
\>[16]\AgdaSymbol{(}\AgdaFunction{[} \AgdaFunction{[} \AgdaBound{B} \AgdaFunction{]fm} \AgdaSymbol{(}\AgdaBound{y} \AgdaInductiveConstructor{,} \AgdaBound{b}\AgdaSymbol{)} \AgdaFunction{]trans} \<[42]%
\>[42]\<%
\\
\>[0]\AgdaIndent{16}{}\<[16]%
\>[16]\AgdaSymbol{(}\AgdaFunction{[} \AgdaFunction{[} \AgdaBound{B} \AgdaFunction{]fm} \AgdaSymbol{(}\AgdaBound{y} \AgdaInductiveConstructor{,} \AgdaBound{b}\AgdaSymbol{)} \AgdaFunction{]sym} \AgdaSymbol{(}\AgdaFunction{[} \AgdaBound{B} \AgdaFunction{]trans*} \AgdaSymbol{\_))} \<[57]%
\>[57]\<%
\\
\>[0]\AgdaIndent{16}{}\<[16]%
\>[16]\AgdaBound{pre}\AgdaSymbol{))\}}\<%
\\
%
\\
%
\\
\>[0]\AgdaIndent{2}{}\<[2]%
\>[2]\AgdaSymbol{;} \AgdaField{subst*} \AgdaSymbol{=} \AgdaSymbol{λ} \AgdaBound{\_} \AgdaBound{q} \AgdaBound{\_} \AgdaSymbol{→} \AgdaFunction{[} \AgdaBound{B} \AgdaFunction{]subst*} \AgdaSymbol{\_} \AgdaSymbol{(}\AgdaBound{q} \AgdaSymbol{\_)}\<%
\\
\>[0]\AgdaIndent{2}{}\<[2]%
\>[2]\AgdaSymbol{;} \AgdaField{refl*} \AgdaSymbol{=} \AgdaSymbol{λ} \AgdaSymbol{\{}\AgdaBound{x}\AgdaSymbol{\}} \AgdaSymbol{\{}\AgdaBound{a}\AgdaSymbol{\}} \AgdaBound{ax} \<[25]%
\>[25]\<%
\\
\>[2]\AgdaIndent{10}{}\<[10]%
\>[10]\AgdaSymbol{→} \AgdaKeyword{let} \AgdaBound{rsp} \AgdaSymbol{=} \AgdaFunction{prj₂} \AgdaBound{a} \AgdaKeyword{in} \AgdaSymbol{(}\AgdaBound{rsp} \AgdaSymbol{\_} \AgdaSymbol{\_} \AgdaFunction{[} \AgdaBound{A} \AgdaFunction{]refl*}\AgdaSymbol{)}\<%
\\
\>[0]\AgdaIndent{2}{}\<[2]%
\>[2]\AgdaSymbol{;} \AgdaField{trans*} \AgdaSymbol{=} \<[14]%
\>[14]\AgdaSymbol{λ} \AgdaSymbol{\{(}\AgdaBound{f} \AgdaInductiveConstructor{,} \AgdaBound{rsp}\AgdaSymbol{)} \AgdaBound{a} \AgdaSymbol{→}\<%
\\
\>[0]\AgdaIndent{13}{}\<[13]%
\>[13]\AgdaFunction{[} \AgdaFunction{[} \AgdaBound{B} \AgdaFunction{]fm} \AgdaSymbol{\_} \AgdaFunction{]trans} \<[32]%
\>[32]\<%
\\
\>[0]\AgdaIndent{13}{}\<[13]%
\>[13]\AgdaSymbol{(}\AgdaFunction{[} \AgdaFunction{[} \AgdaBound{B} \AgdaFunction{]fm} \AgdaSymbol{\_} \AgdaFunction{]trans} \<[33]%
\>[33]\<%
\\
\>[0]\AgdaIndent{13}{}\<[13]%
\>[13]\AgdaSymbol{(}\AgdaFunction{[} \AgdaBound{B} \AgdaFunction{]trans*} \AgdaSymbol{\_)} \<[29]%
\>[29]\<%
\\
\>[0]\AgdaIndent{13}{}\<[13]%
\>[13]\AgdaSymbol{(}\AgdaFunction{[} \AgdaFunction{[} \AgdaBound{B} \AgdaFunction{]fm} \AgdaSymbol{\_} \AgdaFunction{]sym} \AgdaSymbol{(}\AgdaFunction{[} \AgdaBound{B} \AgdaFunction{]trans*} \AgdaSymbol{\_)))}\<%
\\
\>[0]\AgdaIndent{13}{}\<[13]%
\>[13]\AgdaSymbol{(}\AgdaFunction{[} \AgdaBound{B} \AgdaFunction{]subst*} \AgdaSymbol{\_} \AgdaSymbol{(}\AgdaBound{rsp} \AgdaSymbol{\_} \AgdaSymbol{\_} \AgdaSymbol{(}\AgdaFunction{[} \AgdaBound{A} \AgdaFunction{]trans*} \AgdaSymbol{\_)} \AgdaSymbol{))} \AgdaSymbol{\}}\<%
\\
\>[0]\AgdaIndent{2}{}\<[2]%
\>[2]\AgdaSymbol{\}}\<%
\\
%
\\
%
\\
\>\AgdaFunction{lam} \AgdaSymbol{:} \AgdaSymbol{\{}\AgdaBound{Γ} \AgdaSymbol{:} \AgdaFunction{Con}\AgdaSymbol{\}\{}\AgdaBound{A} \AgdaSymbol{:} \AgdaRecord{Ty} \AgdaBound{Γ}\AgdaSymbol{\}\{}\AgdaBound{B} \AgdaSymbol{:} \AgdaRecord{Ty} \AgdaSymbol{(}\AgdaBound{Γ} \AgdaFunction{\&} \AgdaBound{A}\AgdaSymbol{)\}} \AgdaSymbol{→} \AgdaRecord{Tm} \AgdaBound{B} \AgdaSymbol{→} \AgdaRecord{Tm} \AgdaSymbol{(}\AgdaFunction{Π} \AgdaBound{A} \AgdaBound{B}\AgdaSymbol{)}\<%
\\
\>\AgdaFunction{lam} \AgdaSymbol{\{}\AgdaBound{Γ}\AgdaSymbol{\}} \AgdaSymbol{\{}\AgdaBound{A}\AgdaSymbol{\}} \AgdaSymbol{(}\AgdaInductiveConstructor{tm:} \AgdaBound{tm} \AgdaInductiveConstructor{resp:} \AgdaBound{respt}\AgdaSymbol{)} \AgdaSymbol{=} \<[35]%
\>[35]\<%
\\
\>[0]\AgdaIndent{2}{}\<[2]%
\>[2]\AgdaKeyword{record} \AgdaSymbol{\{} \AgdaField{tm} \AgdaSymbol{=} \AgdaSymbol{λ} \AgdaBound{x} \AgdaSymbol{→} \AgdaSymbol{(λ} \AgdaBound{a} \AgdaSymbol{→} \AgdaBound{tm} \AgdaSymbol{(}\AgdaBound{x} \AgdaInductiveConstructor{,} \AgdaBound{a}\AgdaSymbol{))} \<[41]%
\>[41]\<%
\\
\>[2]\AgdaIndent{16}{}\<[16]%
\>[16]\AgdaInductiveConstructor{,} \AgdaSymbol{(λ} \AgdaBound{a} \AgdaBound{b} \AgdaBound{p} \AgdaSymbol{→} \AgdaBound{respt} \AgdaSymbol{(}\AgdaFunction{[} \AgdaBound{Γ} \AgdaFunction{]refl} \AgdaInductiveConstructor{,} \<[48]%
\>[48]\<%
\\
\>[2]\AgdaIndent{16}{}\<[16]%
\>[16]\AgdaFunction{[} \AgdaFunction{[} \AgdaBound{A} \AgdaFunction{]fm} \AgdaBound{x} \AgdaFunction{]trans} \AgdaFunction{[} \AgdaBound{A} \AgdaFunction{]refl*} \AgdaBound{p}\AgdaSymbol{))}\<%
\\
\>[5]\AgdaIndent{9}{}\<[9]%
\>[9]\AgdaSymbol{;} \AgdaField{respt} \AgdaSymbol{=} \AgdaSymbol{λ} \AgdaBound{p} \AgdaBound{\_} \AgdaSymbol{→} \AgdaBound{respt} \AgdaSymbol{(}\AgdaBound{p} \AgdaInductiveConstructor{,} \AgdaFunction{[} \AgdaBound{A} \AgdaFunction{]tr*}\AgdaSymbol{)}\<%
\\
\>[0]\AgdaIndent{9}{}\<[9]%
\>[9]\AgdaSymbol{\}}\<%
\\
%
\\
%
\\
\>\AgdaFunction{app} \AgdaSymbol{:} \AgdaSymbol{\{}\AgdaBound{Γ} \AgdaSymbol{:} \AgdaFunction{Con}\AgdaSymbol{\}\{}\AgdaBound{A} \AgdaSymbol{:} \AgdaRecord{Ty} \AgdaBound{Γ}\AgdaSymbol{\}\{}\AgdaBound{B} \AgdaSymbol{:} \AgdaRecord{Ty} \AgdaSymbol{(}\AgdaBound{Γ} \AgdaFunction{\&} \AgdaBound{A}\AgdaSymbol{)\}} \AgdaSymbol{→} \AgdaRecord{Tm} \AgdaSymbol{(}\AgdaFunction{Π} \AgdaBound{A} \AgdaBound{B}\AgdaSymbol{)} \AgdaSymbol{→} \AgdaRecord{Tm} \AgdaBound{B}\<%
\\
\>\AgdaFunction{app} \AgdaSymbol{\{}\AgdaBound{Γ}\AgdaSymbol{\}} \AgdaSymbol{\{}\AgdaBound{A}\AgdaSymbol{\}} \AgdaSymbol{\{}\AgdaBound{B}\AgdaSymbol{\}} \AgdaSymbol{(}\AgdaInductiveConstructor{tm:} \AgdaBound{tm} \AgdaInductiveConstructor{resp:} \AgdaBound{respt}\AgdaSymbol{)} \AgdaSymbol{=} \<[39]%
\>[39]\<%
\\
\>[0]\AgdaIndent{2}{}\<[2]%
\>[2]\AgdaKeyword{record} \AgdaSymbol{\{} \AgdaField{tm} \AgdaSymbol{=} \AgdaSymbol{λ} \AgdaSymbol{\{(}\AgdaBound{x} \AgdaInductiveConstructor{,} \AgdaBound{a}\AgdaSymbol{)} \AgdaSymbol{→} \AgdaFunction{prj₁} \AgdaSymbol{(}\AgdaBound{tm} \AgdaBound{x}\AgdaSymbol{)} \AgdaBound{a}\AgdaSymbol{\}}\<%
\\
\>[0]\AgdaIndent{9}{}\<[9]%
\>[9]\AgdaSymbol{;} \AgdaField{respt} \AgdaSymbol{=} \AgdaSymbol{λ} \AgdaSymbol{\{}\AgdaBound{x}\AgdaSymbol{\}} \AgdaSymbol{\{}\AgdaBound{y}\AgdaSymbol{\}} \AgdaSymbol{→} \AgdaSymbol{λ} \AgdaSymbol{\{(}\AgdaBound{p} \AgdaInductiveConstructor{,} \AgdaBound{tr}\AgdaSymbol{)} \AgdaSymbol{→} \<[45]%
\>[45]\<%
\\
\>[9]\AgdaIndent{17}{}\<[17]%
\>[17]\AgdaKeyword{let} \AgdaBound{fresp} \AgdaSymbol{=} \AgdaFunction{prj₂} \AgdaSymbol{(}\AgdaBound{tm} \AgdaSymbol{(}\AgdaFunction{proj₁} \AgdaBound{x}\AgdaSymbol{))} \AgdaKeyword{in}\<%
\\
\>[9]\AgdaIndent{17}{}\<[17]%
\>[17]\AgdaFunction{[} \AgdaFunction{[} \AgdaBound{B} \AgdaFunction{]fm} \AgdaSymbol{\_} \AgdaFunction{]trans} \<[36]%
\>[36]\<%
\\
\>[9]\AgdaIndent{17}{}\<[17]%
\>[17]\AgdaSymbol{(}\AgdaFunction{[} \AgdaBound{B} \AgdaFunction{]subst*} \AgdaSymbol{(}\AgdaBound{p} \AgdaInductiveConstructor{,} \AgdaBound{tr}\AgdaSymbol{)} \<[39]%
\>[39]\<%
\\
\>[9]\AgdaIndent{17}{}\<[17]%
\>[17]\AgdaSymbol{(}\AgdaFunction{[} \AgdaFunction{[} \AgdaBound{B} \AgdaFunction{]fm} \AgdaSymbol{\_} \AgdaFunction{]sym} \AgdaFunction{[} \AgdaBound{B} \AgdaFunction{]refl*}\AgdaSymbol{))}\<%
\\
\>[9]\AgdaIndent{17}{}\<[17]%
\>[17]\AgdaSymbol{(}\AgdaFunction{[} \AgdaFunction{[} \AgdaBound{B} \AgdaFunction{]fm} \AgdaSymbol{\_} \AgdaFunction{]trans}\<%
\\
\>[9]\AgdaIndent{17}{}\<[17]%
\>[17]\AgdaSymbol{(}\AgdaFunction{[} \AgdaBound{B} \AgdaFunction{]trans*} \AgdaSymbol{\{}p \AgdaSymbol{=} \AgdaSymbol{(}\AgdaFunction{[} \AgdaBound{Γ} \AgdaFunction{]refl} \AgdaInductiveConstructor{,} \AgdaFunction{[} \AgdaBound{A} \AgdaFunction{]refl*}\AgdaSymbol{)\}} \AgdaSymbol{\_)}\<%
\\
\>[0]\AgdaIndent{8}{}\<[8]%
\>[8]\<%
\\
\>[0]\AgdaIndent{17}{}\<[17]%
\>[17]\AgdaSymbol{(}\AgdaFunction{[} \AgdaFunction{[} \AgdaBound{B} \AgdaFunction{]fm} \AgdaSymbol{\_} \AgdaFunction{]trans} \<[37]%
\>[37]\<%
\\
\>[0]\AgdaIndent{17}{}\<[17]%
\>[17]\AgdaSymbol{(}\AgdaFunction{[} \AgdaFunction{[} \AgdaBound{B} \AgdaFunction{]fm} \AgdaSymbol{\_} \AgdaFunction{]sym} \<[35]%
\>[35]\<%
\\
\>[0]\AgdaIndent{17}{}\<[17]%
\>[17]\AgdaSymbol{(}\AgdaFunction{[} \AgdaBound{B} \AgdaFunction{]trans*} \AgdaSymbol{\{}q \AgdaSymbol{=} \AgdaSymbol{(}\AgdaBound{p} \AgdaInductiveConstructor{,} \AgdaFunction{[} \AgdaBound{A} \AgdaFunction{]tr*}\AgdaSymbol{)\}} \AgdaSymbol{\_))}\<%
\\
\>[0]\AgdaIndent{17}{}\<[17]%
\>[17]\AgdaSymbol{(}\AgdaFunction{[} \AgdaFunction{[} \AgdaBound{B} \AgdaFunction{]fm} \AgdaSymbol{\_} \AgdaFunction{]trans} \<[37]%
\>[37]\<%
\\
\>[0]\AgdaIndent{17}{}\<[17]%
\>[17]\AgdaSymbol{(}\AgdaFunction{[} \AgdaBound{B} \AgdaFunction{]subst*} \AgdaSymbol{\_} \AgdaSymbol{(}\AgdaBound{fresp} \AgdaSymbol{\_} \AgdaSymbol{\_} \<[43]%
\>[43]\<%
\\
\>[17]\AgdaIndent{19}{}\<[19]%
\>[19]\AgdaSymbol{(}\AgdaFunction{[} \AgdaFunction{[} \AgdaBound{A} \AgdaFunction{]fm} \AgdaSymbol{\_} \AgdaFunction{]trans} \AgdaSymbol{(}\AgdaFunction{[} \AgdaFunction{[} \AgdaBound{A} \AgdaFunction{]fm} \AgdaSymbol{\_} \AgdaFunction{]sym} \AgdaFunction{[} \AgdaBound{A} \AgdaFunction{]tr*}\AgdaSymbol{)} \<[67]%
\>[67]\<%
\\
\>[17]\AgdaIndent{19}{}\<[19]%
\>[19]\AgdaSymbol{(}\AgdaFunction{[} \AgdaBound{A} \AgdaFunction{]subst*} \AgdaSymbol{(}\AgdaFunction{[} \AgdaBound{Γ} \AgdaFunction{]sym} \AgdaBound{p}\AgdaSymbol{)} \AgdaBound{tr}\AgdaSymbol{))))}\<%
\\
\>[0]\AgdaIndent{17}{}\<[17]%
\>[17]\AgdaSymbol{(}\AgdaBound{respt} \AgdaBound{p} \AgdaSymbol{\_))))} \AgdaSymbol{\}}\<%
\\
\>[0]\AgdaIndent{9}{}\<[9]%
\>[9]\AgdaSymbol{\}}\<%
\\
%
\\
\>\AgdaFunction{\_⇒\_} \AgdaSymbol{:} \AgdaSymbol{\{}\AgdaBound{Γ} \AgdaSymbol{:} \AgdaFunction{Con}\AgdaSymbol{\}(}\AgdaBound{A} \AgdaBound{B} \AgdaSymbol{:} \AgdaRecord{Ty} \AgdaBound{Γ}\AgdaSymbol{)} \AgdaSymbol{→} \AgdaRecord{Ty} \AgdaBound{Γ}\<%
\\
\>\AgdaBound{A} \AgdaFunction{⇒} \AgdaBound{B} \AgdaSymbol{=} \AgdaFunction{Π} \AgdaBound{A} \AgdaSymbol{(}\AgdaBound{B} \AgdaFunction{[} \AgdaFunction{fst\&} \AgdaSymbol{\{}A \AgdaSymbol{=} \AgdaBound{A}\AgdaSymbol{\}} \AgdaFunction{]T}\AgdaSymbol{)}\<%
\\
%
\\
\>\AgdaKeyword{infixr} \AgdaNumber{6} \_⇒\_\<%
\\
\>\<\end{code}

Simpler definition for functions

\begin{code}\>\<%
\\
\>\AgdaFunction{[\_,\_]\_⇒fm\_} \AgdaSymbol{:} \AgdaSymbol{(}\AgdaBound{Γ} \AgdaSymbol{:} \AgdaFunction{Con}\AgdaSymbol{)(}\AgdaBound{x} \AgdaSymbol{:} \AgdaFunction{∣} \AgdaBound{Γ} \AgdaFunction{∣}\AgdaSymbol{)} \<[34]%
\>[34]\<%
\\
\>[9]\AgdaIndent{11}{}\<[11]%
\>[11]\AgdaSymbol{→} \AgdaRecord{Setoid} \AgdaSymbol{→} \AgdaRecord{Setoid} \AgdaSymbol{→} \AgdaRecord{Setoid}\<%
\\
\>\AgdaFunction{[} \AgdaBound{Γ} \AgdaFunction{,} \AgdaBound{x} \AgdaFunction{]} \AgdaBound{Ax} \AgdaFunction{⇒fm} \AgdaBound{Bx} \<[20]%
\>[20]\<%
\\
\>[0]\AgdaIndent{2}{}\<[2]%
\>[2]\AgdaSymbol{=} \AgdaKeyword{record}\<%
\\
\>[0]\AgdaIndent{6}{}\<[6]%
\>[6]\AgdaSymbol{\{} \AgdaField{Carrier} \AgdaSymbol{=} \AgdaRecord{Σ[} \AgdaBound{fn} \AgdaRecord{∶} \AgdaSymbol{(}\AgdaFunction{∣} \AgdaBound{Ax} \AgdaFunction{∣} \AgdaSymbol{→} \AgdaFunction{∣} \AgdaBound{Bx} \AgdaFunction{∣}\AgdaSymbol{)} \AgdaRecord{]} \AgdaSymbol{((}\AgdaBound{a} \AgdaBound{b} \AgdaSymbol{:} \AgdaFunction{∣} \AgdaBound{Ax} \AgdaFunction{∣}\AgdaSymbol{)}\<%
\\
\>[6]\AgdaIndent{18}{}\<[18]%
\>[18]\AgdaSymbol{(}\AgdaBound{p} \AgdaSymbol{:} \AgdaFunction{[} \AgdaBound{Ax} \AgdaFunction{]} \AgdaBound{a} \AgdaFunction{≈} \AgdaBound{b}\AgdaSymbol{)} \AgdaSymbol{→} \AgdaFunction{[} \AgdaBound{Bx} \AgdaFunction{]} \AgdaBound{fn} \AgdaBound{a} \AgdaFunction{≈} \AgdaBound{fn} \AgdaBound{b}\AgdaSymbol{)}\<%
\\
\>[0]\AgdaIndent{6}{}\<[6]%
\>[6]\AgdaSymbol{;} \AgdaField{\_≈\_} \<[15]%
\>[15]\AgdaSymbol{=} \AgdaSymbol{λ\{(}\AgdaBound{f} \AgdaInductiveConstructor{,} \AgdaSymbol{\_)} \AgdaSymbol{(}\AgdaBound{g} \AgdaInductiveConstructor{,} \AgdaSymbol{\_)} \AgdaSymbol{→} \AgdaSymbol{∀} \AgdaBound{a} \AgdaSymbol{→} \AgdaFunction{[} \AgdaBound{Bx} \AgdaFunction{]} \AgdaBound{f} \AgdaBound{a} \AgdaFunction{≈} \AgdaBound{g} \AgdaBound{a} \AgdaSymbol{\}}\<%
\\
\>[0]\AgdaIndent{6}{}\<[6]%
\>[6]\AgdaSymbol{;} \AgdaField{refl} \<[16]%
\>[16]\AgdaSymbol{=} \AgdaSymbol{λ} \AgdaBound{\_} \AgdaSymbol{→} \AgdaFunction{[} \AgdaBound{Bx} \AgdaFunction{]refl} \<[35]%
\>[35]\<%
\\
\>[0]\AgdaIndent{6}{}\<[6]%
\>[6]\AgdaSymbol{;} \AgdaField{sym} \<[16]%
\>[16]\AgdaSymbol{=} \AgdaSymbol{λ} \AgdaBound{f} \AgdaBound{a} \AgdaSymbol{→} \AgdaFunction{[} \AgdaBound{Bx} \AgdaFunction{]sym} \AgdaSymbol{(}\AgdaBound{f} \AgdaBound{a}\AgdaSymbol{)}\<%
\\
\>[0]\AgdaIndent{6}{}\<[6]%
\>[6]\AgdaSymbol{;} \AgdaField{trans} \<[16]%
\>[16]\AgdaSymbol{=} \AgdaSymbol{λ} \AgdaBound{f} \AgdaBound{g} \AgdaBound{a} \AgdaSymbol{→} \AgdaFunction{[} \AgdaBound{Bx} \AgdaFunction{]trans} \AgdaSymbol{(}\AgdaBound{f} \AgdaBound{a}\AgdaSymbol{)} \AgdaSymbol{(}\AgdaBound{g} \AgdaBound{a}\AgdaSymbol{)}\<%
\\
\>[0]\AgdaIndent{6}{}\<[6]%
\>[6]\AgdaSymbol{\}}\<%
\\
\>\<\end{code}

$\Sigma$-types (object level)

\begin{code}\>\<%
\\
%
\\
\>\AgdaFunction{Σ'} \AgdaSymbol{:} \AgdaSymbol{\{}\AgdaBound{Γ} \AgdaSymbol{:} \AgdaFunction{Con}\AgdaSymbol{\}(}\AgdaBound{A} \AgdaSymbol{:} \AgdaRecord{Ty} \AgdaBound{Γ}\AgdaSymbol{)(}\AgdaBound{B} \AgdaSymbol{:} \AgdaRecord{Ty} \AgdaSymbol{(}\AgdaBound{Γ} \AgdaFunction{\&} \AgdaBound{A}\AgdaSymbol{))} \AgdaSymbol{→} \AgdaRecord{Ty} \AgdaBound{Γ}\<%
\\
\>\AgdaFunction{Σ'} \AgdaSymbol{\{}\AgdaBound{Γ}\AgdaSymbol{\}} \AgdaBound{A} \AgdaBound{B} \AgdaSymbol{=} \AgdaKeyword{record} \<[20]%
\>[20]\<%
\\
\>[6]\AgdaIndent{8}{}\<[8]%
\>[8]\AgdaSymbol{\{} \AgdaField{fm} \AgdaSymbol{=} \AgdaSymbol{λ} \AgdaBound{x} \AgdaSymbol{→} \AgdaKeyword{let} \AgdaBound{Ax} \AgdaSymbol{=} \AgdaFunction{[} \AgdaBound{A} \AgdaFunction{]fm} \AgdaBound{x} \AgdaKeyword{in}\<%
\\
\>[8]\AgdaIndent{15}{}\<[15]%
\>[15]\AgdaKeyword{let} \AgdaBound{Bx} \AgdaSymbol{=} \AgdaSymbol{λ} \AgdaBound{a} \AgdaSymbol{→} \AgdaFunction{[} \AgdaBound{B} \AgdaFunction{]fm} \AgdaSymbol{(}\AgdaBound{x} \AgdaInductiveConstructor{,} \AgdaBound{a}\AgdaSymbol{)} \AgdaKeyword{in}\<%
\\
\>[-2]\AgdaIndent{9}{}\<[9]%
\>[9]\AgdaKeyword{record}\<%
\\
\>[0]\AgdaIndent{11}{}\<[11]%
\>[11]\AgdaSymbol{\{} \AgdaField{Carrier} \AgdaSymbol{=} \AgdaRecord{Σ[} \AgdaBound{a} \AgdaRecord{∶} \AgdaFunction{∣} \AgdaBound{Ax} \AgdaFunction{∣} \AgdaRecord{]} \AgdaFunction{∣} \AgdaBound{Bx} \AgdaBound{a} \AgdaFunction{∣}\<%
\\
%
\\
\>[0]\AgdaIndent{11}{}\<[11]%
\>[11]\AgdaSymbol{;} \AgdaField{\_≈\_} \<[20]%
\>[20]\AgdaSymbol{=} \AgdaSymbol{λ\{(}\AgdaBound{a₁} \AgdaInductiveConstructor{,} \AgdaBound{b₁}\AgdaSymbol{)} \AgdaSymbol{(}\AgdaBound{a₂} \AgdaInductiveConstructor{,} \AgdaBound{b₂}\AgdaSymbol{)} \AgdaSymbol{→} \<[46]%
\>[46]\<%
\\
\>[11]\AgdaIndent{22}{}\<[22]%
\>[22]\AgdaRecord{Subset} \AgdaSymbol{(}\AgdaFunction{[} \AgdaBound{Ax} \AgdaFunction{]} \AgdaBound{a₁} \AgdaFunction{≈} \AgdaBound{a₂}\AgdaSymbol{)} \<[46]%
\>[46]\<%
\\
\>[11]\AgdaIndent{22}{}\<[22]%
\>[22]\AgdaSymbol{(λ} \AgdaBound{eq₁} \AgdaSymbol{→} \AgdaFunction{[} \AgdaBound{Bx} \AgdaSymbol{\_} \AgdaFunction{]} \AgdaFunction{[} \AgdaBound{B} \AgdaFunction{]subst} \<[51]%
\>[51]\<%
\\
\>[22]\AgdaIndent{23}{}\<[23]%
\>[23]\AgdaSymbol{(}\AgdaFunction{[} \AgdaBound{Γ} \AgdaFunction{]refl} \AgdaInductiveConstructor{,} \AgdaFunction{[} \AgdaFunction{[} \AgdaBound{A} \AgdaFunction{]fm} \AgdaBound{x} \AgdaFunction{]trans} \<[55]%
\>[55]\<%
\\
\>[22]\AgdaIndent{23}{}\<[23]%
\>[23]\AgdaFunction{[} \AgdaBound{A} \AgdaFunction{]refl*} \AgdaBound{eq₁}\AgdaSymbol{)} \AgdaBound{b₁} \AgdaFunction{≈} \AgdaBound{b₂}\AgdaSymbol{)}\<%
\\
\>[-10]\AgdaIndent{11}{}\<[11]%
\>[11]\AgdaSymbol{\}}\<%
\\
%
\\
\>[0]\AgdaIndent{11}{}\<[11]%
\>[11]\AgdaSymbol{;} \AgdaField{refl} \<[21]%
\>[21]\AgdaSymbol{=} \AgdaSymbol{λ} \AgdaSymbol{\{}\AgdaBound{t}\AgdaSymbol{\}} \AgdaSymbol{→} \AgdaFunction{[} \AgdaBound{Ax} \AgdaFunction{]refl} \AgdaInductiveConstructor{,} \AgdaFunction{[} \AgdaBound{B} \AgdaFunction{]refl*}\<%
\\
%
\\
\>[0]\AgdaIndent{11}{}\<[11]%
\>[11]\AgdaSymbol{;} \AgdaField{sym} \<[21]%
\>[21]\AgdaSymbol{=} \AgdaSymbol{λ} \AgdaSymbol{\{(}\AgdaBound{p} \AgdaInductiveConstructor{,} \AgdaBound{q}\AgdaSymbol{)} \AgdaSymbol{→} \AgdaSymbol{(}\AgdaFunction{[} \AgdaBound{Ax} \AgdaFunction{]sym} \AgdaBound{p}\AgdaSymbol{)} \AgdaInductiveConstructor{,} \<[52]%
\>[52]\<%
\\
\>[11]\AgdaIndent{23}{}\<[23]%
\>[23]\AgdaFunction{[} \AgdaBound{Bx} \AgdaSymbol{\_} \AgdaFunction{]trans} \AgdaSymbol{(}\AgdaFunction{[} \AgdaBound{B} \AgdaFunction{]subst*} \AgdaSymbol{\_} \<[52]%
\>[52]\<%
\\
\>[11]\AgdaIndent{23}{}\<[23]%
\>[23]\AgdaSymbol{(}\AgdaFunction{[} \AgdaBound{Bx} \AgdaSymbol{\_} \AgdaFunction{]sym} \AgdaBound{q}\AgdaSymbol{))} \AgdaFunction{[} \AgdaBound{B} \AgdaFunction{]tr*}\AgdaSymbol{\}}\<%
\\
%
\\
\>[0]\AgdaIndent{11}{}\<[11]%
\>[11]\AgdaSymbol{;} \AgdaField{trans} \<[21]%
\>[21]\AgdaSymbol{=} \AgdaSymbol{λ} \AgdaSymbol{\{(}\AgdaBound{p} \AgdaInductiveConstructor{,} \AgdaBound{q}\AgdaSymbol{)} \AgdaSymbol{(}\AgdaBound{r} \AgdaInductiveConstructor{,} \AgdaBound{s}\AgdaSymbol{)} \AgdaSymbol{→} \AgdaSymbol{(}\AgdaFunction{[} \AgdaBound{Ax} \AgdaFunction{]trans} \AgdaBound{p} \AgdaBound{r}\AgdaSymbol{)} \AgdaInductiveConstructor{,}\<%
\\
\>[0]\AgdaIndent{23}{}\<[23]%
\>[23]\AgdaFunction{[} \AgdaBound{Bx} \AgdaSymbol{\_} \AgdaFunction{]trans} \AgdaSymbol{(}\AgdaFunction{[} \AgdaBound{Bx} \AgdaSymbol{\_} \AgdaFunction{]trans} \<[52]%
\>[52]\<%
\\
\>[0]\AgdaIndent{23}{}\<[23]%
\>[23]\AgdaSymbol{(}\AgdaFunction{[} \AgdaBound{Bx} \AgdaSymbol{\_} \AgdaFunction{]sym} \AgdaSymbol{(}\AgdaFunction{[} \AgdaBound{B} \AgdaFunction{]trans*} \AgdaSymbol{\_))} \<[53]%
\>[53]\<%
\\
\>[0]\AgdaIndent{23}{}\<[23]%
\>[23]\AgdaSymbol{(}\AgdaFunction{[} \AgdaBound{B} \AgdaFunction{]subst*} \AgdaSymbol{\_} \AgdaBound{q}\AgdaSymbol{))} \AgdaBound{s}\AgdaSymbol{\}}\<%
\\
\>[0]\AgdaIndent{11}{}\<[11]%
\>[11]\AgdaSymbol{\}}\<%
\\
%
\\
\>[0]\AgdaIndent{8}{}\<[8]%
\>[8]\AgdaSymbol{;} \AgdaField{substT} \AgdaSymbol{=} \AgdaSymbol{λ} \AgdaBound{x≈y} \AgdaSymbol{→} \AgdaSymbol{λ} \AgdaSymbol{\{(}\AgdaBound{p} \AgdaInductiveConstructor{,} \AgdaBound{q}\AgdaSymbol{)} \AgdaSymbol{→} \<[40]%
\>[40]\<%
\\
\>[8]\AgdaIndent{19}{}\<[19]%
\>[19]\AgdaSymbol{(}\AgdaFunction{[} \AgdaBound{A} \AgdaFunction{]subst} \AgdaBound{x≈y} \AgdaBound{p}\AgdaSymbol{)} \AgdaInductiveConstructor{,} \AgdaFunction{[} \AgdaBound{B} \AgdaFunction{]subst} \AgdaSymbol{(}\AgdaBound{x≈y} \AgdaInductiveConstructor{,} \<[58]%
\>[58]\<%
\\
\>[8]\AgdaIndent{19}{}\<[19]%
\>[19]\AgdaFunction{[} \AgdaFunction{[} \AgdaBound{A} \AgdaFunction{]fm} \AgdaSymbol{\_} \AgdaFunction{]refl}\AgdaSymbol{)} \AgdaBound{q}\AgdaSymbol{\}}\<%
\\
%
\\
\>[0]\AgdaIndent{8}{}\<[8]%
\>[8]\AgdaSymbol{;} \AgdaField{subst*} \AgdaSymbol{=} \AgdaSymbol{λ} \AgdaBound{x≈y} \AgdaSymbol{→} \AgdaSymbol{λ} \AgdaSymbol{\{(}\AgdaBound{p} \AgdaInductiveConstructor{,} \AgdaBound{q}\AgdaSymbol{)} \AgdaSymbol{→} \AgdaFunction{[} \AgdaBound{A} \AgdaFunction{]subst*} \AgdaBound{x≈y} \AgdaBound{p} \AgdaInductiveConstructor{,} \<[60]%
\>[60]\<%
\\
\>[0]\AgdaIndent{19}{}\<[19]%
\>[19]\AgdaFunction{[} \AgdaFunction{[} \AgdaBound{B} \AgdaFunction{]fm} \AgdaSymbol{\_} \AgdaFunction{]trans} \AgdaSymbol{(}\AgdaFunction{[} \AgdaFunction{[} \AgdaBound{B} \AgdaFunction{]fm} \AgdaSymbol{\_} \AgdaFunction{]trans} \<[58]%
\>[58]\<%
\\
\>[0]\AgdaIndent{19}{}\<[19]%
\>[19]\AgdaSymbol{(}\AgdaFunction{[} \AgdaBound{B} \AgdaFunction{]trans*} \AgdaSymbol{\_)}\<%
\\
\>[0]\AgdaIndent{19}{}\<[19]%
\>[19]\AgdaSymbol{(}\AgdaFunction{[} \AgdaFunction{[} \AgdaBound{B} \AgdaFunction{]fm} \AgdaSymbol{\_} \AgdaFunction{]sym} \AgdaSymbol{(}\AgdaFunction{[} \AgdaBound{B} \AgdaFunction{]trans*} \AgdaSymbol{\_)))} \<[55]%
\>[55]\<%
\\
\>[0]\AgdaIndent{19}{}\<[19]%
\>[19]\AgdaSymbol{(}\AgdaFunction{[} \AgdaBound{B} \AgdaFunction{]subst*} \AgdaSymbol{(}\AgdaBound{x≈y} \AgdaInductiveConstructor{,} \AgdaFunction{[} \AgdaFunction{[} \AgdaBound{A} \AgdaFunction{]fm} \AgdaSymbol{\_} \AgdaFunction{]refl}\AgdaSymbol{)} \AgdaBound{q}\AgdaSymbol{)} \AgdaSymbol{\}}\<%
\\
\>[0]\AgdaIndent{8}{}\<[8]%
\>[8]\AgdaSymbol{;} \AgdaField{refl*} \AgdaSymbol{=} \AgdaSymbol{λ} \AgdaSymbol{\{}\AgdaBound{x}\AgdaSymbol{\}} \AgdaSymbol{\{}\AgdaBound{a}\AgdaSymbol{\}} \AgdaSymbol{→} \<[30]%
\>[30]\<%
\\
\>[0]\AgdaIndent{18}{}\<[18]%
\>[18]\AgdaKeyword{let} \AgdaSymbol{(}\AgdaBound{p} \AgdaInductiveConstructor{,} \AgdaBound{q}\AgdaSymbol{)} \AgdaSymbol{=} \AgdaBound{a} \AgdaKeyword{in} \AgdaFunction{[} \AgdaBound{A} \AgdaFunction{]refl*} \AgdaInductiveConstructor{,} \AgdaFunction{[} \AgdaBound{B} \AgdaFunction{]tr*}\<%
\\
\>[0]\AgdaIndent{8}{}\<[8]%
\>[8]\AgdaSymbol{;} \AgdaField{trans*} \AgdaSymbol{=} \<[20]%
\>[20]\AgdaSymbol{λ} \AgdaSymbol{\{(}\AgdaBound{p} \AgdaInductiveConstructor{,} \AgdaBound{q}\AgdaSymbol{)} \<[32]%
\>[32]\AgdaSymbol{→} \AgdaSymbol{(}\AgdaFunction{[} \AgdaBound{A} \AgdaFunction{]trans*} \AgdaSymbol{\_)} \AgdaInductiveConstructor{,} \<[52]%
\>[52]\<%
\\
\>[0]\AgdaIndent{20}{}\<[20]%
\>[20]\AgdaSymbol{(}\AgdaFunction{[} \AgdaFunction{[} \AgdaBound{B} \AgdaFunction{]fm} \AgdaSymbol{\_} \AgdaFunction{]trans}\<%
\\
\>[0]\AgdaIndent{20}{}\<[20]%
\>[20]\AgdaSymbol{(}\AgdaFunction{[} \AgdaBound{B} \AgdaFunction{]trans*} \AgdaSymbol{\_)} \AgdaSymbol{(}\AgdaFunction{[} \AgdaBound{B} \AgdaFunction{]trans*} \AgdaSymbol{\_))} \AgdaSymbol{\}}\<%
\\
\>[0]\AgdaIndent{8}{}\<[8]%
\>[8]\AgdaSymbol{\}}\<%
\\
%
\\
\>\<\end{code}

Binary relation

\begin{code}\>\<%
\\
\>\AgdaFunction{Rel} \AgdaSymbol{:} \AgdaSymbol{\{}\AgdaBound{Γ} \AgdaSymbol{:} \AgdaFunction{Con}\AgdaSymbol{\}} \AgdaSymbol{→} \AgdaRecord{Ty} \AgdaBound{Γ} \AgdaSymbol{→} \AgdaPrimitiveType{Set₁}\<%
\\
\>\AgdaFunction{Rel} \AgdaSymbol{\{}\AgdaBound{Γ}\AgdaSymbol{\}} \AgdaBound{A} \AgdaSymbol{=} \AgdaRecord{Ty} \AgdaSymbol{(}\AgdaBound{Γ} \AgdaFunction{\&} \AgdaBound{A} \AgdaFunction{\&} \AgdaBound{A} \AgdaFunction{[} \AgdaFunction{fst\&} \AgdaSymbol{\{}A \AgdaSymbol{=} \AgdaBound{A}\AgdaSymbol{\}} \AgdaFunction{]T}\AgdaSymbol{)}\<%
\\
\>\<\end{code}

Natural numbers

Axiom: irrelevant:

\begin{code}\>\<%
\\
\>\AgdaKeyword{postulate}\<%
\\
\>[0]\AgdaIndent{4}{}\<[4]%
\>[4]\AgdaSymbol{.}\AgdaPostulate{irrelevant} \AgdaSymbol{:} \AgdaSymbol{\{}\AgdaBound{A} \AgdaSymbol{:} \AgdaPrimitiveType{Set}\AgdaSymbol{\}} \AgdaSymbol{→} \AgdaSymbol{.}\AgdaBound{A} \AgdaSymbol{→} \AgdaBound{A}\<%
\\
\>\<\end{code}


\begin{code}\>\<%
\\
\>\AgdaKeyword{module} \AgdaModule{Natural} \AgdaSymbol{(}\AgdaBound{Γ} \AgdaSymbol{:} \AgdaFunction{Con}\AgdaSymbol{)} \AgdaKeyword{where}\<%
\\
%
\\
\>[0]\AgdaIndent{2}{}\<[2]%
\>[2]\AgdaFunction{\_≈nat\_} \AgdaSymbol{:} \AgdaDatatype{ℕ} \AgdaSymbol{→} \AgdaDatatype{ℕ} \AgdaSymbol{→} \AgdaPrimitiveType{Set}\<%
\\
\>[0]\AgdaIndent{2}{}\<[2]%
\>[2]\AgdaInductiveConstructor{zero} \AgdaFunction{≈nat} \AgdaInductiveConstructor{zero} \AgdaSymbol{=} \AgdaRecord{⊤}\<%
\\
\>[0]\AgdaIndent{2}{}\<[2]%
\>[2]\AgdaInductiveConstructor{zero} \AgdaFunction{≈nat} \AgdaInductiveConstructor{suc} \AgdaBound{n} \AgdaSymbol{=} \AgdaDatatype{⊥}\<%
\\
\>[0]\AgdaIndent{2}{}\<[2]%
\>[2]\AgdaInductiveConstructor{suc} \AgdaBound{m} \AgdaFunction{≈nat} \AgdaInductiveConstructor{zero} \AgdaSymbol{=} \AgdaDatatype{⊥}\<%
\\
\>[0]\AgdaIndent{2}{}\<[2]%
\>[2]\AgdaInductiveConstructor{suc} \AgdaBound{m} \AgdaFunction{≈nat} \AgdaInductiveConstructor{suc} \AgdaBound{n} \AgdaSymbol{=} \AgdaBound{m} \AgdaFunction{≈nat} \AgdaBound{n}\<%
\\
\>[0]\AgdaIndent{2}{}\<[2]%
\>[2]\<%
\\
\>[0]\AgdaIndent{2}{}\<[2]%
\>[2]\AgdaFunction{reflNat} \AgdaSymbol{:} \AgdaSymbol{\{}\AgdaBound{x} \AgdaSymbol{:} \AgdaDatatype{ℕ}\AgdaSymbol{\}} \AgdaSymbol{→} \AgdaBound{x} \AgdaFunction{≈nat} \AgdaBound{x}\<%
\\
\>[0]\AgdaIndent{2}{}\<[2]%
\>[2]\AgdaFunction{reflNat} \AgdaSymbol{\{}\AgdaInductiveConstructor{zero}\AgdaSymbol{\}} \AgdaSymbol{=} \AgdaInductiveConstructor{tt}\<%
\\
\>[0]\AgdaIndent{2}{}\<[2]%
\>[2]\AgdaFunction{reflNat} \AgdaSymbol{\{}\AgdaInductiveConstructor{suc} \AgdaBound{n}\AgdaSymbol{\}} \AgdaSymbol{=} \AgdaFunction{reflNat} \AgdaSymbol{\{}\AgdaBound{n}\AgdaSymbol{\}}\<%
\\
%
\\
\>[0]\AgdaIndent{2}{}\<[2]%
\>[2]\AgdaFunction{symNat} \AgdaSymbol{:} \AgdaSymbol{\{}\AgdaBound{x} \AgdaBound{y} \AgdaSymbol{:} \AgdaDatatype{ℕ}\AgdaSymbol{\}} \AgdaSymbol{→} \AgdaBound{x} \AgdaFunction{≈nat} \AgdaBound{y} \AgdaSymbol{→} \AgdaBound{y} \AgdaFunction{≈nat} \AgdaBound{x}\<%
\\
\>[0]\AgdaIndent{2}{}\<[2]%
\>[2]\AgdaFunction{symNat} \AgdaSymbol{\{}\AgdaInductiveConstructor{zero}\AgdaSymbol{\}} \AgdaSymbol{\{}\AgdaInductiveConstructor{zero}\AgdaSymbol{\}} \AgdaBound{eq} \AgdaSymbol{=} \AgdaInductiveConstructor{tt}\<%
\\
\>[0]\AgdaIndent{2}{}\<[2]%
\>[2]\AgdaFunction{symNat} \AgdaSymbol{\{}\AgdaInductiveConstructor{zero}\AgdaSymbol{\}} \AgdaSymbol{\{}\AgdaInductiveConstructor{suc} \AgdaSymbol{\_\}} \AgdaBound{eq} \AgdaSymbol{=} \AgdaBound{eq}\<%
\\
\>[0]\AgdaIndent{2}{}\<[2]%
\>[2]\AgdaFunction{symNat} \AgdaSymbol{\{}\AgdaInductiveConstructor{suc} \AgdaSymbol{\_\}} \AgdaSymbol{\{}\AgdaInductiveConstructor{zero}\AgdaSymbol{\}} \AgdaBound{eq} \AgdaSymbol{=} \AgdaBound{eq}\<%
\\
\>[0]\AgdaIndent{2}{}\<[2]%
\>[2]\AgdaFunction{symNat} \AgdaSymbol{\{}\AgdaInductiveConstructor{suc} \AgdaBound{x}\AgdaSymbol{\}} \AgdaSymbol{\{}\AgdaInductiveConstructor{suc} \AgdaBound{y}\AgdaSymbol{\}} \AgdaBound{eq} \AgdaSymbol{=} \AgdaFunction{symNat} \AgdaSymbol{\{}\AgdaBound{x}\AgdaSymbol{\}} \AgdaSymbol{\{}\AgdaBound{y}\AgdaSymbol{\}} \AgdaBound{eq}\<%
\\
%
\\
\>[0]\AgdaIndent{2}{}\<[2]%
\>[2]\AgdaFunction{transNat} \AgdaSymbol{:} \AgdaSymbol{\{}\AgdaBound{x} \AgdaBound{y} \AgdaBound{z} \AgdaSymbol{:} \AgdaDatatype{ℕ}\AgdaSymbol{\}}\<%
\\
\>[2]\AgdaIndent{11}{}\<[11]%
\>[11]\AgdaSymbol{→} \AgdaBound{x} \AgdaFunction{≈nat} \AgdaBound{y} \AgdaSymbol{→} \AgdaBound{y} \AgdaFunction{≈nat} \AgdaBound{z} \AgdaSymbol{→} \AgdaBound{x} \AgdaFunction{≈nat} \AgdaBound{z}\<%
\\
\>[0]\AgdaIndent{2}{}\<[2]%
\>[2]\AgdaFunction{transNat} \AgdaSymbol{\{}\AgdaInductiveConstructor{zero}\AgdaSymbol{\}} \AgdaSymbol{\{}\AgdaInductiveConstructor{zero}\AgdaSymbol{\}} \AgdaBound{xy} \AgdaBound{yz} \AgdaSymbol{=} \AgdaBound{yz}\<%
\\
\>[0]\AgdaIndent{2}{}\<[2]%
\>[2]\AgdaFunction{transNat} \AgdaSymbol{\{}\AgdaInductiveConstructor{zero}\AgdaSymbol{\}} \AgdaSymbol{\{}\AgdaInductiveConstructor{suc} \AgdaSymbol{\_\}} \AgdaSymbol{()} \AgdaBound{yz}\<%
\\
\>[0]\AgdaIndent{2}{}\<[2]%
\>[2]\AgdaFunction{transNat} \AgdaSymbol{\{}\AgdaInductiveConstructor{suc} \AgdaSymbol{\_\}} \AgdaSymbol{\{}\AgdaInductiveConstructor{zero}\AgdaSymbol{\}} \AgdaSymbol{()} \AgdaBound{yz}\<%
\\
\>[0]\AgdaIndent{2}{}\<[2]%
\>[2]\AgdaFunction{transNat} \AgdaSymbol{\{}\AgdaInductiveConstructor{suc} \AgdaSymbol{\_\}} \AgdaSymbol{\{}\AgdaInductiveConstructor{suc} \AgdaSymbol{\_\}} \AgdaSymbol{\{}\AgdaInductiveConstructor{zero}\AgdaSymbol{\}} \AgdaBound{xy} \AgdaBound{yz} \AgdaSymbol{=} \AgdaBound{yz}\<%
\\
\>[0]\AgdaIndent{2}{}\<[2]%
\>[2]\AgdaFunction{transNat} \AgdaSymbol{\{}\AgdaInductiveConstructor{suc} \AgdaBound{x}\AgdaSymbol{\}} \AgdaSymbol{\{}\AgdaInductiveConstructor{suc} \AgdaBound{y}\AgdaSymbol{\}} \AgdaSymbol{\{}\AgdaInductiveConstructor{suc} \AgdaBound{z}\AgdaSymbol{\}} \AgdaBound{xy} \AgdaBound{yz} \AgdaSymbol{=}\<%
\\
\>[2]\AgdaIndent{11}{}\<[11]%
\>[11]\AgdaFunction{transNat} \AgdaSymbol{\{}\AgdaBound{x}\AgdaSymbol{\}} \AgdaSymbol{\{}\AgdaBound{y}\AgdaSymbol{\}} \AgdaSymbol{\{}\AgdaBound{z}\AgdaSymbol{\}} \AgdaBound{xy} \AgdaBound{yz}\<%
\\
\>[6]\AgdaIndent{3}{}\<[3]%
\>[3]\<%
\\
%
\\
\>[0]\AgdaIndent{2}{}\<[2]%
\>[2]\AgdaFunction{⟦Nat⟧} \AgdaSymbol{:} \AgdaRecord{Ty} \AgdaBound{Γ}\<%
\\
\>[0]\AgdaIndent{2}{}\<[2]%
\>[2]\AgdaFunction{⟦Nat⟧} \AgdaSymbol{=} \AgdaKeyword{record} \<[17]%
\>[17]\<%
\\
\>[2]\AgdaIndent{4}{}\<[4]%
\>[4]\AgdaSymbol{\{} \AgdaField{fm} \AgdaSymbol{=} \AgdaSymbol{λ} \AgdaBound{γ} \AgdaSymbol{→} \AgdaKeyword{record}\<%
\\
\>[4]\AgdaIndent{9}{}\<[9]%
\>[9]\AgdaSymbol{\{} \AgdaField{Carrier} \AgdaSymbol{=} \AgdaDatatype{ℕ}\<%
\\
\>[4]\AgdaIndent{9}{}\<[9]%
\>[9]\AgdaSymbol{;} \AgdaField{\_≈\_} \AgdaSymbol{=} \AgdaFunction{\_≈nat\_}\<%
\\
\>[4]\AgdaIndent{9}{}\<[9]%
\>[9]\AgdaSymbol{;} \AgdaField{refl} \AgdaSymbol{=} \AgdaSymbol{λ} \AgdaSymbol{\{}\AgdaBound{n}\AgdaSymbol{\}} \AgdaSymbol{→} \AgdaFunction{reflNat} \AgdaSymbol{\{}\AgdaBound{n}\AgdaSymbol{\}}\<%
\\
\>[4]\AgdaIndent{9}{}\<[9]%
\>[9]\AgdaSymbol{;} \AgdaField{sym} \AgdaSymbol{=} \AgdaSymbol{λ} \AgdaSymbol{\{}\AgdaBound{x}\AgdaSymbol{\}} \AgdaSymbol{\{}\AgdaBound{y}\AgdaSymbol{\}} \AgdaSymbol{→} \AgdaFunction{symNat} \AgdaSymbol{\{}\AgdaBound{x}\AgdaSymbol{\}} \AgdaSymbol{\{}\AgdaBound{y}\AgdaSymbol{\}}\<%
\\
\>[4]\AgdaIndent{9}{}\<[9]%
\>[9]\AgdaSymbol{;} \AgdaField{trans} \AgdaSymbol{=} \AgdaSymbol{λ} \AgdaSymbol{\{}\AgdaBound{x}\AgdaSymbol{\}} \AgdaSymbol{\{}\AgdaBound{y}\AgdaSymbol{\}} \AgdaSymbol{\{}\AgdaBound{z}\AgdaSymbol{\}} \AgdaSymbol{→} \AgdaFunction{transNat} \AgdaSymbol{\{}\AgdaBound{x}\AgdaSymbol{\}} \AgdaSymbol{\{}\AgdaBound{y}\AgdaSymbol{\}} \AgdaSymbol{\{}\AgdaBound{z}\AgdaSymbol{\}}\<%
\\
\>[4]\AgdaIndent{9}{}\<[9]%
\>[9]\AgdaSymbol{\}}\<%
\\
\>[0]\AgdaIndent{4}{}\<[4]%
\>[4]\AgdaSymbol{;} \AgdaField{substT} \AgdaSymbol{=} \AgdaSymbol{λ} \AgdaBound{\_} \AgdaBound{n} \AgdaSymbol{→} \AgdaBound{n}\<%
\\
\>[0]\AgdaIndent{4}{}\<[4]%
\>[4]\AgdaSymbol{;} \AgdaField{subst*} \AgdaSymbol{=} \AgdaSymbol{λ} \AgdaBound{\_} \AgdaBound{x} \AgdaSymbol{→} \AgdaPostulate{irrelevant} \AgdaBound{x}\<%
\\
\>[0]\AgdaIndent{4}{}\<[4]%
\>[4]\AgdaSymbol{;} \AgdaField{refl*} \AgdaSymbol{=} \AgdaSymbol{λ} \AgdaSymbol{\{}\AgdaBound{x}\AgdaSymbol{\}} \AgdaSymbol{\{}\AgdaBound{a}\AgdaSymbol{\}} \AgdaSymbol{→} \AgdaFunction{reflNat} \AgdaSymbol{\{}\AgdaBound{a}\AgdaSymbol{\}}\<%
\\
\>[0]\AgdaIndent{4}{}\<[4]%
\>[4]\AgdaSymbol{;} \AgdaField{trans*} \AgdaSymbol{=} \AgdaSymbol{λ} \AgdaBound{a} \AgdaSymbol{→} \AgdaFunction{reflNat} \AgdaSymbol{\{}\AgdaBound{a}\AgdaSymbol{\}} \<[33]%
\>[33]\<%
\\
\>[0]\AgdaIndent{4}{}\<[4]%
\>[4]\AgdaSymbol{\}}\<%
\\
%
\\
\>[0]\AgdaIndent{2}{}\<[2]%
\>[2]\AgdaFunction{⟦0⟧} \AgdaSymbol{:} \AgdaRecord{Tm} \AgdaFunction{⟦Nat⟧}\<%
\\
\>[0]\AgdaIndent{2}{}\<[2]%
\>[2]\AgdaFunction{⟦0⟧} \AgdaSymbol{=} \AgdaKeyword{record}\<%
\\
\>[2]\AgdaIndent{6}{}\<[6]%
\>[6]\AgdaSymbol{\{} \AgdaField{tm} \AgdaSymbol{=} \AgdaSymbol{λ} \AgdaBound{\_} \AgdaSymbol{→} \AgdaNumber{0}\<%
\\
\>[2]\AgdaIndent{6}{}\<[6]%
\>[6]\AgdaSymbol{;} \AgdaField{respt} \AgdaSymbol{=} \AgdaSymbol{λ} \AgdaBound{p} \AgdaSymbol{→} \AgdaInductiveConstructor{tt}\<%
\\
\>[2]\AgdaIndent{6}{}\<[6]%
\>[6]\AgdaSymbol{\}}\<%
\\
%
\\
\>[0]\AgdaIndent{2}{}\<[2]%
\>[2]\AgdaFunction{⟦s⟧} \AgdaSymbol{:} \AgdaRecord{Tm} \AgdaFunction{⟦Nat⟧} \AgdaSymbol{→} \AgdaRecord{Tm} \AgdaFunction{⟦Nat⟧}\<%
\\
\>[0]\AgdaIndent{2}{}\<[2]%
\>[2]\AgdaFunction{⟦s⟧} \AgdaSymbol{(}\AgdaInductiveConstructor{tm:} \AgdaBound{t} \AgdaInductiveConstructor{resp:} \AgdaBound{respt}\AgdaSymbol{)} \<[26]%
\>[26]\<%
\\
\>[2]\AgdaIndent{6}{}\<[6]%
\>[6]\AgdaSymbol{=} \AgdaKeyword{record}\<%
\\
\>[2]\AgdaIndent{6}{}\<[6]%
\>[6]\AgdaSymbol{\{} \AgdaField{tm} \AgdaSymbol{=} \AgdaInductiveConstructor{suc} \AgdaFunction{∘} \AgdaBound{t}\<%
\\
\>[2]\AgdaIndent{6}{}\<[6]%
\>[6]\AgdaSymbol{;} \AgdaField{respt} \AgdaSymbol{=} \AgdaBound{respt}\<%
\\
\>[2]\AgdaIndent{6}{}\<[6]%
\>[6]\AgdaSymbol{\}}\<%
\\
\>\<\end{code}

Simply typed universe

\AgdaHide{
\begin{code}\>\<%
\\
\>[0]\AgdaIndent{2}{}\<[2]%
\>[2]\AgdaKeyword{data} \AgdaDatatype{⟦U⟧⁰} \AgdaSymbol{:} \AgdaPrimitiveType{Set} \AgdaKeyword{where}\<%
\\
\>[0]\AgdaIndent{4}{}\<[4]%
\>[4]\AgdaInductiveConstructor{nat} \AgdaSymbol{:} \AgdaDatatype{⟦U⟧⁰}\<%
\\
\>[0]\AgdaIndent{4}{}\<[4]%
\>[4]\AgdaInductiveConstructor{arr<\_,\_>} \AgdaSymbol{:} \AgdaSymbol{(}\AgdaBound{a} \AgdaBound{b} \AgdaSymbol{:} \AgdaDatatype{⟦U⟧⁰}\AgdaSymbol{)} \AgdaSymbol{→} \AgdaDatatype{⟦U⟧⁰}\<%
\\
%
\\
\>[0]\AgdaIndent{2}{}\<[2]%
\>[2]\AgdaFunction{\_\textasciitilde⟦U⟧\_} \AgdaSymbol{:} \AgdaDatatype{⟦U⟧⁰} \AgdaSymbol{→} \AgdaDatatype{⟦U⟧⁰} \AgdaSymbol{→} \AgdaPrimitiveType{Set} \AgdaComment{-- HProp}\<%
\\
\>[0]\AgdaIndent{2}{}\<[2]%
\>[2]\AgdaInductiveConstructor{nat} \AgdaFunction{\textasciitilde⟦U⟧} \AgdaInductiveConstructor{nat} \AgdaSymbol{=} \AgdaRecord{⊤}\<%
\\
\>[0]\AgdaIndent{2}{}\<[2]%
\>[2]\AgdaInductiveConstructor{nat} \AgdaFunction{\textasciitilde⟦U⟧} \AgdaInductiveConstructor{arr<} \AgdaBound{a} \AgdaInductiveConstructor{,} \AgdaBound{b} \AgdaInductiveConstructor{>} \AgdaSymbol{=} \AgdaDatatype{⊥}\<%
\\
\>[0]\AgdaIndent{2}{}\<[2]%
\>[2]\AgdaInductiveConstructor{arr<} \AgdaBound{a} \AgdaInductiveConstructor{,} \AgdaBound{b} \AgdaInductiveConstructor{>} \AgdaFunction{\textasciitilde⟦U⟧} \AgdaInductiveConstructor{nat} \AgdaSymbol{=} \AgdaDatatype{⊥}\<%
\\
\>[0]\AgdaIndent{2}{}\<[2]%
\>[2]\AgdaInductiveConstructor{arr<} \AgdaBound{a} \AgdaInductiveConstructor{,} \AgdaBound{b} \AgdaInductiveConstructor{>} \AgdaFunction{\textasciitilde⟦U⟧} \AgdaInductiveConstructor{arr<} \AgdaBound{a'} \AgdaInductiveConstructor{,} \AgdaBound{b'} \AgdaInductiveConstructor{>} \AgdaSymbol{=} \AgdaBound{a} \AgdaFunction{\textasciitilde⟦U⟧} \AgdaBound{a'} \AgdaFunction{×} \AgdaBound{b} \AgdaFunction{\textasciitilde⟦U⟧} \AgdaBound{b'}\<%
\\
%
\\
\>[0]\AgdaIndent{2}{}\<[2]%
\>[2]\AgdaFunction{reflU} \AgdaSymbol{:} \<[11]%
\>[11]\AgdaSymbol{\{}\AgdaBound{x} \AgdaSymbol{:} \AgdaDatatype{⟦U⟧⁰}\AgdaSymbol{\}} \AgdaSymbol{→} \AgdaBound{x} \AgdaFunction{\textasciitilde⟦U⟧} \AgdaBound{x}\<%
\\
\>[0]\AgdaIndent{2}{}\<[2]%
\>[2]\AgdaFunction{reflU} \AgdaSymbol{\{}\AgdaInductiveConstructor{nat}\AgdaSymbol{\}} \AgdaSymbol{=} \AgdaInductiveConstructor{tt}\<%
\\
\>[0]\AgdaIndent{2}{}\<[2]%
\>[2]\AgdaFunction{reflU} \AgdaSymbol{\{}\AgdaInductiveConstructor{arr<} \AgdaBound{a} \AgdaInductiveConstructor{,} \AgdaBound{b} \AgdaInductiveConstructor{>}\AgdaSymbol{\}} \AgdaSymbol{=} \AgdaFunction{reflU} \AgdaSymbol{\{}\AgdaBound{a}\AgdaSymbol{\}} \AgdaInductiveConstructor{,} \AgdaFunction{reflU} \AgdaSymbol{\{}\AgdaBound{b}\AgdaSymbol{\}}\<%
\\
%
\\
\>[0]\AgdaIndent{2}{}\<[2]%
\>[2]\AgdaFunction{symU} \AgdaSymbol{:} \AgdaSymbol{\{}\AgdaBound{x} \AgdaBound{y} \AgdaSymbol{:} \AgdaDatatype{⟦U⟧⁰}\AgdaSymbol{\}} \AgdaSymbol{→} \AgdaBound{x} \AgdaFunction{\textasciitilde⟦U⟧} \AgdaBound{y} \AgdaSymbol{→} \AgdaBound{y} \AgdaFunction{\textasciitilde⟦U⟧} \AgdaBound{x}\<%
\\
\>[0]\AgdaIndent{2}{}\<[2]%
\>[2]\AgdaFunction{symU} \AgdaSymbol{\{}\AgdaInductiveConstructor{nat}\AgdaSymbol{\}} \AgdaSymbol{\{}\AgdaInductiveConstructor{nat}\AgdaSymbol{\}} \AgdaBound{eq} \AgdaSymbol{=} \AgdaInductiveConstructor{tt}\<%
\\
\>[0]\AgdaIndent{2}{}\<[2]%
\>[2]\AgdaFunction{symU} \AgdaSymbol{\{}\AgdaInductiveConstructor{nat}\AgdaSymbol{\}} \AgdaSymbol{\{}\AgdaInductiveConstructor{arr<} \AgdaBound{a} \AgdaInductiveConstructor{,} \AgdaBound{b} \AgdaInductiveConstructor{>}\AgdaSymbol{\}} \AgdaBound{eq} \AgdaSymbol{=} \AgdaBound{eq}\<%
\\
\>[0]\AgdaIndent{2}{}\<[2]%
\>[2]\AgdaFunction{symU} \AgdaSymbol{\{}\AgdaInductiveConstructor{arr<} \AgdaBound{a} \AgdaInductiveConstructor{,} \AgdaBound{b} \AgdaInductiveConstructor{>}\AgdaSymbol{\}} \AgdaSymbol{\{}\AgdaInductiveConstructor{nat}\AgdaSymbol{\}} \AgdaBound{eq} \AgdaSymbol{=} \AgdaBound{eq}\<%
\\
\>[0]\AgdaIndent{2}{}\<[2]%
\>[2]\AgdaFunction{symU} \AgdaSymbol{\{}\AgdaInductiveConstructor{arr<} \AgdaBound{a} \AgdaInductiveConstructor{,} \AgdaBound{b} \AgdaInductiveConstructor{>}\AgdaSymbol{\}} \AgdaSymbol{\{}\AgdaInductiveConstructor{arr<} \AgdaBound{a'} \AgdaInductiveConstructor{,} \AgdaBound{b'} \AgdaInductiveConstructor{>}\AgdaSymbol{\}} \AgdaSymbol{(}\AgdaBound{p} \AgdaInductiveConstructor{,} \AgdaBound{q}\AgdaSymbol{)} \AgdaSymbol{=} \AgdaSymbol{(}\AgdaFunction{symU} \AgdaSymbol{\{}\AgdaBound{a}\AgdaSymbol{\}} \AgdaSymbol{\{}\AgdaBound{a'}\AgdaSymbol{\}} \AgdaBound{p}\AgdaSymbol{)} \<[67]%
\>[67]\<%
\\
\>[2]\AgdaIndent{47}{}\<[47]%
\>[47]\AgdaInductiveConstructor{,} \AgdaSymbol{(}\AgdaFunction{symU} \AgdaSymbol{\{}\AgdaBound{b}\AgdaSymbol{\}} \AgdaSymbol{\{}\AgdaBound{b'}\AgdaSymbol{\}} \AgdaBound{q}\AgdaSymbol{)}\<%
\\
%
\\
\>[0]\AgdaIndent{2}{}\<[2]%
\>[2]\AgdaFunction{transU} \AgdaSymbol{:} \AgdaSymbol{\{}\AgdaBound{x} \AgdaBound{y} \AgdaBound{z} \AgdaSymbol{:} \AgdaDatatype{⟦U⟧⁰}\AgdaSymbol{\}} \AgdaSymbol{→} \AgdaBound{x} \AgdaFunction{\textasciitilde⟦U⟧} \AgdaBound{y} \AgdaSymbol{→} \AgdaBound{y} \AgdaFunction{\textasciitilde⟦U⟧} \AgdaBound{z} \AgdaSymbol{→} \AgdaBound{x} \AgdaFunction{\textasciitilde⟦U⟧} \AgdaBound{z}\<%
\\
\>[0]\AgdaIndent{2}{}\<[2]%
\>[2]\AgdaFunction{transU} \AgdaSymbol{\{}\AgdaInductiveConstructor{nat}\AgdaSymbol{\}} \AgdaSymbol{\{}\AgdaInductiveConstructor{nat}\AgdaSymbol{\}} \AgdaBound{eq1} \AgdaBound{eq2} \AgdaSymbol{=} \AgdaBound{eq2}\<%
\\
\>[0]\AgdaIndent{2}{}\<[2]%
\>[2]\AgdaFunction{transU} \AgdaSymbol{\{}\AgdaInductiveConstructor{nat}\AgdaSymbol{\}} \AgdaSymbol{\{}\AgdaInductiveConstructor{arr<} \AgdaBound{a} \AgdaInductiveConstructor{,} \AgdaBound{b} \AgdaInductiveConstructor{>}\AgdaSymbol{\}} \AgdaSymbol{()} \AgdaBound{eq2}\<%
\\
\>[0]\AgdaIndent{2}{}\<[2]%
\>[2]\AgdaFunction{transU} \AgdaSymbol{\{}\AgdaInductiveConstructor{arr<} \AgdaBound{a} \AgdaInductiveConstructor{,} \AgdaBound{b} \AgdaInductiveConstructor{>}\AgdaSymbol{\}} \AgdaSymbol{\{}\AgdaInductiveConstructor{nat}\AgdaSymbol{\}} \AgdaSymbol{()} \AgdaBound{eq2}\<%
\\
\>[0]\AgdaIndent{2}{}\<[2]%
\>[2]\AgdaFunction{transU} \AgdaSymbol{\{}\AgdaInductiveConstructor{arr<} \AgdaBound{a} \AgdaInductiveConstructor{,} \AgdaBound{b} \AgdaInductiveConstructor{>}\AgdaSymbol{\}} \AgdaSymbol{\{}\AgdaInductiveConstructor{arr<} \AgdaBound{a'} \AgdaInductiveConstructor{,} \AgdaBound{b'} \AgdaInductiveConstructor{>}\AgdaSymbol{\}} \AgdaSymbol{\{}\AgdaInductiveConstructor{nat}\AgdaSymbol{\}} \AgdaBound{eq1} \AgdaBound{eq2} \AgdaSymbol{=} \AgdaBound{eq2}\<%
\\
\>[0]\AgdaIndent{2}{}\<[2]%
\>[2]\AgdaFunction{transU} \AgdaSymbol{\{}\AgdaInductiveConstructor{arr<} \AgdaBound{a} \AgdaInductiveConstructor{,} \AgdaBound{b} \AgdaInductiveConstructor{>}\AgdaSymbol{\}} \AgdaSymbol{\{}\AgdaInductiveConstructor{arr<} \AgdaBound{a'} \AgdaInductiveConstructor{,} \AgdaBound{b'} \AgdaInductiveConstructor{>}\AgdaSymbol{\}} \AgdaSymbol{\{}\AgdaInductiveConstructor{arr<} \AgdaBound{a0} \AgdaInductiveConstructor{,} \AgdaBound{b0} \AgdaInductiveConstructor{>}\AgdaSymbol{\}} \AgdaSymbol{(}\AgdaBound{p1} \AgdaInductiveConstructor{,} \AgdaBound{q1}\AgdaSymbol{)} \<[68]%
\>[68]\<%
\\
\>[2]\AgdaIndent{9}{}\<[9]%
\>[9]\AgdaSymbol{(}\AgdaBound{p2} \AgdaInductiveConstructor{,} \AgdaBound{q2}\AgdaSymbol{)} \AgdaSymbol{=} \AgdaSymbol{(}\AgdaFunction{transU} \AgdaSymbol{\{}\AgdaBound{a}\AgdaSymbol{\}} \AgdaSymbol{\{}\AgdaBound{a'}\AgdaSymbol{\}} \AgdaSymbol{\{}\AgdaBound{a0}\AgdaSymbol{\}} \AgdaBound{p1} \AgdaBound{p2}\AgdaSymbol{)} \<[50]%
\>[50]\<%
\\
\>[2]\AgdaIndent{9}{}\<[9]%
\>[9]\AgdaInductiveConstructor{,} \AgdaFunction{transU} \AgdaSymbol{\{}\AgdaBound{b}\AgdaSymbol{\}} \AgdaSymbol{\{}\AgdaBound{b'}\AgdaSymbol{\}} \AgdaSymbol{\{}\AgdaBound{b0}\AgdaSymbol{\}} \AgdaBound{q1} \AgdaBound{q2}\<%
\\
%
\\
\>[0]\AgdaIndent{2}{}\<[2]%
\>[2]\AgdaFunction{⟦U⟧} \AgdaSymbol{:} \AgdaRecord{Ty} \AgdaBound{Γ}\<%
\\
\>[0]\AgdaIndent{2}{}\<[2]%
\>[2]\AgdaFunction{⟦U⟧} \AgdaSymbol{=} \AgdaKeyword{record} \<[15]%
\>[15]\<%
\\
\>[2]\AgdaIndent{4}{}\<[4]%
\>[4]\AgdaSymbol{\{} \AgdaField{fm} \AgdaSymbol{=} \AgdaSymbol{λ} \AgdaBound{γ} \AgdaSymbol{→} \AgdaKeyword{record}\<%
\\
\>[4]\AgdaIndent{9}{}\<[9]%
\>[9]\AgdaSymbol{\{} \AgdaField{Carrier} \AgdaSymbol{=} \AgdaDatatype{⟦U⟧⁰}\<%
\\
\>[4]\AgdaIndent{9}{}\<[9]%
\>[9]\AgdaSymbol{;} \AgdaField{\_≈\_} \AgdaSymbol{=} \AgdaFunction{\_\textasciitilde⟦U⟧\_}\<%
\\
\>[4]\AgdaIndent{9}{}\<[9]%
\>[9]\AgdaSymbol{;} \AgdaField{refl} \AgdaSymbol{=} \AgdaSymbol{λ} \AgdaSymbol{\{}\AgdaBound{x}\AgdaSymbol{\}} \AgdaSymbol{→} \AgdaFunction{reflU} \AgdaSymbol{\{}\AgdaBound{x}\AgdaSymbol{\}}\<%
\\
\>[4]\AgdaIndent{9}{}\<[9]%
\>[9]\AgdaSymbol{;} \AgdaField{sym} \AgdaSymbol{=} \AgdaSymbol{λ} \AgdaSymbol{\{}\AgdaBound{x}\AgdaSymbol{\}} \AgdaSymbol{\{}\AgdaBound{y}\AgdaSymbol{\}} \AgdaSymbol{→} \AgdaFunction{symU} \AgdaSymbol{\{}\AgdaBound{x}\AgdaSymbol{\}} \AgdaSymbol{\{}\AgdaBound{y}\AgdaSymbol{\}}\<%
\\
\>[4]\AgdaIndent{9}{}\<[9]%
\>[9]\AgdaSymbol{;} \AgdaField{trans} \AgdaSymbol{=} \AgdaSymbol{λ} \AgdaSymbol{\{}\AgdaBound{x}\AgdaSymbol{\}} \AgdaSymbol{\{}\AgdaBound{y}\AgdaSymbol{\}} \AgdaSymbol{\{}\AgdaBound{z}\AgdaSymbol{\}} \AgdaSymbol{→} \AgdaFunction{transU} \AgdaSymbol{\{}\AgdaBound{x}\AgdaSymbol{\}} \AgdaSymbol{\{}\AgdaBound{y}\AgdaSymbol{\}} \AgdaSymbol{\{}\AgdaBound{z}\AgdaSymbol{\}}\<%
\\
\>[4]\AgdaIndent{9}{}\<[9]%
\>[9]\AgdaSymbol{\}}\<%
\\
\>[0]\AgdaIndent{4}{}\<[4]%
\>[4]\AgdaSymbol{;} \AgdaField{substT} \AgdaSymbol{=} \AgdaSymbol{λ} \AgdaBound{\_} \AgdaBound{x} \AgdaSymbol{→} \AgdaBound{x}\<%
\\
\>[0]\AgdaIndent{4}{}\<[4]%
\>[4]\AgdaSymbol{;} \AgdaField{subst*} \AgdaSymbol{=} \<[16]%
\>[16]\AgdaSymbol{λ} \AgdaBound{\_} \AgdaBound{x} \AgdaSymbol{→} \AgdaPostulate{irrelevant} \AgdaBound{x}\<%
\\
\>[0]\AgdaIndent{4}{}\<[4]%
\>[4]\AgdaSymbol{;} \AgdaField{refl*} \AgdaSymbol{=} \AgdaSymbol{λ} \AgdaSymbol{\{}\AgdaBound{x}\AgdaSymbol{\}} \AgdaSymbol{\{}\AgdaBound{a}\AgdaSymbol{\}} \AgdaSymbol{→} \AgdaFunction{reflU} \AgdaSymbol{\{}\AgdaBound{a}\AgdaSymbol{\}}\<%
\\
\>[0]\AgdaIndent{4}{}\<[4]%
\>[4]\AgdaSymbol{;} \AgdaField{trans*} \AgdaSymbol{=} \AgdaSymbol{λ} \AgdaBound{a} \AgdaSymbol{→} \AgdaFunction{reflU} \AgdaSymbol{\{}\AgdaBound{a}\AgdaSymbol{\}}\<%
\\
\>[0]\AgdaIndent{4}{}\<[4]%
\>[4]\AgdaSymbol{\}}\<%
\\
%
\\
\>[0]\AgdaIndent{2}{}\<[2]%
\>[2]\AgdaFunction{elfm} \AgdaSymbol{:} \AgdaRecord{Σ} \AgdaFunction{∣} \AgdaBound{Γ} \AgdaFunction{∣} \AgdaSymbol{(λ} \AgdaBound{x} \AgdaSymbol{→} \AgdaDatatype{⟦U⟧⁰}\AgdaSymbol{)} \AgdaSymbol{→} \AgdaRecord{Setoid}\<%
\\
\>[0]\AgdaIndent{2}{}\<[2]%
\>[2]\AgdaFunction{elfm} \AgdaSymbol{(}\AgdaBound{γ} \AgdaInductiveConstructor{,} \AgdaInductiveConstructor{nat}\AgdaSymbol{)} \AgdaSymbol{=} \AgdaFunction{[} \AgdaFunction{⟦Nat⟧} \AgdaFunction{]fm} \AgdaBound{γ}\<%
\\
\>[0]\AgdaIndent{2}{}\<[2]%
\>[2]\AgdaFunction{elfm} \AgdaSymbol{(}\AgdaBound{γ} \AgdaInductiveConstructor{,} \AgdaInductiveConstructor{arr<} \AgdaBound{a} \AgdaInductiveConstructor{,} \AgdaBound{b} \AgdaInductiveConstructor{>}\AgdaSymbol{)} \AgdaSymbol{=} \AgdaFunction{[} \AgdaBound{Γ} \AgdaFunction{,} \AgdaBound{γ} \AgdaFunction{]} \AgdaFunction{elfm} \AgdaSymbol{(}\AgdaBound{γ} \AgdaInductiveConstructor{,} \AgdaBound{a}\AgdaSymbol{)} \AgdaFunction{⇒fm} \AgdaFunction{elfm} \AgdaSymbol{(}\AgdaBound{γ} \AgdaInductiveConstructor{,} \AgdaBound{b}\AgdaSymbol{)}\<%
\\
\>\<\end{code}
}

\section{What we can do in this model}

\section{Examples of types}

\AgdaHide{
\begin{code}\>\<%
\\
%
\\
\>\AgdaSymbol{\{-\#} \AgdaKeyword{OPTIONS} --type-in-type \AgdaSymbol{\#-\}}\<%
\\
%
\\
\>\AgdaKeyword{import} \AgdaModule{Level}\<%
\\
\>\AgdaKeyword{open} \AgdaKeyword{import} \AgdaModule{Relation.Binary.PropositionalEquality} \AgdaSymbol{as} \AgdaModule{PE} \AgdaKeyword{hiding} \AgdaSymbol{(}refl \AgdaSymbol{;} sym \AgdaSymbol{;} trans\AgdaSymbol{;} isEquivalence\AgdaSymbol{;} [\_]\AgdaSymbol{)}\<%
\\
%
\\
\>\AgdaKeyword{module} \AgdaModule{CwF-ctd} \AgdaSymbol{(}\AgdaBound{ext} \AgdaSymbol{:} \AgdaFunction{Extensionality} \AgdaPrimitive{Level.zero} \AgdaPrimitive{Level.zero}\AgdaSymbol{)} \AgdaKeyword{where}\<%
\\
%
\\
\>\AgdaKeyword{open} \AgdaKeyword{import} \AgdaModule{Data.Unit}\<%
\\
\>\AgdaKeyword{open} \AgdaKeyword{import} \AgdaModule{Function}\<%
\\
\>\AgdaKeyword{open} \AgdaKeyword{import} \AgdaModule{Data.Product}\<%
\\
%
\\
\>\AgdaKeyword{open} \AgdaKeyword{import} \AgdaModule{CwF-setoidwo} \AgdaBound{ext} \AgdaKeyword{public}\<%
\\
%
\\
\>\AgdaKeyword{open} \AgdaKeyword{import} \AgdaModule{Data.Nat}\<%
\\
%
\\
\>\<\end{code}
}

Binary relation

\begin{code}\>\<%
\\
\>\AgdaFunction{Rel} \AgdaSymbol{:} \AgdaSymbol{\{}\AgdaBound{Γ} \AgdaSymbol{:} \AgdaFunction{Con}\AgdaSymbol{\}} \AgdaSymbol{→} \AgdaRecord{Ty} \AgdaBound{Γ} \AgdaSymbol{→} \AgdaPrimitiveType{Set₁}\<%
\\
\>\AgdaFunction{Rel} \AgdaSymbol{\{}\AgdaBound{Γ}\AgdaSymbol{\}} \AgdaBound{A} \AgdaSymbol{=} \AgdaRecord{Ty} \AgdaSymbol{(}\AgdaBound{Γ} \AgdaFunction{\&} \AgdaBound{A} \AgdaFunction{\&} \AgdaBound{A} \AgdaFunction{[} \AgdaFunction{fst\&} \AgdaSymbol{\{}A \AgdaSymbol{=} \AgdaBound{A}\AgdaSymbol{\}} \AgdaFunction{]T}\AgdaSymbol{)}\<%
\\
\>\<\end{code}

Natural numbers

\begin{code}\>\<%
\\
\>\AgdaKeyword{module} \AgdaModule{Natural} \AgdaSymbol{(}\AgdaBound{Γ} \AgdaSymbol{:} \AgdaFunction{Con}\AgdaSymbol{)} \AgdaKeyword{where}\<%
\\
%
\\
\>[0]\AgdaIndent{2}{}\<[2]%
\>[2]\AgdaFunction{\_≈nat\_} \AgdaSymbol{:} \AgdaDatatype{ℕ} \AgdaSymbol{→} \AgdaDatatype{ℕ} \AgdaSymbol{→} \AgdaRecord{HProp}\<%
\\
\>[0]\AgdaIndent{2}{}\<[2]%
\>[2]\AgdaInductiveConstructor{zero} \AgdaFunction{≈nat} \AgdaInductiveConstructor{zero} \AgdaSymbol{=} \AgdaFunction{⊤'}\<%
\\
\>[0]\AgdaIndent{2}{}\<[2]%
\>[2]\AgdaInductiveConstructor{zero} \AgdaFunction{≈nat} \AgdaInductiveConstructor{suc} \AgdaBound{n} \AgdaSymbol{=} \AgdaFunction{⊥'}\<%
\\
\>[0]\AgdaIndent{2}{}\<[2]%
\>[2]\AgdaInductiveConstructor{suc} \AgdaBound{m} \AgdaFunction{≈nat} \AgdaInductiveConstructor{zero} \AgdaSymbol{=} \AgdaFunction{⊥'}\<%
\\
\>[0]\AgdaIndent{2}{}\<[2]%
\>[2]\AgdaInductiveConstructor{suc} \AgdaBound{m} \AgdaFunction{≈nat} \AgdaInductiveConstructor{suc} \AgdaBound{n} \AgdaSymbol{=} \AgdaBound{m} \AgdaFunction{≈nat} \AgdaBound{n}\<%
\\
\>[0]\AgdaIndent{2}{}\<[2]%
\>[2]\<%
\\
\>[0]\AgdaIndent{2}{}\<[2]%
\>[2]\AgdaFunction{reflNat} \AgdaSymbol{:} \AgdaSymbol{\{}\AgdaBound{x} \AgdaSymbol{:} \AgdaDatatype{ℕ}\AgdaSymbol{\}} \AgdaSymbol{→} \AgdaFunction{<} \AgdaBound{x} \AgdaFunction{≈nat} \AgdaBound{x} \AgdaFunction{>} \<[35]%
\>[35]\<%
\\
\>[0]\AgdaIndent{2}{}\<[2]%
\>[2]\AgdaFunction{reflNat} \AgdaSymbol{\{}\AgdaInductiveConstructor{zero}\AgdaSymbol{\}} \AgdaSymbol{=} \AgdaInductiveConstructor{tt}\<%
\\
\>[0]\AgdaIndent{2}{}\<[2]%
\>[2]\AgdaFunction{reflNat} \AgdaSymbol{\{}\AgdaInductiveConstructor{suc} \AgdaBound{n}\AgdaSymbol{\}} \AgdaSymbol{=} \AgdaFunction{reflNat} \AgdaSymbol{\{}\AgdaBound{n}\AgdaSymbol{\}}\<%
\\
%
\\
\>[0]\AgdaIndent{2}{}\<[2]%
\>[2]\AgdaFunction{symNat} \AgdaSymbol{:} \AgdaSymbol{\{}\AgdaBound{x} \AgdaBound{y} \AgdaSymbol{:} \AgdaDatatype{ℕ}\AgdaSymbol{\}} \AgdaSymbol{→} \AgdaFunction{<} \AgdaBound{x} \AgdaFunction{≈nat} \AgdaBound{y} \AgdaFunction{>} \AgdaSymbol{→} \AgdaFunction{<} \AgdaBound{y} \AgdaFunction{≈nat} \AgdaBound{x} \AgdaFunction{>}\<%
\\
\>[0]\AgdaIndent{2}{}\<[2]%
\>[2]\AgdaFunction{symNat} \AgdaSymbol{\{}\AgdaInductiveConstructor{zero}\AgdaSymbol{\}} \AgdaSymbol{\{}\AgdaInductiveConstructor{zero}\AgdaSymbol{\}} \AgdaBound{eq} \AgdaSymbol{=} \AgdaInductiveConstructor{tt}\<%
\\
\>[0]\AgdaIndent{2}{}\<[2]%
\>[2]\AgdaFunction{symNat} \AgdaSymbol{\{}\AgdaInductiveConstructor{zero}\AgdaSymbol{\}} \AgdaSymbol{\{}\AgdaInductiveConstructor{suc} \AgdaSymbol{\_\}} \AgdaBound{eq} \AgdaSymbol{=} \AgdaBound{eq}\<%
\\
\>[0]\AgdaIndent{2}{}\<[2]%
\>[2]\AgdaFunction{symNat} \AgdaSymbol{\{}\AgdaInductiveConstructor{suc} \AgdaSymbol{\_\}} \AgdaSymbol{\{}\AgdaInductiveConstructor{zero}\AgdaSymbol{\}} \AgdaBound{eq} \AgdaSymbol{=} \AgdaBound{eq}\<%
\\
\>[0]\AgdaIndent{2}{}\<[2]%
\>[2]\AgdaFunction{symNat} \AgdaSymbol{\{}\AgdaInductiveConstructor{suc} \AgdaBound{x}\AgdaSymbol{\}} \AgdaSymbol{\{}\AgdaInductiveConstructor{suc} \AgdaBound{y}\AgdaSymbol{\}} \AgdaBound{eq} \AgdaSymbol{=} \AgdaFunction{symNat} \AgdaSymbol{\{}\AgdaBound{x}\AgdaSymbol{\}} \AgdaSymbol{\{}\AgdaBound{y}\AgdaSymbol{\}} \AgdaBound{eq}\<%
\\
%
\\
\>[0]\AgdaIndent{2}{}\<[2]%
\>[2]\AgdaFunction{transNat} \AgdaSymbol{:} \AgdaSymbol{\{}\AgdaBound{x} \AgdaBound{y} \AgdaBound{z} \AgdaSymbol{:} \AgdaDatatype{ℕ}\AgdaSymbol{\}} \AgdaSymbol{→} \AgdaFunction{<} \AgdaBound{x} \AgdaFunction{≈nat} \AgdaBound{y} \AgdaFunction{>} \AgdaSymbol{→} \AgdaFunction{<} \AgdaBound{y} \AgdaFunction{≈nat} \AgdaBound{z} \AgdaFunction{>} \AgdaSymbol{→} \AgdaFunction{<} \AgdaBound{x} \AgdaFunction{≈nat} \AgdaBound{z} \AgdaFunction{>}\<%
\\
\>[0]\AgdaIndent{2}{}\<[2]%
\>[2]\AgdaFunction{transNat} \AgdaSymbol{\{}\AgdaInductiveConstructor{zero}\AgdaSymbol{\}} \AgdaSymbol{\{}\AgdaInductiveConstructor{zero}\AgdaSymbol{\}} \AgdaBound{xy} \AgdaBound{yz} \AgdaSymbol{=} \AgdaBound{yz}\<%
\\
\>[0]\AgdaIndent{2}{}\<[2]%
\>[2]\AgdaFunction{transNat} \AgdaSymbol{\{}\AgdaInductiveConstructor{zero}\AgdaSymbol{\}} \AgdaSymbol{\{}\AgdaInductiveConstructor{suc} \AgdaSymbol{\_\}} \AgdaSymbol{()} \AgdaBound{yz}\<%
\\
\>[0]\AgdaIndent{2}{}\<[2]%
\>[2]\AgdaFunction{transNat} \AgdaSymbol{\{}\AgdaInductiveConstructor{suc} \AgdaSymbol{\_\}} \AgdaSymbol{\{}\AgdaInductiveConstructor{zero}\AgdaSymbol{\}} \AgdaSymbol{()} \AgdaBound{yz}\<%
\\
\>[0]\AgdaIndent{2}{}\<[2]%
\>[2]\AgdaFunction{transNat} \AgdaSymbol{\{}\AgdaInductiveConstructor{suc} \AgdaSymbol{\_\}} \AgdaSymbol{\{}\AgdaInductiveConstructor{suc} \AgdaSymbol{\_\}} \AgdaSymbol{\{}\AgdaInductiveConstructor{zero}\AgdaSymbol{\}} \AgdaBound{xy} \AgdaBound{yz} \AgdaSymbol{=} \AgdaBound{yz}\<%
\\
\>[0]\AgdaIndent{2}{}\<[2]%
\>[2]\AgdaFunction{transNat} \AgdaSymbol{\{}\AgdaInductiveConstructor{suc} \AgdaBound{x}\AgdaSymbol{\}} \AgdaSymbol{\{}\AgdaInductiveConstructor{suc} \AgdaBound{y}\AgdaSymbol{\}} \AgdaSymbol{\{}\AgdaInductiveConstructor{suc} \AgdaBound{z}\AgdaSymbol{\}} \AgdaBound{xy} \AgdaBound{yz} \AgdaSymbol{=} \AgdaFunction{transNat} \AgdaSymbol{\{}\AgdaBound{x}\AgdaSymbol{\}} \AgdaSymbol{\{}\AgdaBound{y}\AgdaSymbol{\}} \AgdaSymbol{\{}\AgdaBound{z}\AgdaSymbol{\}} \AgdaBound{xy} \AgdaBound{yz}\<%
\\
%
\\
\>[0]\AgdaIndent{2}{}\<[2]%
\>[2]\AgdaFunction{⟦Nat⟧} \AgdaSymbol{:} \AgdaRecord{Ty} \AgdaBound{Γ}\<%
\\
\>[0]\AgdaIndent{2}{}\<[2]%
\>[2]\AgdaFunction{⟦Nat⟧} \AgdaSymbol{=} \AgdaKeyword{record} \<[17]%
\>[17]\<%
\\
\>[2]\AgdaIndent{4}{}\<[4]%
\>[4]\AgdaSymbol{\{} \AgdaField{fm} \AgdaSymbol{=} \AgdaSymbol{λ} \AgdaBound{γ} \AgdaSymbol{→} \AgdaKeyword{record}\<%
\\
\>[4]\AgdaIndent{9}{}\<[9]%
\>[9]\AgdaSymbol{\{} \AgdaField{Carrier} \AgdaSymbol{=} \AgdaDatatype{ℕ}\<%
\\
\>[4]\AgdaIndent{9}{}\<[9]%
\>[9]\AgdaSymbol{;} \AgdaField{\_≈h\_} \AgdaSymbol{=} \AgdaFunction{\_≈nat\_}\<%
\\
\>[4]\AgdaIndent{9}{}\<[9]%
\>[9]\AgdaSymbol{;} \AgdaField{refl} \AgdaSymbol{=} \AgdaSymbol{λ} \AgdaSymbol{\{}\AgdaBound{n}\AgdaSymbol{\}} \AgdaSymbol{→} \AgdaFunction{reflNat} \AgdaSymbol{\{}\AgdaBound{n}\AgdaSymbol{\}}\<%
\\
\>[4]\AgdaIndent{9}{}\<[9]%
\>[9]\AgdaSymbol{;} \AgdaField{sym} \AgdaSymbol{=} \AgdaSymbol{λ} \AgdaSymbol{\{}\AgdaBound{x}\AgdaSymbol{\}} \AgdaSymbol{\{}\AgdaBound{y}\AgdaSymbol{\}} \AgdaSymbol{→} \AgdaFunction{symNat} \AgdaSymbol{\{}\AgdaBound{x}\AgdaSymbol{\}} \AgdaSymbol{\{}\AgdaBound{y}\AgdaSymbol{\}}\<%
\\
\>[4]\AgdaIndent{9}{}\<[9]%
\>[9]\AgdaSymbol{;} \AgdaField{trans} \AgdaSymbol{=} \AgdaSymbol{λ} \AgdaSymbol{\{}\AgdaBound{x}\AgdaSymbol{\}} \AgdaSymbol{\{}\AgdaBound{y}\AgdaSymbol{\}} \AgdaSymbol{\{}\AgdaBound{z}\AgdaSymbol{\}} \AgdaSymbol{→} \AgdaFunction{transNat} \AgdaSymbol{\{}\AgdaBound{x}\AgdaSymbol{\}} \AgdaSymbol{\{}\AgdaBound{y}\AgdaSymbol{\}} \AgdaSymbol{\{}\AgdaBound{z}\AgdaSymbol{\}}\<%
\\
\>[4]\AgdaIndent{9}{}\<[9]%
\>[9]\AgdaSymbol{\}}\<%
\\
\>[0]\AgdaIndent{4}{}\<[4]%
\>[4]\AgdaSymbol{;} \AgdaField{substT} \AgdaSymbol{=} \AgdaSymbol{λ} \AgdaBound{\_} \AgdaSymbol{→} \AgdaFunction{id}\<%
\\
\>[0]\AgdaIndent{4}{}\<[4]%
\>[4]\AgdaSymbol{;} \AgdaField{subst*} \AgdaSymbol{=} \AgdaSymbol{λ} \AgdaBound{\_} \AgdaSymbol{→} \AgdaFunction{id}\<%
\\
\>[0]\AgdaIndent{4}{}\<[4]%
\>[4]\AgdaSymbol{;} \AgdaField{refl*} \AgdaSymbol{=} \AgdaSymbol{λ} \AgdaBound{x} \AgdaBound{a} \AgdaSymbol{→} \AgdaFunction{reflNat} \AgdaSymbol{\{}\AgdaBound{a}\AgdaSymbol{\}}\<%
\\
\>[0]\AgdaIndent{4}{}\<[4]%
\>[4]\AgdaSymbol{;} \AgdaField{trans*} \AgdaSymbol{=} \AgdaSymbol{λ} \AgdaBound{a} \AgdaSymbol{→} \AgdaFunction{reflNat} \AgdaSymbol{\{}\AgdaBound{a}\AgdaSymbol{\}} \<[33]%
\>[33]\<%
\\
\>[0]\AgdaIndent{4}{}\<[4]%
\>[4]\AgdaSymbol{\}}\<%
\\
%
\\
\>[0]\AgdaIndent{2}{}\<[2]%
\>[2]\AgdaFunction{⟦0⟧} \AgdaSymbol{:} \AgdaRecord{Tm} \AgdaFunction{⟦Nat⟧}\<%
\\
\>[0]\AgdaIndent{2}{}\<[2]%
\>[2]\AgdaFunction{⟦0⟧} \AgdaSymbol{=} \AgdaKeyword{record}\<%
\\
\>[2]\AgdaIndent{6}{}\<[6]%
\>[6]\AgdaSymbol{\{} \AgdaField{tm} \AgdaSymbol{=} \AgdaSymbol{λ} \AgdaBound{\_} \AgdaSymbol{→} \AgdaNumber{0}\<%
\\
\>[2]\AgdaIndent{6}{}\<[6]%
\>[6]\AgdaSymbol{;} \AgdaField{respt} \AgdaSymbol{=} \AgdaSymbol{λ} \AgdaBound{p} \AgdaSymbol{→} \AgdaInductiveConstructor{tt}\<%
\\
\>[2]\AgdaIndent{6}{}\<[6]%
\>[6]\AgdaSymbol{\}}\<%
\\
%
\\
\>[0]\AgdaIndent{2}{}\<[2]%
\>[2]\AgdaFunction{⟦s⟧} \AgdaSymbol{:} \AgdaRecord{Tm} \AgdaFunction{⟦Nat⟧} \AgdaSymbol{→} \AgdaRecord{Tm} \AgdaFunction{⟦Nat⟧}\<%
\\
\>[0]\AgdaIndent{2}{}\<[2]%
\>[2]\AgdaFunction{⟦s⟧} \AgdaSymbol{(}\AgdaInductiveConstructor{tm:} \AgdaBound{t} \AgdaInductiveConstructor{resp:} \AgdaBound{respt}\AgdaSymbol{)} \<[26]%
\>[26]\<%
\\
\>[2]\AgdaIndent{6}{}\<[6]%
\>[6]\AgdaSymbol{=} \AgdaKeyword{record}\<%
\\
\>[2]\AgdaIndent{6}{}\<[6]%
\>[6]\AgdaSymbol{\{} \AgdaField{tm} \AgdaSymbol{=} \AgdaInductiveConstructor{suc} \AgdaFunction{∘} \AgdaBound{t}\<%
\\
\>[2]\AgdaIndent{6}{}\<[6]%
\>[6]\AgdaSymbol{;} \AgdaField{respt} \AgdaSymbol{=} \AgdaBound{respt}\<%
\\
\>[2]\AgdaIndent{6}{}\<[6]%
\>[6]\AgdaSymbol{\}}\<%
\\
\>\<\end{code}

Simply typed universe

\AgdaHide{
\begin{code}\>\<%
\\
\>[0]\AgdaIndent{2}{}\<[2]%
\>[2]\AgdaKeyword{data} \AgdaDatatype{⟦U⟧⁰} \AgdaSymbol{:} \AgdaPrimitiveType{Set} \AgdaKeyword{where}\<%
\\
\>[0]\AgdaIndent{4}{}\<[4]%
\>[4]\AgdaInductiveConstructor{nat} \AgdaSymbol{:} \AgdaDatatype{⟦U⟧⁰}\<%
\\
\>[0]\AgdaIndent{4}{}\<[4]%
\>[4]\AgdaInductiveConstructor{arr<\_,\_>} \AgdaSymbol{:} \AgdaSymbol{(}\AgdaBound{a} \AgdaBound{b} \AgdaSymbol{:} \AgdaDatatype{⟦U⟧⁰}\AgdaSymbol{)} \AgdaSymbol{→} \AgdaDatatype{⟦U⟧⁰}\<%
\\
%
\\
\>[0]\AgdaIndent{2}{}\<[2]%
\>[2]\AgdaFunction{\_\textasciitilde⟦U⟧\_} \AgdaSymbol{:} \AgdaDatatype{⟦U⟧⁰} \AgdaSymbol{→} \AgdaDatatype{⟦U⟧⁰} \AgdaSymbol{→} \AgdaRecord{HProp}\<%
\\
\>[0]\AgdaIndent{2}{}\<[2]%
\>[2]\AgdaInductiveConstructor{nat} \AgdaFunction{\textasciitilde⟦U⟧} \AgdaInductiveConstructor{nat} \AgdaSymbol{=} \AgdaFunction{⊤'}\<%
\\
\>[0]\AgdaIndent{2}{}\<[2]%
\>[2]\AgdaInductiveConstructor{nat} \AgdaFunction{\textasciitilde⟦U⟧} \AgdaInductiveConstructor{arr<} \AgdaBound{a} \AgdaInductiveConstructor{,} \AgdaBound{b} \AgdaInductiveConstructor{>} \AgdaSymbol{=} \AgdaFunction{⊥'}\<%
\\
\>[0]\AgdaIndent{2}{}\<[2]%
\>[2]\AgdaInductiveConstructor{arr<} \AgdaBound{a} \AgdaInductiveConstructor{,} \AgdaBound{b} \AgdaInductiveConstructor{>} \AgdaFunction{\textasciitilde⟦U⟧} \AgdaInductiveConstructor{nat} \AgdaSymbol{=} \AgdaFunction{⊥'}\<%
\\
\>[0]\AgdaIndent{2}{}\<[2]%
\>[2]\AgdaInductiveConstructor{arr<} \AgdaBound{a} \AgdaInductiveConstructor{,} \AgdaBound{b} \AgdaInductiveConstructor{>} \AgdaFunction{\textasciitilde⟦U⟧} \AgdaInductiveConstructor{arr<} \AgdaBound{a'} \AgdaInductiveConstructor{,} \AgdaBound{b'} \AgdaInductiveConstructor{>} \AgdaSymbol{=} \AgdaBound{a} \AgdaFunction{\textasciitilde⟦U⟧} \AgdaBound{a'} \AgdaFunction{∧} \AgdaBound{b} \AgdaFunction{\textasciitilde⟦U⟧} \AgdaBound{b'}\<%
\\
%
\\
\>[0]\AgdaIndent{2}{}\<[2]%
\>[2]\AgdaFunction{reflU} \AgdaSymbol{:} \<[11]%
\>[11]\AgdaSymbol{\{}\AgdaBound{x} \AgdaSymbol{:} \AgdaDatatype{⟦U⟧⁰}\AgdaSymbol{\}} \AgdaSymbol{→} \AgdaFunction{<} \AgdaBound{x} \AgdaFunction{\textasciitilde⟦U⟧} \AgdaBound{x} \AgdaFunction{>}\<%
\\
\>[0]\AgdaIndent{2}{}\<[2]%
\>[2]\AgdaFunction{reflU} \AgdaSymbol{\{}\AgdaInductiveConstructor{nat}\AgdaSymbol{\}} \AgdaSymbol{=} \AgdaInductiveConstructor{tt}\<%
\\
\>[0]\AgdaIndent{2}{}\<[2]%
\>[2]\AgdaFunction{reflU} \AgdaSymbol{\{}\AgdaInductiveConstructor{arr<} \AgdaBound{a} \AgdaInductiveConstructor{,} \AgdaBound{b} \AgdaInductiveConstructor{>}\AgdaSymbol{\}} \AgdaSymbol{=} \AgdaFunction{reflU} \AgdaSymbol{\{}\AgdaBound{a}\AgdaSymbol{\}} \AgdaInductiveConstructor{,} \AgdaFunction{reflU} \AgdaSymbol{\{}\AgdaBound{b}\AgdaSymbol{\}}\<%
\\
%
\\
\>[0]\AgdaIndent{2}{}\<[2]%
\>[2]\AgdaFunction{symU} \AgdaSymbol{:} \AgdaSymbol{\{}\AgdaBound{x} \AgdaBound{y} \AgdaSymbol{:} \AgdaDatatype{⟦U⟧⁰}\AgdaSymbol{\}} \AgdaSymbol{→} \AgdaFunction{<} \AgdaBound{x} \AgdaFunction{\textasciitilde⟦U⟧} \AgdaBound{y} \AgdaFunction{>} \AgdaSymbol{→} \AgdaFunction{<} \AgdaBound{y} \AgdaFunction{\textasciitilde⟦U⟧} \AgdaBound{x} \AgdaFunction{>}\<%
\\
\>[0]\AgdaIndent{2}{}\<[2]%
\>[2]\AgdaFunction{symU} \AgdaSymbol{\{}\AgdaInductiveConstructor{nat}\AgdaSymbol{\}} \AgdaSymbol{\{}\AgdaInductiveConstructor{nat}\AgdaSymbol{\}} \AgdaBound{eq} \AgdaSymbol{=} \AgdaInductiveConstructor{tt}\<%
\\
\>[0]\AgdaIndent{2}{}\<[2]%
\>[2]\AgdaFunction{symU} \AgdaSymbol{\{}\AgdaInductiveConstructor{nat}\AgdaSymbol{\}} \AgdaSymbol{\{}\AgdaInductiveConstructor{arr<} \AgdaBound{a} \AgdaInductiveConstructor{,} \AgdaBound{b} \AgdaInductiveConstructor{>}\AgdaSymbol{\}} \AgdaBound{eq} \AgdaSymbol{=} \AgdaBound{eq}\<%
\\
\>[0]\AgdaIndent{2}{}\<[2]%
\>[2]\AgdaFunction{symU} \AgdaSymbol{\{}\AgdaInductiveConstructor{arr<} \AgdaBound{a} \AgdaInductiveConstructor{,} \AgdaBound{b} \AgdaInductiveConstructor{>}\AgdaSymbol{\}} \AgdaSymbol{\{}\AgdaInductiveConstructor{nat}\AgdaSymbol{\}} \AgdaBound{eq} \AgdaSymbol{=} \AgdaBound{eq}\<%
\\
\>[0]\AgdaIndent{2}{}\<[2]%
\>[2]\AgdaFunction{symU} \AgdaSymbol{\{}\AgdaInductiveConstructor{arr<} \AgdaBound{a} \AgdaInductiveConstructor{,} \AgdaBound{b} \AgdaInductiveConstructor{>}\AgdaSymbol{\}} \AgdaSymbol{\{}\AgdaInductiveConstructor{arr<} \AgdaBound{a'} \AgdaInductiveConstructor{,} \AgdaBound{b'} \AgdaInductiveConstructor{>}\AgdaSymbol{\}} \AgdaSymbol{(}\AgdaBound{p} \AgdaInductiveConstructor{,} \AgdaBound{q}\AgdaSymbol{)} \AgdaSymbol{=} \AgdaSymbol{(}\AgdaFunction{symU} \AgdaSymbol{\{}\AgdaBound{a}\AgdaSymbol{\}} \AgdaSymbol{\{}\AgdaBound{a'}\AgdaSymbol{\}} \AgdaBound{p}\AgdaSymbol{)} \<[67]%
\>[67]\<%
\\
\>[2]\AgdaIndent{47}{}\<[47]%
\>[47]\AgdaInductiveConstructor{,} \AgdaSymbol{(}\AgdaFunction{symU} \AgdaSymbol{\{}\AgdaBound{b}\AgdaSymbol{\}} \AgdaSymbol{\{}\AgdaBound{b'}\AgdaSymbol{\}} \AgdaBound{q}\AgdaSymbol{)}\<%
\\
%
\\
\>[0]\AgdaIndent{2}{}\<[2]%
\>[2]\AgdaFunction{transU} \AgdaSymbol{:} \AgdaSymbol{\{}\AgdaBound{x} \AgdaBound{y} \AgdaBound{z} \AgdaSymbol{:} \AgdaDatatype{⟦U⟧⁰}\AgdaSymbol{\}} \AgdaSymbol{→} \AgdaFunction{<} \AgdaBound{x} \AgdaFunction{\textasciitilde⟦U⟧} \AgdaBound{y} \AgdaFunction{>} \AgdaSymbol{→} \AgdaFunction{<} \AgdaBound{y} \AgdaFunction{\textasciitilde⟦U⟧} \AgdaBound{z} \AgdaFunction{>} \AgdaSymbol{→} \AgdaFunction{<} \AgdaBound{x} \AgdaFunction{\textasciitilde⟦U⟧} \AgdaBound{z} \AgdaFunction{>}\<%
\\
\>[0]\AgdaIndent{2}{}\<[2]%
\>[2]\AgdaFunction{transU} \AgdaSymbol{\{}\AgdaInductiveConstructor{nat}\AgdaSymbol{\}} \AgdaSymbol{\{}\AgdaInductiveConstructor{nat}\AgdaSymbol{\}} \AgdaBound{eq1} \AgdaBound{eq2} \AgdaSymbol{=} \AgdaBound{eq2}\<%
\\
\>[0]\AgdaIndent{2}{}\<[2]%
\>[2]\AgdaFunction{transU} \AgdaSymbol{\{}\AgdaInductiveConstructor{nat}\AgdaSymbol{\}} \AgdaSymbol{\{}\AgdaInductiveConstructor{arr<} \AgdaBound{a} \AgdaInductiveConstructor{,} \AgdaBound{b} \AgdaInductiveConstructor{>}\AgdaSymbol{\}} \AgdaSymbol{()} \AgdaBound{eq2}\<%
\\
\>[0]\AgdaIndent{2}{}\<[2]%
\>[2]\AgdaFunction{transU} \AgdaSymbol{\{}\AgdaInductiveConstructor{arr<} \AgdaBound{a} \AgdaInductiveConstructor{,} \AgdaBound{b} \AgdaInductiveConstructor{>}\AgdaSymbol{\}} \AgdaSymbol{\{}\AgdaInductiveConstructor{nat}\AgdaSymbol{\}} \AgdaSymbol{()} \AgdaBound{eq2}\<%
\\
\>[0]\AgdaIndent{2}{}\<[2]%
\>[2]\AgdaFunction{transU} \AgdaSymbol{\{}\AgdaInductiveConstructor{arr<} \AgdaBound{a} \AgdaInductiveConstructor{,} \AgdaBound{b} \AgdaInductiveConstructor{>}\AgdaSymbol{\}} \AgdaSymbol{\{}\AgdaInductiveConstructor{arr<} \AgdaBound{a'} \AgdaInductiveConstructor{,} \AgdaBound{b'} \AgdaInductiveConstructor{>}\AgdaSymbol{\}} \AgdaSymbol{\{}\AgdaInductiveConstructor{nat}\AgdaSymbol{\}} \AgdaBound{eq1} \AgdaBound{eq2} \AgdaSymbol{=} \AgdaBound{eq2}\<%
\\
\>[0]\AgdaIndent{2}{}\<[2]%
\>[2]\AgdaFunction{transU} \AgdaSymbol{\{}\AgdaInductiveConstructor{arr<} \AgdaBound{a} \AgdaInductiveConstructor{,} \AgdaBound{b} \AgdaInductiveConstructor{>}\AgdaSymbol{\}} \AgdaSymbol{\{}\AgdaInductiveConstructor{arr<} \AgdaBound{a'} \AgdaInductiveConstructor{,} \AgdaBound{b'} \AgdaInductiveConstructor{>}\AgdaSymbol{\}} \AgdaSymbol{\{}\AgdaInductiveConstructor{arr<} \AgdaBound{a0} \AgdaInductiveConstructor{,} \AgdaBound{b0} \AgdaInductiveConstructor{>}\AgdaSymbol{\}} \AgdaSymbol{(}\AgdaBound{p1} \AgdaInductiveConstructor{,} \AgdaBound{q1}\AgdaSymbol{)} \<[68]%
\>[68]\<%
\\
\>[2]\AgdaIndent{9}{}\<[9]%
\>[9]\AgdaSymbol{(}\AgdaBound{p2} \AgdaInductiveConstructor{,} \AgdaBound{q2}\AgdaSymbol{)} \AgdaSymbol{=} \AgdaSymbol{(}\AgdaFunction{transU} \AgdaSymbol{\{}\AgdaBound{a}\AgdaSymbol{\}} \AgdaSymbol{\{}\AgdaBound{a'}\AgdaSymbol{\}} \AgdaSymbol{\{}\AgdaBound{a0}\AgdaSymbol{\}} \AgdaBound{p1} \AgdaBound{p2}\AgdaSymbol{)} \<[50]%
\>[50]\<%
\\
\>[2]\AgdaIndent{9}{}\<[9]%
\>[9]\AgdaInductiveConstructor{,} \AgdaFunction{transU} \AgdaSymbol{\{}\AgdaBound{b}\AgdaSymbol{\}} \AgdaSymbol{\{}\AgdaBound{b'}\AgdaSymbol{\}} \AgdaSymbol{\{}\AgdaBound{b0}\AgdaSymbol{\}} \AgdaBound{q1} \AgdaBound{q2}\<%
\\
%
\\
\>[0]\AgdaIndent{2}{}\<[2]%
\>[2]\AgdaFunction{⟦U⟧} \AgdaSymbol{:} \AgdaRecord{Ty} \AgdaBound{Γ}\<%
\\
\>[0]\AgdaIndent{2}{}\<[2]%
\>[2]\AgdaFunction{⟦U⟧} \AgdaSymbol{=} \AgdaKeyword{record} \<[15]%
\>[15]\<%
\\
\>[2]\AgdaIndent{4}{}\<[4]%
\>[4]\AgdaSymbol{\{} \AgdaField{fm} \AgdaSymbol{=} \AgdaSymbol{λ} \AgdaBound{γ} \AgdaSymbol{→} \AgdaKeyword{record}\<%
\\
\>[4]\AgdaIndent{9}{}\<[9]%
\>[9]\AgdaSymbol{\{} \AgdaField{Carrier} \AgdaSymbol{=} \AgdaDatatype{⟦U⟧⁰}\<%
\\
\>[4]\AgdaIndent{9}{}\<[9]%
\>[9]\AgdaSymbol{;} \AgdaField{\_≈h\_} \AgdaSymbol{=} \AgdaFunction{\_\textasciitilde⟦U⟧\_}\<%
\\
\>[4]\AgdaIndent{9}{}\<[9]%
\>[9]\AgdaSymbol{;} \AgdaField{refl} \AgdaSymbol{=} \AgdaSymbol{λ} \AgdaSymbol{\{}\AgdaBound{x}\AgdaSymbol{\}} \AgdaSymbol{→} \AgdaFunction{reflU} \AgdaSymbol{\{}\AgdaBound{x}\AgdaSymbol{\}}\<%
\\
\>[4]\AgdaIndent{9}{}\<[9]%
\>[9]\AgdaSymbol{;} \AgdaField{sym} \AgdaSymbol{=} \AgdaSymbol{λ} \AgdaSymbol{\{}\AgdaBound{x}\AgdaSymbol{\}} \AgdaSymbol{\{}\AgdaBound{y}\AgdaSymbol{\}} \AgdaSymbol{→} \AgdaFunction{symU} \AgdaSymbol{\{}\AgdaBound{x}\AgdaSymbol{\}} \AgdaSymbol{\{}\AgdaBound{y}\AgdaSymbol{\}}\<%
\\
\>[4]\AgdaIndent{9}{}\<[9]%
\>[9]\AgdaSymbol{;} \AgdaField{trans} \AgdaSymbol{=} \AgdaSymbol{λ} \AgdaSymbol{\{}\AgdaBound{x}\AgdaSymbol{\}} \AgdaSymbol{\{}\AgdaBound{y}\AgdaSymbol{\}} \AgdaSymbol{\{}\AgdaBound{z}\AgdaSymbol{\}} \AgdaSymbol{→} \AgdaFunction{transU} \AgdaSymbol{\{}\AgdaBound{x}\AgdaSymbol{\}} \AgdaSymbol{\{}\AgdaBound{y}\AgdaSymbol{\}} \AgdaSymbol{\{}\AgdaBound{z}\AgdaSymbol{\}}\<%
\\
\>[4]\AgdaIndent{9}{}\<[9]%
\>[9]\AgdaSymbol{\}}\<%
\\
\>[0]\AgdaIndent{4}{}\<[4]%
\>[4]\AgdaSymbol{;} \AgdaField{substT} \AgdaSymbol{=} \AgdaSymbol{λ} \AgdaBound{\_} \AgdaSymbol{→} \AgdaFunction{id}\<%
\\
\>[0]\AgdaIndent{4}{}\<[4]%
\>[4]\AgdaSymbol{;} \AgdaField{subst*} \AgdaSymbol{=} \AgdaSymbol{λ} \AgdaBound{\_} \AgdaSymbol{→} \AgdaFunction{id}\<%
\\
\>[0]\AgdaIndent{4}{}\<[4]%
\>[4]\AgdaSymbol{;} \AgdaField{refl*} \AgdaSymbol{=} \AgdaSymbol{λ} \AgdaBound{x} \AgdaBound{a} \AgdaSymbol{→} \AgdaFunction{reflU} \AgdaSymbol{\{}\AgdaBound{a}\AgdaSymbol{\}}\<%
\\
\>[0]\AgdaIndent{4}{}\<[4]%
\>[4]\AgdaSymbol{;} \AgdaField{trans*} \AgdaSymbol{=} \AgdaSymbol{λ} \AgdaBound{a} \AgdaSymbol{→} \AgdaFunction{reflU} \AgdaSymbol{\{}\AgdaBound{a}\AgdaSymbol{\}}\<%
\\
\>[0]\AgdaIndent{4}{}\<[4]%
\>[4]\AgdaSymbol{\}}\<%
\\
%
\\
\>[0]\AgdaIndent{2}{}\<[2]%
\>[2]\AgdaFunction{elfm} \AgdaSymbol{:} \AgdaRecord{Σ} \AgdaFunction{∣} \AgdaBound{Γ} \AgdaFunction{∣} \AgdaSymbol{(λ} \AgdaBound{x} \AgdaSymbol{→} \AgdaDatatype{⟦U⟧⁰}\AgdaSymbol{)} \AgdaSymbol{→} \AgdaRecord{HSetoid}\<%
\\
\>[0]\AgdaIndent{2}{}\<[2]%
\>[2]\AgdaFunction{elfm} \AgdaSymbol{(}\AgdaBound{γ} \AgdaInductiveConstructor{,} \AgdaInductiveConstructor{nat}\AgdaSymbol{)} \AgdaSymbol{=} \AgdaFunction{[} \AgdaFunction{⟦Nat⟧} \AgdaFunction{]fm} \AgdaBound{γ}\<%
\\
\>[0]\AgdaIndent{2}{}\<[2]%
\>[2]\AgdaFunction{elfm} \AgdaSymbol{(}\AgdaBound{γ} \AgdaInductiveConstructor{,} \AgdaInductiveConstructor{arr<} \AgdaBound{a} \AgdaInductiveConstructor{,} \AgdaBound{b} \AgdaInductiveConstructor{>}\AgdaSymbol{)} \AgdaSymbol{=} \AgdaFunction{[} \AgdaBound{Γ} \AgdaFunction{,} \AgdaBound{γ} \AgdaFunction{]} \AgdaFunction{elfm} \AgdaSymbol{(}\AgdaBound{γ} \AgdaInductiveConstructor{,} \AgdaBound{a}\AgdaSymbol{)} \AgdaFunction{⇒fm} \AgdaFunction{elfm} \AgdaSymbol{(}\AgdaBound{γ} \AgdaInductiveConstructor{,} \AgdaBound{b}\AgdaSymbol{)}\<%
\\
\>\<\end{code}
}

\AgdaHide{
\begin{code}\>\<%
\\
%
\\
\>\AgdaComment{\{- To do : To find the way to extract the substT from ->

  elsubstT : \{x y : Σ ∣ Γ ∣ (λ x' → ⟦U⟧⁰)\} →
      Σ < [ Γ ] proj₁ x ≈h proj₁ y > (λ x' → < proj₂ x \textasciitilde⟦U⟧ proj₂ y >) →
      ∣ elfm x ∣ → ∣ elfm y ∣
  elsubstT \{\_ , nat\} \{\_ , nat\} \_ x' = x'
  elsubstT \{\_ , nat\} \{\_ , arr< a , b >\} (p , ()) x'
  elsubstT \{\_ , arr< a , b >\} \{\_ , nat\} (p , ()) x'
  elsubstT \{γ , arr< a , b >\} \{γ' , arr< a' , b' >\} (p , qa , qb) (s1 , s2) = 
   \{!!\}

  ⟦El⟧ : Ty (Γ \& ⟦U⟧)
  ⟦El⟧ = record 
       \{ fm = elfm
       ; substT = elsubstT
       ; subst* = \{!!\}
       ; refl* = \{!!\}
       ; trans* = \{!!\} 
       \}

-\}}\<%
\\
\>\<\end{code}
}

The equality type

\begin{code}\>\<%
\\
\>\AgdaKeyword{module} \AgdaModule{Equality-Type} \AgdaSymbol{(}\AgdaBound{Γ} \AgdaSymbol{:} \AgdaFunction{Con}\AgdaSymbol{)(}\AgdaBound{A} \AgdaSymbol{:} \AgdaRecord{Ty} \AgdaBound{Γ}\AgdaSymbol{)} \AgdaKeyword{where}\<%
\\
%
\\
\>[0]\AgdaIndent{2}{}\<[2]%
\>[2]\AgdaFunction{⟦Id⟧} \AgdaSymbol{:} \AgdaFunction{Rel} \AgdaBound{A}\<%
\\
\>[0]\AgdaIndent{2}{}\<[2]%
\>[2]\AgdaFunction{⟦Id⟧} \AgdaSymbol{=} \AgdaKeyword{record} \<[16]%
\>[16]\<%
\\
\>[2]\AgdaIndent{4}{}\<[4]%
\>[4]\AgdaSymbol{\{} \AgdaField{fm} \AgdaSymbol{=} \AgdaSymbol{λ} \AgdaSymbol{\{((}\AgdaBound{x} \AgdaInductiveConstructor{,} \AgdaBound{a}\AgdaSymbol{)} \AgdaInductiveConstructor{,} \AgdaBound{b}\AgdaSymbol{)} \AgdaSymbol{→} \<[30]%
\>[30]\<%
\\
\>[4]\AgdaIndent{13}{}\<[13]%
\>[13]\AgdaKeyword{record} \<[20]%
\>[20]\<%
\\
\>[4]\AgdaIndent{13}{}\<[13]%
\>[13]\AgdaSymbol{\{} \AgdaField{Carrier} \AgdaSymbol{=} \AgdaFunction{[} \AgdaFunction{[} \AgdaBound{A} \AgdaFunction{]fm} \AgdaBound{x} \AgdaFunction{]} \AgdaBound{a} \AgdaFunction{≈} \AgdaBound{b}\<%
\\
\>[4]\AgdaIndent{13}{}\<[13]%
\>[13]\AgdaSymbol{;} \AgdaField{\_≈h\_} \AgdaSymbol{=} \AgdaSymbol{λ} \AgdaBound{\_} \AgdaBound{\_} \AgdaSymbol{→} \AgdaKeyword{record} \AgdaSymbol{\{} \AgdaField{prf} \AgdaSymbol{=} \AgdaRecord{⊤} \AgdaSymbol{;} \AgdaField{Uni} \AgdaSymbol{=} \AgdaInductiveConstructor{PE.refl} \AgdaSymbol{\}}\<%
\\
\>[4]\AgdaIndent{13}{}\<[13]%
\>[13]\AgdaSymbol{;} \AgdaField{refl} \AgdaSymbol{=} \AgdaInductiveConstructor{tt} \<[25]%
\>[25]\<%
\\
\>[4]\AgdaIndent{13}{}\<[13]%
\>[13]\AgdaSymbol{;} \AgdaField{sym} \AgdaSymbol{=} \AgdaFunction{id}\<%
\\
\>[4]\AgdaIndent{13}{}\<[13]%
\>[13]\AgdaSymbol{;} \AgdaField{trans} \AgdaSymbol{=} \AgdaSymbol{λ} \AgdaBound{\_} \AgdaBound{\_} \AgdaSymbol{→} \AgdaInductiveConstructor{tt}\<%
\\
\>[4]\AgdaIndent{13}{}\<[13]%
\>[13]\AgdaSymbol{\}}\<%
\\
\>[4]\AgdaIndent{13}{}\<[13]%
\>[13]\AgdaSymbol{\}}\<%
\\
\>[0]\AgdaIndent{4}{}\<[4]%
\>[4]\AgdaSymbol{;} \AgdaField{substT} \AgdaSymbol{=} \AgdaSymbol{λ} \AgdaSymbol{\{((}\AgdaBound{x} \AgdaInductiveConstructor{,} \AgdaBound{a}\AgdaSymbol{)} \AgdaInductiveConstructor{,} \AgdaBound{b}\AgdaSymbol{)} \AgdaBound{x0} \AgdaSymbol{→} \<[37]%
\>[37]\<%
\\
\>[0]\AgdaIndent{15}{}\<[15]%
\>[15]\AgdaFunction{[} \AgdaFunction{[} \AgdaBound{A} \AgdaFunction{]fm} \AgdaSymbol{\_} \AgdaFunction{]trans} \<[34]%
\>[34]\<%
\\
\>[0]\AgdaIndent{15}{}\<[15]%
\>[15]\AgdaSymbol{(}\AgdaFunction{[} \AgdaFunction{[} \AgdaBound{A} \AgdaFunction{]fm} \AgdaSymbol{\_} \AgdaFunction{]sym} \AgdaBound{a}\AgdaSymbol{)} \<[36]%
\>[36]\<%
\\
\>[0]\AgdaIndent{15}{}\<[15]%
\>[15]\AgdaSymbol{(}\AgdaFunction{[} \AgdaFunction{[} \AgdaBound{A} \AgdaFunction{]fm} \AgdaSymbol{\_} \AgdaFunction{]trans} \<[35]%
\>[35]\<%
\\
\>[0]\AgdaIndent{15}{}\<[15]%
\>[15]\AgdaSymbol{(}\AgdaFunction{[} \AgdaBound{A} \AgdaFunction{]subst*} \AgdaSymbol{\_} \AgdaBound{x0}\AgdaSymbol{)} \AgdaBound{b}\AgdaSymbol{)} \<[37]%
\>[37]\<%
\\
\>[0]\AgdaIndent{15}{}\<[15]%
\>[15]\AgdaSymbol{\}}\<%
\\
\>[0]\AgdaIndent{4}{}\<[4]%
\>[4]\AgdaSymbol{;} \AgdaField{subst*} \AgdaSymbol{=} \AgdaSymbol{λ} \AgdaBound{\_} \AgdaBound{\_} \AgdaSymbol{→} \AgdaInductiveConstructor{tt}\<%
\\
\>[0]\AgdaIndent{4}{}\<[4]%
\>[4]\AgdaSymbol{;} \AgdaField{refl*} \AgdaSymbol{=} \AgdaSymbol{λ} \AgdaBound{\_} \AgdaBound{\_} \AgdaSymbol{→} \AgdaInductiveConstructor{tt}\<%
\\
\>[0]\AgdaIndent{4}{}\<[4]%
\>[4]\AgdaSymbol{;} \AgdaField{trans*} \AgdaSymbol{=} \AgdaSymbol{λ} \AgdaBound{\_} \AgdaSymbol{→} \AgdaInductiveConstructor{tt} \<[24]%
\>[24]\<%
\\
\>[0]\AgdaIndent{4}{}\<[4]%
\>[4]\AgdaSymbol{\}}\<%
\\
%
\\
\>[0]\AgdaIndent{2}{}\<[2]%
\>[2]\AgdaFunction{⟦refl⟧⁰} \AgdaSymbol{:} \AgdaRecord{Tm} \AgdaSymbol{\{}\AgdaBound{Γ} \AgdaFunction{\&} \AgdaBound{A}\AgdaSymbol{\}} \AgdaSymbol{(}\AgdaFunction{⟦Id⟧} \AgdaFunction{[} \AgdaKeyword{record} \AgdaSymbol{\{} \AgdaField{fn} \AgdaSymbol{=} \AgdaSymbol{λ} \AgdaBound{x'} \AgdaSymbol{→} \AgdaBound{x'} \AgdaInductiveConstructor{,} \AgdaFunction{proj₂} \AgdaBound{x'} \<[66]%
\>[66]\<%
\\
\>[0]\AgdaIndent{23}{}\<[23]%
\>[23]\AgdaSymbol{;} \AgdaField{resp} \AgdaSymbol{=} \AgdaSymbol{λ} \AgdaBound{x'} \AgdaSymbol{→} \AgdaBound{x'} \AgdaInductiveConstructor{,} \AgdaFunction{proj₂} \AgdaBound{x'} \AgdaSymbol{\}} \AgdaFunction{]T}\AgdaSymbol{)} \<[59]%
\>[59]\<%
\\
\>[0]\AgdaIndent{2}{}\<[2]%
\>[2]\AgdaFunction{⟦refl⟧⁰} \AgdaSymbol{=} \AgdaKeyword{record}\<%
\\
\>[0]\AgdaIndent{11}{}\<[11]%
\>[11]\AgdaSymbol{\{} \AgdaField{tm} \AgdaSymbol{=} \AgdaSymbol{λ} \AgdaSymbol{\{(}\AgdaBound{x} \AgdaInductiveConstructor{,} \AgdaBound{a}\AgdaSymbol{)} \AgdaSymbol{→} \AgdaFunction{[} \AgdaFunction{[} \AgdaBound{A} \AgdaFunction{]fm} \AgdaBound{x} \AgdaFunction{]refl} \AgdaSymbol{\{}\AgdaBound{a}\AgdaSymbol{\}} \AgdaSymbol{\}}\<%
\\
\>[0]\AgdaIndent{11}{}\<[11]%
\>[11]\AgdaSymbol{;} \AgdaField{respt} \AgdaSymbol{=} \AgdaSymbol{λ} \AgdaBound{p} \AgdaSymbol{→} \AgdaInductiveConstructor{tt}\<%
\\
\>[0]\AgdaIndent{11}{}\<[11]%
\>[11]\AgdaSymbol{\}}\<%
\\
%
\\
\>[0]\AgdaIndent{2}{}\<[2]%
\>[2]\AgdaFunction{⟦refl⟧} \AgdaSymbol{=} \<[12]%
\>[12]\AgdaFunction{lam} \AgdaSymbol{\{}\AgdaBound{Γ}\AgdaSymbol{\}} \AgdaSymbol{\{}\AgdaBound{A}\AgdaSymbol{\}} \AgdaFunction{⟦refl⟧⁰}\<%
\\
\>\<\end{code}

Subst using equality types

\begin{code}\>\<%
\\
%
\\
\>[0]\AgdaIndent{2}{}\<[2]%
\>[2]\AgdaKeyword{module} \AgdaModule{substIn} \AgdaSymbol{(}\AgdaBound{B} \AgdaSymbol{:} \AgdaRecord{Ty} \AgdaSymbol{(}\AgdaBound{Γ} \AgdaFunction{\&} \AgdaBound{A}\AgdaSymbol{))} \AgdaKeyword{where}\<%
\\
\>[0]\AgdaIndent{2}{}\<[2]%
\>[2]\<%
\\
\>[2]\AgdaIndent{4}{}\<[4]%
\>[4]\AgdaFunction{⟦subst⟧⁰} \AgdaSymbol{:} \AgdaRecord{Tm} \AgdaSymbol{\{}\AgdaBound{Γ} \AgdaFunction{\&} \AgdaBound{A} \AgdaFunction{\&} \AgdaSymbol{(}\AgdaBound{A} \AgdaFunction{[} \AgdaFunction{fst\&} \AgdaSymbol{\{}A \AgdaSymbol{=} \AgdaBound{A}\AgdaSymbol{\}} \AgdaFunction{]T}\AgdaSymbol{)} \<[49]%
\>[49]\<%
\\
\>[4]\AgdaIndent{15}{}\<[15]%
\>[15]\AgdaFunction{\&} \AgdaFunction{⟦Id⟧} \AgdaFunction{\&} \AgdaBound{B} \AgdaFunction{[} \AgdaFunction{fst\&} \AgdaSymbol{\{}A \AgdaSymbol{=} \AgdaBound{A} \AgdaFunction{[} \AgdaFunction{fst\&} \AgdaSymbol{\{}A \AgdaSymbol{=} \AgdaBound{A}\AgdaSymbol{\}} \AgdaFunction{]T}\AgdaSymbol{\}} \<[60]%
\>[60]\AgdaFunction{]T} \<[63]%
\>[63]\<%
\\
\>[4]\AgdaIndent{15}{}\<[15]%
\>[15]\AgdaFunction{[} \AgdaFunction{fst\&} \AgdaSymbol{\{}A \AgdaSymbol{=} \AgdaFunction{⟦Id⟧}\AgdaSymbol{\}} \AgdaFunction{]T}\AgdaSymbol{\}} \<[37]%
\>[37]\<%
\\
\>[-2]\AgdaIndent{13}{}\<[13]%
\>[13]\AgdaSymbol{(}\AgdaBound{B} \AgdaFunction{[} \AgdaKeyword{record} \AgdaSymbol{\{} \AgdaField{fn} \AgdaSymbol{=} \AgdaSymbol{λ} \AgdaBound{x} \AgdaSymbol{→} \AgdaSymbol{(}\AgdaFunction{proj₁} \AgdaSymbol{(}\AgdaFunction{proj₁} \AgdaSymbol{(}\AgdaFunction{proj₁} \AgdaSymbol{(}\AgdaFunction{proj₁} \AgdaBound{x}\AgdaSymbol{))))} \<[72]%
\>[72]\<%
\\
\>[0]\AgdaIndent{13}{}\<[13]%
\>[13]\AgdaInductiveConstructor{,} \AgdaSymbol{(}\AgdaFunction{proj₂} \AgdaSymbol{(}\AgdaFunction{proj₁} \AgdaSymbol{(}\AgdaFunction{proj₁} \AgdaBound{x}\AgdaSymbol{)))} \<[41]%
\>[41]\<%
\\
\>[0]\AgdaIndent{13}{}\<[13]%
\>[13]\AgdaSymbol{;} \AgdaField{resp} \AgdaSymbol{=} \AgdaSymbol{λ} \AgdaBound{x} \AgdaSymbol{→} \AgdaFunction{proj₁} \AgdaSymbol{(}\AgdaFunction{proj₁} \AgdaSymbol{(}\AgdaFunction{proj₁} \AgdaSymbol{(}\AgdaFunction{proj₁} \AgdaBound{x}\AgdaSymbol{)))} \<[60]%
\>[60]\<%
\\
\>[0]\AgdaIndent{13}{}\<[13]%
\>[13]\AgdaInductiveConstructor{,} \AgdaFunction{proj₂} \AgdaSymbol{(}\AgdaFunction{proj₁} \AgdaSymbol{(}\AgdaFunction{proj₁} \AgdaBound{x}\AgdaSymbol{))} \AgdaSymbol{\}} \AgdaFunction{]T}\AgdaSymbol{)}\<%
\\
%
\\
\>[0]\AgdaIndent{4}{}\<[4]%
\>[4]\AgdaFunction{⟦subst⟧⁰} \AgdaSymbol{=} \AgdaKeyword{record}\<%
\\
\>[0]\AgdaIndent{11}{}\<[11]%
\>[11]\AgdaSymbol{\{} \AgdaField{tm} \AgdaSymbol{=} \AgdaSymbol{λ} \AgdaSymbol{\{((((}\AgdaBound{x} \AgdaInductiveConstructor{,} \AgdaBound{a}\AgdaSymbol{)} \AgdaInductiveConstructor{,} \AgdaBound{b}\AgdaSymbol{)} \AgdaInductiveConstructor{,} \AgdaBound{p}\AgdaSymbol{)} \AgdaInductiveConstructor{,} \AgdaBound{PA}\AgdaSymbol{)} \AgdaSymbol{→} \AgdaFunction{[} \AgdaBound{B} \AgdaFunction{]subst} \<[61]%
\>[61]\<%
\\
\>[11]\AgdaIndent{18}{}\<[18]%
\>[18]\AgdaSymbol{(}\AgdaFunction{[} \AgdaBound{Γ} \AgdaFunction{]refl} \AgdaInductiveConstructor{,} \AgdaFunction{[} \AgdaFunction{[} \AgdaBound{A} \AgdaFunction{]fm} \AgdaSymbol{\_} \AgdaFunction{]trans} \<[50]%
\>[50]\<%
\\
\>[11]\AgdaIndent{18}{}\<[18]%
\>[18]\AgdaSymbol{(}\AgdaFunction{[} \AgdaBound{A} \AgdaFunction{]refl*} \AgdaSymbol{\_} \AgdaSymbol{\_)} \AgdaBound{p}\AgdaSymbol{)} \AgdaBound{PA} \AgdaSymbol{\}}\<%
\\
\>[0]\AgdaIndent{11}{}\<[11]%
\>[11]\AgdaSymbol{;} \AgdaField{respt} \AgdaSymbol{=} \AgdaSymbol{λ} \AgdaSymbol{\{((((}\AgdaBound{m} \AgdaInductiveConstructor{,} \AgdaBound{a}\AgdaSymbol{)} \AgdaInductiveConstructor{,} \AgdaBound{b}\AgdaSymbol{)} \AgdaInductiveConstructor{,} \AgdaBound{p}\AgdaSymbol{)} \AgdaInductiveConstructor{,} \AgdaBound{PA}\AgdaSymbol{)} \AgdaSymbol{→} \<[53]%
\>[53]\<%
\\
\>[0]\AgdaIndent{13}{}\<[13]%
\>[13]\AgdaFunction{[} \AgdaFunction{[} \AgdaBound{B} \AgdaFunction{]fm} \AgdaSymbol{\_} \AgdaFunction{]trans} \<[32]%
\>[32]\<%
\\
\>[0]\AgdaIndent{13}{}\<[13]%
\>[13]\AgdaSymbol{(}\AgdaFunction{[} \AgdaBound{B} \AgdaFunction{]trans*} \AgdaSymbol{\_)} \<[29]%
\>[29]\<%
\\
\>[13]\AgdaIndent{14}{}\<[14]%
\>[14]\AgdaSymbol{(}\AgdaFunction{[} \AgdaFunction{[} \AgdaBound{B} \AgdaFunction{]fm} \AgdaSymbol{\_} \AgdaFunction{]trans} \<[34]%
\>[34]\<%
\\
\>[0]\AgdaIndent{13}{}\<[13]%
\>[13]\AgdaFunction{[} \AgdaBound{B} \AgdaFunction{]subst-pi} \<[27]%
\>[27]\<%
\\
\>[0]\AgdaIndent{13}{}\<[13]%
\>[13]\AgdaSymbol{(}\AgdaFunction{[} \AgdaFunction{[} \AgdaBound{B} \AgdaFunction{]fm} \AgdaSymbol{\_} \AgdaFunction{]trans} \<[33]%
\>[33]\<%
\\
\>[0]\AgdaIndent{13}{}\<[13]%
\>[13]\AgdaSymbol{(}\AgdaFunction{[} \AgdaFunction{[} \AgdaBound{B} \AgdaFunction{]fm} \AgdaSymbol{\_} \AgdaFunction{]sym} \AgdaSymbol{(}\AgdaFunction{[} \AgdaBound{B} \AgdaFunction{]trans*} \AgdaSymbol{\_))}\<%
\\
\>[0]\AgdaIndent{13}{}\<[13]%
\>[13]\AgdaSymbol{(}\AgdaFunction{[} \AgdaBound{B} \AgdaFunction{]subst*} \AgdaSymbol{\_} \AgdaBound{PA}\AgdaSymbol{)} \AgdaSymbol{))} \AgdaSymbol{\}}\<%
\\
\>[0]\AgdaIndent{11}{}\<[11]%
\>[11]\AgdaSymbol{\}}\<%
\\
%
\\
\>[0]\AgdaIndent{4}{}\<[4]%
\>[4]\AgdaFunction{⟦subst⟧} \AgdaSymbol{=} \AgdaFunction{lam} \AgdaSymbol{\{}A \AgdaSymbol{=} \AgdaSymbol{(}\AgdaBound{A} \AgdaFunction{[} \AgdaFunction{fst\&} \AgdaSymbol{\{}A \AgdaSymbol{=} \AgdaBound{A}\AgdaSymbol{\}} \AgdaFunction{]T}\AgdaSymbol{)\}} \<[46]%
\>[46]\<%
\\
\>[4]\AgdaIndent{14}{}\<[14]%
\>[14]\AgdaSymbol{(}\AgdaFunction{lam} \AgdaSymbol{\{}A \AgdaSymbol{=} \AgdaFunction{⟦Id⟧}\AgdaSymbol{\}} \<[30]%
\>[30]\<%
\\
\>[4]\AgdaIndent{14}{}\<[14]%
\>[14]\AgdaSymbol{(}\AgdaFunction{lam} \AgdaSymbol{\{}A \AgdaSymbol{=} \AgdaBound{B} \AgdaFunction{[} \AgdaFunction{fst\&} \AgdaSymbol{\{}A \AgdaSymbol{=} \AgdaBound{A} \AgdaFunction{[} \AgdaFunction{fst\&} \AgdaSymbol{\{}A \AgdaSymbol{=} \AgdaBound{A}\AgdaSymbol{\}} \AgdaFunction{]T}\AgdaSymbol{\}}\AgdaFunction{]T} \<[61]%
\>[61]\<%
\\
\>[14]\AgdaIndent{15}{}\<[15]%
\>[15]\AgdaFunction{[} \AgdaFunction{fst\&} \AgdaSymbol{\{}A \AgdaSymbol{=} \AgdaFunction{⟦Id⟧}\AgdaSymbol{\}} \AgdaFunction{]T}\AgdaSymbol{\}} \AgdaFunction{⟦subst⟧⁰}\AgdaSymbol{))}\<%
\\
%
\\
\>\<\end{code}


\AgdaHide{
\begin{code}\>\<%
\\
\>\AgdaKeyword{import} \AgdaModule{Level}\<%
\\
\>\AgdaKeyword{open} \AgdaKeyword{import} \AgdaModule{Relation.Binary.PropositionalEquality} \AgdaSymbol{as} \AgdaModule{PE} \AgdaKeyword{hiding} \AgdaSymbol{(}refl \AgdaSymbol{;} sym \AgdaSymbol{;} trans\AgdaSymbol{;} isEquivalence\AgdaSymbol{;} [\_]\AgdaSymbol{)}\<%
\\
%
\\
\>\AgdaKeyword{module} \AgdaModule{CwF-quotient} \AgdaSymbol{(}\AgdaBound{ext} \AgdaSymbol{:} \AgdaFunction{Extensionality} \AgdaPrimitive{Level.zero} \AgdaPrimitive{Level.zero}\AgdaSymbol{)} \AgdaKeyword{where}\<%
\\
%
\\
\>\AgdaKeyword{open} \AgdaKeyword{import} \AgdaModule{Data.Unit}\<%
\\
\>\AgdaKeyword{open} \AgdaKeyword{import} \AgdaModule{Function}\<%
\\
\>\AgdaKeyword{open} \AgdaKeyword{import} \AgdaModule{Data.Product}\<%
\\
\>\AgdaKeyword{open} \AgdaKeyword{import} \AgdaModule{Data.Bool}\<%
\\
\>\AgdaKeyword{open} \AgdaKeyword{import} \AgdaModule{CategoryofSetoid}\<%
\\
\>\AgdaKeyword{open} \AgdaKeyword{import} \AgdaModule{CwF-setoid}\<%
\\
%
\\
\>\<\end{code}
}

Quotient types

\begin{code}\>\<%
\\
\>\AgdaKeyword{module} \AgdaModule{Q} \AgdaSymbol{(}\AgdaBound{Γ} \AgdaSymbol{:} \AgdaFunction{Con}\AgdaSymbol{)(}\AgdaBound{A} \AgdaSymbol{:} \AgdaRecord{Ty} \AgdaBound{Γ}\AgdaSymbol{)}\<%
\\
%
\\
\>[0]\AgdaIndent{9}{}\<[9]%
\>[9]\AgdaSymbol{(}\AgdaBound{R} \AgdaSymbol{:} \AgdaSymbol{(}\AgdaBound{γ} \AgdaSymbol{:} \AgdaFunction{∣} \AgdaBound{Γ} \AgdaFunction{∣}\AgdaSymbol{)} \AgdaSymbol{→} \AgdaFunction{∣} \AgdaFunction{[} \AgdaBound{A} \AgdaFunction{]fm} \AgdaBound{γ} \AgdaFunction{∣} \AgdaSymbol{→} \AgdaFunction{∣} \AgdaFunction{[} \AgdaBound{A} \AgdaFunction{]fm} \AgdaBound{γ} \AgdaFunction{∣} \AgdaSymbol{→} \AgdaPrimitiveType{Set}\AgdaSymbol{)}\<%
\\
%
\\
\>[0]\AgdaIndent{9}{}\<[9]%
\>[9]\AgdaSymbol{.(}\AgdaBound{Rrespt} \AgdaSymbol{:} \AgdaSymbol{∀\{}\AgdaBound{γ} \AgdaBound{γ'} \AgdaSymbol{:} \AgdaFunction{∣} \AgdaBound{Γ} \AgdaFunction{∣}\AgdaSymbol{\}}\<%
\\
\>[9]\AgdaIndent{19}{}\<[19]%
\>[19]\AgdaSymbol{(}\AgdaBound{p} \AgdaSymbol{:} \AgdaFunction{[} \AgdaBound{Γ} \AgdaFunction{]} \AgdaBound{γ} \AgdaFunction{≈} \AgdaBound{γ'}\AgdaSymbol{)}\<%
\\
\>[9]\AgdaIndent{19}{}\<[19]%
\>[19]\AgdaSymbol{(}\AgdaBound{a} \AgdaBound{b} \AgdaSymbol{:} \AgdaFunction{∣} \AgdaFunction{[} \AgdaBound{A} \AgdaFunction{]fm} \AgdaBound{γ} \AgdaFunction{∣}\AgdaSymbol{)} \AgdaSymbol{→}\<%
\\
\>[9]\AgdaIndent{19}{}\<[19]%
\>[19]\AgdaSymbol{.(}\AgdaBound{R} \AgdaBound{γ} \AgdaBound{a} \AgdaBound{b}\AgdaSymbol{)} \AgdaSymbol{→} \<[32]%
\>[32]\<%
\\
\>[9]\AgdaIndent{19}{}\<[19]%
\>[19]\AgdaBound{R} \AgdaBound{γ'} \AgdaSymbol{(}\AgdaFunction{[} \AgdaBound{A} \AgdaFunction{]subst} \AgdaBound{p} \AgdaBound{a}\AgdaSymbol{)} \AgdaSymbol{(}\AgdaFunction{[} \AgdaBound{A} \AgdaFunction{]subst} \AgdaBound{p} \AgdaBound{b}\AgdaSymbol{))}\<%
\\
%
\\
\>[0]\AgdaIndent{9}{}\<[9]%
\>[9]\AgdaSymbol{.(}\AgdaBound{Rrsp} \AgdaSymbol{:} \AgdaSymbol{∀} \AgdaSymbol{\{}\AgdaBound{γ} \AgdaBound{a} \AgdaBound{b}\AgdaSymbol{\}} \AgdaSymbol{→} \AgdaSymbol{.(}\AgdaFunction{[} \AgdaFunction{[} \AgdaBound{A} \AgdaFunction{]fm} \AgdaBound{γ} \AgdaFunction{]} \AgdaBound{a} \AgdaFunction{≈} \AgdaBound{b}\AgdaSymbol{)} \AgdaSymbol{→} \AgdaBound{R} \AgdaBound{γ} \AgdaBound{a} \AgdaBound{b}\AgdaSymbol{)}\<%
\\
%
\\
\>[0]\AgdaIndent{9}{}\<[9]%
\>[9]\AgdaSymbol{.(}\AgdaBound{Rref} \AgdaSymbol{:} \AgdaSymbol{∀} \AgdaSymbol{\{}\AgdaBound{γ} \AgdaBound{a}\AgdaSymbol{\}} \AgdaSymbol{→} \AgdaBound{R} \AgdaBound{γ} \AgdaBound{a} \AgdaBound{a}\AgdaSymbol{)}\<%
\\
\>[0]\AgdaIndent{9}{}\<[9]%
\>[9]\AgdaSymbol{.(}\AgdaBound{Rsym} \AgdaSymbol{:} \AgdaSymbol{(∀} \AgdaSymbol{\{}\AgdaBound{γ} \AgdaBound{a} \AgdaBound{b}\AgdaSymbol{\}} \AgdaSymbol{→} \AgdaSymbol{.(}\AgdaBound{R} \AgdaBound{γ} \AgdaBound{a} \AgdaBound{b}\AgdaSymbol{)} \AgdaSymbol{→} \AgdaBound{R} \AgdaBound{γ} \AgdaBound{b} \AgdaBound{a}\AgdaSymbol{))}\<%
\\
\>[0]\AgdaIndent{9}{}\<[9]%
\>[9]\AgdaSymbol{.(}\AgdaBound{Rtrn} \AgdaSymbol{:} \<[19]%
\>[19]\AgdaSymbol{(∀} \AgdaSymbol{\{}\AgdaBound{γ} \AgdaBound{a} \AgdaBound{b} \AgdaBound{c}\AgdaSymbol{\}} \AgdaSymbol{→} \AgdaSymbol{.(}\AgdaBound{R} \AgdaBound{γ} \AgdaBound{a} \AgdaBound{b}\AgdaSymbol{)}\<%
\\
\>[9]\AgdaIndent{17}{}\<[17]%
\>[17]\AgdaSymbol{→} \<[20]%
\>[20]\AgdaSymbol{.(}\AgdaBound{R} \AgdaBound{γ} \AgdaBound{b} \AgdaBound{c}\AgdaSymbol{)} \AgdaSymbol{→} \AgdaBound{R} \AgdaBound{γ} \AgdaBound{a} \AgdaBound{c}\AgdaSymbol{))}\<%
\\
\>[0]\AgdaIndent{9}{}\<[9]%
\>[9]\AgdaKeyword{where}\<%
\\
%
\\
\>[0]\AgdaIndent{2}{}\<[2]%
\>[2]\AgdaFunction{⟦Q⟧₀} \AgdaSymbol{:} \AgdaFunction{∣} \AgdaBound{Γ} \AgdaFunction{∣} \AgdaSymbol{→} \AgdaRecord{Setoid}\<%
\\
\>[0]\AgdaIndent{2}{}\<[2]%
\>[2]\AgdaFunction{⟦Q⟧₀} \AgdaBound{γ} \AgdaSymbol{=} \AgdaKeyword{record}\<%
\\
\>[2]\AgdaIndent{9}{}\<[9]%
\>[9]\AgdaSymbol{\{} \AgdaField{Carrier} \AgdaSymbol{=} \AgdaFunction{∣} \AgdaFunction{[} \AgdaBound{A} \AgdaFunction{]fm} \AgdaBound{γ} \AgdaFunction{∣}\<%
\\
\>[2]\AgdaIndent{9}{}\<[9]%
\>[9]\AgdaSymbol{;} \AgdaField{\_≈\_} \AgdaSymbol{=} \AgdaBound{R} \AgdaBound{γ}\<%
\\
\>[2]\AgdaIndent{9}{}\<[9]%
\>[9]\AgdaSymbol{;} \AgdaField{refl} \AgdaSymbol{=} \AgdaBound{Rref}\<%
\\
\>[2]\AgdaIndent{9}{}\<[9]%
\>[9]\AgdaSymbol{;} \AgdaField{sym} \AgdaSymbol{=} \AgdaBound{Rsym}\<%
\\
\>[2]\AgdaIndent{9}{}\<[9]%
\>[9]\AgdaSymbol{;} \AgdaField{trans} \AgdaSymbol{=} \AgdaBound{Rtrn}\<%
\\
\>[2]\AgdaIndent{9}{}\<[9]%
\>[9]\AgdaSymbol{\}}\<%
\\
%
\\
%
\\
\>[0]\AgdaIndent{2}{}\<[2]%
\>[2]\AgdaFunction{⟦Q⟧} \AgdaSymbol{:} \AgdaRecord{Ty} \AgdaBound{Γ}\<%
\\
\>[0]\AgdaIndent{2}{}\<[2]%
\>[2]\AgdaFunction{⟦Q⟧} \AgdaSymbol{=} \AgdaKeyword{record} \<[15]%
\>[15]\<%
\\
\>[2]\AgdaIndent{4}{}\<[4]%
\>[4]\AgdaSymbol{\{} \AgdaField{fm} \AgdaSymbol{=} \AgdaFunction{⟦Q⟧₀}\<%
\\
\>[2]\AgdaIndent{4}{}\<[4]%
\>[4]\AgdaSymbol{;} \AgdaField{substT} \AgdaSymbol{=} \AgdaFunction{[} \AgdaBound{A} \AgdaFunction{]subst}\<%
\\
\>[2]\AgdaIndent{4}{}\<[4]%
\>[4]\AgdaSymbol{;} \AgdaField{subst*} \AgdaSymbol{=} \AgdaSymbol{λ} \AgdaBound{p} \AgdaBound{q} \AgdaSymbol{→} \AgdaBound{Rrespt} \AgdaBound{p} \AgdaSymbol{\_} \AgdaSymbol{\_} \AgdaBound{q}\<%
\\
\>[2]\AgdaIndent{4}{}\<[4]%
\>[4]\AgdaSymbol{;} \AgdaField{refl*} \AgdaSymbol{=} \AgdaBound{Rrsp} \AgdaFunction{[} \AgdaBound{A} \AgdaFunction{]refl*}\<%
\\
\>[2]\AgdaIndent{4}{}\<[4]%
\>[4]\AgdaSymbol{;} \AgdaField{trans*} \AgdaSymbol{=} \AgdaSymbol{λ} \AgdaBound{a} \AgdaSymbol{→} \AgdaBound{Rrsp} \AgdaSymbol{(}\AgdaFunction{[} \AgdaBound{A} \AgdaFunction{]trans*} \AgdaSymbol{\_)}\<%
\\
\>[2]\AgdaIndent{4}{}\<[4]%
\>[4]\AgdaSymbol{\}}\<%
\\
%
\\
\>[0]\AgdaIndent{2}{}\<[2]%
\>[2]\AgdaFunction{⟦[\_]⟧} \AgdaSymbol{:} \AgdaRecord{Tm} \AgdaBound{A} \AgdaSymbol{→} \AgdaRecord{Tm} \AgdaFunction{⟦Q⟧}\<%
\\
\>[0]\AgdaIndent{2}{}\<[2]%
\>[2]\AgdaFunction{⟦[} \AgdaBound{x} \AgdaFunction{]⟧} \AgdaSymbol{=} \AgdaKeyword{record}\<%
\\
\>[2]\AgdaIndent{11}{}\<[11]%
\>[11]\AgdaSymbol{\{} \AgdaField{tm} \AgdaSymbol{=} \AgdaFunction{[} \AgdaBound{x} \AgdaFunction{]tm}\<%
\\
\>[2]\AgdaIndent{11}{}\<[11]%
\>[11]\AgdaSymbol{;} \AgdaField{respt} \AgdaSymbol{=} \AgdaSymbol{λ} \AgdaBound{p} \AgdaSymbol{→} \AgdaBound{Rrsp} \AgdaSymbol{(}\AgdaFunction{[} \AgdaBound{x} \AgdaFunction{]respt} \AgdaBound{p}\AgdaSymbol{)}\<%
\\
\>[2]\AgdaIndent{11}{}\<[11]%
\>[11]\AgdaSymbol{\}}\<%
\\
\>[0]\AgdaIndent{2}{}\<[2]%
\>[2]\<%
\\
\>[0]\AgdaIndent{2}{}\<[2]%
\>[2]\AgdaFunction{⟦[\_]⟧'} \AgdaSymbol{:} \AgdaRecord{Tm} \AgdaSymbol{(}\AgdaBound{A} \AgdaFunction{⇒} \AgdaFunction{⟦Q⟧}\AgdaSymbol{)}\<%
\\
\>[0]\AgdaIndent{2}{}\<[2]%
\>[2]\AgdaFunction{⟦[\_]⟧'} \AgdaSymbol{=} \AgdaKeyword{record}\<%
\\
\>[2]\AgdaIndent{11}{}\<[11]%
\>[11]\AgdaSymbol{\{} \AgdaField{tm} \AgdaSymbol{=} \AgdaSymbol{λ} \AgdaBound{x} \AgdaSymbol{→} \AgdaSymbol{(λ} \AgdaBound{a} \AgdaSymbol{→} \AgdaBound{a}\AgdaSymbol{)} \AgdaInductiveConstructor{,} \<[36]%
\>[36]\<%
\\
\>[11]\AgdaIndent{18}{}\<[18]%
\>[18]\AgdaSymbol{(λ} \AgdaBound{a} \AgdaBound{b} \AgdaBound{p} \AgdaSymbol{→} \<[29]%
\>[29]\<%
\\
\>[11]\AgdaIndent{18}{}\<[18]%
\>[18]\AgdaBound{Rrsp} \AgdaSymbol{(}\AgdaFunction{[} \AgdaFunction{[} \AgdaBound{A} \AgdaFunction{]fm} \AgdaSymbol{\_} \AgdaFunction{]trans} \AgdaFunction{[} \AgdaBound{A} \AgdaFunction{]refl*} \AgdaBound{p}\AgdaSymbol{))}\<%
\\
\>[0]\AgdaIndent{11}{}\<[11]%
\>[11]\AgdaSymbol{;} \AgdaField{respt} \AgdaSymbol{=} \AgdaSymbol{λ} \AgdaBound{p} \AgdaBound{a} \AgdaSymbol{→} \AgdaBound{Rrsp} \AgdaFunction{[} \AgdaBound{A} \AgdaFunction{]tr*}\<%
\\
\>[0]\AgdaIndent{11}{}\<[11]%
\>[11]\AgdaSymbol{\}}\<%
\\
\>\<\end{code}

\begin{code}\>\>[0]\AgdaIndent{2}{}\<[2]%
\>[2]\<%
\\
\>[0]\AgdaIndent{2}{}\<[2]%
\>[2]\AgdaSymbol{.}\AgdaFunction{Q-Ax} \AgdaSymbol{:} \AgdaSymbol{∀} \AgdaBound{γ} \AgdaBound{a} \AgdaBound{b} \AgdaSymbol{→} \AgdaFunction{[} \AgdaFunction{[} \AgdaBound{A} \AgdaFunction{]fm} \AgdaBound{γ} \AgdaFunction{]} \AgdaBound{a} \AgdaFunction{≈} \AgdaBound{b} \AgdaSymbol{→} \AgdaFunction{[} \AgdaFunction{[} \AgdaFunction{⟦Q⟧} \AgdaFunction{]fm} \AgdaSymbol{\_} \AgdaFunction{]} \AgdaBound{a} \AgdaFunction{≈} \AgdaBound{b}\<%
\\
\>[0]\AgdaIndent{2}{}\<[2]%
\>[2]\AgdaFunction{Q-Ax} \AgdaBound{γ} \AgdaBound{a} \AgdaBound{b} \AgdaSymbol{=} \AgdaBound{Rrsp}\<%
\\
%
\\
\>[0]\AgdaIndent{2}{}\<[2]%
\>[2]\AgdaFunction{Q-elim} \AgdaSymbol{:} \AgdaSymbol{(}\AgdaBound{B} \AgdaSymbol{:} \AgdaRecord{Ty} \AgdaBound{Γ}\AgdaSymbol{)} \<[22]%
\>[22]\<%
\\
\>[2]\AgdaIndent{9}{}\<[9]%
\>[9]\AgdaSymbol{→} \AgdaSymbol{(}\AgdaBound{f} \AgdaSymbol{:} \AgdaRecord{Tm} \AgdaSymbol{(}\AgdaBound{A} \AgdaFunction{⇒} \AgdaBound{B}\AgdaSymbol{))}\<%
\\
\>[2]\AgdaIndent{9}{}\<[9]%
\>[9]\AgdaSymbol{→} \AgdaSymbol{(}\AgdaBound{frespR} \AgdaSymbol{:} \AgdaSymbol{∀} \AgdaBound{γ} \AgdaBound{a} \AgdaBound{b}\<%
\\
\>[9]\AgdaIndent{18}{}\<[18]%
\>[18]\AgdaSymbol{→} \AgdaSymbol{(}\AgdaBound{R} \AgdaBound{γ} \AgdaBound{a} \AgdaBound{b}\AgdaSymbol{)}\<%
\\
\>[9]\AgdaIndent{18}{}\<[18]%
\>[18]\AgdaSymbol{→} \AgdaFunction{[} \AgdaFunction{[} \AgdaBound{B} \AgdaFunction{]fm} \AgdaBound{γ} \AgdaFunction{]} \AgdaFunction{prj₁} \AgdaSymbol{(}\AgdaFunction{[} \AgdaBound{f} \AgdaFunction{]tm} \AgdaBound{γ}\AgdaSymbol{)} \AgdaBound{a} \<[53]%
\>[53]\<%
\\
\>[18]\AgdaIndent{20}{}\<[20]%
\>[20]\AgdaFunction{≈} \<[23]%
\>[23]\AgdaFunction{prj₁} \AgdaSymbol{(}\AgdaFunction{[} \AgdaBound{f} \AgdaFunction{]tm} \AgdaBound{γ}\AgdaSymbol{)} \AgdaBound{b}\AgdaSymbol{)}\<%
\\
\>[-7]\AgdaIndent{9}{}\<[9]%
\>[9]\AgdaSymbol{→} \AgdaRecord{Tm} \AgdaSymbol{(}\AgdaFunction{⟦Q⟧} \AgdaFunction{⇒} \AgdaBound{B}\AgdaSymbol{)}\<%
\\
\>[0]\AgdaIndent{2}{}\<[2]%
\>[2]\AgdaFunction{Q-elim} \AgdaBound{B} \AgdaBound{f} \AgdaBound{frespR} \AgdaSymbol{=} \AgdaKeyword{record}\<%
\\
\>[2]\AgdaIndent{11}{}\<[11]%
\>[11]\AgdaSymbol{\{} \AgdaField{tm} \AgdaSymbol{=} \AgdaSymbol{λ} \AgdaBound{γ} \AgdaSymbol{→} \AgdaFunction{prj₁} \AgdaSymbol{(}\AgdaFunction{[} \AgdaBound{f} \AgdaFunction{]tm} \AgdaBound{γ}\AgdaSymbol{)} \AgdaInductiveConstructor{,} \AgdaSymbol{(λ} \AgdaBound{a} \AgdaBound{b} \AgdaBound{p} \AgdaSymbol{→} \<[54]%
\>[54]\<%
\\
\>[11]\AgdaIndent{18}{}\<[18]%
\>[18]\AgdaFunction{[} \AgdaFunction{[} \AgdaBound{B} \AgdaFunction{]fm} \AgdaSymbol{\_} \AgdaFunction{]trans} \AgdaFunction{[} \AgdaBound{B} \AgdaFunction{]refl*} \AgdaSymbol{(}\AgdaBound{frespR} \AgdaSymbol{\_} \AgdaSymbol{\_} \AgdaSymbol{\_} \AgdaBound{p}\AgdaSymbol{))}\<%
\\
\>[0]\AgdaIndent{11}{}\<[11]%
\>[11]\AgdaSymbol{;} \AgdaField{respt} \AgdaSymbol{=} \AgdaSymbol{λ} \AgdaSymbol{\{}\AgdaBound{γ}\AgdaSymbol{\}} \AgdaSymbol{\{}\AgdaBound{γ'}\AgdaSymbol{\}} \AgdaBound{p} \AgdaBound{a} \AgdaSymbol{→} \AgdaFunction{[} \AgdaBound{f} \AgdaFunction{]respt} \AgdaBound{p} \AgdaBound{a}\<%
\\
\>[0]\AgdaIndent{11}{}\<[11]%
\>[11]\AgdaSymbol{\}}\<%
\\
%
\\
\>[0]\AgdaIndent{2}{}\<[2]%
\>[2]\<%
\\
\>[0]\AgdaIndent{2}{}\<[2]%
\>[2]\AgdaFunction{substQ} \AgdaSymbol{:} \AgdaSymbol{(}\AgdaBound{Γ} \AgdaFunction{\&} \AgdaBound{A}\AgdaSymbol{)} \AgdaRecord{⇉} \AgdaSymbol{(}\AgdaBound{Γ} \AgdaFunction{\&} \AgdaFunction{⟦Q⟧}\AgdaSymbol{)}\<%
\\
\>[0]\AgdaIndent{2}{}\<[2]%
\>[2]\AgdaFunction{substQ} \AgdaSymbol{=} \AgdaKeyword{record}\<%
\\
\>[2]\AgdaIndent{11}{}\<[11]%
\>[11]\AgdaSymbol{\{} \AgdaField{fn} \AgdaSymbol{=} \AgdaSymbol{λ} \AgdaSymbol{\{(}\AgdaBound{x} \AgdaInductiveConstructor{,} \AgdaBound{a}\AgdaSymbol{)} \AgdaSymbol{→} \AgdaBound{x} \AgdaInductiveConstructor{,} \AgdaBound{a}\AgdaSymbol{\}}\<%
\\
\>[2]\AgdaIndent{11}{}\<[11]%
\>[11]\AgdaSymbol{;} \AgdaField{resp} \AgdaSymbol{=} \AgdaSymbol{λ\{} \AgdaSymbol{(}\AgdaBound{p} \AgdaInductiveConstructor{,} \AgdaBound{q}\AgdaSymbol{)} \AgdaSymbol{→} \AgdaBound{p} \AgdaInductiveConstructor{,} \AgdaSymbol{(}\AgdaBound{Rrsp} \AgdaBound{q}\AgdaSymbol{)\}}\<%
\\
\>[2]\AgdaIndent{11}{}\<[11]%
\>[11]\AgdaSymbol{\}}\<%
\\
%
\\
\>[0]\AgdaIndent{2}{}\<[2]%
\>[2]\AgdaFunction{Q-ind} \AgdaSymbol{:} \AgdaSymbol{(}\AgdaBound{P} \AgdaSymbol{:} \AgdaRecord{Ty} \AgdaSymbol{(}\AgdaBound{Γ} \AgdaFunction{\&} \AgdaFunction{⟦Q⟧}\AgdaSymbol{))}\<%
\\
\>[0]\AgdaIndent{9}{}\<[9]%
\>[9]\AgdaSymbol{→} \AgdaSymbol{(}\AgdaBound{isProp} \AgdaSymbol{:} \AgdaSymbol{∀} \AgdaSymbol{\{}\AgdaBound{x} \AgdaBound{a}\AgdaSymbol{\}} \AgdaSymbol{(}\AgdaBound{r} \AgdaBound{s} \AgdaSymbol{:} \AgdaFunction{∣} \AgdaFunction{[} \AgdaBound{P} \AgdaFunction{]fm} \AgdaSymbol{(}\AgdaBound{x} \AgdaInductiveConstructor{,} \AgdaBound{a}\AgdaSymbol{)} \AgdaFunction{∣}\AgdaSymbol{)} \AgdaSymbol{→} \<[59]%
\>[59]\<%
\\
\>[9]\AgdaIndent{22}{}\<[22]%
\>[22]\AgdaFunction{[} \AgdaFunction{[} \AgdaBound{P} \AgdaFunction{]fm} \AgdaSymbol{(}\AgdaBound{x} \AgdaInductiveConstructor{,} \AgdaBound{a}\AgdaSymbol{)} \AgdaFunction{]} \AgdaBound{r} \AgdaFunction{≈} \AgdaBound{s} \AgdaSymbol{)}\<%
\\
\>[0]\AgdaIndent{9}{}\<[9]%
\>[9]\AgdaSymbol{→} \AgdaSymbol{(}\AgdaBound{h} \AgdaSymbol{:} \AgdaRecord{Tm} \AgdaSymbol{(}\AgdaFunction{Π} \AgdaBound{A} \AgdaSymbol{(}\AgdaBound{P} \AgdaFunction{[} \AgdaFunction{substQ} \AgdaFunction{]T}\AgdaSymbol{)))}\<%
\\
\>[0]\AgdaIndent{9}{}\<[9]%
\>[9]\AgdaSymbol{→} \AgdaRecord{Tm} \AgdaSymbol{(}\AgdaFunction{Π} \AgdaFunction{⟦Q⟧} \AgdaBound{P}\AgdaSymbol{)}\<%
\\
\>[0]\AgdaIndent{2}{}\<[2]%
\>[2]\AgdaFunction{Q-ind} \AgdaBound{P} \AgdaBound{isProp} \AgdaBound{h} \AgdaSymbol{=} \AgdaKeyword{record}\<%
\\
\>[0]\AgdaIndent{11}{}\<[11]%
\>[11]\AgdaSymbol{\{} \AgdaField{tm} \AgdaSymbol{=} \AgdaSymbol{λ} \AgdaBound{x} \AgdaSymbol{→} \AgdaSymbol{(}\AgdaFunction{prj₁} \AgdaSymbol{(}\AgdaFunction{[} \AgdaBound{h} \AgdaFunction{]tm} \AgdaBound{x}\AgdaSymbol{))} \AgdaInductiveConstructor{,} \<[45]%
\>[45]\<%
\\
\>[11]\AgdaIndent{18}{}\<[18]%
\>[18]\AgdaSymbol{(λ} \AgdaBound{a} \AgdaBound{b} \AgdaBound{p} \AgdaSymbol{→} \AgdaBound{isProp} \AgdaSymbol{\{}\AgdaBound{x}\AgdaSymbol{\}} \AgdaSymbol{\{}\AgdaBound{b}\AgdaSymbol{\}} \AgdaSymbol{\_} \AgdaSymbol{\_)}\<%
\\
\>[0]\AgdaIndent{11}{}\<[11]%
\>[11]\AgdaSymbol{;} \AgdaField{respt} \AgdaSymbol{=} \AgdaFunction{[} \AgdaBound{h} \AgdaFunction{]respt}\<%
\\
\>[0]\AgdaIndent{11}{}\<[11]%
\>[11]\AgdaSymbol{\}}\<%
\\
\>\<\end{code}






\section{Quotient types in setoid model}





\section{Observational equality}

%definitional distinct types

Later in in \cite{alti:ott-conf}, Altenkirch and McBride further
simplifies the setoid model by adopting McBride's heterogeneous
approach to equality. They identifies values up to observation rather than
  construction which is called observational equality. It is the
  propositional equality induced by the Setoid model.  In general we have a heterogeneous equality which
  compares terms of types which are different in construction. It only
  make sense when we can prove the types are the same. It helps us
  avoids the heavy use of $subst$ which makes formalisation and
  reasoning involved. We could simplify the setoid model by adapting this
  approach and the implementation could be easier.
